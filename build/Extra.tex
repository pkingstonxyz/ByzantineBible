\section{ΨΑΛΜΟΙ ΣΟΛΟΜΩΝΤΟΣ}
ἐβόησα πρὸς κύριον ἐν τῷ θλίβεσθαί με εἰς τέλος πρὸς τὸν θεὸν ἐν τῷ ἐπιθέσθαι ἁμαρτωλούς
ἐξάπινα ἠκούσθη κραυγὴ πολέμου ἐνώπιόν μου εἶπα ἐπακούσεταί μου ὅτι ἐπλήσθην δικαιοσύνης
ἐλογισάμην ἐν καρδίᾳ μου ὅτι ἐπλήσθην δικαιοσύνης ἐν τῷ εὐθηνῆσαί με καὶ πολλὴν γενέσθαι ἐν τέκνοις
ὁ πλοῦτος αὐτῶν διεδόθη εἰς πᾶσαν τὴν γῆν καὶ ἡ δόξα αὐτῶν ἕως ἐσχάτου τῆς γῆς
ὑψώθησαν ἕως τῶν ἄστρων εἶπαν οὐ μὴ πέσωσιν
καὶ ἐξύβρισαν ἐν τοῖς ἀγαθοῖς αὐτῶν καὶ οὐκ ἤνεγκαν
αἱ ἁμαρτίαι αὐτῶν ἐν ἀποκρύφοις καὶ ἐγὼ οὐκ ᾔδειν
αἱ ἀνομίαι αὐτῶν ὑπὲρ τὰ πρὸ αὐτῶν ἔθνη ἐβεβήλωσαν τὰ ἅγια κυρίου ἐν βεβηλώσει
ψαλμὸς τῷ Σαλωμων περὶ Ιερουσαλημ
ἐν τῷ ὑπερηφανεύεσθαι τὸν ἁμαρτωλὸν ἐν κριῷ κατέβαλε τείχη ὀχυρά καὶ οὐκ ἐκώλυσας
ἀνέβησαν ἐπὶ τὸ θυσιαστήριόν σου ἔθνη ἀλλότρια κατεπατοῦσαν ἐν ὑποδήμασιν αὐτῶν ἐν ὑπερηφανίᾳ
ἀνθ᾽ ὧν οἱ υἱοὶ Ιερουσαλημ ἐμίαναν τὰ ἅγια κυρίου ἐβεβηλοῦσαν τὰ δῶρα τοῦ θεοῦ ἐν ἀνομίαις
ἕνεκεν τούτων εἶπεν ἀπορρίψατε αὐτὰ μακρὰν ἀπ᾽ ἐμοῦ οὐκ εὐδοκῶ ἐν αὐτοῖς
τὸ κάλλος τῆς δόξης αὐτῆς ἐξουθενώθη ἐνώπιον τοῦ θεοῦ ἠτιμώθη ἕως εἰς τέλος
οἱ υἱοὶ καὶ αἱ θυγατέρες ἐν αἰχμαλωσίᾳ πονηρᾷ ἐν σφραγῖδι ὁ τράχηλος αὐτῶν ἐν ἐπισήμῳ ἐν τοῖς ἔθνεσιν
κατὰ τὰς ἁμαρτίας αὐτῶν ἐποίησεν αὐτοῖς ὅτι ἐγκατέλιπεν αὐτοὺς εἰς χεῖρας κατισχυόντων
ἀπέστρεψεν γὰρ τὸ πρόσωπον αὐτοῦ ἀπὸ ἐλέους αὐτῶν νέον καὶ πρεσβύτην καὶ τέκνα αὐτῶν εἰς ἅπαξ ὅτι πονηρὰ ἐποίησαν εἰς ἅπαξ τοῦ μὴ ἀκούειν
καὶ ὁ οὐρανὸς ἐβαρυθύμησεν καὶ ἡ γῆ ἐβδελύξατο αὐτούς ὅτι οὐκ ἐποίησε πᾶς ἄνθρωπος ἐπ᾽ αὐτῆς ὅσα ἐποίησαν
καὶ γνώσεται ἡ γῆ τὰ κρίματά σου πάντα τὰ δίκαια ὁ θεός
ἔστησαν τοὺς υἱοὺς Ιερουσαλημ εἰς ἐμπαιγμὸν ἀντὶ πορνῶν ἐν αὐτῇ πᾶς ὁ παραπορευόμενος εἰσεπορεύετο κατέναντι τοῦ ἡλίου
ἐνέπαιζον ταῖς ἀνομίαις αὐτῶν καθὰ ἐποίουν αὐτοί ἀπέναντι τοῦ ἡλίου παρεδειγμάτισαν ἀδικίας αὐτῶν
καὶ θυγατέρες Ιερουσαλημ βέβηλοι κατὰ τὸ κρίμα σου ἀνθ᾽ ὧν αὐταὶ ἐμιαίωσαν αὑτὰς ἐν φυρμῷ ἀναμείξεως
τὴν κοιλίαν μου καὶ τὰ σπλάγχνα μου πονῶ ἐπὶ τούτοις
ἐγὼ δικαιώσω σε ὁ θεός ἐν εὐθύτητι καρδίας ὅτι ἐν τοῖς κρίμασίν σου ἡ δικαιοσύνη σου ὁ θεός
ὅτι ἀπέδωκας τοῖς ἁμαρτωλοῖς κατὰ τὰ ἔργα αὐτῶν καὶ κατὰ τὰς ἁμαρτίας αὐτῶν τὰς πονηρὰς σφόδρα
ἀνεκάλυψας τὰς ἁμαρτίας αὐτῶν ἵνα φανῇ τὸ κρίμα σου ἐξήλειψας τὸ μνημόσυνον αὐτῶν ἀπὸ τῆς γῆς
ὁ θεὸς κριτὴς δίκαιος καὶ οὐ θαυμάσει πρόσωπον
ὠνείδισαν γὰρ ἔθνη Ιερουσαλημ ἐν καταπατήσει κατεσπάσθη τὸ κάλλος αὐτῆς ἀπὸ θρόνου δόξης
περιεζώσατο σάκκον ἀντὶ ἐνδύματος εὐπρεπείας σχοινίον περὶ τὴν κεφαλὴν αὐτῆς ἀντὶ στεφάνου
περιείλατο μίτραν δόξης ἣν περιέθηκεν αὐτῇ ὁ θεός ἐν ἀτιμίᾳ τὸ κάλλος αὐτῆς ἀπερρίφη ἐπὶ τὴν γῆν
καὶ ἐγὼ εἶδον καὶ ἐδεήθην τοῦ προσώπου κυρίου καὶ εἶπον ἱκάνωσον κύριε τοῦ βαρύνεσθαι χεῖρά σου ἐπὶ Ιερουσαλημ ἐν ἐπαγωγῇ ἐθνῶν
ὅτι ἐνέπαιξαν καὶ οὐκ ἐφείσαντο ἐν ὀργῇ καὶ θυμῷ μετὰ μηνίσεως καὶ συντελεσθήσονται ἐὰν μὴ σύ κύριε ἐπιτιμήσῃς αὐτοῖς ἐν ὀργῇ σου
ὅτι οὐκ ἐν ζήλει ἐποίησαν ἀλλ᾽ ἐν ἐπιθυμίᾳ ψυχῆς ἐκχέαι τὴν ὀργὴν αὐτῶν εἰς ἡμᾶς ἐν ἁρπάγματι
μὴ χρονίσῃς ὁ θεός τοῦ ἀποδοῦναι αὐτοῖς εἰς κεφαλάς τοῦ εἰπεῖν τὴν ὑπερηφανίαν τοῦ δράκοντος ἐν ἀτιμίᾳ
καὶ οὐκ ἐχρόνισα ἕως ἔδειξέν μοι ὁ θεὸς τὴν ὕβριν αὐτοῦ ἐκκεκεντημένον ἐπὶ τῶν ὀρέων Αἰγύπτου ὑπὲρ ἐλάχιστον ἐξουδενωμένον ἐπὶ γῆς καὶ θαλάσσης
τὸ σῶμα αὐτοῦ διαφερόμενον ἐπὶ κυμάτων ἐν ὕβρει πολλῇ καὶ οὐκ ἦν ὁ θάπτων ὅτι ἐξουθένωσεν αὐτὸν ἐν ἀτιμίᾳ
οὐκ ἐλογίσατο ὅτι ἄνθρωπός ἐστιν καὶ τὸ ὕστερον οὐκ ἐλογίσατο
εἶπεν ἐγὼ κύριος γῆς καὶ θαλάσσης ἔσομαι καὶ οὐκ ἐπέγνω ὅτι ὁ θεὸς μέγας κραταιὸς ἐν ἰσχύι αὐτοῦ τῇ μεγάλῃ
αὐτὸς βασιλεὺς ἐπὶ τῶν οὐρανῶν καὶ κρίνων βασιλεῖς καὶ ἀρχάς
ὁ ἀνιστῶν ἐμὲ εἰς δόξαν καὶ κοιμίζων ὑπερηφάνους εἰς ἀπώλειαν αἰῶνος ἐν ἀτιμίᾳ ὅτι οὐκ ἔγνωσαν αὐτόν
καὶ νῦν ἴδετε οἱ μεγιστᾶνες τῆς γῆς τὸ κρίμα τοῦ κυρίου ὅτι μέγας βασιλεὺς καὶ δίκαιος κρίνων τὴν ὑπ᾽ οὐρανόν
εὐλογεῖτε τὸν θεόν οἱ φοβούμενοι τὸν κύριον ἐν ἐπιστήμῃ ὅτι τὸ ἔλεος κυρίου ἐπὶ τοὺς φοβουμένους αὐτὸν μετὰ κρίματος
τοῦ διαστεῖλαι ἀνὰ μέσον δικαίου καὶ ἁμαρτωλοῦ ἀποδοῦναι ἁμαρτωλοῖς εἰς τὸν αἰῶνα κατὰ τὰ ἔργα αὐτῶν
καὶ ἐλεῆσαι δίκαιον ἀπὸ ταπεινώσεως ἁμαρτωλοῦ καὶ ἀποδοῦναι ἁμαρτωλῷ ἀνθ᾽ ὧν ἐποίησεν δικαίῳ
ὅτι χρηστὸς ὁ κύριος τοῖς ἐπικαλουμένοις αὐτὸν ἐν ὑπομονῇ ποιῆσαι κατὰ τὸ ἔλεος αὐτοῦ τοῖς ὁσίοις αὐτοῦ παρεστάναι διὰ παντὸς ἐνώπιον αὐτοῦ ἐν ἰσχύι
εὐλογητὸς κύριος εἰς τὸν αἰῶνα ἐνώπιον δούλων αὐτοῦ
ψαλμὸς τῷ Σαλωμων περὶ δικαίων
ἵνα τί ὑπνοῖς ψυχή καὶ οὐκ εὐλογεῖς τὸν κύριον ὕμνον καινὸν ψάλατε τῷ θεῷ τῷ αἰνετῷ
ψάλλε καὶ γρηγόρησον ἐπὶ τὴν γρηγόρησιν αὐτοῦ ὅτι ἀγαθὸς ψαλμὸς τῷ θεῷ ἐξ ἀγαθῆς καρδίας
δίκαιοι μνημονεύουσιν διὰ παντὸς τοῦ κυρίου ἐν ἐξομολογήσει καὶ δικαιώσει τὰ κρίματα κυρίου
οὐκ ὀλιγωρήσει δίκαιος παιδευόμενος ὑπὸ κυρίου ἡ εὐδοκία αὐτοῦ διὰ παντὸς ἔναντι κυρίου
προσέκοψεν ὁ δίκαιος καὶ ἐδικαίωσεν τὸν κύριον ἔπεσεν καὶ ἀποβλέπει τί ποιήσει αὐτῷ ὁ θεός ἀποσκοπεύει ὅθεν ἥξει σωτηρία αὐτοῦ
ἀλήθεια τῶν δικαίων παρὰ θεοῦ σωτῆρος αὐτῶν οὐκ αὐλίζεται ἐν οἴκῳ δικαίου ἁμαρτία ἐφ᾽ ἁμαρτίαν
ἐπισκέπτεται διὰ παντὸς τὸν οἶκον αὐτοῦ ὁ δίκαιος τοῦ ἐξᾶραι ἀδικίαν ἐν παραπτώματι αὐτοῦ
ἐξιλάσατο περὶ ἀγνοίας ἐν νηστείᾳ καὶ ταπεινώσει ψυχῆς αὐτοῦ καὶ ὁ κύριος καθαρίζει πᾶν ἄνδρα ὅσιον καὶ τὸν οἶκον αὐτοῦ
προσέκοψεν ἁμαρτωλὸς καὶ καταρᾶται ζωὴν αὐτοῦ τὴν ἡμέραν γενέσεως αὐτοῦ καὶ ὠδῖνας μητρός
προσέθηκεν ἁμαρτίας ἐφ᾽ ἁμαρτίας τῇ ζωῇ αὐτοῦ ἔπεσεν ὅτι πονηρὸν τὸ πτῶμα αὐτοῦ καὶ οὐκ ἀναστήσεται
ἡ ἀπώλεια τοῦ ἁμαρτωλοῦ εἰς τὸν αἰῶνα καὶ οὐ μνησθήσεται ὅταν ἐπισκέπτηται δικαίους
αὕτη ἡ μερὶς τῶν ἁμαρτωλῶν εἰς τὸν αἰῶνα οἱ δὲ φοβούμενοι τὸν κύριον ἀναστήσονται εἰς ζωὴν αἰώνιον καὶ ἡ ζωὴ αὐτῶν ἐν φωτὶ κυρίου καὶ οὐκ ἐκλείψει ἔτι
διαλογὴ τοῦ Σαλωμων τοῖς ἀνθρωπαρέσκοις
ἵνα τί σύ βέβηλε κάθησαι ἐν συνεδρίῳ ὁσίων καὶ ἡ καρδία σου μακρὰν ἀφέστηκεν ἀπὸ τοῦ κυρίου ἐν παρανομίαις παροργίζων τὸν θεὸν Ισραηλ
περισσὸς ἐν λόγοις περισσὸς ἐν σημειώσει ὑπὲρ πάντας ὁ σκληρὸς ἐν λόγοις κατακρῖναι ἁμαρτωλοὺς ἐν κρίσει
καὶ ἡ χεὶρ αὐτοῦ ἐν πρώτοις ἐπ᾽ αὐτὸν ὡς ἐν ζήλει καὶ αὐτὸς ἔνοχος ἐν ποικιλίᾳ ἁμαρτιῶν καὶ ἐν ἀκρασίαις
οἱ ὀφθαλμοὶ αὐτοῦ ἐπὶ πᾶσαν γυναῖκα ἄνευ διαστολῆς ἡ γλῶσσα αὐτοῦ ψευδὴς ἐν συναλλάγματι μεθ᾽ ὅρκου
ἐν νυκτὶ καὶ ἐν ἀποκρύφοις ἁμαρτάνει ὡς οὐχ ὁρώμενος ἐν ὀφθαλμοῖς αὐτοῦ λαλεῖ πάσῃ γυναικὶ ἐν συνταγῇ κακίας ταχὺς εἰσόδῳ εἰς πᾶσαν οἰκίαν ἐν ἱλαρότητι ὡς ἄκακος
ἐξάραι ὁ θεὸς τοὺς ἐν ὑποκρίσει ζῶντας μετὰ ὁσίων ἐν φθορᾷ σαρκὸς αὐτοῦ καὶ πενίᾳ τὴν ζωὴν αὐτοῦ
ἀνακαλύψαι ὁ θεὸς τὰ ἔργα ἀνθρώπων ἀνθρωπαρέσκων ἐν καταγέλωτι καὶ μυκτηρισμῷ τὰ ἔργα αὐτοῦ
καὶ δικαιώσαισαν ὅσιοι τὸ κρίμα τοῦ θεοῦ αὐτῶν ἐν τῷ ἐξαίρεσθαι ἁμαρτωλοὺς ἀπὸ προσώπου δικαίου ἀνθρωπάρεσκον λαλοῦντα νόμον μετὰ δόλου
καὶ οἱ ὀφθαλμοὶ αὐτῶν ἐπ᾽ οἶκον ἀνδρὸς ἐν εὐσταθείᾳ ὡς ὄφις διαλῦσαι σοφίαν ἀλλήλων ἐν λόγοις παρανόμων
οἱ λόγοι αὐτοῦ παραλογισμοὶ εἰς πρᾶξιν ἐπιθυμίας ἀδίκου οὐκ ἀπέστη ἕως ἐνίκησεν σκορπίσαι ὡς ἐν ὀρφανίᾳ
καὶ ἠρήμωσεν οἶκον ἕνεκεν ἐπιθυμίας παρανόμου παρελογίσατο ἐν λόγοις ὅτι οὐκ ἔστιν ὁρῶν καὶ κρίνων
ἐπλήσθη ἐν παρανομίᾳ ἐν ταύτῃ καὶ οἱ ὀφθαλμοὶ αὐτοῦ ἐπ᾽ οἶκον ἕτερον ὀλεθρεῦσαι ἐν λόγοις ἀναπτερώσεως
οὐκ ἐμπίπλαται ἡ ψυχὴ αὐτοῦ ὡς ᾅδης ἐν πᾶσι τούτοις
γένοιτο κύριε ἡ μερὶς αὐτοῦ ἐν ἀτιμίᾳ ἐνώπιόν σου ἡ ἔξοδος αὐτοῦ ἐν στεναγμοῖς καὶ ἡ εἴσοδος αὐτοῦ ἐν ἀρᾷ
ἐν ὀδύναις καὶ πενίᾳ καὶ ἀπορίᾳ ἡ ζωὴ αὐτοῦ κύριε ὁ ὕπνος αὐτοῦ ἐν λύπαις καὶ ἡ ἐξέγερσις αὐτοῦ ἐν ἀπορίαις
ἀφαιρεθείη ὕπνος ἀπὸ κροτάφων αὐτοῦ ἐν νυκτί ἀποπέσοι ἀπὸ παντὸς ἔργου χειρῶν αὐτοῦ ἐν ἀτιμίᾳ
κενὸς χερσὶν αὐτοῦ εἰσέλθοι εἰς τὸν οἶκον αὐτοῦ καὶ ἐλλιπὴς ὁ οἶκος αὐτοῦ ἀπὸ παντός οὗ ἐμπλήσει ψυχὴν αὐτοῦ
ἐν μονώσει ἀτεκνίας τὸ γῆρας αὐτοῦ εἰς ἀνάλημψιν
σκορπισθείησαν σάρκες ἀνθρωπαρέσκων ὑπὸ θηρίων καὶ ὀστᾶ παρανόμων κατέναντι τοῦ ἡλίου ἐν ἀτιμίᾳ
ὀφθαλμοὺς ἐκκόψαισαν κόρακες ὑποκρινομένων ὅτι ἠρήμωσαν οἴκους πολλοὺς ἀνθρώπων ἐν ἀτιμίᾳ καὶ ἐσκόρπισαν ἐν ἐπιθυμίᾳ
καὶ οὐκ ἐμνήσθησαν θεοῦ καὶ οὐκ ἐφοβήθησαν τὸν θεὸν ἐν ἅπασι τούτοις καὶ παρώργισαν τὸν θεὸν καὶ παρώξυναν
ἐξάραι αὐτοὺς ἀπὸ τῆς γῆς ὅτι ψυχὰς ἀκάκων παραλογισμῷ ὑπεκρίνοντο
μακάριοι οἱ φοβούμενοι τὸν κύριον ἐν ἀκακίᾳ αὐτῶν ὁ κύριος ῥύσεται αὐτοὺς ἀπὸ ἀνθρώπων δολίων καὶ ἁμαρτωλῶν καὶ ῥύσεται ἡμᾶς ἀπὸ παντὸς σκανδάλου παρανόμου
ἐξάραι ὁ θεὸς τοὺς ποιοῦντας ἐν ὑπερηφανίᾳ πᾶσαν ἀδικίαν ὅτι κριτὴς μέγας καὶ κραταιὸς κύριος ὁ θεὸς ἡμῶν ἐν δικαιοσύνῃ
γένοιτο κύριε τὸ ἔλεός σου ἐπὶ πάντας τοὺς ἀγαπῶντάς σε
ψαλμὸς τῷ Σαλωμων
κύριε ὁ θεός αἰνέσω τῷ ὀνόματί σου ἐν ἀγαλλιάσει ἐν μέσῳ ἐπισταμένων τὰ κρίματά σου τὰ δίκαια
ὅτι σὺ χρηστὸς καὶ ἐλεήμων ἡ καταφυγὴ τοῦ πτωχοῦ ἐν τῷ κεκραγέναι με πρὸς σὲ μὴ παρασιωπήσῃς ἀπ᾽ ἐμοῦ
οὐ γὰρ λήψεταί τις σκῦλα παρὰ ἀνδρὸς δυνατοῦ καὶ τίς λήψεται ἀπὸ πάντων ὧν ἐποίησας ἐὰν μὴ σὺ δῷς
ὅτι ἄνθρωπος καὶ ἡ μερὶς αὐτοῦ παρὰ σοῦ ἐν σταθμῷ οὐ προσθήσει τοῦ πλεονάσαι παρὰ τὸ κρίμα σου ὁ θεός
ἐν τῷ θλίβεσθαι ἡμᾶς ἐπικαλεσόμεθά σε εἰς βοήθειαν καὶ σὺ οὐκ ἀποστρέψῃ τὴν δέησιν ἡμῶν ὅτι σὺ ὁ θεὸς ἡμῶν εἶ
μὴ βαρύνῃς τὴν χεῖρά σου ἐφ᾽ ἡμᾶς ἵνα μὴ δι᾽ ἀνάγκην ἁμάρτωμεν
καὶ ἐὰν μὴ ἐπιστρέψῃς ἡμᾶς οὐκ ἀφεξόμεθα ἀλλ᾽ ἐπὶ σὲ ἥξομεν
ἐὰν γὰρ πεινάσω πρὸς σὲ κεκράξομαι ὁ θεός καὶ σὺ δώσεις μοι
τὰ πετεινὰ καὶ τοὺς ἰχθύας σὺ τρέφεις ἐν τῷ διδόναι σε ὑετὸν ἐρήμοις εἰς ἀνατολὴν χλόης
ἡτοίμασας χορτάσματα ἐν ἐρήμῳ παντὶ ζῶντι καὶ ἐὰν πεινάσωσιν πρὸς σὲ ἀροῦσιν πρόσωπον αὐτῶν
τοὺς βασιλεῖς καὶ ἄρχοντας καὶ λαοὺς σὺ τρέφεις ὁ θεός καὶ πτωχοῦ καὶ πένητος ἡ ἐλπὶς τίς ἐστιν εἰ μὴ σύ κύριε
καὶ σὺ ἐπακούσῃ ὅτι τίς χρηστὸς καὶ ἐπιεικὴς ἀλλ᾽ ἢ σὺ εὐφρᾶναι ψυχὴν ταπεινοῦ ἐν τῷ ἀνοῖξαι χεῖρά σου ἐν ἐλέει
ἡ χρηστότης ἀνθρώπου ἐν φειδοῖ καὶ ἡ αὔριον καὶ ἐὰν δευτερώσῃ ἄνευ γογγυσμοῦ καὶ τοῦτο θαυμάσειας
τὸ δὲ δόμα σου πολὺ μετὰ χρηστότητος καὶ πλούσιον καὶ οὗ ἐστιν ἡ ἐλπὶς ἐπὶ σέ οὐ φείσεται ἐν δόματι
ἐπὶ πᾶσαν τὴν γῆν τὸ ἔλεός σου κύριε ἐν χρηστότητι
μακάριος οὗ μνημονεύει ὁ θεὸς ἐν συμμετρίᾳ αὐταρκείας ἐὰν ὑπερπλεονάσῃ ὁ ἄνθρωπος ἐξαμαρτάνει
ἱκανὸν τὸ μέτριον ἐν δικαιοσύνῃ καὶ ἐν τούτῳ ἡ εὐλογία κυρίου εἰς πλησμονὴν ἐν δικαιοσύνῃ
εὐφρανθείησαν οἱ φοβούμενοι κύριον ἐν ἀγαθοῖς καὶ ἡ χρηστότης σου ἐπὶ Ισραηλ ἐν τῇ βασιλείᾳ σου
εὐλογημένη ἡ δόξα κυρίου ὅτι αὐτὸς βασιλεὺς ἡμῶν
ἐν ἐλπίδι τῷ Σαλωμων
μακάριος ἀνήρ οὗ ἡ καρδία αὐτοῦ ἑτοίμη ἐπικαλέσασθαι τὸ ὄνομα κυρίου ἐν τῷ μνημονεύειν αὐτὸν τὸ ὄνομα κυρίου σωθήσεται
αἱ ὁδοὶ αὐτοῦ κατευθύνονται ὑπὸ κυρίου καὶ πεφυλαγμένα ἔργα χειρῶν αὐτοῦ ὑπὸ κυρίου θεοῦ αὐτοῦ
ἀπὸ ὁράσεως πονηρῶν ἐνυπνίων αὐτοῦ οὐ ταραχθήσεται ἡ ψυχὴ αὐτοῦ ἐν διαβάσει ποταμῶν καὶ σάλῳ θαλασσῶν οὐ πτοηθήσεται
ἐξανέστη ἐξ ὕπνου αὐτοῦ καὶ ηὐλόγησεν τῷ ὀνόματι κυρίου ἐπ᾽ εὐσταθείᾳ καρδίας αὐτοῦ ἐξύμνησεν τῷ ὀνόματι τοῦ θεοῦ αὐτοῦ
καὶ ἐδεήθη τοῦ προσώπου κυρίου περὶ παντὸς τοῦ οἴκου αὐτοῦ καὶ κύριος εἰσήκουσεν προσευχὴν παντὸς ἐν φόβῳ θεοῦ
καὶ πᾶν αἴτημα ψυχῆς ἐλπιζούσης πρὸς αὐτὸν ἐπιτελεῖ ὁ κύριος εὐλογητὸς κύριος ὁ ποιῶν ἔλεος τοῖς ἀγαπῶσιν αὐτὸν ἐν ἀληθείᾳ
τῷ Σαλωμων ἐπιστροφῆς
μὴ ἀποσκηνώσῃς ἀφ᾽ ἡμῶν ὁ θεός ἵνα μὴ ἐπιθῶνται ἡμῖν οἳ ἐμίσησαν ἡμᾶς δωρεάν
ὅτι ἀπώσω αὐτούς ὁ θεός μὴ πατησάτω ὁ ποὺς αὐτῶν κληρονομίαν ἁγιάσματός σου
σὺ ἐν θελήματί σου παίδευσον ἡμᾶς καὶ μὴ δῷς ἔθνεσιν
ἐὰν γὰρ ἀποστείλῃς θάνατον σὺ ἐντελῇ αὐτῷ περὶ ἡμῶν
ὅτι σὺ ἐλεήμων καὶ οὐκ ὀργισθήσῃ τοῦ συντελέσαι ἡμᾶς
ἐν τῷ κατασκηνοῦν τὸ ὄνομά σου ἐν μέσῳ ἡμῶν ἐλεηθησόμεθα καὶ οὐκ ἰσχύσει πρὸς ἡμᾶς ἔθνος
ὅτι σὺ ὑπερασπιστὴς ἡμῶν καὶ ἡμεῖς ἐπικαλεσόμεθά σε καὶ σὺ ἐπακούσῃ ἡμῶν
ὅτι σὺ οἰκτιρήσεις τὸ γένος Ισραηλ εἰς τὸν αἰῶνα καὶ οὐκ ἀπώσῃ
καὶ ἡμεῖς ὑπὸ ζυγόν σου τὸν αἰῶνα καὶ μάστιγα παιδείας σου
κατευθυνεῖς ἡμᾶς ἐν καιρῷ ἀντιλήψεώς σου τοῦ ἐλεῆσαι τὸν οἶκον Ιακωβ εἰς ἡμέραν ἐν ᾗ ἐπηγγείλω αὐτοῖς
τῷ Σαλωμων εἰς νεῖκος
θλῖψιν καὶ φωνὴν πολέμου ἤκουσεν τὸ οὖς μου φωνὴν σάλπιγγος ἠχούσης σφαγὴν καὶ ὄλεθρον
φωνὴ λαοῦ πολλοῦ ὡς ἀνέμου πολλοῦ σφόδρα ὡς καταιγὶς πυρὸς πολλοῦ φερομένου δι᾽ ἐρήμου
καὶ εἶπα ἐν τῇ καρδίᾳ μου ποῦ ἄρα κρινεῖ αὐτὸν ὁ θεός
φωνὴν ἤκουσα εἰς Ιερουσαλημ πόλιν ἁγιάσματος
συνετρίβη ἡ ὀσφύς μου ἀπὸ ἀκοῆς παρελύθη γόνατά μου ἐφοβήθη ἡ καρδία μου ἐταράχθη τὰ ὀστᾶ μου ὡς λίνον
εἶπα κατευθυνοῦσιν ὁδοὺς αὐτῶν ἐν δικαιοσύνῃ
ἀνελογισάμην τὰ κρίματα τοῦ θεοῦ ἀπὸ κτίσεως οὐρανοῦ καὶ γῆς ἐδικαίωσα τὸν θεὸν ἐν τοῖς κρίμασιν αὐτοῦ τοῖς ἀπ᾽ αἰῶνος
ἀνεκάλυψεν ὁ θεὸς τὰς ἁμαρτίας αὐτῶν ἐναντίον τοῦ ἡλίου ἔγνω πᾶσα ἡ γῆ τὰ κρίματα τοῦ θεοῦ τὰ δίκαια
ἐν καταγαίοις κρυφίοις αἱ παρανομίαι αὐτῶν ἐν παροργισμῷ υἱὸς μετὰ μητρὸς καὶ πατὴρ μετὰ θυγατρὸς συνεφύροντο
ἐμοιχῶντο ἕκαστος τὴν γυναῖκα τοῦ πλησίον αὐτοῦ συνέθεντο αὑτοῖς συνθήκας μετὰ ὅρκου περὶ τούτων
τὰ ἅγια τοῦ θεοῦ διηρπάζοσαν ὡς μὴ ὄντος κληρονόμου λυτρουμένου
ἐπατοῦσαν τὸ θυσιαστήριον κυρίου ἀπὸ πάσης ἀκαθαρσίας καὶ ἐν ἀφέδρῳ αἵματος ἐμίαναν τὰς θυσίας ὡς κρέα βέβηλα
οὐ παρέλιπον ἁμαρτίαν ἣν οὐκ ἐποίησαν ὑπὲρ τὰ ἔθνη
διὰ τοῦτο ἐκέρασεν αὐτοῖς ὁ θεὸς πνεῦμα πλανήσεως ἐπότισεν αὐτοὺς ποτήριον οἴνου ἀκράτου εἰς μέθην
ἤγαγεν τὸν ἀπ᾽ ἐσχάτου τῆς γῆς τὸν παίοντα κραταιῶς ἔκρινεν τὸν πόλεμον ἐπὶ Ιερουσαλημ καὶ τὴν γῆν αὐτῆς
ἀπήντησαν αὐτῷ οἱ ἄρχοντες τῆς γῆς μετὰ χαρᾶς εἶπαν αὐτῷ ἐπευκτὴ ἡ ὁδός σου δεῦτε εἰσέλθατε μετ᾽ εἰρήνης
ὡμάλισαν ὁδοὺς τραχείας ἀπὸ εἰσόδου αὐτοῦ ἤνοιξαν πύλας ἐπὶ Ιερουσαλημ ἐστεφάνωσαν τείχη αὐτῆς
εἰσῆλθεν ὡς πατὴρ εἰς οἶκον υἱῶν αὐτοῦ μετ᾽ εἰρήνης ἔστησεν τοὺς πόδας αὐτοῦ μετὰ ἀσφαλείας πολλῆς
κατελάβετο τὰς πυργοβάρεις αὐτῆς καὶ τὸ τεῖχος Ιερουσαλημ ὅτι ὁ θεὸς ἤγαγεν αὐτὸν μετὰ ἀσφαλείας ἐν τῇ πλανήσει αὐτῶν
ἀπώλεσεν ἄρχοντας αὐτῶν καὶ πᾶν σοφὸν ἐν βουλῇ ἐξέχεεν τὸ αἷμα τῶν οἰκούντων Ιερουσαλημ ὡς ὕδωρ ἀκαθαρσίας
ἀπήγαγεν τοὺς υἱοὺς καὶ τὰς θυγατέρας αὐτῶν ἃ ἐγέννησαν ἐν βεβηλώσει
ἐποίησαν κατὰ τὰς ἀκαθαρσίας αὐτῶν καθὼς οἱ πατέρες αὐτῶν ἐμίαναν Ιερουσαλημ καὶ τὰ ἡγιασμένα τῷ ὀνόματι τοῦ θεοῦ
ἐδικαιώθη ὁ θεὸς ἐν τοῖς κρίμασιν αὐτοῦ ἐν τοῖς ἔθνεσιν τῆς γῆς καὶ οἱ ὅσιοι τοῦ θεοῦ ὡς ἀρνία ἐν ἀκακίᾳ ἐν μέσῳ αὐτῶν
αἰνετὸς κύριος ὁ κρίνων πᾶσαν τὴν γῆν ἐν δικαιοσύνῃ αὐτοῦ
ἰδοὺ δή ὁ θεός ἔδειξας ἡμῖν τὸ κρίμα σου ἐν τῇ δικαιοσύνῃ σου εἴδοσαν οἱ ὀφθαλμοὶ ἡμῶν τὰ κρίματά σου ὁ θεός
ἐδικαιώσαμεν τὸ ὄνομά σου τὸ ἔντιμον εἰς αἰῶνας ὅτι σὺ ὁ θεὸς τῆς δικαιοσύνης κρίνων τὸν Ισραηλ ἐν παιδείᾳ
ἐπίστρεψον ὁ θεός τὸ ἔλεός σου ἐφ᾽ ἡμᾶς καὶ οἰκτίρησον ἡμᾶς
συνάγαγε τὴν διασπορὰν Ισραηλ μετὰ ἐλέους καὶ χρηστότητος ὅτι ἡ πίστις σου μεθ᾽ ἡμῶν
καὶ ἡμεῖς ἐσκληρύναμεν τὸν τράχηλον ἡμῶν καὶ σὺ παιδευτὴς ἡμῶν εἶ
μὴ ὑπερίδῃς ἡμᾶς ὁ θεὸς ἡμῶν ἵνα μὴ καταπίωσιν ἡμᾶς ἔθνη ὡς μὴ ὄντος λυτρουμένου
καὶ σὺ ὁ θεὸς ἡμῶν ἀπ᾽ ἀρχῆς καὶ ἐπὶ σὲ ἡ ἐλπὶς ἡμῶν κύριε
καὶ ἡμεῖς οὐκ ἀφεξόμεθά σου ὅτι χρηστὰ τὰ κρίματά σου ἐφ᾽ ἡμᾶς
ἡμῖν καὶ τοῖς τέκνοις ἡμῶν ἡ εὐδοκία εἰς τὸν αἰῶνα κύριε σωτὴρ ἡμῶν οὐ σαλευθησόμεθα ἔτι τὸν αἰῶνα χρόνον
αἰνετὸς κύριος ἐν τοῖς κρίμασιν αὐτοῦ ἐν στόματι ὁσίων καὶ εὐλογημένος Ισραηλ ὑπὸ κυρίου εἰς τὸν αἰῶνα
τῷ Σαλωμων εἰς ἔλεγχον
ἐν τῷ ἀπαχθῆναι Ισραηλ ἐν ἀποικεσίᾳ εἰς γῆν ἀλλοτρίαν ἐν τῷ ἀποστῆναι αὐτοὺς ἀπὸ κυρίου τοῦ λυτρωσαμένου αὐτοὺς ἀπερρίφησαν ἀπὸ κληρονομίας ἧς ἔδωκεν αὐτοῖς κύριος
ἐν παντὶ ἔθνει ἡ διασπορὰ τοῦ Ισραηλ κατὰ τὸ ῥῆμα τοῦ θεοῦ ἵνα δικαιωθῇς ὁ θεός ἐν τῇ δικαιοσύνῃ σου ἐν ταῖς ἀνομίαις ἡμῶν ὅτι σὺ κριτὴς δίκαιος ἐπὶ πάντας τοὺς λαοὺς τῆς γῆς
οὐ γὰρ κρυβήσεται ἀπὸ τῆς γνώσεώς σου πᾶς ποιῶν ἄδικα καὶ αἱ δικαιοσύναι τῶν ὁσίων σου ἐνώπιόν σου κύριε καὶ ποῦ κρυβήσεται ἄνθρωπος ἀπὸ τῆς γνώσεώς σου ὁ θεός
τὰ ἔργα ἡμῶν ἐν ἐκλογῇ καὶ ἐξουσίᾳ τῆς ψυχῆς ἡμῶν τοῦ ποιῆσαι δικαιοσύνην καὶ ἀδικίαν ἐν ἔργοις χειρῶν ἡμῶν καὶ ἐν τῇ δικαιοσύνῃ σου ἐπισκέπτῃ υἱοὺς ἀνθρώπων
ὁ ποιῶν δικαιοσύνην θησαυρίζει ζωὴν αὑτῷ παρὰ κυρίῳ καὶ ὁ ποιῶν ἀδικίαν αὐτὸς αἴτιος τῆς ψυχῆς ἐν ἀπωλείᾳ τὰ γὰρ κρίματα κυρίου ἐν δικαιοσύνῃ κατ᾽ ἄνδρα καὶ οἶκον
τίνι χρηστεύσῃ ὁ θεός εἰ μὴ τοῖς ἐπικαλουμένοις τὸν κύριον καθαριεῖς ἐν ἁμαρτίαις ψυχὴν ἐν ἐξομολογήσει ἐν ἐξαγορίαις ὅτι αἰσχύνη ἡμῖν καὶ τοῖς προσώποις ἡμῶν περὶ ἁπάντων
καὶ τίνι ἀφήσεις ἁμαρτίας εἰ μὴ τοῖς ἡμαρτηκόσιν δικαίους εὐλογήσεις καὶ οὐκ εὐθυνεῖς περὶ ὧν ἡμάρτοσαν καὶ ἡ χρηστότης σου ἐπὶ ἁμαρτάνοντας ἐν μεταμελείᾳ
καὶ νῦν σὺ ὁ θεός καὶ ἡμεῖς λαός ὃν ἠγάπησας ἰδὲ καὶ οἰκτίρησον ὁ θεὸς Ισραηλ ὅτι σοί ἐσμεν καὶ μὴ ἀποστήσῃς ἔλεός σου ἀφ᾽ ἡμῶν ἵνα μὴ ἐπιθῶνται ἡμῖν
ὅτι σὺ ᾑρετίσω τὸ σπέρμα Αβρααμ παρὰ πάντα τὰ ἔθνη καὶ ἔθου τὸ ὄνομά σου ἐφ᾽ ἡμᾶς κύριε καὶ οὐκ ἀπώσῃ εἰς τὸν αἰῶνα
ἐν διαθήκῃ διέθου τοῖς πατράσιν ἡμῶν περὶ ἡμῶν καὶ ἡμεῖς ἐλπιοῦμεν ἐπὶ σὲ ἐν ἐπιστροφῇ ψυχῆς ἡμῶν
τοῦ κυρίου ἡ ἐλεημοσύνη ἐπὶ οἶκον Ισραηλ εἰς τὸν αἰῶνα καὶ ἔτι
ἐν ὕμνοις τῷ Σαλωμων
μακάριος ἀνήρ οὗ ὁ κύριος ἐμνήσθη ἐν ἐλεγμῷ καὶ ἐκυκλώθη ἀπὸ ὁδοῦ πονηρᾶς ἐν μάστιγι καθαρισθῆναι ἀπὸ ἁμαρτίας τοῦ μὴ πληθῦναι
ὁ ἑτοιμάζων νῶτον εἰς μάστιγας καθαρισθήσεται χρηστὸς γὰρ ὁ κύριος τοῖς ὑπομένουσιν παιδείαν
ὀρθώσει γὰρ ὁδοὺς δικαίων καὶ οὐ διαστρέψει ἐν παιδείᾳ καὶ τὸ ἔλεος κυρίου ἐπὶ τοὺς ἀγαπῶντας αὐτὸν ἐν ἀληθείᾳ
καὶ μνησθήσεται κύριος τῶν δούλων αὐτοῦ ἐν ἐλέει ἡ γὰρ μαρτυρία ἐν νόμῳ διαθήκης αἰωνίου ἡ μαρτυρία κυρίου ἐπὶ ὁδοὺς ἀνθρώπων ἐν ἐπισκοπῇ
δίκαιος καὶ ὅσιος ὁ κύριος ἡμῶν ἐν κρίμασιν αὐτοῦ εἰς τὸν αἰῶνα καὶ Ισραηλ αἰνέσει τῷ ὀνόματι κυρίου ἐν εὐφροσύνῃ
καὶ ὅσιοι ἐξομολογήσονται ἐν ἐκκλησίᾳ λαοῦ καὶ πτωχοὺς ἐλεήσει ὁ θεὸς ἐν εὐφροσύνῃ Ισραηλ
ὅτι χρηστὸς καὶ ἐλεήμων ὁ θεὸς εἰς τὸν αἰῶνα καὶ συναγωγαὶ Ισραηλ δοξάσουσιν τὸ ὄνομα κυρίου
τοῦ κυρίου ἡ σωτηρία ἐπὶ οἶκον Ισραηλ εἰς εὐφροσύνην αἰώνιον
τῷ Σαλωμων εἰς προσδοκίαν
σαλπίσατε ἐν Σιων ἐν σάλπιγγι σημασίας ἁγίων κηρύξατε ἐν Ιερουσαλημ φωνὴν εὐαγγελιζομένου ὅτι ἠλέησεν ὁ θεὸς Ισραηλ ἐν τῇ ἐπισκοπῇ αὐτῶν
στῆθι Ιερουσαλημ ἐφ᾽ ὑψηλοῦ καὶ ἰδὲ τὰ τέκνα σου ἀπὸ ἀνατολῶν καὶ δυσμῶν συνηγμένα εἰς ἅπαξ ὑπὸ κυρίου
ἀπὸ βορρᾶ ἔρχονται τῇ εὐφροσύνῃ τοῦ θεοῦ αὐτῶν ἐκ νήσων μακρόθεν συνήγαγεν αὐτοὺς ὁ θεός
ὄρη ὑψηλὰ ἐταπείνωσεν εἰς ὁμαλισμὸν αὐτοῖς οἱ βουνοὶ ἐφύγοσαν ἀπὸ εἰσόδου αὐτῶν
οἱ δρυμοὶ ἐσκίασαν αὐτοῖς ἐν τῇ παρόδῳ αὐτῶν πᾶν ξύλον εὐωδίας ἀνέτειλεν αὐτοῖς ὁ θεός
ἵνα παρέλθῃ Ισραηλ ἐν ἐπισκοπῇ δόξης θεοῦ αὐτῶν
ἔνδυσαι Ιερουσαλημ τὰ ἱμάτια τῆς δόξης σου ἑτοίμασον τὴν στολὴν τοῦ ἁγιάσματός σου ὅτι ὁ θεὸς ἐλάλησεν ἀγαθὰ Ισραηλ εἰς τὸν αἰῶνα καὶ ἔτι
ποιήσαι κύριος ἃ ἐλάλησεν ἐπὶ Ισραηλ καὶ Ιερουσαλημ ἀναστήσαι κύριος τὸν Ισραηλ ἐν ὀνόματι δόξης αὐτοῦ
τοῦ κυρίου τὸ ἔλεος ἐπὶ τὸν Ισραηλ εἰς τὸν αἰῶνα καὶ ἔτι
τῷ Σαλωμων ἐν γλώσσῃ παρανόμων
κύριε ῥῦσαι τὴν ψυχήν μου ἀπὸ ἀνδρὸς παρανόμου καὶ πονηροῦ ἀπὸ γλώσσης παρανόμου καὶ ψιθύρου καὶ λαλούσης ψευδῆ καὶ δόλια
ἐν ποικιλίᾳ στροφῆς οἱ λόγοι τῆς γλώσσης ἀνδρὸς πονηροῦ ὥσπερ ἐν λαῷ πῦρ ἀνάπτον καλλονὴν αὐτοῦ
ἡ παροικία αὐτοῦ ἐμπρῆσαι οἴκους ἐν γλώσσῃ ψευδεῖ ἐκκόψαι δένδρα εὐφροσύνης φλογιζούσης παρανόμους συγχέαι οἴκους ἐν πολέμῳ χείλεσιν ψιθύροις
μακρύναι ὁ θεὸς ἀπὸ ἀκάκων χείλη παρανόμων ἐν ἀπορίᾳ καὶ σκορπισθείησαν ὀστᾶ ψιθύρων ἀπὸ φοβουμένων κύριον ἐν πυρὶ φλογὸς γλῶσσα ψίθυρος ἀπόλοιτο ἀπὸ ὁσίων
φυλάξαι κύριος ψυχὴν ἡσύχιον μισοῦσαν ἀδίκους καὶ κατευθύναι κύριος ἄνδρα ποιοῦντα εἰρήνην ἐν οἴκῳ
τοῦ κυρίου ἡ σωτηρία ἐπὶ Ισραηλ παῖδα αὐτοῦ εἰς τὸν αἰῶνα καὶ ἀπόλοιντο οἱ ἁμαρτωλοὶ ἀπὸ προσώπου κυρίου ἅπαξ καὶ ὅσιοι κυρίου κληρονομήσαισαν ἐπαγγελίας κυρίου
τῷ Σαλωμων ψαλμός παράκλησις τῶν δικαίων
δεξιὰ κυρίου ἐσκέπασέν με δεξιὰ κυρίου ἐφείσατο ἡμῶν
ὁ βραχίων κυρίου ἔσωσεν ἡμᾶς ἀπὸ ῥομφαίας διαπορευομένης ἀπὸ λιμοῦ καὶ θανάτου ἁμαρτωλῶν
θηρία ἐπεδράμοσαν αὐτοῖς πονηρά ἐν τοῖς ὀδοῦσιν αὐτῶν ἐτίλλοσαν σάρκας αὐτῶν καὶ ἐν ταῖς μύλαις ἔθλων ὀστᾶ αὐτῶν
καὶ ἐκ τούτων ἁπάντων ἐρρύσατο ἡμᾶς κύριος
ἐταράχθη ὁ εὐσεβὴς διὰ τὰ παραπτώματα αὐτοῦ μήποτε συμπαραληφθῇ μετὰ τῶν ἁμαρτωλῶν
ὅτι δεινὴ ἡ καταστροφὴ τοῦ ἁμαρτωλοῦ καὶ οὐχ ἅψεται δικαίου οὐδὲν ἐκ πάντων τούτων
ὅτι οὐχ ὁμοία ἡ παιδεία τῶν δικαίων ἐν ἀγνοίᾳ καὶ ἡ καταστροφὴ τῶν ἁμαρτωλῶν
ἐν περιστολῇ παιδεύεται δίκαιος ἵνα μὴ ἐπιχαρῇ ὁ ἁμαρτωλὸς τῷ δικαίῳ
ὅτι νουθετήσει δίκαιον ὡς υἱὸν ἀγαπήσεως καὶ ἡ παιδεία αὐτοῦ ὡς πρωτοτόκου
ὅτι φείσεται κύριος τῶν ὁσίων αὐτοῦ καὶ τὰ παραπτώματα αὐτῶν ἐξαλείψει ἐν παιδείᾳ
ἡ γὰρ ζωὴ τῶν δικαίων εἰς τὸν αἰῶνα ἁμαρτωλοὶ δὲ ἀρθήσονται εἰς ἀπώλειαν καὶ οὐχ εὑρεθήσεται μνημόσυνον αὐτῶν ἔτι
ἐπὶ δὲ τοὺς ὁσίους τὸ ἔλεος κυρίου καὶ ἐπὶ τοὺς φοβουμένους αὐτὸν τὸ ἔλεος αὐτοῦ
\section{ΜΑΚΚΑΒΑΙΩΝ Δʹ}
ΦΙΛΟΣΟΦΩΤΑΤΟΝ λόγον ἐπιδείκνυσθαι μέλλων, εἰ αὐτοδέσποτός ἐστιν τῶν παθῶν ὁ εὐσεβὴς λογισμός· συμβουλεύσαιμ' ἂν ὑμῖν ὀρθῶς, ὅπως προθύμως προσέχητε τῇ φιλοσοφίᾳ. 
Καὶ γὰρ ἀναγκαῖος εἰς ἐπιστήμην παντὶ ὁ λόγος, καὶ ἄλλως τῆς μεγίστης ἀρετῆς ἀρετῆς, λέγω δὴ φρονήσεως, περιέχει ἔπαινον· 
Εἰ ἄρα τῶν σωφροσύνης κωλυτικῶν παθῶν ὁ λογισμὸς φαίνεται ἐπικρατεῖν, γαστριμαργίας τε καὶ ἐπιθυμίας· 
ἀλλὰ καὶ τῶν τῆς δικαιοσύνης ἐμποδιστικῶν παθῶν κυριεύειν ἀναφαίνεται, οἷον κακοηθείας· καὶ τῶν τῆς ἀνδρείας ἐμποδιστικῶν παθῶν, θυμοῦ τε, καὶ πόνου καὶ φόβου. 
Πῶς οὖν, ἴσως εἴποιεν ἄν τινες, εἰ τῶν παθῶν ὁ λογισμὸς κρατεῖ, λήθης καὶ ἀγνοίας οὐ δεσπόζει; 
γελοῖον ἐπιχειροῦντες λέγειν· οὐ γὰρ τῶν ἑαυτοῦ παθῶν ὁ λογισμὸς κρατεῖ, ἀλλὰ τῶν τῆς δικαιοσύνης καὶ ἀνδρείας καὶ σωφροσύνης, καὶ φρονήσεως ἐναντίων· καὶ τούτων, οὐχ ὥστε αὐτὰ καταλῦσαι, ἀλλ' ὥστε αὐτοῖς μὴ εἶξαι. 
Πολλαχόθεν μὲν οὖν καὶ ἀλλαχόθεν ἔχοιμ' ἂν ὑμῖν ἐπιδεῖξαι, ὅτι αὐτοκράτωρ ἐστὶν τῶν παθῶν ὁ εὐσεβὴς λογισμός. 
Πολὺ δὲ πλέον τοῦτο ἀποδείξαιμι ἀπὸ τῆς ἀνδραγαθείας τῶν ὑπὲρ ἀρετὴν ἀποθανόντων, Ἐλεαζάρου τε καὶ ἑπτὰ ἀδελφῶν καὶ τῆς τούτων μητρός. 
Ἅπαντες γὰρ οὗτοι τῶν ἕως θανάτου πόνων ὑπεριδοντες, ὑπεριδόντες ἐπεδείξαντο ὅτι περικρατεῖ τῶν παθῶν ὁ λογισμός. 
Τῶν μὲν οὖν ἀρετῶν, ἔπεστί μοι ἐπαινεῖν τοὺς κατὰ τοῦτον τὸν καιρὸν ὑπὲρ τῆς καλοκᾳγαθίας ἀποθανόντας μετὰ τῆς μητρὸς ἄνδρας· 
τῶν δὲ τιμῶν μακαρίσαιμ' ἄν· θαυμασθέντες γὰρ ἐκεῖνοι οὐ μόνον ὑπὸ πάντων ἀνθρώπων ἐπὶ τῇ ἀνδρείᾳ καὶ τῇ ὑπομονῇ, ἀλλὰ καὶ ὑπὸ τῶν αἰκισαμένων, αἴτιοι κατέστησαν τοῦ καταλυθῆναι τὴν κατὰ τοῦ ἔθνους τυραννίδα, νικήσαντες τὸν τύραννον τῇ ὑπομονῇ, ὥστε δι' αὐτῶν καθαρισθῆναι τὴν πατρίδα. 
Ἀλλὰ καὶ περὶ τούτου νῦν αὐτίκα δὴ λέγειν ἐξέσται, ἀρξαμένων τῆς ὑποθέσεως, ὥσπερ εἴωθα ποιεῖν, καὶ οὕτως εἰς τὸν περὶ αὐτῶν τρέψομαι λόγον, δόξαν διδοὺς τῷ πανσόφῳ Θεῷ. 
Ζητοῦμεν δὴ τοίνυν, εἰ αὐτοκράτωρ ἐστὶν παθῶν ὁ λογισμός. 
Διακρίνωμεν δὲ, τί ποτέ ἐστιν λογισμός; καὶ τί πάθος; καὶ πόσαι παθῶν ἰδέαι; καὶ εἰ πάντων ἐπικρατεῖ τούτων ὁ λογισμός; 
Λογισμὸς μὲν δὴ τοίνυν ἐστὶν νοῦς μετὰ ὀρθοῦς βίου· πρωτιμῶν τὸν σοφίας λόγον. 
Σοφία δὴ τοὶνυν ἐστὶν γνῶσις θείων καὶ ἀνθρωπίνων πραγμάτων, καὶ τῶν τούτων αἰτίων. 
Αὕτη δὴ τοίνυν ἐστὶν ἠ τοῦ νόμου παιδσεία· δι' ἧς τὰ θεῖα σεμνῶς, καὶ τὰ ἀνθρώπινα συμφερόντως μανθάνομεν. 
Τῆς δὲ σοφίας ἰδέαι καθεστᾶσιν, φρόνησις καὶ δικαιοσύνη καὶ ἀνδρεια καὶ σωφροσύνη. 
Κυριωτάτη πάντων ἡ φρόνησις· ἐξ ἧς δὴ τῶν παθῶν ὁ λογισμὸς ἐπικρατεῖ. 
Παθῶν δὲ φύσεις εἰσὶν αἱ περιεκτικώταται δύο, ἡδονή τε καὶ πόνος· τούτων δὲ ἐκάτερον καὶ περὶ τὴν ψυχὴν πέφυκεν. 
Πολλαὶ δὲ καὶ περὶ τὴν ἡδονὴν καὶ τὸν πόνον παθῶν εἰσὶν ἀκολουθίαι. 
Πρὸ μὲν οὖν τῆς ἡδονῆς ἐστιν ἐπιθυμία· μετὰ δὲ τὴν ἡδονὴν, χαρά. 
Πρὸ δὲ τοῦ πόνου ἐστὶν φόβος· μετὰ δὲ τὸν πόνον, λύπη. 
Θυμὸς δὲ κοινὸν πάθος ἐστὶν ἡδονῆς καὶ πόνου, ἐὰν ἐννοηθῇ τις ὃτε αὐτῷ περιέπεσεν. 
Ἐν δὲ τῇ ἡδονῇ ἐστιν καὶ ἡ κακοήθης διάθεσις, πολυτροπωτάτη πάντων τῶν παθῶν οὖσα. 
Κατὰ μὲν ψυχῆς ἀλαζονεία, καὶ φιλαργυρία, καὶ φιλοδοξία, καὶ φιλονεικία, ἀπιστία καὶ βασκανία· 
κατὰ δὲ τὸ σῶμα, παντοφαγία, καὶ λαιμαργία, καὶ νομοφαγία. 
Καθάπερ οὖν δυοῖν τοῦ σώματος καὶ τῆς ψυχῆς φυτῶν ὄντων ἡδονῆς τε καὶ πὸνου, πολλαὶ τούτων τῶν παθῶν εἰσιν παραφυάδες. 
Ὧν ἕκαστος ὁ πανγέωργος λογισμὸς περικαθαίρων τε καὶ ἀποκνίζων, καὶ περιπλέκων, καὶ ἐπάρδων, καὶ πάντα τρόπον μεταχέων, ἐξημεροῖ τὰς τῶν ἠθῶν καὶ παθῶν ὕλας. 
Ὁ γὰρ λογισμὸς τῶν μὲν ἀρετῶν ἐστιν ἡγεμῶν, τῶν δὲ παθῶν αὐτοκράτωρ. Ἐπιθεώρει γε τοίνυν πρῶτον δι' αὐτῶν κωλυτικῶν τῆς σωφροσύνης ἔργων, ὅτι αὐτοδέσποτός ἐστιν τῶν παθῶν ὁ λογισμός. 
Σωφροσύνη δὴ τοίνυν ἐστὶν ἐπικράτεια τῶν ἐπιθυμιῶν. 
Τῶν δὲ ἐπιθυμιῶν αἱ μέν εἰσιν ψυχικαὶ, αἱ δὲ σωματικαί· καὶ τούτων ἀμφοτέρων ὁ λογισμὸς ἐπικρατεῖν φαίνεται. 
Ἐπεὶ πόθεν κινούμενοι πρὸς τὰς ἀπειρημένας τοοφὰς, ἀποτρεπόμεθα τὰς ἐξ ἑαυτῶν ἡδονάς; οὐχ ὅτι δύναται τῶν ὀρὲξεων ἐπικρατεῖν ὁ λογισμός; ἐγὼ μὲν οἶμαι. 
Τοιγαροῦν ἐνύδρων ἐπιθυμοῦντες καὶ ὀρνέων καὶ τετραπόδων, παντοίων βρωμάτων τὼν ἀπηγορευμένων ἡμῖν κατὰ τὸν νόμον ἀπεχόμεθα διὰ τὴν τοῦ λογισμοῦ ἐπικράτειαν. 
Ἀντέχεται γὰρ τὰ τῶν ὀρέξεων πάθη ὑπὸ τοῦ σώφρονος νοὸς ἀνακαμπτόμενα· καὶ φιλοτιμοῦνται πάντα τὰ τοῦ σώματος κινήματα ὑπὸ τοῦ λογισμοῦ. 
Καὶ τὶ θαυμαστὸν; εἰ αἱ τῆς ψυχῆς ἐπιθυμίαι πρὸς τὴν τοῦ κάλλους μετουσίαν ἀκυροῦνται. 
Ταύτῃ γοῦν ὁ σώφρων Ἰωσὴφ ἐπαινεῖται, ὅτι τῷ λογισμῷ, διανοίᾳ περιεκράτησεν τῆς ἠδυπαθείας. 
Νέος γὰρ ὢν καὶ ἀκμάζων πρὸς συνουσιασμὸν ἠκύρωσεν τῷ λογισμῷ τὸν τῶν παθῶν οἶστρον. 
Οὐ μόνον δὲ τὴν τῆς ἡδυπαθείας οἰστρηλασίαν ἐπικρατεῖν ὁ λογισμὸς φαίνεται, ἀλλὰ καὶ πάσης ἐπιθυμίας. 
Λέγει γοῦν ὁ νόμος· οὐκ ἐπιθυμήσεις τὴν γυναῖκα τοῦ πλησίον σου, οὐδὲ ὅσα τῷ πλησίον σου ἐστίν. 
Καίτοι ὅτε μὴ ἐπιθυμεῖν εἴρηκεν ἡμᾶς ὁ νόμος, πολὺ πλέον πείσαιμ' ἂν ὑμᾶς, ὅτι τῶν ἐπιθυμιῶν κρατεῖν δύναται ὁ λογισμὸς, ὥσπερ καὶ τῶν κωλυτικῶν τῆς δικαιοσύνης παθῶν. 
Ἐπεὶ τίνα τρόπον μονοφάγος τις ὢν τὸ ἦθος, καὶ γαστρίμαργος, καὶ μέθυσος, μεταπαιδεύεται, εἰ μὴ δῆλον, ὅτι κύριός ἐστιν τῶν παθῶν ὁ λογισμός; 
Αὐτίκα γοῦν τῷ νόμῳ πολιτευόμενος, κᾂν φιλάργυρός τις εἴη, βιάζεται τὸν ἑαυτοῦ τρόπον, τοῖς δεομένοις δανείζων χωρὶς τόκων, καὶ τὸ δάνειον τῶν ἑβδομάδων ἐντάσσων χρεοκοπούμενος. 
Κᾂν φειδωλός τις ᾖ, ὑπὸ τοῦ νόμου κρατεῖται διὰ τὸν λογισμὸν, μήτε ἐπικαρπούμενος τοὺς ἀμητοὺς, μήτε ἐπιῤῥωγολογούμενος τοὺς ἀμπελῶνας, καὶ ἐπὶ τῶν ἐτέρων ἔστιν ἐπιγνῶναι τοῦτο, ὅτι τῶν παθῶν ἐστιν ὁ λογισμὸς κρατῶν. 
Ὁ γὰρ νόμος καὶ τῆς πρὸς γονεῖς εὐνοίας κρατεῖ, μὴ καταπροδιδοὺς τὴν ἀρετὴν δι' αὐτούς· 
καὶ τῆς προσγαμετῆς φιλίας ἐπικρατεῖ, διὰ παρανομίαν αὐτὴν ἀπελέγχων. 
Καὶ τῆς τέκνων φιλίας κυριεύει, διὰ κακίαν αὐτῶν κολάζων, καὶ τῆς φίλων συνηθείας δεσπόζει, διὰ πονηρίας αὐτοὺς ἐξελέγχων. 
Καὶ μὴ νομίσητε παράδοξον εἶναι, ὅπου καὶ ἔχθραν ὁ λογισμὸς ἐπινομίσητε παράδοξον εἶναι, ὅπου καὶ ἔχθραν ὁ λογισμὸς ἐπινομισητε παράδοξον εἶναι, ὅπου καὶ ἔχθραν ὁ λογισμὸς ἐπικρατεῖν δύναται διὰ τὸν νόμον, 
μηδὲ δενδροτομῶν τὰ ἥμερα τῶν πολεμίων φυτὰ, τὰ δὲ τῶν ἐχθρῶν τοῖς ἀπολέσασιν διασώζων, καὶ τὰ πεπτωκότα συνεγείρων. 
Καὶ τῶν βιοτέρων δὲ παθῶν κρατεῖν ὁ λογισμὸς φαίνεται, φιλαρχίας, καὶ κενοδοξίας, καὶ ἀλαζονείας, καὶ μεγαλαυχίας, καὶ βασκανίας. 
Πάντα γὰρ ταῦτα τὰ κακοήθη πάθη ὁ σώφρων νοῦς ἀπωθεῖται, ὥσπερ καὶ τὸν θυμόν· καὶ γὰρ τοῦτο δεσπόζει. 
Θυμούμενος γέ τοι Μωσῆς κατὰ Δαθὰν καὶ Ἀβειρῶν, οὐ θυμῷ τι κατ' αὐτῶν ἐποίησεν, ἀλλὰ λογισμῷ τὸν θυμὸν διῄτησεν. 
Δυνατὸς γὰρ ὁ σώφρων νοῦς, ὡς ἔφην, κατὰ τῶν παθῶι ἀριστεῦσαι, καὶ τὰ μὲν αὐτῶν μεταθεῖναι, τὰ δὲ καὶ ἀκυρῶσαι. 
Ἐπεὶ διατί ὁ πάνσοφος ἡμῶν πατὴρ Ἰακὼβ τοὺς περὶ Συμεὼν καὶ Λευὶν αἰτιᾶται, μὴ λογισμῷ τοὺς Σικιμίτας ἐθνηδὸν ἀποσφάξαντας, λέγων, ἐπικατάρατος ὁ θυμὸς αὐτῶν; 
Εἰ μὴ γὰρ ἐδύνετο τῶν θυμῶν ὁ λογισμὸς κρατεῖν, οὐκ ἂν εἶπεν οὑτως. 
Ὁπηνίκα γὰρ ὁ Θεὸς τὸν ἄνθρωπον κατεσκεύαζεν, τὰ πάθη αὐτοῦ καὶ τὰ ἤθη περιεφύτευσεν. 
Καὶ τηνικαῦτα δὲ περὶ πάντων τὸν ἱερὸν ἡγεμόνα νοῦν διὰ τῶν αἰσθητηρίων ἐνεθρόνισεν· 
καὶ τούτῳ νόμον ἔδωκεν, καθ' ὃν πολιτευόμενος βασιλεύσει βασιλείαν σώφρονά τε, καὶ δικαίαν, καὶ ἀγαθὴν, καὶ ἀνδρείαν. 
Πῶς οὖν, εἴποι τις ἂν, εἰ τῶν παθῶν ὁ λογισμὸς κρατεῖ, λήθης καὶ ἀγνοίας οὐ κρατεῖ; 
Ἐστὶ δὲ κομιδῆ γελοῖος ὁ λογισμός οὐ γὰρ τῶν ἑαυτοῦ παθῶν ὁ λογισμὸς ἐπικρατεῖν φαίνεται, ἀλλὰ τῶν σωματικῶν. 
Οἷον ἐπιθυμίαν τις ὑμῶν οὐ δύναται ἐκκόψαι, ἀλλὰ μὴ δουλωθῆναι τῇ ἐπιθυμίᾳ δύναται ὁ λογισμὸς παρασχέσθαι. 
Θυμόν τις οὐ δύναται ἐκκόψαι ἡμῶν τῆς ψυχῆς, ἀλλὰ τῷ θυμῷ δυνατὸν βοηθῆσαι. 
Κακοήθειάν τις ὑμῶν οὐ δύναται ἐκκόψαι, ἀλλὰ τὸ μὴ καμφθῆναι τῇ κακοηθείᾀ δυνατὸν ὁ λογισμὸς συμμαχῆσαι. 
Οὐ γὰρ ἐκριζωτὴς τῶν παθῶν ὁ λογισμός ἐστιν, ἀλλ' ἀνταγωνιστής. 
Ἔστιν γοῦν τοῦτο διὰ τῆς Δαυεὶδ τοῦ βασιλέως δίψης σαφέστερον ἐπιλογίσασθαι. 
Ἐπεὶ γὰρ δι' ὅλης ἡμέρας προσβαλὼν τοῖς ἀλλοφύλοις ὁ Δαυὶδ, πολλοὺς αὐτῶν ἀπέκτεινεν μετὰ τῶν τοῦ ἔθνους στρατιωτῶν· 
τότε δὲ γενομένης ἑσπέρας, ὑδρῶν καὶ σφόδρα κεκμηκὼς, ἐπὶ τὴν βασίλειον σκηνὴν ἦλθεν, περὶ ἣν ὁ πᾶς τῶν προγόνων στρατὸς ἐστρατοπέδευκεν. 
Οἱ μὲν οὖν ἄλλοι πάντες ἐπὶ τὸ δεῖπνον ἦσαν. 
Ὁ δὲ βασιλεὺς ὡς μάλιστα διψῶν, καίπερ ἀφθόνους ἔχων πηγὰς, οὐκ ἠδύνατο δι' αὐτῶν ἰάσασθαι τὴν δίψαν· 
ἀλλά τις αὐτὸν ἀλόγιστος ἐπιθυμία τοῦ παρὰ τοῖς πολεμίοις ὕδατος ἐπιτείνουσα συνέφρυγεν, καὶ λύουσα κατέφλεγεν. 
Ὅθεν τῶν ὑπερασπιστῶν ἐπὶ τῇ τοῦ βασιλέως ἐπιθυμία σχετλιαζόντων, δύο νεανίσκοι στρατιῶται καρτεροὶ καταιδεσθέντες τὴν τοῦ βασιλέως ἐπιθυμίαν, τὰς πανοπλίας καθωπλίσαντο, καὶ κάλπην λαβόντες ὑπερέβησαν τοὺς τῶν πολεμίων χάρακας· 
καὶ λαθόντες τοὺς τῶν πυλῶν ἀκροφύλακας, διεξῄεσαν εὑράμενοι κατὰ πᾶν τὸ τῶν πολεμίων στρατόπεδον. 
Καὶ ἀνευράμενοι θαῤῥαλέως τὴν πηγὴν, ἐξ αὐτῆς ἐγέμισαν τῷ βασιλεῖ τὸ ποτόν. 
Ὁ δὲ καὶ περὶ τὴν δίψαν διαπυρούμενος, ἐλογίσατο πάνδεινον εἶναι κίνδυνον τῇ ψυχῇ λογισθὲν ἰσοδύναμον τὸ ποτὸν αἵματι. 
Ὅθεν ἀντιθεὶς τῇ ἐπιθυμίᾳ τὸν λογισμὸν, ἔσπεισεν τὸ πόμα τῷ Θεῷ. 
Δυνατὸς γὰρ ὁ σώφρων νοῦς νικῆσαι τὰς τῶν παθῶν ἀνάγκας, καὶ σβέσαι τὰς τῶν οἴστρων φλεγμονὰς, καὶ τὰς τῶν σωμάτων ἀλγηδόνας καθ' ὑπερβολὴν οὔσας καταπαλαῖσαι, 
καὶ τῆς καλοκᾳγαθίας τοῦ λογισμοῦ ἀποπτῦσαι πάσας τὰς τῶν παθῶν ἐπικρατείας. 
Ἤδη δὲ καὶ ὁ καιρὸς ἡμᾶς καλεῖ ἐπὶ τὴν ἀπόδειξιν τῆς ἱστορίας τοῦ σώφρονος λογισμοῦ. 
Ἐπειδὴ γὰρ βαθεῖαν εἰρήνην διὰ τὴν εὐνομίαν οἱ πατέρες ἡμῶν εἶχον, καὶ ἔπραττον καλῶς, ὥστε καὶ τὸν τῆς Ἀσίας βασιλέα Σέλευκον τὸν Νικάνορα καὶ χρήματα εἰς τὴν ἱερουργίαν αὐτοῖς ἀποφορίσαι, καὶ τὴν πολιτείαν αὐτῶν ἀποδέχεσθαι· 
τότε δή τινες πρὸς τὴν κοινὴν νεωτερίσαντες ὁμόνοιαν, πυλυτρόπως ἐχρήσαντο συμφοραῖς. 
Σίμων γάρ τις πρὸς Ὀνίαν ἀντιπολιτεύομενος τόν ποτε τὴν ἀρχιερωσύνην ἔχοντα διὰ βίου, καλὸν καὶ ἀγαθὸν ἄνδρα, ἐπειδὴ πάντα τρόπον διαβάλλων ὑπὲρ τοῦ ἔθνους οὐκ ἰσχυσεν κακῶσαι, φυγὰς ᾤχετο, τὴν πατρίδα προδώσων. 
Ὅθεν ἥκων πρὸς Ἀπολλώνιος, τὸν Συρίας τε καὶ Φοινίκης καὶ Κιλικίας στρατηγὸν, ἔλεγεν, 
εὔνους ὢν τοῖς τοῦ βασιλέως πράγμασιν ἥκω, μηνύων πολλὰς ἰδιωτικῶν χρημάτων μυριάδας ἐν τοῖς Ἱεροσολύμων γαζοφυλακίοις τεθησαύρισται, τῷ ἱερῷ μὴ ἐπικοινωνούσας, ἀλλὰ προσήκειν ταῦτα Σελεύκῳ τῷ βασιλεῖ. 
Τούτων ἕκαστα γνοὺς ὁ Ἀπολλώνιος, τὸν μὲν Σίμωνα τῆς εἰς τὸν βασιλέα κηδεμονίας ἐπαινεῖ, πρὸς δὲ τὸν Σέλευκον ἀναβὰς κατεμήνυε τὸν τῶν χρημάτων θησαυρόν· 
καὶ λαβὼν τὴν περὶ αὐτῶν ἐξουσίαν, ταχὺ εἰς τὴν πατρίδα ἡμῶν μετὰ τοῦ καταράτου Σίμωνος καὶ βαρυτάτου στρατοῦ προσελθὼν, 
ταῖς τοῦ βασιλέως ἐντολαῖς ἥκειν ἔλεγεν, ὅπως τὰ ἰδιωτικὰ τοῦ γαζοφυλακίου λάβοι χρήματα. 
Καὶ τοῦ ἔθνους πρὸς τὸν λόγον σχετλιάζοντος, ἀντιλέγοντός τε, πάνδεινον εἶναι νομίσαντες, εἰ οἱ τὰς παρακαταθήκας πιστεύσαντας τῷ ἱερῷ θησαυρῷ στερηθήσονται, ὡς οἷόν τε ἦν ἐκώλυον. 
Μετὰ ἀπειλῆς δὲ ὁ Ἀπολλώνιος ἀπῄει εἰς τὸ ἱερόν. 
Τῶν δὲ ἱερέων μετὰ γυναικῶν καὶ παιδίων ἐν τῷ ἱερῷ ἱκετευσάντων τὸν Θεὸν ὑπερασπίσαι τοῦ ἱεροῦ καταφρονουμένου τόπου. 
Ἀνιόντος τε μετὰ καθωπλισμένης τῆς στρατιᾶς τοῦ Ἀπολλωνίου πρὸς τὴν τῶν χρημάτων ἀρπαγὴν οὐρανόθεν ἔφιπποι προϋφάνησαν ἄγγελοι περιαστράπτοντες τοῖς ὅπλοις, καὶ πολὺν αὐτοῖς φόβον τε καὶ τρόμον ἐνιόντες. 
Καταπεσὼν γέ τοι ἡμιθανὴς ὁ Ἀπολλώνιος ἐπὶ τὸν πάμφυλον τοῦ ἱεροῦ περίβολον, τὰς χεῖρας ἐξέτεινεν εἰς τὸν οὐρανὸν, μετὰ διακρύων τοὺς Ἑβραίους παρεκάλει, ὅπως περὶ αὐτοῦ εὐξόμενοι, τὸν ἐπουράνιον ἐξευμενίσωνται στρατόν. 
Ἔλεγεν γὰρ ἡμαρτηκὼς, ὥστε καὶ ἀποθανεῖν ἄξιος ὑπάρχειν, πᾶσίν τε ἀνθρώποις ὑμνήσειν σωθεὶς τὴν τοῦ ἱεροῦ τόπου μακαριότητα. 
Τούτοις ἐπαχθεὶς τοῖς λόγοις Ὀνίας ὁ ἀρχιερεὺς, καίπερ ἄλλως εὐλαβηθεὶς, μή ποτε νομίσειεν ὁ βασιλεὺς Σέλευκος ἐξ ἀνθρωπίνης ἐπιβουλῆς καὶ μὴ θείας δίκης ἀνῃρήσασθαι τὸν Ἀπωλλώνιον, ηὔξατο περὶ αὐτοῦ. 
Καὶ ὁ μὲν παραδὸξως διασωθεὶς ᾤχετο, δηλώσων τῷ βασιλεῖ τὰ συμβάντα αὐτῷ. 
Τελευτήσαντος δὲ Σελεύκου τοῦ βασιλέως διαδέχεται τὴν ἀρχὴν ὁ υἱὸς αὐτοῦ Ἀντίοχος Ἐπιφανὴς, ἀνὴρ ὑπερήφανος καὶ δεινὸς. 
Ὃς καταλύσας τὸν Ὀνίαν τῆς ἀρχιερωσύνης, 
Ἰάσονα τὸν ἀδελφὸν αὐτοῦ κατέστησεν ἀρχιερέα, συνθέμενον δώσειν, εἰ ἐπιτρέψειεν αὐτῷ τὴν ἀρχὴν, κατ' ἐνιαυτὸν τρισχίλια ἐξακόσια ἑξήκοντα τάλαντα. 
Ὁ δὲ ἐπέτρεψεν αὐτῷ ἀρχιερᾶσθαι καὶ τοῦ ἔθνους ἀφηγεῖσθαι. 
Ὃς καὶ ἐξεζήτησεν τὸ ἔθνος, καὶ ἐξεπολίτευσεν ἐπὶ πᾶσαν παρανομίαν. 
Ὥστε μὴ μόνον ἐπ' αὐτῇ τῆ ἄκρᾳ τῆς πατρίδος ἡμῶν γυμνάσιον κατασκευάσαι, τὴν τοῦ ἱεροῦ κηδεμονίαν. 
Ἐφ' οἷς ἀγανακτήσασα ἡ θεία δίκη αὐτόν τοι τὸν Ἀντίοχον ἐπολέμησεν. 
Ἐπειδὴ γὰρ πολεμῶν ἦν κατ' Αἴγυπ τον Πτολεμαίῳ, ἤκουσέν τε, ὅτι φήμης διαδοθείσης περὶ τιῦ τεθνάναι αὐτὸν, ὡς ἔνι μάλιστα χαίροιεν οἱ Ἱεροσολυμῖται, ταχέως ἐπ' αὐτοὺς ἀνέζευξεν. 
Καὶ ὡς ἐπόρθσεν αὐτοὺς, δόγμα ἔθετο, ὅπως εἴ τινες αὐτῶν φάνοιεν τῷ πατρίῳ πολιτευόμενοι νόμῳ θάνοιεν. 
Καὶ ἐπεὶ κατὰ μηδένα τρόπον ἴσχυεν καταλῦσαι διὰ τῶν δογμάτων τὴν τοῦ ἔθνους εὔνοιαν, ἀλλὰ πάσας τὰς ἑαυτοῦ ἀπειλὰς καὶ τιμωρίας ἑώρα καταλυομένας, 
ὥστε καὶ γυναῖκας, ὅτι περιέτεμον τὰ παιδία, μετὰ τῶν βρεφῶν κατακρημνισθῆναι, προειδυίας ὅτι τοῦτο πείσονται· 
ἐπεὶ οὖν τὰ δόγματα αὑτοῦ κατεφρονεῖτο ὑπὸ τοῦ λαοῦ, αὐτὸς διὰ βασάνων ἕνα ἕκαστον τούτου ἔθνους ἠνάγκαζεν μικρῶν ἀπογευομένους τροφῶν, ἐξόμνυσθαι τὸν Ἰουδαϊσμόν. 
Προκαθίσας γέ τοι μετὰ τῶν συνέδρων ὁ τύραννος Ἀντίοχος ἐπί τινος ὑψηλοῦ τόπου, 
καὶ τῶν στρατευμάτων αὐτῶν ἐνόπλων κυκλόθεν παρεστηκότων παρεκέλευεν τοῖς δορυφόροις ἕνα ἕκαστον τῶν Ἑβραίων περισπᾶσθαι καὶ κρεῶν ὑείων καὶ εἰδωλοθύτων ἀναγκάζειν ἀπογεύεσθαι. 
Εἰ δὲ τινες μὴ θέλοιεν μιαροφαγῆσαι, τούτους τροχισθέντας ἀναιρεθῆναι. 
Πολλῶν δὲ συναρπασθέντων, εἷς πρῶτος ἐκ τῆς ἀγέλης Ἑβραῖος ὀνόματι Ἐλεάζαρος, τὸ γένος ἱερεὺς, τὴν ἐπιστήμην νομικὸς, καὶ τὴν ἡλικίαν προήκων, καὶ πολλοῖς τῶν περὶ τὸν τύραννον διὰ τὴν ἡλικίαν γνώριμος, παρήχθη πλησίον αὐτοῦ. 
Καὶ αὐτὸν ἰδὼν ὁ Ἀντίοχος, ἔφη, 
ἐγὼ πρὶν ἂρξασθαι τῶν κατὰ σοῦ βασάνων, ὦ πρεσβύτα, συμβουλεύσαιμ' ἄν σοι ταῦτα ὅπως ἀπογευσάμενος τῶν ὑείων σώζοιο· αἰδοῦμαι γάρ σου τὴν ἡλικίαν καὶ τὴν πολιὰν, ἥν μετὰ τοσοῦτον ἔχων χρόνον, οὔ μοι δοκεῖς φιλοσοφεῖν, τῇ Ἰουδαιων χρώμενος θρησκείᾳ. 
Διατί γὰρ τῆς φύσεως κεχαρισμένης καλλίστην τὴν τοῦδε τοῦ ζώου σαρκοφαγίαν βδελύττῃ; 
Καὶ γὰρ ἀνόητον τοῦτο τὸ μὴ ἀπολαύειν τῶν χωρὶς ὀνείδους ἡδέων, καὶ δι' ἄδικον ἀποστρέφεσθαι τὰς τῆς φύσεως χάριτας. 
Σὺ δέ μοι καὶ ἀνοητότερον ποιήσειν δοκεῖς, εἰ κενοδοξῶν περὶ τὸ ἀληθὲς, 
ἔτι κᾀμοῦ καταφρονήσεις ἐπὶ τῇ ἰδίᾳ τιμωρίᾳ· οὐκ ἐξυπνώσεις ἀπὸ τῆς φλυάρου φιλοσοφίας ὑμῶν; 
Καὶ ἀποσκεδάσεις ψῶν λογισμῶν σου τὸν λῆρον, καὶ ἄξιον τῆς ἡλικίας ἀναλαβῶν νοῦν φιλοσοφήσεις τήν τοῦ συμφέροντος ἀλήθειαν; 
καὶ προσκυνήσας μου τὴν φιλάνθρωπον παρηγορίαν οἰκτειρήσεις τὸ σεαυτοῦ γῆρας; 
καὶ γὰρ ἐνθυμήθητι, ὡς εἰ καί τις ἐστιν τῆσδε τῆς θρησκείας ἐποπτικὴ δύναμις, συγνωμονήσειν σοι ἐπὶ πᾶσιν δι' ἀνάγκην παρανομίᾳ γεινομένῃ. 
Τοῦτον τὸν τρόπον ἐπὶ τὴν ἔκθεσμον σαρκοφαγίαν ἐποτρύνοντος τοῦ τυράννου, λόγον ᾔτησεν ὁ Ἐλεάζαρος. 
Καὶ λαβὼν τοῦ λέγειν ἐξουσίαν, ἤρξατο δημηγορεῖν οὕτως· 
ἡμεῖς, Αντίοχε, θείῳ πεπεισμένοι νόμῳ πολιτευεσθαι, οὐδεμίαν ἀνάγκην βιαιοτέραν εἶναι νομίζομεν τῆς πρὸς τὸν νόμον ἡμῶν εὐπειθείας. 
Διὸ δὲ κατ' οὐδένα τρόπον παρανομεῖν ἀξιοῦμεν. 
Καί τοι εἰ καὶ κατὰ ἀλήθειαν μὴ ἦν ὁ νόμος ἡμῶν, ὡς σὺ ὑπολαμβάνεις, θεῖος, (ἄλλως δὲ νομίζομεν αὐτὸν εἶναι θεῖον) οὐδὲ οὕτως ἐξὸν ἡμῖν ἦν τὴν ἐπὶ τῇ εὐσεβείᾳ δόκαν ἀκυρῶσαι. 
Μὴ μικρὰν οὖν εἶναι νομίσῃς ταύτην, εἰ μιαροφαγήσεμεν, ἁμαρτίαν. 
Τὸ γὰρ ἐν μικροῖς καὶ ἐν μεγάλοις παρανομεῖν ἰσοδύναμόν ἐστιν· 
δι' ἑκατέρου γὰρ ὡς ὁμοίως ὁ νόμος ὑπερηφανεῖται. 
Χλευάζεις δὲ ἡμῶν τὴν φιλοσοφίαν, ὥσπερ οὐ μετὰ εὐλογιστίας ἐν αὐτῇ βιούντων. 
Σωφροσύνην τε γὰρ ἡμᾶς ἐκδιδάσκει, ὥστε πασῶν τῶν ἡδονῶν καὶ ἐπιθυμιῶν κρατεῖν, καὶ ἀνδρείαν ἐξασκεῖν, ὥστε πάντα πόνον ἑκουσίως ὑπομένειν· 
καὶ δικαιοσύνην παιδεύει, ὥστε διὰ πάντων τῶν ἠθῶν ἰσονομεῖν καὶ εὐσέβειαν διδάσκειν, ὥστε μόνον τὸν ὄντα Θεὸν σέβειν μεγαλοπρεπῶς. 
Διὸ οὐ μιαροφαγοῦμεν· πιστεύοντες γὰρ Θεοῦ καθεστᾶναι τὸν νόμον, οἴδαμεν ὅτι καὶ κατὰ φύσιν ἡμῖν συμπαθεῖ νομοθετῶν ὁ τοῦ κόσμου κτίστης· 
τὰ μὲν οἰκειωθωσόμενα ἡμῶν ταῖς ψυχαῖς ἐπέτρεψεν ἐσθίειν, τὰ δὲ ἐναντιωθησόμενα ἐκώλυσεν σαρκοφαγεῖν. 
Τυραννικὸν δὲ, οὐ μόνον ἀναγκάζεις ἡμᾶς παρανομεῖν, ἀλλὰ καὶ ἐσθίειν, ὅπως τῇ ἐχθίστῃ ἡμῶν μιαροφαγίᾳ ταύτῃ ἔτι ἐγγελάσῃς. 
Ἀλλ' οὐ γελάσεις κατ' ἐμοῦ τοῦτον τὸν γέλωτα· 
οὔτε τοὺς ἱεροὺς τῶν προγόνων περὶ τοῦ φυλάξαι τὸν νόμον ὅρκους οὐ παρήσω. 
Οὐδ' ἂν ἐκκόψεις μου τὰ ὄμματα, καὶ τὰ σπλάγχνα μου τήξεις. 
Οὐχ οὕτως εἰμὶ γέρων ἐγὼ καὶ ἄνανδρος, ὥστε μοι διὰ τὴν εὐσέβειαν μὴ νεάζειν τὸν λογισμόν. 
Πρὸς ταῦτα τροχοὺς εὐτρέπιζε, καὶ τὸ πῦρ ἐκφύσα σφοδρότερον. 
Οὐχ οὕτως οἰκτειρήσω τὸ ἐμαυτοῦ γῆρας, ὥστε με δι' ἐμαυτοῦ τὸν πάτριον καταλῦσαι νόμον. 
Οὐ ψεύσομαί σε, παιδευτὰ νόμε, οὐδὲ φεύξομαί σε, φίλη ἐγκράτεια. 
Οὐδὲ καταισχυνῶ σε, φιλόσοφε λόγε, οὐδὲ ἐξαρνήσεμαί σε, ἱερωσύνη τιμία, καὶ νομοθεσίας ἐπιστήμη· 
οὐδὲ μιανεῖς μου τὸ σεμνὸν γήρηως στόμα, οὐδὲ νομίμου βίου ἡλικίαν. 
Ἁγνόν με οἱ πατέρες προσδέξονται, μὴ φοβηθέντα σου τὰς μέχρι θανάτου ἀνάγκας. 
Ἀσεβῶν μὲν γὰρ τυραννήσεις· τῶν δὲ ἐμῶν περὶ τῆς εὐσεβείας λογισμῶν οὔτε λόγοις δεσπόσεις, οὔτε δι' ἔργων. 
Τοῦτον τὸν τρόπον ἀντιρητορεύσαντα ταῖς τοῦ τυράννου παρηγορίαις, παραστάντες οἱ δορυφόροι πικρῶς ἔσυραν ἐπὶ τὰ βασανιστήρια τὸν Ἐλεάζαρον. 
Καὶ πρῶτον μὲν περιέδυσαν τὸν γηραιὸν ἐκκεκοσμημένον περὶ τὴν εὐσέβειαν εὐσχημοσύνην. 
Ἔπειτα περιαγκωνίσαντες ἑκατέρωθεν, μάστιξιν κατῇκιζον· 
πείσθητι ταῖς τοῦ βασιλέως ἐντολαῖς, ἑτέρωθεν κήρυκος ἐπιβοῶντος. 
Ὁ δὲ μεγαλόφρων καὶ εὐγενὴς ὡς ἀληθῶς Ἐλεάζαρος, ὥσπερ ἐν ὀνείρω βασανιζόμενος κατ' οὐδένα τρόπον μετετρέπετο. 
Ἀλλὰ ὑψηλοὺς ἀνατείνας εἰς τὸν οὐρανὸν τοὺς ὀφθαλμοὺς, ἀπεξαίνετο ταῖς μάστιξιν τὰς σάρκας ὁ γέρων, καὶ κατεῤῥεῖτο τῷ αἵματι, 
καὶ τὰ πλευρὰ κατετιτρώσκετο, καὶ πίπτων εἰς τὸ ἔδαφος, ἀπὸ τοῦ μὴ φέρειν τὸ σῶμα τὰς ἀλγηδόνας, ὀρθὸν εἶχεν καὶ ἀκλινῆ τὸν λογισμόν. 
Λὰξ γέ τοι τῶν πικρῶν τις δορυφόρων, εἰς τοὺς κενεῶνας ἐναλλόμενος ἔτυπτεν, ὅπως ἐξανίσταιτο πίπτων. 
Ὁ δὲ ὑπέμενεν τοὺς πόνους, καὶ περιεφρόνει τῆς ἀνάγκης, καὶ διεκαρτέρει τοὺς αἰκισμοὺς, 
καὶ καθάπερ γενναῖος ἀθλητὴς τυπτόμενος ἐνίκα τοὺς βασανίζοντας ὁ γέρων. 
Ἱδρῶν γέ τοι τὸ πρόσωπον, καὶ ἐπασθμαίνων σφοδρῶς, καὶ ὑπ' αὐτῶν τῶν βασανιζόντων ἐθαυμάζετο ἐπὶ τῇ εὐτυχίᾳ. 
Ὅθεν τὰ μὲν ἐλεοῦντες τὰ τοῦ γήρως αὐτοῦ, τὰ δὲ ἐν συμπαθείᾳ τῆς συνηθείας ὄντες, 
τὰ δὲ ἐν θαυμαστῷ τῆς καρτερίας προσιόντες αὐτῷ τινὲς τῶν τοῦ βασιλέως ἔλεγον, 
τί τοῖς κακοῖς τούτοις σεαντὸν ἀλογίστως ἀπολλεῖς, 
Ἐλεάζαρ; ἡμεῖς μὲν τῶν ἡψημένων βρωμάτων παραθήσομεν· σὺ δὲ ὑποκρινόμενος τῶν ὑείων ἀπογεύσασθαι, σώθητι. 
Καὶ ὁ Ἐλεάζαρος, ὥσπερ πικρότερον διὰ τῆς συμβουλίας αἰκισθεὶς, ἀνεβόησεν, 
μὴ οὕτως κακῶς φρονήσαιμεν οἱ Ἁβραὰμ παῖδες, ὥστε μαλακοψυχήσαντας ἀπρεπὲς ἡμῖν δρᾶμα ὑποκρίνασθαι. 
Καὶ γὰρ ἀλόγιστον, εἰ πρὸς ἀλήθειαν ζήσαντες τὸν μέχρι γήρως βίον, καὶ τὴν ἐπ' αὐτῶν δόξαν νομίμως φυλάσσοντες, νῦν μεταβαλοίμεθα, 
καὶ αὐτοὶ μὲν ἡμεῖς γενοίμεθα τοῖς νέοις ἀσεβείας τύπος, ἵνα παράδειγμα γενώμεθα τῆς μιεροφαγίας. 
Αἰσχρὸν γὰρ εἰ ἐπιβιώσωμεν ἀλίγον χρόνον, 
καὶ τοῦτον καταγελώμενοι πρὸς ἁπάντων ἐπὶ δειλίᾳ· καὶ ὑπὸ μὲν τοῦ τυράννου καταφρονηθῶμεν ὡς ἄνανδροι, τὸν δὲ θεῖον ἡμῶν νόμον μέχρι θανάτου μὴ προασπίσαιμεν. 
Πρὸς ταῦτα ὑμεῖς μὲν, ὦ Ἁβραὰμ παῖδες, εὐγενῶς ὑπὲρ τῆς εὐσεβείας τελευτᾶτε. 
Οἱ δὲ τοῦ τυράννου δορυφόροι, τί μέλλετε; 
Πρὸς τὰς ἀνάγκας οὕτως μεγαλοφρονοῦντα αὐτὸν ἰδόντες καὶ μηδὲ πρὸς τὸν οἰκτιρμὸν αὐτῶν μεταβαλλόμενον, ἐπὶ πῦρ αὐτὸν ἤγαγον. 
Ἔνθα διὰ κακοτέχνων ὀργάνων καταφλέγοντες αὐτὸν ὑπερέπτοσαν, καὶ δυσώδεις χυλοὺς εἰς τοὺς μυκτῆρας αὐτοῦ κατέχεον. 
Ὁ δὲ μέχρι τῶν ὀστέων ἤδη κατακεκαυμένος καὶ μέλλων λιποθυμεῖν, ἀνέτεινεν τὰ ὄμματα πρὸς τὸν Θεὸν, καὶ εἶπεν, σὺ οἶσθα, Θεὲ, παρόν μοι σώζεσθαι, 
βασάνοις καυστικαῖς ἀποθνήσκω διὰ τὸν νόμον. 
Ἵλεως γενοῦ τῷ ἔθνει σου, ἀρκεσθεὶς τῇ ἡμετέρᾳ περὶ αὐτῶν δίκῃ. 
Καθάρσιον αὐτῶν ποίησον τὸ ἐμὸν αἷμα, καὶ ἀντίψυχον αὐτῶν λαβὲ τὴν ἐμὴν ψυχήν. 
Καὶ ταῦτα εἰπὼν ὁ ἱερὸς ἀνὴρ εὐγενῶς ταῖς βασάνοις ἐναπέθανεν, καὶ μέχρι τῶν τοῦ θανάτου βασάνων ἀντέστη τῷ λσγισμῷ διὰ τὸν νόμον. 
Ὁμολογουμένως οἶν δεσπότης ἐστὶν τῶν παθῶν ὁ εὐσεβὴς λογισμός. 
Εἰ γὰρ τὰ πάθη τοῦ λογισμοῦ κεκρατήκει, τούτοις ἂν ἀπεδόμην τὴν τῆς ἐπικρατείας μαρτυρίαν. 
Νυνὶ δὲ τοῦ λογισμοῦ τὰ πάθη νικήσαντος, αὐτῷ προσηκόντως τὴν τῆς ἡγεμονίας προσνέμομεν ἐξουσίαν. 
Καὶ δίκαιόν ἐστιν ὁμολογεῖν ἡμᾶς, τὸ κράτος εἶναι τοῦ λογισμοῦ, ὅπου γε καὶ τῶν ἔξωθεν ἀλγηδόνων ἐπικρατεῖ. 
Ἐπεὶ καὶ γελοῖον· καὶ οὐ μόνον τῶν ἀλγηδόνων ἐπιδείκνυμι κεκρατηκέναι τὸν λογισμὸν, ἀλλὰ καὶ τῶν ἡδονῶν κρατεῖν, μηδὲ αὐταῖς ὑπείκειν. 
Ὥσπερ καὶ ἄριστος κυβερνήτης ὁ τοῦ πατρὸς ἡμῶν Ἐλεαζάρου λογισμὸς, πηδαλιουεχῶν τὴν τῆς εὐσεβείας ναῦν ἐν τῷ τῶν παθῶν πελάγει, 
καὶ καταικιζόμενος ταῖς τοῦ τυράννου ἀπειλαῖς, καὶ καταντλούμενος ταῖς τῶν βασάνων τρικυμίαις, 
κατ' οὐδένα τρόπον μετέτρεψεν τοὺς τῆς εὐσεβείας οἴακας, ἕως οὗ ἔπλευσεν ἐπὶ τὸν τῆς θανάτου νίκης λιμένα. 
Οὐχ οὕτως πόλις πολλοῖς καὶ ποικίλοις μηχανήμασιν ἀντέσχεν ποτὲ πολιορκουμένη, ὡς ὁ πανάγιος ἐκεῖνος τὴν ἱερὰν ψυχὴν αἰκισμοῖς τε καὶ στρέβλαις πυρπολούμενος, ἐκίνησεν τοὺς πολιορκοῦντας, διὰ τὸν ὑπερασπίζοντα τῆς εὐσεβείας λογισμόν. 
Ὥσπερ γὰρ πρόκρημνον ἄκραν, τὴν ἑαυτοῦ διὰνοιαν ὁ πατὴρ Ἐλεάζαρος ἐκτείνας, περιέκλασεν τοὺς μαινομένους τῶν παθῶν κλύδωνας. 
Ὦ ἄξιε τῆς ἱερωσύνης ἱερεῦ, οὐκ ἐμίανας τοὺς ἱεροὺς ὀδόντας, οὐδὲ τὴν θεοσέβειαν καὶ καθαρισμὸν χωρήσασαν γαστέρα ἐκοινώνησας μιεροφαγίᾳ· 
Ὦ σύμφωνε νόμου, καὶ φιλόσοφε θείου βίου. 
Τοίουτους δεῖ εἶναι τοὺς δημιουργοῦντας τὸν νόμον ἰδίῳ αἵματι, καὶ γενναίῳ ἱδρῶτι τοῖς μέχρι θανάτου πάθεσιν ὑπερασπίζοντας. 
Σὺ πάτερ, τὴν εὐνομίαν ἡμῶν διὰ τῶν ὑπομονῶν εἰς δόξαν ἐκύρωσας, καὶ τὴν ἁγιαστίαν σεμνολογήσας οὐ κατέλυσας, καὶ διὰ τῶν ἔργων ἐπιστοποίησας τοὺς τῆς φιλοσοφίας λὸγους. 
Ὦ βασάνων βιότερε γέρων, πυρὸς εὐτονώτερε πρεσβύτα, καὶ παθῶν μέγιστε βασιλεῦ Ἐλεάζαρ. 
Ὥσπερ γὰρ ὁ πατὴρ Ἀαρὸν τῷ θυμιατηρίῳ κατθωπλισμένος, διὰ τοῦ ἐθνοπλήθου ἐπιτρέχων τὸν ἐμπυριστὴν ἐνίκησεν ἄγγελον. 
Οὕτως ὁ Ἀαρωνίδης Ἐλεάζαρος διὰ τοῦ πυρὸς ὑπερτηκόμενος οὐ μετετράπη τὸν λογισμόν. 
Καίτοι τὸ θαυμασιώτατον, γέρων ὢν, λελυμένων μὲν ἤδη τῶν τοῦ σώματος πόνων, καὶ περιεχαλασμένων δὲ τῶν σαρκῶν, κεκμηκότων δὲ καὶ τῶν νεύρων, ἀνενέασεν. 
Τῷ πνεύματι τοῦ λογισμοῦ, καὶ τῷ Ἰσακείῳ λογισμῷ τὴν πολυκέφαλον στρέβλαν ἠκύρωσεν. 
Ὦ μακαρίου γήρως, καὶ σεμνῆς πολιᾶς, καὶ βίου νομίμου, ὃν πιστὴ θανάτου σφραγὶς ἐτελείωσεν. 
Εἰ δὲ τοίνυν γέρων τῶν μέχρι θανάτου βασάνων περιεφρόνησεν δι' εὐσέβειαν, ὁμολογουμένως ἡγεμών ἐστιν τῶν παθῶν ὁ εὐσεβὴς λογισμός. 
Ἴσως δ' ἂν εἴποιέν τινες, τῶν παθῶν οὐ πάντες περικρατοῦσιν, ὅτι οὐδὲ πάντες φρόνιμον ἔχουσιν τὸν λογισμόν. 
Ἀλλ' ὅσοι εὑσεβείας προνοοῦσιν ἐξ ὅλης καρδίας, οὗτοι μόνοι δύνανται κρατεῖν τῶν τῆς σαρκὸς παθῶν· 
οἱ πιστεύοντες, ὅτι Θεῷ οὐκ ἀποθνήσκουσιν, ὥσπερ γὰρ οἱ πατριάρχαι ἡμῶν Ἁβραὰμ, Ἰσαὰκ, Ἰακὼβ, ζῶσι τῷ Θεῷ. 
Οὐδὲν οὖν ἐναντιοῦται τὸ φαίνεσθαί τινας παθοκρατεῖσθαι διὰ τὸν ἀσθενῆ λογισμόν. 
Ἐπεὶ τίς πρὸς ὅλον τὸν τῆς φιλοσοφίας κανόνα εὐσεβῶς φιλοσοφῶν, καὶ πεπιστευκὼς Θεῷ, 
καὶ εἰδὼς ὅτι διὰ τὴν ἀρετὴν πάντα πόνον ὑπομένειν μακάριόν ἐστιν, οὐκ ἂν περικρατήσειεν τῶν παθῶν διὰ τὴν εὐσέβειαν; 
μόνος γὰρ ὁ σοφὸς καὶ σώφρων ἀνδρεῖός ἐστιν τῶν παθῶν κύριος. 
Διὰ τοῦτο γέ τοι καὶ μειρακίσκοι τῷ τῆς εὐσεβείας λογισμῷ φιλοσοφουντες χαλεπωτερων βασανιστηρίων ἐπεκράτησαν. 
Ἐπειδὴ γὰρ κατὰ τὴν πρώτην πεῖραν ἐνικήθη περιφανὴς ὁ τύραννος, μὴ δυνηθεὶς ἀναγκάσαι γέροντα μιαιροφαγῆσαι. 
Τὸ δὲ δὴ σφόδρα περιπαθῶς ἐκέλευσεν ἄλλους ἐκ τῆς ἠλικίας τῶν Ἑβραίων ἀγαγεῖν· καὶ εἰ μὲν μιεροφαγήσαιεν, ἀπολύειν φάγοντας· εἰ δὲ ἀντιλέγοιεν, πικρότερον βασανίζειν. 
Ταῦτα διαδεξαμένου τοῦ τυράννου, παρῆσαν ἀγόμενοι μετὰ γηραιᾶς μητρὸς ἑπτὰ ἀδελφοὶ, καλοί τε καὶ αἰδήμονες καὶ γενναῖοι καὶ ἐν παντὶ χαριέντες. 
Οὓς ἰδὼν ὁ τύραννος καθάπερ ἐν χορῷ περιέχοντας μέσην τὴν μητέρα, ἤσθετο ἐπ' αὐτοῖς, καὶ τῆς εὐπρεπείας ἐκπλαγεὶς καὶ τῆς εὐγενείας προσεμειδίασεν αὐτοῖς, καὶ πλησίον καλέσας, ἔφη, 
Ὦ νεανίαι φιλοφρόνως ἐγὼ καθ' ἐνὸς ἑκάστου ὑμῶν θαυμάζω τὸ κάλλος· καὶ τὸ πλῆθος τοσούτων ἀδελφῶν ὑπερτιμῶν, οὐ μόνον συμβουλεύω μὴ μανῆναι τὴν αὐτὴν τῷ προβασανισθέντι γέροντι μανίαν· 
ἀλλὰ καὶ παρακαλῶ συνείξαντας τῆς ἐμῆς ἀπολαῦσαι φιλίας· δυναίμην γὰρ ὥσπερ κολάζειν τοὺς ἐπιτάγμασιν, σὕτως καὶ εὐεργετεῖν τοὺς εὐπειθοῦντάς μοι. 
Πιστεύσατε οὖν, καὶ ἀρχὰς ἐπὶ τῶν ἐμῶν πραγμάτων ἡγεμονικὰς λήψεσθε, ἀρνησάμενοι τὸν πάτριον ἡμῶν τῆς πολιτείας θεσμόν· 
καὶ μεταλαβόντες Ἑλληνικοῦ βίου, καὶ μεταδιαιτηθέντες ἐντρυφήσατε ταῖς νεότησιν ὑμῶν. 
Ἐπεὶ ἐὰν ὀργίλως με διάθησθε διὰ τῆς ἀπειθείας ὑμῶν, ἀναγκάσετέ με ἐπὶ δειναῖς κολάσεσιν ἕνα ἕκαστον ὑμῶν διὰ τῶν βασάνων ἀπολέσαι. 
Κατελεήσατε οὖν ἑαυτοὺς, οὕς καὶ ὁ πολέμιος ἔγωγε καὶ τῆς ἡλικίας καὶ τῆς εὐμορφίας οἰκτείρομαι. 
Οὐ διαλογιεῖσθε τοῦτο, ὅτι οὐδὲν ὑμῖν ἀπειθήσασιν πλὴν τοῦ μετὰ στρεβλῶν ἀποθανεῖν ἀπόκειται; 
Ταῦτα δὲ λέγων, ἐκέλευσεν εἰς τὸ ἔμπροσθεν προτεθῆναι τὰ βασανιστήρια, ὅπως καὶ διὰ τοῦ φόβου πείσειεν αὐτοὺς μιεροφαγῆσαι. 
Ὡς δὲ τροχούς τε καὶ ἀρθενβόλους στρεβλωτήρια, καὶ τροχαντῆρας καὶ καταπέλτας καὶ λέβητας, τήγανά τε καὶ δακτυλήθρας, καὶ χεῖρας σιδηρᾶς καὶ σφῆνας, καὶ τὰ ζώπυρα τοῦ πυρὸς οἱ δορυφόροι προέθησαν, ὑπολαβὼν δὲ ὁ τύραννος, ἔφη, μειράκια φοβήθητε, 
καὶ ἥν σέβεσθε δίκην, ἵλεως ὑμῖν ἔσται δι' ἀνάγκην παρανομήσασιν. 
Οἱ δὲ ἀκούσαντες ἐπαγωγὰ, καὶ ὁρῶντες δεινὰ, οὐ μόνον οὐκ ἐφοβήθησαν, ἀλλὰ καὶ ἀντεφιλοσόφῃσαν τῷ τυράννῳ, καὶ διὰ τῆς εὐλογιστίας τὴν τυραννίδα αὐτοῦ κατέλυσαν. 
Καί τοι λογισώμεθα· εἰ δειλόψυχοί τινες ἦσαν, καὶ ἄνανδροι ἐν αὐτοῖς, ποίοις ἂν ἐχρήσαντο λόγοις; οὐχὶ τούτοις; 
Ὦ τάλανες ἡμεῖς, καὶ λίαν ἀνόητοι· βασιλέως ἡμᾶς παρακαλοῦντος, καὶ ἐπὶ εὐεργεσίᾳ φωνοῦντος, μὴ πεισθείημεν αὐτῷ; 
Τί βουλήμασιν κενοῖς ἐαυτοὺς εὐφραίνομεν, καὶ θανατηφόρον ἀπείθειαν τολμῶμεν; 
Οὐ φοβησόμεθα, ἄνδρες ἀδελφοί, τὰ βασανιστήρια, καὶ λογιούμεθα τὰς τῶν βασάνων ἀπειλὰς, καὶ φευξόμεθα τὴν κενοδοξίαν ταύτην καὶ ὀλεθροφόρον ἀλαζονείαν; 
Ἐλεήσωμεν τὰς ἑαυτῶν ἡλικίας, καὶ κατοικτειρήσωμεν τὸ τῆς μητρὸς γῆρας· 
καὶ ἐνθυμηθῶμεν, ὅτι ἀπειθοῦντες τεθνηξόμεθα. 
Συγγνώσεται δὲ ἡμῖν καὶ ἡ θεία δίκη δι' ἀνάγκην τὸν βασιλὲα φοβηθεῖσιν. 
Τί ἐξάγομεν ἑαυτοὺς τοῦ ἡδίστου βίου, καὶ ἐπιστεροῦμεν ἐαυτοὺς τοῦ γλυκέος κόσμου; 
Μὴ βιαζώμεθα τὴν ἀνάγκην, μηδὲ κενοδοξήσωμεν ἐπ' τῇ ἑαυτῶν στρέβλῃ. 
Οὐδὲ αὐτὸς ὁ ναὸς ἑκουσίως ἡμᾶς θανατοῖ φοβηθέντας τὰ βασανιστήρια. 
Πόθεν ἡμῖν ἡ τοσαύτη ἐντέηκεν φιλονεικία, καὶ ἡ θανατεφόρος ἀρέσκει καρτερία, παρὸν μετὰ ἀταραξίας χρὴ τῷ βασιλεῖ πεισθέντας; 
Ἀλλὰ τούτων οὐδὲν εἶπον οἱ νεανίαι βασανίζεσθαι μέλλοντες, οὐδὲ ἐνεθυμήθησαν. 
Ἦσαν γὰρ περιφρονες τῶν παθῶν, καὶ αὐτηκράτορες τῶν ἀλγηδόνων. Ὥστε ἅμα τῷ παύσασθαι τὸν τύραννον συμβουλεύοντα αὐτοῖς μιεροφαγῆσαι, πάντες διὰ μιᾶς φωνῆς ὁμοῦ, ὥσπερ ἀπὸ τῆς αὐτῆς ψυχῆς, εἶπον, 
Τί μέλλεις, ὦ τύραννε; ἕτοιμοι γάρ ἐσμεν ἀποθνήσκειν, ἢ παραβαίνειν τὰς πατρίους ἡμῶν ἐντολάς. 
Καὶ αἰσχυνόμεθα γὰρ τοὺς προγόνους εἰκότως, εἰ μὴ τῇ τοῦ νόμου εὐπειθείᾳ καὶ συμβούλῳ γνώσει χρησαίμεθα. 
Σύμβουλε τύραννε παρανομίας, μὴ ἡμᾶς μισῶν ὑπὲρ αὐτοὺς ἡμᾶς ἐλέα. 
Χαλεπώτερον γὰρ αὐτοὺς τοῦ θανάτου νομίζομεν εἶναί σου τὸν ἐπὶ τῇ παρανόμῳ σωτηρίᾳ ἡμῶν ἔλεον. 
Ἐκφοβεῖς δὲ ἡμᾶς, τὸν διὰ τῶν βασάνων ἡμῖν θάνατον ἀπειλῶν, ὥσπερ οὐχὶ πρὸ βραχέως παρὰ Ἐλεαζάρου μαθών. 
Εἰ δ' οἱ γέροντες τῶν Ἑβραίων διὰ τὴν εὐσέβειαν καὶ βασανισμὸυς ὑπομείναντες ἀπέθανον, ἀποθάνοιμεν ἂν δικαιότερον ἡμεῖς οἱ νέοι, τὰς βασάνους τῶν σῶν ἀναγκῶν ὑπεριδόντες, ἃς καὶ ὁ παιδευτὴς γέρων ἐνίκησεν. 
Πείραζε γαροῦν τύραννε· καὶ τὰς ἡμῶν ψυχὰς εἰ θανατώσεις διὰ τὴν εὐσέβειαν, μὴ νομίσῃς ἡμᾶς βλάπτειν βασανίζων. 
Ἡμεῖς μὲν γὰρ διὰ τῆσδε τῆς κακοπαθείας καὶ ὑπομονῆς, τὰ τῆς ἀρετῆς ἆθλα οἴσομεν. 
Σὺ δὲ διὰ τὴν ἡμῶν μιαροφονίαν αὐτάρχη καρτερήσεις περὶ τῆς θείας δίκης αἰώνιον βάσανον διὰ πυρός. 
Ταῦτα αὐτῶν εἰπόντων, οὐ μόνον ὡς κατὰ ἀπειθούντων ἐχαλέπαινεν ὁ τύραννος, ἀλλ' ὡς καὶ κατὰ ἀχαρίστων ὠργίσθη. 
Ὅθεν τὸν πρεσβύτατον αὐτῶν κελευθέντες παρήγαγον οἱ μαστισταὶ, καὶ διαῤῥήξαντες τὸν χιτῶνα διέδησαν τὰς χεῖρας αὐτοῦ καὶ τοὺς βραχίονας ἱμᾶσιν ἑκατέρωθεν. 
Ὡς δὲ τύπτοντες ταῖς μάστιξιν ἐκοπίασαν, μηδὲν ἀνύοντες, ἀνέβαλον αὐτὸν ἐπὶ τὸν τροχόν. 
Περὶ ὃν κατατεινόμενος ὁ εὐγενὴς νεανίας, ἔξαρθρος ἐγίνετο. 
Καὶ κατὰ πᾶν μέλος κλώμενος κατηγόρει, λέγων, 
Τύραννε μιαιρώτατε, καὶ τῆς οὐρανίου δίκης ἐχθρὲ, καὶ ὠμόφρον, οὐκ ἀνδροφονήσαντά με τοῦτον καταικίζεις τὸν τρόπον, οὐδὲ ἀσεβήσαντα, ἀλλα θείου νόμου προασπίζοντα. 
Καὶ τῶν δορυφόρων λεγόντων, ὁμολόγησον φαγεῖν, οὕπως ἀπαλλαγῇς τῶν βασάνων, 
ὁ δὲ εἶπεν, οὐχ οὕτως ἰσχυρὸς ὑμῶν ἐστιν ὁ τρόπος, ὦ μιαιροὶ διὰκονοι, ὥστε μου τὸν λογισμὸν ἄξαι· τέμνετέ μου μέλη, καὶ πυροῦτε τὰς σάρκας, καὶ στρεβλοῦτε τὰ ἄρθρα. 
Διὰ πασῶν γὰρ ὑμᾶς πείσω τῶν βασάνων· ὅτι μόνοι παῖδες Ἑβραίων ὑπὲρ ἀρετῆς εἰσιν ἀνίκητοι. 
Ταῦτα λέγοντες εἰς πῦρ ἐπέτρωσαν, καὶ διερεθίζοντες, τὸν τροχὸν προσεπικατέτεινον. 
Ἐμολύνετο δὲ πάντοθεν αἵματι ὁ τρόχος, καὶ ὁ σωρὸς τῆς ἀνθρακιᾶς τοῖς τῶν ἰχώρων ἐσβέννυτο σταλαγμοῖς, καὶ περὶ τοὺς αὔξονας τοῦ ὀργάνου περιέῤῥεον αἱ σάρκες. 
Καὶ περιτετηκμένον ἤδη ἔχων τὸ τῶν ὀστέων πῆγμα ὁ μεγαλόφρων καὶ Ἀβραμιαῖος νεανίας οὐκ ἐστέναξεν. 
Ἀλλ' ὥσπερ ἐν πυρὶ μετασχηματιζόμενος εἰς ἀφθαρσίαν, ὑπέμεινεν εὐγενῶς τὰς στρέβλας. 
Μιμήσασθέ με, ἀδελφοὶ, λέγων· μή μου τὸν αἰῶνα λειποτακτήσητε, μηδ' ἐξομόσησθέ μου τὴν τῆς εὐψυχίας ἀδελφότητα· ἱερὰν καὶ εὐγενῆ στρατείαν στρατεύσασθε περὶ τῆς εὐσεβείας. 
Δι' ἧς ἷλεως ἡ δικαία καὶ πάτριος ἡμῶν πρόνοια τῷ ἔθνει γενηθεῖσα τιμωρήσειεν τὸν ἀλάστορα τύραννον. 
Καὶ ταῦτα εἰπὼν ὁ ἱεροπρεπὴς νεανίας, ἀπέῤῥηξεν τὴν ψυχήν. 
Θαυμασάντων δὲ πάντων τὴν καρτεροψυχίαν αὐτοῦ, ἦγον οἱ δορυφόροι τὸν καθ' ἡλικίαν τοῶ προτέρου δεύτερον, καὶ σιδηρᾶς ἐναρμοσάμενοί χεῖρας, ὀξέσιν τοῖς ὄνυξιν, τοῖς ὀργάνοις καταπέλτῃ προσέδησαν αὐτόν. 
Ὡς δὲ, εἰ φαγεῖν βούλοιτο πρὶν βασανίζεσθαι πυνθανόμενοι, τὴν εὐγενῆ γνώμην ἤκουσαν· 
ἀπὸ τῶν τενόντων ταῖς σιδηραῖς χερσὶν ἐπισπασάμενοι, μέχρι γε τῶν γενείων τὴν σάρκα πᾶσαν καὶ τὴν τῆς κεφαλῆς δορὰν οἱ παρδάλειοι θῆρες ἀπέσυραν· ὁ δὲ ταύτην βαρέως τὴν ἀλγηδόνα καρτερῶν, ἔλεγεν, 
Ὡς ἡδὺς πᾶς τρόπος θανάτου, διὰ τὴν πάτριον ἡμῶν εὐσέβειαν· ἔφη τε πρὸς τὸν τύραννον, 
Οὐ δοκεῖς, πάντων ὠμότατε τύραννε, πλεῖων ἐμοῦ σε νὺν βασανίζεσθαι, ὁρῶν σου νικώμενον τὸν τῆς τυραννίδος ὑπερήφανον λογισμὸν ὑπὸ τῆς διὰ τὴν εὐσέβειαν ἡμῶν ὑπομονῆς. 
Ἐγὼ μὲν γὰρ ταῖς διὰ τὴν ἀρετὴν ἡδοναῖς τὸν πόνον ἐπικουφίζομαι. 
Σὺ δὲ ἐν ταῖς τῆς ἀσεβείας ἀπειλαῖς βασανίζῃ· οὐκ ἐκφεύξῃ δὲ, μιαιρότατε τύραννε, τὰς τῆς θείας ὀργῆς δίκας. 
Καὶ τούτου τὸν ἀοίδιμον θάνατον καρτερήσαντος, ὁ τρίτος ἤγετο, παρακαλούμενος πολλὰ ὑπὸ πολλῶν ὅπως ἀπογευσάμενος σώζοιτο. 
Ὁ δὲ ἀναβοήσας, ἔφη, ἤ ἀγνοεῖτε, ὅτι αὐτός με τοῖς ἀποθανοῦσιν ἔσπειρεν πατὴρ, καὶ ἡ αὐτὴ μήτηρ ἐγέννεσιν, καὶ ἐπὶ τοῖς αὐτοῖς ἀνετράφην δόγμασιν; 
Οὐκ ἐξόμνυμαι τὴν εὐγενῆ τῆς ἀδελφότητος συγγένειαν. 
Πρὸς ταῦτα εἴ τι ἔχετε κολαστήριον προσαγάγετε τῷ σώματί μου· τῆς γὰρ ψυχῆς μου, οὐδ' ἂν θέλητε, ἅψασθαι δύνασθε. 
Οἱ δὲ πίκρῶς ἐνέγκαντες τὴν παῤῥησίαν τοῦ ἀνδρὸς, ἀρθρεμβόλοις ὀργάνοις τὰς χεῖρας αὐτοῦ καὶ τοὺς πόδας ἐξήρθρουν, καὶ ἐξ ἁρμῶν ἀναμοχλεύοντες ἐξεμέλιζον· 
καὶ τοὺς δακτύλους, καὶ τοὺς βραχίονας, καὶ τὰ σκέλη, καὶ τοὺς ἀγκῶνας περιέλκων. 
Καὶ κατὰ μηδένα τρόπον ἰσχύοντες αὐτὸν ἄγξαι, περισύραντες τὸ δέρμα σὺν ἄκραις ταῖς τῶν δακτύλων κορυφαῖς ἀπεσκύθιζον, καὶ εὐθέως ἦγον ἐπὶ τὸν τροχόν. 
Περὶ ὃν ἐκ σφονδύλων ἐκμελιζόμενος ἑώρα τὰς ἑαυτοῦ σάρκας περιλακιζομένας καὶ κατὰ σπλάγχνων σταγόνας αἵματος ἀποῤῥεούσας. 
Μέλλων δὲ ἀποθνήσκειν, 
ἔφη, ἡμεῖς μὲν ὦ μιαιρώτατε τύραννε, διὰ παιδείαν καί ἀρετὴν Θεοῦ ταῦτα πάσχομεν. 
Σὺ δὲ διὰ τὴν ἀσέβειαν καὶ μιαιφονίαν, ἀκαταλύτους καρτερήσεις βασάνους. 
Καὶ τούτου θανόντος ἀδελφοπρεπῶς, τὸν τέταρτον ἐπεσπῶντο, λέγοντες, 
Μὴ μανῄς καὶ σὺ τοῖς ἀδελφοῖς σου τὴν αὐτὴν μανίαν· ἀλλὰ πεισθεὶς τῷ βασιλεῖ, σῶζε σεαυτόν. 
Ὁ δὲ αὐτοῖς ἔφη, οὐχ οὕτως καυστικώτερον ἔχετε κατ' ἐμοῦ τὸ πῦρ, ὥστε με δειλανδρῆσαι. 
Μὰ τὸν μακάριον τῶν ἀδελφῶν μου θάνατον, καὶ τὸν αἰώνιον τοῦ τυράννου ὄλεθρον, καὶ τὸν ἀοίδιμον τῶν εὐσεβῶν βίον, οὐκ ἀρνήσομαι τὴν εὐγενῆ ἀδελφότητα. 
Ἐπινόει, τύραννε, βασάνους· ἵνα καὶ διὰ τούτων μάθῃς, ὅτι ἀδελφός εἰμι τῶν προβεβανασισθέντων. 
Ταῦτα ἀκούσας ὁ αἱμοβόρος καὶ φονώδης καὶ πανμιαιρώτατος Ἀντίοχος, ἐκέλευσεν τὴν γλῶτταν αὐτοῦ ἐκτεμεῖν. 
Ὁ δὲ ἔφη, κᾂν ἀφέλῃς τὸ τῆς φωνῆς ὄργανον, καὶ σιωπώντων ἀκούει ὁ Θεός. 
Ἰδοὺ κεχάλασται ἡ γλῶσσα· τέμνε· οὐ γὰρ παρὰ τοῦτο τὸν λογισμὸν ἡμῶν γλωσσοτομήσεις. 
Ἡδέως ὑπὲρ τοῦ Θεοῦ τὰ τοῦ σώματος μέλη ἀκρωτηριαζόμενα. 
Σὲ δὲ ταχέως μετελεύσεται ὁ Θεός· τὴν γὰρ τῶν θείων ὕμνων μελῳδὸν γλῶτταν ἐκτέμνεις. 
Ὡς δὲ καὶ οὗτος ταῖς βασάνοις καταικισθεὶς ἐναπέθανεν, ὁ πέμπτος παρεπήδησεν, λέγων, 
Οὐ μέλλω, τύραννε, πρὸς τὸν ὑπὲρ τῆς ἀρετῆς βασανισμὸν παραιτεῖσθαι. 
Αὐτὸς δ' ἀπ' ἐμαυτοῦ παρῆλθον, ὅπως κᾀμὲ κατακτείνας, περὶ πλειόνων ἀδικημάτων ὀφειλήσῃς τῇ οὐρανίῳ δίκῃ τιμωρίαν. 
Ὦ μισάρετε καὶ μισάνθρωπε, τὶ δράσαντας ἡμᾶς τοῦτον πορθεῖς τὸν τρόπον; 
Ἢ κακόν σοι δοκεῖ, ὅτι τὸν πάντων κτιστὴν εὐσεβοῦμεν, καὶ κατὰ τὸν ἐνάρετον αὐτοῦ ζῶμεν νόμον; 
Ἀλλὰ ταῦτα τιμῶν, οὐ βασάνων ἐστὶν ἄξια. 
Εἴπερ ᾐσθάνου ἀνθρώπου πόθων, καὶ ἐλπίδα εἶχες παρὰ Θεῷ σωτηρίου· 
νῦν ἰδὲ ἀλλότριος ὢν Θεοῦ, πολεμεῖς τοὺς εὐσεβοῦντας εἰς τὸν Θεόν. 
Τοιαῦτα λέγοντα οἱ δορυφόροι δήσαντες, αὐτὸν εἷλκον ἐπὶ τὸν καταπέλτην· 
ἐφ' ὃ δήσαντες αὐτὸν ἐπὶ τὰ γόνατα, καὶ ταῦτα ποδάγραις σιδηραῖς ἐφορμάσαντες τὴν ὀσφὺν αὐτοῦ ἐπὶ τὸν τροχιαῖον σφῆνα κατέκαμψαν· περὶ ὃν ὅλος ἐπὶ τὸν τρονὸν σκορπίου τρόπον ἀνακλώμενος ἐξεμελίζετο. 
Κατὰ τοῦτον τὸν τρόπον καὶ τὸ πνεῦμα στενοχωρούμενος, καὶ τὸ σῶμα ἀγχόμενος, καλὰς, ἔλεγεν, ἄκων, 
ὦ τύραννε, χάριτας ἡμῖν χαρίζῃ διὰ γενναιοτέρων πόνων ἐπιδείξασθαι παρέχων τὴν εἰς τὸν νόμον ἡμῶν καρτερίαν. 
Τελευτήσαντος δὲ καὶ τούτου, ὁ ἕκτος ἤγετο μειρακίσκος· ὃς πυνθανομένου τοῦ τύραννου εἰ βούλοιτο φαγὼν ἀπολύεσθαι, ὁ δὲ ἔφη, 
Ἐγὼ τῇ μὲν ἡλικίᾳ τῶν ἀδελφῶν μου εἰμὶ νεώτερος, τῇ δὲ διανοίᾳ ἡλικιώτης· 
Εἰς τὰ αὐτὰ γὰρ καὶ γεννηθέντες καὶ τραφέντες, ὑπὲρ τῶν αὐτῶν καὶ ἀποθνήσκειν ὀφείλομεν ὁμοίως. 
Ὥστε εἰ σοὶ δοκεῖ βασανίξειν, μὴ μιαιροφαγοῦτας βασάνιζε. 
Ταῦτα αὐτὸν εἰπόντα παρῆγον ἐπὶ τὸν τροχόν. 
Ἐφ' οὗ κατατεινόμενος εὐμελῶς καὶ ἐκσφονοδυλιζόμενος ὑπεκαίετο. 
Καὶ ὀβελίσκους ὀξεῖς πυρώσαντες, τοῖς νότοις προσέφερον· καὶ τὰ πλευπὰ διαπείραντες, ἀπ' αὐτοῦ σπλάγχνα διέκαιον. 
Ὁ δὲ βασανιζόμενος, ὦ ἱεροπρεποῦς αἰῶνος, ἔλεγεν, ἐφ' ὃν διὰ τὴν εὐσέβειαν εἰς γυμνασίαν πόνων ἀδελφοὶ τοσοῦτοι κληθέντες οὐκ ἐνικήθημεν. 
Ἀνίκητος γάρ ἐστιν, ὦ τύραννε, ἡ εὐσεβὴς ἐπιστήμη. 
Καλοκᾳγαθίᾳ καθωπλισμένος τεθνήξομαι κἀγὼ μετὰ τῶν ἀδελφῶν μοῦ. 
Μέγαν σοὶ προσβάλλων καὶ αὐτὸς ἀλάστορα, καινουργὲ τῶν βασάνων, καὶ πολέμιε τῶν ἀληθῶς εὐσεβούντων. 
Ἓξ μειράκια κατελύσαμέν σου τὴν τυραννίδα. 
Τὸ γὰρ μὴ δυνηθῆναί σε μεταπεῖσαι τὸν λογισμὸν ἡμῶν, μήτε βιάσασθαι πρὸς τὴν μιαιροφαγίαν, οὐ κατάλυσίς ἐστιν σοῦ; 
Τὸ πῦρ σου ψυχρὸν ἡμῖν, καὶ ἄπονοι οἱ καταπέλται, καὶ ἀδύνατος ἡ βία σου. 
Οὐ γὰρ τυράννου, ἀλλὰ θείου νόμου προεστήκασιν ἡμῶν οἱ δορυφίροι· διὰ τοῦτο ἀνίκητον ἕχομεν τὸν λογι σμόν. 
Ὡς δὲ καὶ οὗτος μακαρίως ἐναπέθανεν καταβληθεὶς εἰς λέβητα, ὁ ἕβδομος παρεγίνετο, πάντων νεώτερος. 
Ὅν κατοικτειρήσας ὁ τύραννος, καίπερ δεινῶς ὑπὸ τῶν ἀδελφῶν αὐτοῦ κακισθεὶς, ὁρῶν ἤση τὰ δεσμὰ περικείμενον, 
πλησιέστερον αὐτὸν μετεπέμψατο, καὶ παρηγορεῖν ἐπειρᾶτο, λέγων, 
Τῆς μὲν τῶν ἀδελφῶν σου ἀπονοίας τὸ τὲλος ὁρᾷς· διὰ γὰρ ἀπείθειαν στρεβλωθέντες τεθνήκασιν, σὺ, εἰ μὲν μὴ πεισθείης, τὰλας βασανισθεὶς καὶ πασανισθεὶς καὶ αὐτὸς τεθνήξῃ πρὸ ὥρας. 
Πεισθείης δὲ φίλος ἔσῃ, καὶ τῶν ἐπὶ τῆς βασιλείας ἀφηγήσῃ πραγμάτων. 
Καὶ ταῦτα παρακαλῶν, τὴν μητέρα τοῦ παιδὸς μετεπέμψατο, ὄπως αὐτὴν ἐμεήσαν υἱῶν στερηθῖσαν παρορμήσειεν ἐπὶ τὴν μσωτηρίαν, εὐπειθῆ τὸν περιλειπόμενον. 
Ὁ δὲ τῆν μητρὸς τῇ Ἐβραΐδι φωνῇ προτρεψαμένης· αὐτὸν (ὡς ἐροῦμεν μετὰ μικρὸν ὕστερον.) ἀπολύσατε με, φησίν· 
εἴπω τῷ βασιλεῖ καὶ τοῖς σὺν αὐτῷ φίλοις πᾶσιν. 
Καὶ ἐπιχαρέντες μάλιστα ἐπὶ τῇ ἐπαγγελίᾳ τοῦ παιδὸς ταχέως ἔλυσαν αὐτόν. 
Καὶ σραμὼν ἐπὶ πλησίον τῶν τηγάνων, ἔφη, ἀνόσιε, 
φησὶν, καὶ πάντων τῶν πονηρῶν ἀσεβέστατε τύραννε, οὐκ ᾐδέσθης παρὰ τοῦ Θεοῦ λαβὼν τὰ ἀγαθὰ καὶ τὴν βασιλείαν, τούς θεράποντας αὐτοῦ κατακτεῖναι, καὶ τοὺς τῆς εὐσεβείας στρεβλῶσαι; 
Ἀνθ ὧ ταμιεύεταί σε ἡ θεια δίκη πυκνοτέρῳ καὶ αἰωνίῳ πυεὶ καὶ βασάνοις, αἳ εἰς ὅλον τὸν αιῶνα οὐκ ἀνήσουσίν σε. 
Οὐκ ᾐδέσθης ἄνθρωπος ὢν, θηριωδέστατε, τοὺς ὁμοιοπαθεῖς καὶ ἐκ τῶν αὐτῶν γεγονότας στοιξείων γλωττοτομῆσαι, καὶ τοῦτον καταικίσας τὸν τρόπον βασανίσαι; 
Ἀλλ' οἱ μὴν εὐγενῶς ἀποθανόντες ἐπλήρωσαν τὴν εἰν τὸν Θεὸν εὐσέβειαν. 
Σὺ δὲ κακὸς κακῶς οἰμώξεις, τοὺς τῆς ἀρετῆν ἀγωνιστὰν ἀναιτίως ἀποκτεῖναι. 
Ὅθεν καὶ αὐτὸς ἀποθνήσκειν μέλλων, ἔφη, 
οὐκ ἀπαυτομολῶ τῆς τῶν ἀδελφῶν μου μαρτυρίας. 
Ἐπικαλοῦμαι δὲ τὸν πατρῷον Θεὸν, ὅπως ἵλεως γένει μου. 
Σὲ δὲ καὶ ἐν τῷ νῦν βίῳ καὶ θανόντα τιμωρήσεται. 
Καὶ ταῦτα κατευξάμενος, ἐαυτὸν ἔοιψεν κατὰ τῶν τηγάνων· καὶ οὕτως ἀπέδεκεν. 
Εἰ δὲ τοίνυν τῶν μέχρι θανάτου πόνων ὑπερεφρόνησαν οἱ ἑπτὰ ἀδελφοὶ, συνομολογεῖται πανταχόθεν, ὅτι αὐτοδέσποτός ἐστιν τῶν παθῶν ὁ εὐσεβὴς λογισμός. 
Ὤσπερ γὰρ εἰ τοῖς πάθεσιν σουλωθέντες ἐμιεροφάγησαν, ἐλέγομεν γὰρ αὐτοὺς τούτοις νενικῆσθαι. 
Νυνὶ δὲ οὐχ οὕτως· ἀλλὰ τῷ ἐπαινουμένῳ λογισμῷ παρὰ περιεγένοντο τῶν παθῶν. 
Καὶ οὐκ ἐστὶν παριδεῖν τὴν ἡγεμονίαν· ἐπεκρὰτησεν γὰρ καὶ πὰθους καὶ πὸνων. 
Πῶς οὖν οὐκ ἐστὶν τούτοις τὴν εὐλογιστίας παθοκρὰτειαν ὁμολογεῖν, οἱ τῶν μὲν διὰ πυρὸς ἀλγησόνων οὐκ ἐπεστράφησαν; 
Καθάπερ γὰρ προπλήταις λιμένων πύργοις τὰς κυμάτων ἀπειλὰς ἀνακόπτοντες, γαληνὸν παρὲχουσιν τοῖς εἰσπλέουσιν τὸν ὅρμον. 
Οὕτος ἡ ἑπτάπυργος τῶν νεανίσκων εὐλογιστία τὸν τῆς εὐσεβείας ὀχυρώσασα λιμένα τὴν τῶν παθῶν ἐνίκησεν ἀκολασίαν. 
Ἱερὸν γὰρ εὐσεβείαν στήσαντες χορὸν παρεθάρσυνον ἀλλήλους, λέγοντες, 
ἀδελφικῶς ἀποθάνοιμεν, ἀδελφοὶ, περὶ τοῦ νόμον· μιμησώμεθα τούς τρεῖς τοὺς ἐπὶ Ἀσσυρίας νεανίσκους, οἵ τῆς ἰσεπόλιδος καμίνου κατεφρόνησαν. 
Μὴ δειλανδρήσωμεν πρὸς τὴν τῆς εὐσεβείας ἀπόδειξιν. 
Καὶ ὁ μὲν, θάῤῥει ἀδελφὲ, ἔλεγεν, ὁ δὲ, εὐγενῶς καρτέρησον. 
Ὁ δὲ, ἔλεγεν, μνήσθητε πόθεν ἐστὲ, ἢ τίνος πατρὸς χειρὶ σφαγιασθῆναι διὰ τὴν εὐσέβειαν ὑπέμεινεν ὁ Ἰσαάκ. 
Εἶς δὲ ἕκαστος καὶ ἀλλήλους ὁμοῦ πάντες ἐφόρων φαιδροὶ καὶ μάλα θαῤῥαλέοι, ἐαυτοὺς, ἔλεγον, τῷ Θεῷ ἀφιερώσωμεν ἐξ ὅλης τῆς καρδίας τῷ δόντι τὰς ψυχὰς, καὶ χρήσωμεν τῇ περὶ τὸν νόμον φυλακῇ τὰ σώματα. 
Μὴ φοβηθῶμεν τὸν δοκοῦντα ἀποκτενεῖν 
Μέγας γὰρ ψυξῆς ἀγὼν καὶ κίνδυνος ἐν αἰωνίῳ βασάνῳ κείμενος τοῖς παραβᾶσιν τῆν ἐντολὴν τοῦ Θεοῦ. 
Καθοπλισώμεθα τοιγαροῦν τῇ τοῦ θείου λογισμοῦ παθοκρατείᾳ. 
Οὕτως παθόντας ἡμᾶς Αβραὰμ καὶ Ἰακὼβ ὑποδέξονται, καὶ πάντες οἱ πατέρες ἐπαινέσουσιν. 
Καὶ ἑνὶ ἑκάστῳ τῶν ἀποστωμένων αὐτῶν ἀδελφῶν ἔλεγον οἱ περιλειπόμενοι, μὴ καταισχύνῃς ἡμᾶς ἀδελφὲ, μηδὲ ἀδελφὲ, μηδὲ ψεύσῃ τούς προαποθανόντας. 
Οὐκ ἀγνοεῖτε δὲ τὰ τῆς ἀνθρωπότητος φίλτρα, ἅπερ ἡ θεία καὶ πάνσοφος πρόνοια διὰ τῆς μητρῴας φυτεύσασα γαστρός· 
ἐν ᾗ τὸν ἶσον ἀδελφοὶ κατοικήσαντες χρόνον, καὶ ἐν τῷ αὐτῳ χρόνῳ πλασθέντες, καὶ ἀπὸ τοῦ αὐτῷ αἵματος ἀξηθέντες, καὶ δια τῆς αὐτῆς ψυχῆς τελεσφορηθέντες, 
καὶ διὰ τῶν ἴσων ἀποτεχθέντες χρόνον, καὶ ἀπὸ τῶν αὐτῶν γαλακτοποτοῦντες πηγῶν, ἀφ' οὗ συντέφονται ἐν ἐναγκαλισμάτων φιλάδελφοι ψυχαί· 
καὶ αὔξοντες σφοδρότερον διὰ συντροφίας, καὶ τῆς καθ' ἠμέραν συνηθείας, καὶ τῆς ἄλλης παιδείας, καὶ τῆν ἡμετέρας ἐν νόμῳ Θεοῦ ἀσκήσεως. 
Οὕτως δὲ τοίνυν καθεστηκυίας τῆς φιλαδελφίας συμπαθούσης, οἱ ἑπτὰ ἀδελφοὶ συμπαθέστερον ἔσχον τὴν πρὸς ἀλλήλους ὁμόνοιαν. 
Νόμῳ γὰρ τῷ αὐτῷ παιδευθέντες, καὶ τὰς αὐτὰς ἐξασκήσαντες ἀρετὰς, καὶ τῷ δικαίῳ συντραφέντες βίῳ, μᾶλλον ἐπ' αὐτοὺς ἥγαγον. 
Ἡ γὰρ ὁμοζηλία τῆς καλοκᾳγαθίας ἐπέτεινεν αὐτῶν τὴν πρὸς ἀλλήλους ὁμόνοιαν. 
Σὺν γάρ τῇ εὐσεβείᾳ ποθεινοτέραν αὐτοῖς κατεσκεύαζεν τὴν φιλαδελφίαν. 
Ἀλλ' ὁμοίως καίπερ τῆς φύσεως καὶ τῆς συνηθείας καὶ τῶν τῆς ἀρετῆς ἠθῶν τὰ τῆς ἀδελφότητος αὐτοῖν φίλτρα συναυξόντων, ἀνέσχοντο διὰ τὴν εὐσέβειαν τοὺν ἀδελφοὺς οἱ ὑπολελειμμένοι τοὺς καταικιζομένους, ὁρῶντες μέχρι θανάτου βασανιζομένους. 
Προσέτι καὶ ἐπὶ τὸν αἰκισμὸν ἐποτρύνοντες, ὡς μὴ μόνον τῶν ἀλγηδόνων περιφρονῆσαι αὐτοὺς, ἀλλὰ καὶ τῆς τῶν ἀδελφῶν φιλαδελφίας παθῶν κρατῆσαι. 
Ὦ βασιλέως λογισμοὶ βασιλικώτεροι καὶ ἐλευθέρων ἐλευθερώτεροι. 
Ἱερὰς καὶ ἐναρμόστους περὶ τῆς εὐσεβείας τῶν ἑπτὰ ἀδελφῶν συμφωνίας. 
Οὐδεὶς ἐκ τῶν ἑπτὰ μειρακίων ἐδειλίασεν, οὐδὲ πρὸς τὸν θάνατον ὤκνησεν. 
Ἀλλὰ πάντες, ὥσπερ ἐπ' ἀθανασίας ὁδὸν τρέχοντες, ἐπὶ τὸν διὰ τῶν βασάνων θάνατον ἔσπευδον. 
Καθάπερ γὰρ χεῖρες καὶ πόδες συμφώνως τοῖς τῆς ψυχῆς ἀφηγήμασιν κινοῦνται· οὕτως οἱ ἱεροὶ μείρακες ἐκεῖνοι ὡς ὑπὸ ψυχῆς ἀθανάτου τῆς εὐσεβείας, πρὸς τὸν ὑπὲρ αὐτῆς συνεφώνησαν θάνατον. 
Ὦ παναγία ἡ συμφώνον ἀδελφῶν ἐβδομάς· καθάπερ γὰρ ἑπτὰ τῆς κοσμοποιΐας ἡμέραι περὶ τὴν εὐσέβειαν, 
οὕτος περὶ τὴν ἑβδομάδα χορεύοντες οἱ μείρακες ἐκύκλουν τὸν τῶν βασάνων φόβον καταλύοντες. 
Νῦν ἡμεῖς ἀκούοντες τῆν θλίψιν τῶν νεανίων ἐκείνων, φρίττομεν· οἱ δὲ οὐ μόνον ὁρῶντες, ἀλλ' οὐδὲ μόνον ἀκούοντες τὸν παραχρῆμα ἀπειλῆς λόγον, ἀλλὰ καὶ πάσχοντες, ἐκαρτέρουν καὶ τοῦτο ταῖς διὰ πυρὸς ὀδύναις. 
Ὧν τί γένοιτο ἐπαλγέστερον; ὀξεῖα γὰρ καὶ σύντομος ἡ τοῦ πυρὸς οὖσα δύναμις, ταχέως διέλυσε τὰ σώματα. 
Καὶ μὴ θαυμαστὸν ἡγεῖσθε, εἰ ὁ λογισμὸς περιεκράτησεν τῶν ἀνδρῶν ἐκείνων ἐν ταῖς βασάνοις, ὅπου γε καὶ γυναικὸς νοῦς πολυτροπωτέρον ὑπερεφρόνησεν ἀλγηδόνων. 
Ἡ μήτηρ γὰρ τῶν ἑπτὰ νεανίσκων ὑπήνεγκεν τὰς ἐφ' ἑνὶ ἐκάστῳ τῶν τέκνων στρέβλας. 
Θεωρεῖτε δὲ πῶς πολύπλοκός ἐστιν ἡ τῆς φιλοτεκνίας στοργὴ, ἕλκουσα πάντα πρὸς τὴν τῶν σπλάγχνων συμπάθειαν. 
Ὅπου γε καὶ τὰ ἄλογα ζῶα ὁμοίαν τὴν πρὸς τὰ ἐξ αὐτῶν γεννώμενα συμπάθειαν καὶ στοργὴν ἔχει τοῖς ἀνθρώποις. 
Καὶ γὰρ τῶν πετεινῶν, τὰ μὲν ἥμερα κατὰ τὰς οἰκίας ὀροφοιτοῦντα προασπίζει τῶν νεοττῶν. 
Τὰ δὲ κατὰ τὰς κορυφὰς ὀρέων καὶ φαράγγων ἀποῤῥῶγας καὶ δένδρων ὀπὰς καὶ τὰς τούτων ἄκρας νοσσοποιησάμενα ἀποτίκτει, καὶ τὸν προσιόντα κωλύει. 
Εἰ δὲ καὶ μὴ δύναιντο κωλύειν, περιπτάμενα κυκλόθεν αὐτῶν ἀλγοῦντα τῇ στοργῇ, ἀνακαλούμενα τῇ ἰδίᾳ φωνῇ, καθ' ὃν δύναται τρόπον βοηθεῖ τοῖς τέκνοις. 
Καὶ τί δεῖ τὴν διὰ τῶν ἀλόγων ζώων ἐπιδεικνύναι τὴν πρὸς τὰ τέκνα συμπάθειαν. 
Ὅπου γε καὶ μέλισσαι περὶ τὸν τῆς κηρογονίας καιρὸν ἐπαμύνονται τοὺς προσιόντας, καὶ καθάπερ σιδήρῳ τῷ κέντρῳ πλήσσουσι τοὺς προσιόντας τῇ νοσσιᾷ αὐτῶν, καὶ ἐπαμύνονται ἕως θανάτου. 
Ἀλλ' οὐχὶ τὴν Ἁβραὰμ ὁμόψυχον τῶν νεανίων μητέρα μετεκίνησεν συμπάθεια τῆς συμπαθείας τέκνων. 
Ὦ λογίσμε τέκνων, παθῶν τύραννε, καὶ εὐσέβεια μητρὶ τέκνων ποθεινοτέρα. 
Μήτηρ δυοῖν προκειμένων εὐσεβείας, καὶ τῆς ἑπτὰ υἱῶν σωτηρίας προκαίρους κατὰ τὴν τοῦ τυράννου ὑπόσχεσιν· 
τὴν εὐσέβειαν μᾶλλον ἠγάπησεν τὴν σώζουσαν εἰς αἰώνιον ζωὴν κατὰ Θεόν. 
Ὦ τίνα τρόπον ἠθολογήσαιμι φιλότεκνα γονέων πάθη, ψυχῆς τε καὶ μορφῆς ὁμοιότητα εἰς μικρὸν παιδὸς χαρακτῆρα θαυμάσιον ἐναπεσφράγιζον, μάλιστα διὰ τὸν τῶν παθῶν τοῖς γεννηθεῖσιν τὰς μητέρας καθεστάναι συμπαθευτέρας. 
Ὅσῳ γὰρ καὶ ἀσθενόψυχοι καὶ πολυγονώτεραι ὑπάρχουσιν μητέρες, τοσούτῳ μᾶλλόν εἰσιν φιλοτεκνότεραι. 
Πασῶν δὲ τῶν μητέρων ἐγένετο ἡ τῶν ἑπτὰ μήτηρ φιλοτεκνοτέρα, ἥ τις ἑπτὰ κυοφορίαις τὴν πρὸς αὐτοὺς ἐπιφυτευομένη φιλοστοργία, 
καὶ διὰ πολλὰς τὰς καθ' ἔκαστον αὐτῶν ὠδῖνας ἠναγκασμένην τὴν εἰς αὐτοὺς ἔχειν συμπάθειαν, 
διὰ τὸν πρὸς τὸν Θεὸν φόβον ὑπερεῖδεν τὴν τῶν τέκνων πρόσκαιρον σωτηρίαν. 
Οὐ μὴν δὲ, ἀλλὰ καὶ διὰ τὴν καλοκᾳγαθίαν τῶν υἱῶν, καὶ τὴν πρὸς τὸν νόμον αὐτῶν εὐπείθειαν, μείζων τὴν ἐν αὐτοῖς ἔσχεν φιλοστοργίαν. 
Δίκαιοί τε γὰρ ἦσαν, καὶ σώφρονες, καὶ σώφρονες, καὶ ἀνδρεῖοι, καὶ μεγαλόψυχοι, καὶ φιλάδελφοι, καὶ μεγαλόψυχοι, καὶ μεψαλόψυχοι, καὶ φιλάδελφοι, καὶ φιλομήτορες οὕτως, ὥστε καὶ μέχρι θανάτου τὰ νόμιμα φυλάσσοντες πείθεσθαι αὐτῇ. 
Ἀλλ' ὅμως, καὶ ὑπὲρ τοσούτων ὄντων τῶν περὶ φιλοτεκνίαν εἰς συμπάθειαν ἑλκόντων τὴν μητέρα, ἐπ' οὐδενὸς αὐτῶν τὸν λογισμὸν αὐτῆς αἱ παμποίκιλοι ἴσχυσαν μετατρέψαι. 
Ἀλλὰ καἰ καθ' ἔνα παῖδα καὶ ὁμοῦ πάντας ἡ μήτηρ ἐπὶ τὸν τῆς εὐσεβείας προετρέπετο θάνατον. 
Ὦ φύσις ἱερὰ, καὶ φίλτρα γονέων καὶ γονεῦσιν φιλόστοργε, καὶ τροφεῖα, καὶ μητέρων ἀδάμαστα πάθη. 
Καθ' ἕνα στρεβλούμενον καὶ φλεγόμενον ὁρῶσα υήτηρ, οὐ μετεβάλετο διὰ τὴν εὐσέβεβειαν. 
Τὰς σάρκας τῶν τέκνων ἑώρα περὶ τὸ πῦρ τηκομένας, καὶ τοὺς τὼν ποδῶν καὶ χειρῶν δακτύλους ἐπὶ γῆς σπαίροντας, καὶ τὰς τῶν κεφαλῶν μέχρι τῶν περὶ τὰ γένεια σάρκας, ὥσπερ προσωπεῖα προκειμένας. 
Ὦ πικροτέρων μὲν νῦν μήτηρ πόνων πειρασθεῖσα, ἤπερ τῶν ἐπ' αὐτοῖς ὠδίνων. 
Ὦ μόνη γυνὴ τὴν εὐσέβειαν ὁλόκληρον ἀποκυήσασα. 
Οὐ μετέρεψέν σε πρωτότοκος ἀποπνέων· οὐδὲ δεύτερον εἰς οἶκτρον βλέπων ἐν βασάνοις· οὐδὲ τρίτος ἀποψύχων. 
Οὐδὲ τοὺς ὀφθαλμοὺς ἑνὸς ἑκάστου θεωροῦσα ταυρηδὸν ἐπὶ τῶν βασάνων ὁρῶντας τὸν αὐτὸν αἰκισμὸν, καὶ τοὺς μυκτῆρας προσημειουμένους αὐτῶν τὸν θάνατον, οὐκ ἔκλαυσας. 
Ἐπὶ σαρξὶν τέκνων ὁρῶσα σάρκας τέκνων ἀποκεκομμένας, καὶ ἐπὶ κεφαλαῖς κεφαλὰς ἀποδειροτομουμένας, καὶ ἐπὶ νεκροῖς νεκροὺς πίπτοντας, καὶ πολυάνδριον ὁρῶσα τῶν τέκνων χορεῖον διὰ τῶν βασάνων, οὐκ ἐδάκρυσας. 
Οὐχ οὕτως σειρήνιοι μελῳδίαι, οὐδὲ κύκνειοι πρὸς φιληκοΐαν φωναὶ τοὺς ἀκούοντας ἐφέλκονται, ὦ τέκνων φωναὶ μετὰ βασάνων μητέρα φωνούντων. 
Πηλίκαις καὶ πόσαις τότε ἡ μήτηρ, τῶν υἱῶν βασανιζομένων τροχοῖς τε καὶ καυτερίοις ἐβασανίζετο βασάνοις; 
Ἀλλὰ τὰ σπλάγχνα αὐτῆς ὁ εὐσεβὴς λογισμὸς ἐν αὐτοῖς τοῖς πάθεσιν ἀνδρειώσας ἐπέτεινεν τὴν πρόσκαιρον φιλοτεκνίαν παριδεῖν. 
Καίπερ ἑπτὰ τέκνων ὁρῶσα ἀπώλειαν· ἀσπάσασα ἡ γενναῖα μήτηρ ἐξέδσεν διὰ τὴν πρὸς Θεὸν πίστιν. 
Καθάπερ γὰρ ἐν βουλευτηρίῳ τῇ ἑαυτῆς ψυχῇ δεινοὺς ὁρῶσα συμβούλους, φύσιν καὶ γένεσιν καὶ φιλοτεκνίαν καὶ τέκνων στρέβλαν. 
Δύο ψήφους κρατοῦσα μήτηρ, θανατηφόρον τε καὶ σωτήριον ὑπὲρ τένων· 
Οὐκ ἐπέγνω τὴν σώζουσαν ἑπτὰ υἱοὺς πρὸς ὀλίγον χρόνον σωτηρίαν. 
Ἀλλὰ τῆς θεοσεβοῦς Ἁβραὰμ καρτερίας ἡ θυγάτηρ ἐμνήσθη. 
Ὦ μήτηρ ἔθνους, ἔκδικε τοῦ νόμου, καὶ ὑπερασπίστεια τῆς εὐσεβείας, καὶ τοῦ διὰ σπλάγχνων ἀγῶνος ἀθλοφόρε. 
Ὦ ἀῤῥένων πρὸς καρτερίαν γενναιοτέρα, καὶ ἀνδρῶν πρὸς ὑπομονὴν ἀνδρειοτέρα. 
Καθάπερ γὰρ ἡ Νῶε κιβωτὸς ἐν τῷ κοσμοπληθεῖ κατακλυσμῷ κοσμοφοροῦσα καρτεροὺς ὑπήνεγκεν τοὺς κλύδωνας· 
οὕτως σὺ, ἡ νομοφύλαξ, πανταχόθεν ἐν τῷ τῶν παθῶν περιαντλουμένη κατακλυσμῷ, καὶ καρτεροῖς ἂν λοιμοῖς ταῖς τῶν υἱῶν βασάνοις συνεχομένη, γενναίως ὑπέμεινας τοὺς τῆς εὐσεβείας χειμῶνας. 
Εἰ δὲ τοίνυν καὶ γυνὴ, καὶ γηραιὰ, καὶ ἑπτὰ παὶδων μήτηρ ὑπέμεινε τὰς μέχρι θανάτου βασάνους ὁρῶσα τῶν τέκνων· ὁμολογουμένως αὐτοκράτωρ ἐστὶν τῶν παθῶν ὁ εὐσεβὴς λογισμός. 
Ἀπέδειξα οὖν ὅτι οὐ μόνον τῶν παθῶν ἄνδρες ἐπεκράτησαν, ἀλλὰ καὶ γυνὴ τῶν μεγίστων βασάνων ὑπερεφρόνησεν. 
Καὶ οὐχ οὕτως οἱ περὶ Δανιὴλ λέοντες ἦσαν ἄγριοι, οὐδὲ Μισαὴλ ἐκφλεγομένη κάμινος λαβροτάτῳ πυρὶ, ὡς τῆς φιλοτεκνίας περιέκαιεν ἐκείνη φύσις, ὁρῶσα αὑτῆς τοὺς ἑπτὰ υἱοὺς βασανιζομένους. 
Ἀλλὰ τῷ λογισμῷ τῆς εὐσεβείας κατέσβεσε τοσαῦτα καὶ τηλικαῦτα πάθη ἡ μήτηρ. 
Καὶ γὰρ τοῦτο ἐπιλογίσασθαι, ὅτι εἰ δειλόψυχος ἦν ἡ γυνὴ, καίπερ μήτηρ οὖσα, ὠλοφύρετο ἂν ἐπ' αὐτοῖς· καὶ ἴσως ἄν ταῦτα οὕτως εἶπεν, 
Ὦ μελέα ἔγωγε, καὶ πολλάκις τρισαθλία, ἥτις ἑπτὰ παῖδας τεκοῦσα, οὐδενὸς μήτηρ γεγένημαι. 
Ὦ μάταιοι ἐπτὰ κυοφορίαι, καὶ ἀνόνητοι ἐπτὰ δεκάηνοι, καὶ ἄκαρποι τιθηνίαι, καὶ ταλαίπωροι γαλακτοτροφίοι. 
Μάτην ἐφ' ὑμῖν, ὦ παῖδες, πολλὰς ὑπέμεινα ὠδῖνας καὶ χαλεπωτέρας φροντίδας ἀνατροφῆς. 
Ὦ τῶν ἐμῶν παίδων, οἱ μὲν ἄγαμοι, οἱ δὲ γαμήσαντες ἀνόνητοι, οὐκ ὄψομαι ὑμῶν τέκνα, οὐδὲ μάμμη κληθεῖσα μακαρισθήσομαι. 
Ὦ ἡ πολύπαις καὶ καλλίπαις ἐγὼ γυνὴ χήρα καὶ μόνη πολύθρηνος. 
Οὐδ' ἂν ἀποθάνω, θάπτοντα τῶν υἱῶν ἕξω τινά. 
Ἀλλὰ τούτῳ τῷ θρήνῳ οὐδένα ὠλοφύρετο ἡ ἱερὰ καὶ θεοσεβὴς μήτηρ. Οὐδ' ἵνα μὴ ἀποθάνωσιν ἀπέτρεπεν αὐτῶν τινα, οὐδ' ὡς ἀποθνησκόντων ἐλυπήθη. 
Ἀλλ' ὥσπερ ἀδαμάντινον ἔχουσα τὸν νοῦν, καὶ εἰς ἀθανασίαν ἀνατίκτουσα τὸν τῶν υἱῶν ἀριθμὸν, μᾶλλον ὑπὲρ τῆς εὐσεβείας ἐπὶ τὸν θάνατον αὐτοὺς προετρέπετο ἱκετεύουσα. 
Ὦ δι' εὐσέβειαν Θεοῦ στρατιῶτι, πρεσβύτι καὶ γυνὴ διὰ καρτερίαν καὶ τύραννον ἐνίκησας, καὶ ἔργοις δυνατωτέρα καὶ λόγοις εὑρέθης ἄνανδρος. 
Καὶ γὰν ὃτε συνελήφθης μετὰ τῶν παίδων, εἱστήκεις τὸν Ἐλεάζαρον ὁρῶσα βασανιζόμενον, καὶ ἔλεγες τοῖς παισὶν ἐν τῇ Ἑβοαΐδι φωνῇ, 
Ὦ παῖδες, γενναῖος ὁ ἀγών· εφ ὃν κληθέντες ὑπὲρ τῆς διαμαρτυρίας τοῦ ἔθνους, ἐναγωνίσασθε προθύμως ὑπὲρ τοῦ πατρίουνόμου. 
Καὶ γὰρ αἰσχρὸν τὸν μὴν γέροντα τοῦτον ὑπομένειν τὰς διὰ τὴν εὐσέβειαν ἀληδόνας, ὑμᾶς δὲ τοὺς νεωτέρους καταπλαγῆναι τὰς βασάνους. 
Ἀναμνήσθητε, ὅτι διὰ τὸν Θεὸν τοῦ κόσμου μετελάβετε, καὶ τοῦ βίου ἀπελαύσατέ· 
καὶ διὰ τιῦτο ὀφείλετε πάντα πόνον ὑπομένειν διὰ τὸν Θεόν. 
Δι' ὃν καὶ ὁ πατὴρ ἡμῶν Ἁβραὰμ ἔσπευδεν τὸν ἐθνοπάτορα υἱὸν σφαγιάσαι Ἰσαὰκ, καὶ τὴν πατρῷαν χεῖρα ξιφηφόρον καταφερομένην ἐπ' αὐτὸν ὁρῶν οὐκ ἔπτηξεν. 
Καὶ Δανιὴλ ὁ δίκαιος εἰς λέοντας ἐβλήθη· καὶ Ἀνανίας, καὶ Ἀζαρίας, καὶ Μισαὴλ εἰς κάμινον πυρὸς ἀπεσφενδονήθησαν, καὶ ὑπέμειναν, διὰ τὸν Θεόν. 
Καὶ ὑμεῖς οὖν τὴν αὐτὴν πίστιν πρὸς τὸν Θεὸν ἔχοντες, μὴ χαλεπαίνητε. 
Ἀλόγιστον γὰρ εἰδότας εὐσέβειαν μὴ ἀντιστασθαι τοῖς πόνοις. 
Διὰ τούτων τῶν λόγων ἡ ἑπταμήτωρ ἕνα ἕκαστον τῶν υἱῶν παρακαλοῦσα, ἔπεισε μᾶλλον, ἢ παραβῆναι τὴν ἐντολὴν τοῦ Θεοῦ. 
Ἔτι δὲ καὶ ταῦτα ἰδόντες, ὅτι διὰ τὸν Θεὸν ἀποθανόντες ζῶσιν τῷ Θεῷ, ὥσπερ Ἁβραὰμ καὶ Ἰσαὰκ καὶ Ἰακὼβ, καὶ πάντες οἱ πατριάρχαι. 
Ἔλεγον δὲ καὶ τῶν δορυφόρων τινὲς, ὡς ὅτε ἔμελλεν καὶ αὐτὴ συλλαμβάνεσθαι πρὸς θάνατον, ἵνα μὴ ψαύσειέν τι τοῦ σώματος ἑαυτῆς, ἑαυτὴν ἔῤῥιψεν κατὰ τῆς πυρᾶς. 
Ὦ μήτηρ σὺν ἑπτὰ παισὶν καταλύσασα τὴν τοῦ τυράννου βίαν, καὶ ἀκυρώσασα τὰς κακὰς ἐπινοίας αὐτοῦ, καὶ ἐπιδείξασα τὴν τῆς πίστεως γενναιότητα. 
Καθάπερ γὰρ σὺ στέγη ἐπὶ τοῦ στύλου τῶν παίδων γενναίως ἱδρυμένη, ἀκλινῶς ὑπήνεγκας τὸν διὰ τῶν βασάνων σεισμόν. 
Θάῤῥει τοιγαροῦν, ὦ μήτηρ ἱερόψυχε, τὴν ἐλπίδα τῆς ὑπομονῆς γενναίως ἔχουσα πρὸς Θεόν. 
Οὐχ οὕτω σελήνη κατ' οὐρανὸν σὺν ἄστροις σεμνὴ καθέστηκεν, ὡς σὺ τοὺς εἰς ἀστέρας ἑπτὰ παῖδας φωταγωγήσασα πρὸς τὴν εὐσέβειαν ἔντιμος καθέστηκας Θεῷ, καὶ ἐστήρισαι ἐν οὐρανῷ σὺν αὐτοῖς. 
Ἦν γὰρ ἡ παιδοποιΐα σου ἀπὸ Ἁβραὰμ τοῦ παιδός. 
Εἰ δὲ ἐξὸν ἡμῖν ἦν, ὥσπερ τινὸς ζωγραφῆσαι τὴν τῆς ἱστορίας σου εὐσέβειαν, οὐκ ἂν ἔφριττον οἱ θεωροῦντες μητέρα ἑπτὰ τέκνων δι' εὐσέβειαν ποικίλας βασάνους μέχρι θανάτου ὑπομείνασαν. 
Καὶ γὰρ ἄξιον ἦν καὶ ἐπὶ αὐτοῦ τοῦ ἐπιταφίου ἀναγράψαι καὶ ταῦτα τοῖς ἀπὸ τοῦ ἔθνους εἰς μνείαν λεγόμενα. 
Ἐνταῦθα γέρων ἱερεὺς, καὶ γυμὴ γεραιὰ καὶ ἑπτὰ παῖδες ἐγκεκήδευνται διὰ τυράννου βίαν, τὴν Ἑβραίων πολιτείαν καταλῦσαι θέλοντος. 
Οἳ καὶ ἐξεδίκησαν τὸ ἔθνος εἰς Θεὸν ἀφορῶντες, καὶ μέχρι θανάτου τὰς βασάνους ὑπομείναντες. 
Ἀληθῶς γὰρ ἦν ἀγῶν θεῖος ὁ δι' αὐτῶν γεγενημένος. 
Ἠθλότει γὰρ τότε ἀρετὴ δι' ὑπομονῆς δοκιμάζουσα τὸ νῖκος ἐν ἀφθαρσίᾳ ἐν ζωῇ πολυχρονίῳ. 
Ἐλεάζαρ δὲ προηγωνίζετο· ἡ δὲ μήτηρ τῶν ἑπτὰ παίδων ἐνήθλει· 
οἱ δὲ ἀδελφοὶ ἠγωνίζοντο· ὁ τύραννος ἀντηγωνίζετο· ὁ δὲ κόσμος καὶ ὁ τῶν ἀνθρώπων βίος ἐθεώρει. 
Θεοσέβεια δὲ ἐνίκα, τοὺς ἑαυτῆς ἀθλητὰς στεφανοῦσα. 
Τίνες οὐκ ἐθαύμασαν τοὺς τῆς ἀληθείας νομοθεσίας ἀθλητὰς; τίνες οὐκ ἐξεπλάγησαν; 
Αὐτός γέ τοι ὁ τύραννος καὶ ὅλον τὸν συνέδριον αὐτῶν ἐξεθαύμασαν αὐτῶν τὴν ὑπομονήν. 
Δι' ἣν καὶ τῷ θείῳ νῦν παρεστήκασιν θρόνῳ, καὶ τὸν μακάριον βιοῦσιν αἰῶνα. 
Καὶ γάρ φησιν ὁ Μωσῆς, καὶ πάντες οἱ ἡγιασμένοι ὑπὸ τὰς χεῖράς σου. 
Καὶ οὗτοι οὖν ἁγιασθέντες διὰ Θεὸν τετίμηνται οὐ μόνον οὖν ταύτῃ τῇ τιμῇ, ἀλλὰ καὶ τῷ δι' αὐτοὺς τὸ ἔθνος ἡμῶν τοὺς πολεμίους μὴ ἐπικρατήσας, 
καὶ τὸν τύραννον τιμωρηθῆναι, καὶ τὴν πατρίδα καθαρισθῆναι, 
ὥσπωρ ἀντίψυχον γεγονότας τῆς τοῦ ἔθνους ἁμαρτίας, καὶ διὰ τοῦ αἵματος τῶν εὐσεβῶν ἐκείνων, καὶ τοῦ ἱλαστηρίου θανάτου αὐτῶν, ἡ θεία πρόνοια τὸν Ἰσραὴλ προκακωθέντα διέσωσεν. 
Πρὸς γὰρ τὴν ἀνδρείαν αὐτῶν τῆς ἀρετῆς, καὶ τὴν ἐπὶ ταῖς βασάνοις αὐτῶν ὑπομονὴν ὁ τύραννος ἀφιδὼν Ἀντίοχος ἀνεκήρυξεν τοῖς στρατιώταις αὐτοῦ εἰς ὑπόδειγμα τὴν ἐκείνων ὑπομονήν. 
Ἔσχεν τε αὐτοὺς γενναίους καὶ ἀνδρείους εἰς πεζομαχίαν καὶ πολιορκίαν· καὶ ἐκπορθήσας ἐνίκησεν πάντας τοὺς πολεμίους. 
Ὦ τῶν Ἁβραμιαίων σπερμάτων ἀπόγονοι παῖδες Ἰσραηλῖται, πείθεσθε τῷ νόμῳ τούτῳ, καὶ πάντα τρόπον εὐσεβεῖτε· 
γινώσκοντες, ὅτι τῶν παθῶν δεσπότης ἐστὶν ὁ εὐσεβὴς λογισμός· καὶ οὐ μόνον τῶν ἔνδοθεν, ἀλλὰ καὶ τῶν ἔξωθεν πόνων· 
Ἀνθ' ὦν διὰ τὴν εὐσέβειαν προϊέμενοι τὰ σώματα τοῖς πόνοις ἐκεῖνοι, οὐ μόνον ὑπὸ τῶν ἀνθρώπων ἐθαυμάσθησαν, ἀλλὰ καὶ θείας μερίδος κατηξιώθησαν. 
Καὶ δι' αὐτοὺς εἰρήνευσεν τὸ ἔθνος, καὶ τὴν εὐνομίαν τὴν ἐπὶ τῆς πατρίδος ἀνανεωσάμενος, ἐκπεπολιόρκηκε τοὺς πολεμίους. 
Καὶ ὁ τύραννος Ἀντίοχος καὶ ἐπὶ γῆς τετιμώρηται, καὶ ἀποθανὼν κολάζεται· ὡς γὰρ οὐδὲν οὐδαμῶς ἴσχυσεν ἀναγκάσαι τοὺς Ἱεροσολυμίτας ἀλλοφυλῆσαι, καὶ τῶν πατριῶν ἐθνῶν ἐκδιαιτηθῆναι· 
τότε δὴ ἀπάρας ἀπὸ τῶν Ἱεροσολύμων ἐστρατοπέδευσεν ἐπὶ Πέρσας. 
Ἔλεγεν δὲ ἡ μήτηρ τῶν ἑπτὰ παίδων καὶ ταῦτα ἡ δικαία τοῖς τέκνοις, ὅτι ἐγὼ ἐγενήθην παρθένος ἁγνὴ, καὶ οὐχ ὑπερέβην πατρικὸν οἶκον· ἐφύλασσον δὲ τὴν ᾠκοδομουμένην πλευράν. 
Οὐ διέφθειρέν με λυμεὼν τῆς ἐρημίας φθορεὺς ἐν πεδίῳ· οὐδὲ ἐλυμῄνατό μου τὰ ἁγνὰ τῆς παρθενίας λυμεὼν ἀπατηλὸς ὄφις· ἔμεινα δὲ χρόνον ἀκμῆς σὺν ἀνδρί. 
Τούτων δὲ ἐνελίκων γενομένων ἐτελεύτησεν ὁ πατήρ· μακάριος μὲν ἐκεῖνος· τὸν γὰρ τῆς εὐτεκνίας βίον ἐπιζητήσας, τὸν τῆς ἀτεκνίας οὐκ ὠδυνήθη καιρόν. 
Ὃς ἐδιδασκεν ὑμᾶς, ἔτι ὢν σὺν ὑμῖν, τὸν νόμον καὶ τοὺς προφήτας. 
Τὸν ἀναιρεθέντα Ἀβὲλ ὑπὸ Κάϊν ἀνεγίνωσκεν δὲ ἡμῖν, καὶ τὸν ὁλοκαπούμενον Ἰσαὰκ, καὶ τὸν ἐν φυλακῇ Ἰωσήφ. 
Ἔλεγεν δὲ ἡμῖν τὸν ζηλωτὴν Φινεές· ἐδίδασκεν δὲ ὑμᾶς τοὺς ἐν πυρὶ Ανανίαν, καὶ Ἀζαρίαν, καὶ Μισαήλ. 
Ἐδόξαζεν δὲ καὶ τὸν ἐν λάκκῳ λεόντων Δανιὴλ, ὃν καὶ ἐμακάριζεν. 
Ὑπεμίμνησκεν δὲ ὑμᾶς τῆν Ἠσαΐου γραφὴν τὴν λέγουσαν, κᾂν διὰ πυρὸς διέλθῃς, φλὸξ οὐ κατακαύσει σε. 
Τὸν ὑμνογράφον ἐμελῴδει ὑμῖν Δαρίδ τὸν λέγοντα, πολλαὶ αἱ θλίψεις τῶν δικαίων. 
Τὸν Σαλομῶντα ἐπαροιμίαζεν ἡμῖν τὸν λέγοντα, ξύλον ζωῆς ἐστιν πᾶσιν τοῖς ποιοῦσιν αὐτοῦ τὸ θέλημα. 
Τὸν Ἰεζεκιὴλ ἐπιστοποιεῖτο τὸν λέγοντα, εἰ ζήσεται τὰ ὀστᾶ τὰ ξηρὰ ταῦτα; 
Ὧδην μὲν γὰρ ἣν ἑδίδαξεν Μωϋσῆς οὐκ ἐπελάθετο τὴν διδάσκουσαν, ἐγὼ ἀποκτενῶ καὶ ζῇν ποιήσω. 
Αὗτη ἡ ζωὴ ἡμῶν καὶ ἡ μακαριότης τῶν ἡμερῶν. 
Ὧ πικρᾶς τῆς τότε ἡμέρας, καὶ οὐ πικρᾶς, ὅτε ὁ πικρὸς Ἑλλήμων τύραννος πῦρ φλέξας λέβησιν ὠμοῖς, καὶ ζεουσι θυμοῖς ἀγαγὼν ἐπὶ τὸν καταπέλτην καὶ πάλιν τὰς βασάνους αὐτοῦ τοὺς ἑπτὰ παῖδας τῆς Ἁβρααμίτιδος. 
Τὰς τῶν ὀμμάτων κόρας ἐπήρωσεν, καὶ γλώσσας ἐξέτεμεν, καὶ βασάνοις ποικίλαις ἀπέκτεινεν. 
Ὑπὲρ ὧν ἡ θεία δίκη μετῆλθεν καὶ μετελεύσεται τὸν ἀλάστορα. 
Οἱ δὲ Ἁβραμιαῖοι παῖδες σὺν τῇ ἀθλοφορῳ μητρὶ, εἰς πατέρων χορὸν συναγελάζονται, φυχὰς καὶ ἀθανάτους ἀπειληφότες παρὰ τοῦ Θεοῦ. 
Ὦ ἡ δόξα εἰς τοὺς αἰῶνας τῶν αἰώνων. Ἀμήν. 
\section{ΚΛΗΜΕΝΤΟΣ ΠΡΟΣ ΚΟΡΙΝΘΙΟΥΣ Α}
Διὰ τὰς αἰφνιδίους καὶ ἐπαλλήλους γενομένας ἡμῖν συμφορὰς καὶ περιπτώσεις, βράδιον νομίζομεν ἐπιστροφὴν πεποιῆσθαι περὶ τῶν ἐπιζητουμένων παρ’ ὑμῖν πραγμάτων, ἀγαπητοί, τῆς τε ἀλλοτρίας καὶ ξένης τοῖς ἐκλεκτοῖς τοῦ θεοῦ, μιαρᾶς καὶ ἀνοσίου στάσεως ἣν ὀλίγα πρόσωπα προπετῆ καὶ αὐθάδη ὑπάρχοντα εἰς τοσοῦτον ἀπονοίας ἐξέκαυσαν, ὥστε τὸ σεμνὸν καὶ περιβόητον καὶ πᾶσιν ἀνθρώποις ἀξιαγάπητον ὄνομα ὑμῶν μεγάλως βλασφημηθῆναι. τίς γὰρ παρεπιδημήσας πρὸς ὑμᾶς τὴν πανάρετον καὶ βεβαίαν ὑμῶν πίστιν οὐκ ἐδοκίμασεν; τήν τε σώφρονα καὶ ἐπιεικῆ ἐν Χριστῷ εὐσέβειαν οὐκ ἐθαύμασεν; καὶ τὸ μεγαλοπρεπὲς τῆς φιλοξενίας ὑμῶν ἦθος οὐκ ἐκήρυξεν; καὶ τὴν τελείαν καὶ ἀσφαλῆ γνῶσιν οὐκ ἐμακάρισεν; ἀπροσωπολήμπτως γὰρ πάντα ἐποιεῖτε καὶ ἐν τοῖς νομίμοις τοῦ θεοῦ ἐπορεύσθε, ὑποτασσόμενοι τοῖς παρ’ ὑμῖν πρεσβυτέροις· νέοις τε μέτρια καὶ σεμνὰ νοεῖν ἐπετρέπετε· γυναιξίν τε ἐν ἀμώμῳ καὶ σεμνῇ καὶ ἁγνῇ συνειδήσει πάντα ἐπιτελεῖν παρηγγέλλετε, στεργούσας καθηκόντως τοὺς ἄνδρας ἑαυτῶν· ἔν τε τῷ κανόνι τῆς ὑποταγῆς τὰ κατὰ τὸν οἶκον σεμνῶς οἰκουργεῖν ἐδιδάσκετε, πάνυ σωφρονούσας.
Πάντες τε ἐταπεινοφρονεῖτε μηδὲν ἀλαζονευόμενοι, ὑποτασσόμενοι μᾶλλον ἢ ὑποτάσσοντες, ἥδιον διδόντες ἢ λαμβάνοντες. τοῖς ἐφοδίοις τοῦ Χριστοῦ ἀρκούμενοι, καὶ προσέχοντες τοὺς λόγους αὐτοῦ ἐπιμελῶν ἐνεστερνισμένοι ἦτε τοῖς σπλάγχνοις, καὶ τὰ παθήματα αὐτοῦ ἧν πρὸ ὀφθαλμῶν ὑμῶν. οὕτως εἰρήνη βαθεῖα καὶ λιπαρὰ ἐδέδοτο πᾶσιν καὶ ἀκόρεστος πόθος εἰς ἀγαθοποιΐαν, καὶ πλήρης πνεύματος ἁγίου ἔκχυσις ἐπὶ πάντας ἐγίνετο· μεστοί τε ὁσίας βουλῆς, ἐν ἀγαθῆ προθυμίᾳ μετ’ εὐσεβεοῦς πεποιθήσεως ἐξετείνετε τὰς χεῖρας ὑμῶν πρὸς τὸν παντοκράτορα θεόν, ἱκετεύοντες αὐτὸν ἱλέως γενέσθαι, εἴ τι ἄκοντες ἡμείρτετε. ἀγὼν ἦν ὑμῖν ἡμέρας τε καὶ νυκτὸς ὑπὲρ πάσης τῆς ἀδελφότητος, εἰς τὸ σώζεσθαι μετ’ ἐλέους καὶ συνειδήσεως τὸν ἀριθμόν τῶν ἐκλεκτῶν αὐτοῦ. εἰλικρινεῖς καὶ ἀκέραιοι ἦτε καὶ ἀμνησίκακοι εἰς ἀλλήλους. πᾶσα στάσις καὶ πᾶν σχίσμα βδελυκτὸν ἦν ὑμῖν. ἐπὶ τοῖς παραπτώματα αὐτῶν ἴδια πλησίον ἐπενθεῖτε· τὰ ὑστερήματα αὐτῶν ἴδια ἐκρίνετε. ἀμεταμέλητοι ἦτε ἐπὶ πάσῃ ἀγαθοποιΐᾳ, ἕτοιμοι εἰς πᾶν ἔργον ἀγαθόν. τῇ παναρέτῳ καὶ σεβασμίῳ πολιτείᾳ κεκοσμημένοι πάντα ἐν τῷ φόβῳ αὐτοῦ ἐπετελεῖτε· τὰ προστάγματα καὶ τὰ δικαιωπματα τοῦ κυρίου ἐπὶ τὰ πλάτη τῆς καρδίας ὑμῶν ἐγέγραπτο.
Πᾶσα δόξα καὶ πλατυσμὸς ἐδόθη ὑμῖν, καὶ ἐπετελέσθη τὸ γεγραμμένον· Ἔφαγεν καὶ ἔπιεν, καὶ ἔπλατύνθη, καὶ ἐπαχύνθη, καὶ ἀπελάκτισεν ὁ ἠγαπημένος. ἐκ τούτου ζῆλος καὶ φθόνος, πόλεμος καὶ αἰχμαλωσία. οὕτως ἐπηγέρθησαν οἱ ἄτιμοι ἐπὶ τοὺς ἐντίμους, οἱ ἄδοξοι ἐπὶ τοὺς ἐνδόξους, οἱ ἄφρονες ἐπὶ τοὺς φρονίμους, οἱ νέοι ἐπὶ τοὺς πρεσβυτέρους. διὰ τοῦτο πόρρω ἄπεστιν ἡ δικαιοσύνη καὶ εἰρήνη, ἐν τῷ ἀπολιπεῖν ἕκαστον τὸν φόβον τοῦ θεοῦ καὶ ἐν τῇ πίστει αὐτοῦ ἀμβλυωπῆσαι, μηδὲ ἐν τοῖς νομίμοις τῶν προσταγμάτων αὐτοῦ προεύεσθαι, μηδὲ πολιτεύεσθαι κατὰ τὸ καθῆκον τῷ Χριστῷ, ἀλλὰ ἕκαστον βαδίζειν κατὰ τὰς ἐπιθυμίας τῆς καρδίας αὐτοῦ τῆς πονηρᾶς, ζῆλον ἄδικον καὶ ἀσεβῆ ἀνειληφότας, δι’ οὗ καὶ θάνατος εἰσῆλθεν εἰς τὸν κόσμον.
Γέτραπται γὰρ οὕτως· Καὶ ἐγένετο μεθ’ ἡμέρας, ἤνεγκεν Κάϊν ἀπὸ τῶν καρπῶν τῆς γῆς θυσίαν τῷ θεῷ, καὶ Ἄβελ ἤνεγκεν καὶ αὐτὸς ἀπὸ τῶν πρωτοτόκων τῶν προβάτων καὶ ἀπὸ τῶν στεάτων αὐτῶν. καὶ ἐπεῖδεν ὁ θεὸς ἐπὶ Ἄβελ καὶ ἐπὶ τοῖς δώροις αὐτοῦ, ἐπὶ δὲ Κάϊν καὶ ἐπὶ ταῖς θυσίαις αὐτοῦ οὐ προσέσχεν. καὶ ἐλυπήθη Κάϊν καὶ συνέπεσεν τῷ προσώπῳ αὐτοῦ. καὶ εἶπεν ὁ θεὸς πρὸς Κάϊν· Ἱνατί περίλυπος ἐγένου, καὶ ἱνατί συνέπεσεν τὸ πρόωπόν σου; οὐκ ἐὰν ὀρθῶς προσενέγκῃς, ὀρθῶς δὲ μὴ διέλῃς, ἥμαρτες; ἡσύχασον· πρὸς σὲ ἡ ἀποσροφὴ αὐτοῦ, καὶ σὺ ἄρξεις αὐτοῦ. καὶ εἶπεν Κάϊν πρὸς Ἄβελ τὸν ἀδελφὸν αὐτοῦ· Διέλθωμεν εἰς τὸ πεδίον. καὶ ἐγένετο ἐν τῷ εἶναι αὐτοὺς ἐν τῷ πεδίῳ, ἀνέστη Κάϊν ἐπὶ Ἄβελ τὸν ἀδελφὸν αὐτοῦ καὶ ἀπέκτεινεν αὐτόν. ὁρᾶτε, ἀδελφοί, ζῆλος καὶ φθόνος ἀδελφοκτονίαν κατειργάσατο. διὰ ζῆλος ὁ πατὴρ ἡμῶν Ἰακὼβ ἀπέδρα ἀπὸ προσώπου Ἠσαῦ τοῦ ἀδελφοῦ αὐτοῦ. ζῆλος ἐποίησεν Ἰωσὴφ μέχρι θανάτου διωχθῆναι καὶ μέχρι δουλείας εἰσελθεῖν. 10 ζῆλος φυγεῖν ἠνάγκασεν Μωϋσῆν ἀπὸ προσώπου Φαραὼ βασιλέως Αἰγύπτου ἐν τῷ ἀκοῦσαι αὐτὸν ἀπὸ τοῦ ὁμοφύλου. Τίς σε κατέστησεν κριτὴν ἢ δικαστὴν ἐφ’ ἡμῶν; μὴ ἀνελεῖν με σὺ θέλεις, ὃν τρόπον ἀνεῖλεσ ἐχθὲς τὸν Αἰγύπτιον; 11. διὰ ζῆλος Ἀαρὼν καὶ Μαριὰμ ἔξω τῆς παρεμβολῆς ηὐλίσθησαν. 12. ζῆλος Δαθὰν καὶ Ἀβειρὼν ζῶντας κατήγαγεν εἰς ᾅδου διὰ τὸ στασιάσαι αὐτοὺς πρὸς τὸν θεράποντα τοῦ θεοῦ Μωϋσῆν. 13. διὰ ζῆλος Δαυεὶδ φθόνον ἔσχεν οὐ μόνον ὑπὸ τῶν ἀλλοφύλων, ἀλλὰ καὶ ὑπὸ Σαοὺλ βασιλέως Ἰσραὴλ ἐδιώχθη.
Ἀλλ’ ἵνα τῶν ἀρχαίων ὑποδειγμάτων παυσώμεθα, ἔλθωμεν ἐπὶ τοὺς ἔγγιστα γενομένους ἀθλητάς· λάβωμεν τῆς γενεᾶς ἡμῶν τὰ γενναῖα ὑποδείγματα. διὰ ζῆλον καὶ φθόνον οἱ μέγιστοι καὶ δικαιότατοι στύλοι ἐδιώχθησαν καὶ ἕως θανάτου ἤθλησαν. λάβωμεν πρὸ ὀφθαλμῶν ἡμῶν τοὺς ἀγαθοὺς ἀποστόλους· Πέτρον, ὃς διὰ ζῆλον ἄδικον οὐχ ἕνα οὐδὲ δύο, ἀλλὰ πλείονας ὑπήνεγκεν πόνους καὶ οὕτω μαρτυρήσας ἐπορεύθη εἰς τὸν ὀφειλόμενον τόπον τῆς δόξης. διὰ ζῆλον καὶ ἔριν Παῦλος ὑπομονῆς βραβεῖον ὑπέδειξεν, ἑπτάκις δεσμὰ φορέσας, φυγαδευθείς, λιθασθείς, κήρυξ γενόμενος ἔν τε τῇ ἀναλῇ καὶ ἐν τῇ δύσει, τὸ γενναῖον τῆς πίτεως αὐτοῦ κλέος ἔλαβεν. δικαιοσύνην διδάξας ὅλον τὸν τὸν κόσμον, καὶ ἐπὶ τὸ τέρμα τῆς δυσεως ἐλθὼν καὶ μαρτυρήσας ἐπὶ τῶν ἡγουμένων, οὕτως ἀπηλλάγη τοῦ κοσμου καὶ εἰς τὸν ἅγιον τόπον ἀνελήμφθη, ὑπομονῆς γενόμενος μέγιστος ὑπογραμμός.
Ταῦτα, ἀγαπητοί, οὐ μόνον ὑμᾶς νουθετοῦντες ἐπιστέλλομεν, ἀλλὰ καὶ ἑαυτοὺς ὑπομιμνήσκοντες· ἐν γὰρ τῷ αὐτῷ ἐσμὲν σκάμματι, καὶ ὁ αὐτὸς ἡμῖν ἀγὼν ἐπίκειται. διὸ ἀπολίπωμεν τὰς κενὰς καὶ ματαίας φροντίδας, καὶ ἔλθωμεν ἐπὶ τὸν εὐκλεῆ καὶ σεμνὸν τῆς παραδόσεως ἡμῶν κανόνα, καὶ ἐνώπιον τοῦ ποιήσαντος ἡμᾶς. ἀτενίσωμεν εἰς τὸ αἷμα τοῦ Χριστοῦ καὶ γνῶμεν, ὡς ἔστιν τίμιον τῷ πατρὶ αὐτοῦ, ὅτι διὰ τὴν ἡμετέραν σωτηρίαν ἐκχυθὲν παντὶ τῷ κόσμῳ μετανοίας χάριν ὑήνεγκεν. διέλθωμεν εἰς τὰς γενεὰς πάσας, καὶ καταμάθωμεν ὅτι ἐν γενεᾷ καὶ γενεᾷ μετανοίας τόπον ἔδωκεν ὁ δεσπότης τοῖς βουλομένοις ἐπιστραφῆναι ἐπ’ αὐτόν. Νῶε ἐκήρυξεν μετάνοιαν, καὶ οἱ ὑπακούσαντες ἐσώθησαν. Ἰωνᾶς Νινευΐταις καταστροφὴν ἐκήρυξεν· οἱ δὲ μετανοήσαντες ἐπὶ τοῖς ἁμαρτήμασιν αὐτῶν ἐξιλάσαντο τὸν θεὸν ἱκετεύσαντες καὶ ἔλαβον σωτηρίαν, καιπερ ἀλλότριοι τοῦ θεοῦ ὄντες.
Ταῦτα, ἀγαπητοί, οὐ μόνον ὑμᾶς νουθετοῦντες ἐπιστέλλομεν, ἀλλὰ καὶ ἑαυτοὺς ὑπομιμνήσκοντες· ἐν γὰρ τῷ αὐτῷ ἐσμὲν σκάμματι, καὶ ὁ αὐτὸς ἡμῖν ἀγὼν ἐπίκειται. διὸ ἀπολίπωμεν τὰς κενὰς καὶ ματαίας φροντίδας, καὶ ἔλθωμεν ἐπὶ τὸν εὐκλεῆ καὶ σεμνὸν τῆς παραδόσεως ἡμῶν κανόνα, καὶ ἐνώπιον τοῦ ποιήσαντος ἡμᾶς. ἀτενίσωμεν εἰς τὸ αἷμα τοῦ Χριστοῦ καὶ γνῶμεν, ὡς ἔστιν τίμιον τῷ πατρὶ αὐτοῦ, ὅτι διὰ τὴν ἡμετέραν σωτηρίαν ἐκχυθὲν παντὶ τῷ κόσμῳ μετανοίας χάριν ὑήνεγκεν. διέλθωμεν εἰς τὰς γενεὰς πάσας, καὶ καταμάθωμεν ὅτι ἐν γενεᾷ καὶ γενεᾷ μετανοίας τόπον ἔδωκεν ὁ δεσπότης τοῖς βουλομένοις ἐπιστραφῆναι ἐπ’ αὐτόν. Νῶε ἐκήρυξεν μετάνοιαν, καὶ οἱ ὑπακούσαντες ἐσώθησαν. Ἰωνᾶς Νινευΐταις καταστροφὴν ἐκήρυξεν· οἱ δὲ μετανοήσαντες ἐπὶ τοῖς ἁμαρτήμασιν αὐτῶν ἐξιλάσαντο τὸν θεὸν ἱκετεύσαντες καὶ ἔλαβον σωτηρίαν, καιπερ ἀλλότριοι τοῦ θεοῦ ὄντες.
Οἱ λειτουργοὶ τῆς χάριτος τοῦ θεοῦ διὰ πνεύματος ἁγίου περὶ μετανοίας ἐλάλησαν. καὶ αὐτὸς δὲ ὁ δεσπότης τῶν ἁπάντων περὶ μετανοίας ἐλάλησεν μετὰ ὅρκου· Ζῶ γὰρ ἐγώ, λέγει κύριος, οὐ βούλομαι τὸν θάνατον τοῦ ἁμαρτωλοῦ ὡς τὴν μετάνοιαν, προστιθεὶς καὶ γνώμην ἀγαθήν· Μετανοήσατε, οἶκος Ἰσραήλ, ἀπό τῆς ἀνομίας ὑμῶν· εἶπον τοῖς υἱοῖς τοῦ λαοῦ μου. Ἐὰν ὦσιν αἱ ἁμαρτίαι ὑμῶν ἀπὸ τῆς γῆς ἕως τοῦ οὐρανοῦ καὶ ἐὰν ὦσιν πυρρότεραι κόκκου καὶ μελανώτεραι σάκκου, και ἐπιστραφῆτε πρός με ἐξ ὅλης τῆς καρδίας καὶ εἴπητε· Πάτερ· ἐπακούσομαι ὑμῶν ὡς λαοῦ ἁγίου. καὶ ἐν ἑτέρῳ τόπῳ λέγει οὕτως· Λούσασθε καὶ καθαροὶ γένεσθε, ἀφέλεσθε τὰς πονηρίας ἀπὸ τῶν ψυχῶν ὑμῶν ἀπέναντι τῶν ὀφθαλμῶν μου· παύσασθε ἀπὸ τῶν πονηριῶν ὑμῶν, μάθετε καλὸν ποιεῖν, ἐκζητήσατε κρίσιν, ῥύσασθε ἀδικούμενον, κρίνατε ὀρφανῷ καὶ δικαιώσατε χήρα· καὶ δεῦτε καὶ διελεγχθῶμεν, λέγει κύριος· καὶ ἐὰν ὦσιν αἱ ἁμαρτίαι ὑμῶν ὡς φοινικοῦν, ὡς χιόνα λευκανῶ· ἐὰν δὲ ὦσιν ὡς κόκκινον, ὡς ἔριον λευκανῶ· καὶ ἐὰν θέλητε καὶ εἰσακούσητέ μου, τὰ ἀγαθὰ τῆς γῆς φάγεσθε· ἐὰν δὲ μὴ θέλητε μηδὲ εἰσακούσητέ μου, μάχαιρα ὑμᾶς κατέδεται· τὸ γὰρ στόμα κυρίου ἐλάλησεν ταῦτα. πάντας οὖν τοὺς ἀγαπητοὺς αὐτοῦ βουλόμενος μετανοίας μετασχεῖν ἐστήριξεν τῷ παντοκρατορικῷ βουλήματι αὐτοῦ.
Διὸ ὐπακούσωμεν τῇ μεγαλοπρεπεῖ καί ἐνδόξῳ βουλήσει αὐτοῦ, καὶ ἱκέται βενόμενοι τοῦ ἐλέους καὶ τῆς χρηστότητος αὐτοῦ προσπέσωμεν καὶ ἐπιστρέψωμεν ἐπὶ τοὺς οἰκτιρμοὺς αὐτοῦ, ἀπολιπόντες τὴν ματαιοπονίαν τήν τε ἔριν καὶ τὸ εἰς θάνατον ἄγον ζῆλος. ἀτενίσωμεν εἰς τοὺς τελείως λειτουργήσαντας τῇ μεγαλοπρεπεῖ δόξῃ αὐτοῦ. λάβωμεν Ἐνώχ, ὃς ἐν ὑπακοῇ δίκαιος εὑρεθεὶς μετετέθη, καὶ οὐχ εὑρέθη αὐτοῦ θάνατος. Νῶε πιστὸς εὑρεθεὶς διὰ τῆς λειτουργίας αὐτοῦ παλιγγενσίαν κόσμῳ ἐκήρυξεν, καὶ διέσωσεν δι’ αὐτοῦ ὁ δεσπότης τὰ εἰσελθόντα ἐν ὁμονοίᾳ ζῶα εἰς τὴν κιβωτόν. 
Ἀβραάμ, ὁ φίλος προσαγορευθείς, πιστὸς εὑρέθη ἐν τῷ αὐτὸν ὑπήκοον γενέσθαι τοῖς ῥήμασιν τοῦ θεοῦ. οὗτος δι’ ὑπακοῆς ἐξῆλθεν ἐκ τῆς γῆς αὐτοῦ καὶ ἐκ τῆς συγγενείας αὐτοῦ καὶ ἐκ τοῦ οἴκου τοῦ πατρὸς αὐτοῦ, ὅπως γῆν ὀλίγην καὶ συγγένειαν ἀσθενῆ καὶ οἶκον μικρὸν καταλιπὼν κληρονομήσῃ τὰς ἐαγγελίας τοῦ θεοῦ. λέγει γὰρ αὐτῷ· Ἄπελθε ἐκ τῆς γῆς σου καὶ ἐκ τῆς συγγενείας σου, καὶ ἔσῃ εὐλογημένος· καὶ εὐλογήσω τοὺς εὐλογοῦντάς σε καὶ καταράσομαι τοὺς καταρωμένουσ σε. καὶ εὐλογηθήσονται ἐν σοὶ πᾶσαι αἱ φυλαὶ τῆς γῆς. καὶ πάλιν ἐν τῷ διαχωρισθῆναι αὐτὸν ἀπὸ Λὼτ εἶπεν αὐτῷ ὁ θεός. Ἀναβλέψας τοῖς ὀφθαλμοῖς σου ἴδε ἀπὸ τοῦ τόπου, οὗ νῦν σὺ εἶ, πρὸς βορρᾶν καὶ λίβα καὶ ἀνατολὰς καὶ θάλασσαν, ὅτι´πᾶσαν τὴν γῆν ἣν σὺ ὁρᾷς, σοὶ δώσω αὐτὴν καὶ τῷ σπέρματί σου ἕως αἰῶνος. καὶ ποιήσω τὸ σπέρμα σου ὡς τὴν ἄμμον τῆς γῆς, εἰ δύναταί τις ἐξαριθμῆσαι τὴν ἄμμον τῆς γῆς, καὶ τὸ σπέρμα σου ἐξαριθμηθήσεται. καὶ πάλιν λέγει· Ἐξήγαγεν ὁ θεὸς τὸν Ἀβραὰμ καὶ ἀρίθμησον τοὺς ἀστέρας, εἰ δυνήσῃ ἐξαριθμῆσαι αὐτούς· οὕτως ἔσται τὸ σπέρμα σου. ἐπίστευσεν δὲ Ἀβραὰμ τῷ θεῷ, καὶ ἐλογίσθη αὐτῷ εἰς δικαιοσύνην. διὰ πίστιν καὶ φιλοξενίαν ἐδόθη αὐτῷ υἱὸς ἐν γήρᾳ, καὶ δι’ ὑπακοῆς προσήνεγκεν αὐτὸν θυσίαν τῷ θεῷ πρὸς τὸ ὄρος ὃ ἔδειξεν αὐτῷ.
Διὰ φιλοξενίαν καὶ εὐσέβειαν Λὼτ ἐσώθη ἐκ Σοδόμων, τῆς περιχώρου πάσης κριθείσης διὰ πυρὸς καὶ θείου, πρόδηλον ποιήσας ὁ δεσπότης, ὅτι τοὺς ἐπίζοντας ἐπ’ αὐτὸν οὐκ ἐγκαταλείπει, τοὺς δὲ ἑτεροκλινεῖς ὑπάρχοντας εἰς κόλασιν καὶ αἰκισμὸν τίθησιν. συνεξελθούσης γὰρ αὐτῷ τῆς γυναικὸς ἑτερογνώμονος ὑπαρχούσης καὶ οὐκ ἐν ὁμονοίᾳ, εἰς τοῦτο σημεῖον ἐτέθη, ὥστε γενέσθαι αὐτὴν στήλην ἁλὸς ἕως τῆς ἡμέρας ταύτης, εἰς τὸ γνωστὸν εἶναι πᾶσιν, ὅτι οἱ δίψυχοι καὶ οἱ διστάζοντες περὶ τῆς τοῦ θεοῦ δυνάμεως εἰς κρίμα καὶ εἰς σημείωσιν πάσαις ταῖς γενεαῖς γίνονται.
Διὰ πίστιν καὶ φιλοξενίαν ἐσώθη ῾Ραὰβ ἡ πόρνη. ἐκπεμφθέντων γὰρ ὑπο Ἰησοῦ τοῦ τοῦ Ναυὴ κατασκόπων εἰς τὴν Ἱεριχώ, ἔγνω ὁ βασιλεὺς τῆς γῆς, ὅτι ἥκασιν κατασκοπεῦσαι τὴν χώραν αὐτῶν, καὶ ἐξέπεμψεν ἄνδρας τοὺς συλλημψομένους αὐτούς, ὅπως συλλημφθέντες θανατωθῶσιν. ἡ οὖν φιλόξενος ῾Ραὰβ εἰσδεξαμένη αὐτοὺς ἔκρυψεν εἰς τὸ ὑπερῷον ὑπὸ τὴν λινοκαλάμην. ἐπισταθέντων δὲ τῶν παρὰ τοῦ βασιλέως καὶ λεγόντων· Πρὸς σὲ εἰσῆλθον οἱ βασιλεὺς οὕτως κελεύει, ἥδε ἀπεκρίθη· Εἰσῆλθον μὲν οἱ ἄνδρες, οὓς ζητεῖτε, πρός με, ἀλλ’ εὐθέως ἀπη̈λθον καὶ πορεύονται τῇ ὁδῷ· ὐοδεικνύουσα αὐτοῖς ἐναλλάξ. καὶ εἶπεν πρὸς τοὺς ἄνδρας· Γινώσκοσα γινώσκω ἐγώ, ὅτι κύριος ὁ θεὸς παραδίδωσιν ὑμῖν τὴν τῆν ταύτην· ὁ γὰρ φόβος αὐτήν. ὡς ἐὰν οὖν γένηται λαβεῖν αὐτὴν ὑμᾶς, διασώσατέ με καὶ τὸν οἶκον τοῦ πατρός μου. καὶ εἶπαν αὐτῇ· Ἔσται οὕτως, ὡς ἐλάλησας ἡμῖν, ὡς´πάντας τοὺς σοὺς ὑπὸ τὸ στέγος σου, καὶ διασωθήσονται· ὅσοι γὰρ ἐὰν εὑρεθῶσιν ἔξω τῆς οἰκίας, ἀπολοῦνται. καὶ προσέθεντο αὐτῇ δοῦναι σημεῖον, ὅπως ἐκκρεμάσῃ ἐκ τοῦ οἴκου αὐτῆς κόκκινον, πρόδηλον ποιοῦντες, ὅτι διὰ τοῦ αἵματος τοῦ κυρίου λύτρωσις ἔσται πᾶσιν τοῖς πιστεύουσιν καὶ ἐλπίζουσιν ἐπὶ τὸν θεόν. ὁρᾶτε, ἀγαπητοί, ὅτι οὐ μόνον πίστις, ἀλλὰ καὶ προφητεία ἐν τῇ γυναικὶ γέγονεν.
Ταπεινοφρονήσωμεν οὖν, ἀδελφοί, ἀποθέμενοι πᾶσαν ἀλαζονείαν καὶ τῦφος καί ἀφροσύνην καὶ ὀργάς, καὶ ποιήσωμεν τὸ γεγραμμένον, λέγει γὰρ τὸ πνεῦμα τὸ ἅγιον· Μὴ καυχάσθω ὁ σοφὸς ἐν τῇ σοφίᾳ αὐτοῦ μηδὲ ὁ ἰσχυρὸς ἐν τῷ πλούτῳ αὐτοῦ, ἀλλ’ ἡ ὁ καυχώμενος ἐν κυρίῳ δαυχάσθω, τοῦ ἐκζητεῖν αὐτὸν καὶ ποιεῖν κρίμα καὶ δικαιοσύνην· μάλιστα μεμνημένοι τῶν λόγων τοῦ κυρίου Ἰησοῦ, οὓς ἐλάλησεν δεδάσκων ἐπιείκειαν καὶ μακροθυμίαν. οὕτως γὰρ εἶπεν· Ἐλεᾶτε, ἵνα ἐλεηθῆτε· ἀφίετε, ἵνα ἀφεθῇ ὑμῖν· ὡς χρηστεύεσθε, οὕτως ψρηστευθήσεται ὑμῖν· ᾧ μέτρῳ μετρεῖτε, ἐν αὐτῷ μετρηθήσεται ὑμῖν. ταύτῃ τῇ ἐντολῇ καὶ τοῖς παραγγέλμασιν τούτοις στηρίξωμεν ἑαυτοὺς εἰς λόγοις αὐτοῦ, ταπεινοφρονοῦντες· φησὶν γὰρ ὁ ἅγιος λόγος· Ἐπὶ τίνα ἐπιβλέψω, ἀλλ’ ἢ ἐπὶ τὸν πραῢν καὶ ἡσύχιον καὶ τρέμοντά μου τὰ λόγια.
Δίκαιον οὖν καὶ ὅσιον, ἄνδρες ἀδελφοί, ὑπηκόους ἡμᾶς μᾶλλον γενέσθαι τῷ θεῷ ἢ τοῖς ἐν ἀλαζονείᾳ καὶ ἀκαταστασίᾳ μυσεροῦ ζήλους ἀρχηγοῖς ἐξακολουθεῖν. βλάβην γὰρ οὐ τὴντυχοῦσαν, μᾶλλον δὲ κίνδυνον ὑποίσομεν μέγαν, ἐὰν ῥιψοκινδύνως ἐπιδῶμεν ἑαυτοὺς τοῖς θελήμασιν τῶν ἀνθρώπων, οἵτινες ἐξακοντίζουσιν εἰς ἔριν καὶ στάσεις, εἰς τὸ ἀπαλλοτριῶσαι ἡμᾶς τοῦ καλῶς ἔχοντος. χρηστευσώμεθα ἑαυτοῖς κατὰ τὴν εὐσπλαγχνίαν καὶ γλυκύτητα τοῦ ποιήσαντος ἡμᾶς. γέγραπται γάρ· Χρηστοὶ ἔσονται οἰκήτροες γῆς, ἄκακαοι δὲ ὑπολειφθήσονται ἐπ’ αὐτῆς· οἱ δὲ παρανομοῦντες ἐξολεθρευθήσονται απ’ αὐτῆς. καὶ πάλιν λέγει· Εἶδον ἀσεβῆ ὑπερυψούμενον καὶ ἐπαιρόμενον ὡς τὰς κέδρους τοῦ Λιβάνου· καὶ παρῆλθον, καὶ ἰδοὺ οὐκ ἦν, καὶ ἐξεζήτησα τὸν τόπον αὐτοῦ, καὶ οὐχ εὗρον. φύλασσε ἀκακίαν καὶ ἴδε εὐθύτητα, ὅτι ἐστὶν ἐγκατάλειμμα ἀνθρώπῳ εἰρηνικῷ.
Τοίνυν κολληθῶμεν τοῖς μετ’ εὐσεβείας ἐρηνεύουσιν, καὶ μὴ τοῖς μεθ’ ὑποκρίσεως βουλομένοις ἐρήνην. λέγει γάρ που· Οὗτος ὁ λαὸς τοῖς χείλεσίν με τιμᾷ, ἡ δὲ καρδία αὐτῶν πόρρω ἄπεστιν ἀπ’ ἐμοῦ. καὶ πάλιν· Τῷ στόματι αὐτῶν κατηρῶντο. καὶ πάλιν λέγει· Ἠγάπησαν αὐτὸν τῷ στόματι αὐτῶν καὶ τῇ γλώσσῃ αὐτῶν ἐψεύσαντο αὐτόν, ἡ δὲ καρδία αὐτῶν οὐκ εὐθεῖα μετ’ αὐτοῦ, οὐδὲ ἐπιστώθησαν ἐν τῇ διαθήκῃ αὐτοῦ. διὰ τοῦτο ἄλαλα γενηθήτω τὰ χείλη τὰ δόλια τὰ λαλοῦντα κατὰ τοῦ δικαίου ἀνομίαν. καὶ πάλιν· Ἐξολεθρεύσαι κύριος πάντα τὰ χείλη τὰ δόλια, γλῶσσαν μεγαλορήμονα, τοὺς εἰπόντας· Τὴν γλῶσσαν ἡμῶν μεγαλυνοῦμεν, τὰ χείλη ἡμῶν παρ’ ἡμῖν ἐστιν· τίς ἡμῶν κύριος ἐστιν; ἀπὸ τῆς ταλαιπωρίας τῶν πτωχῶν καὶ τοῦ στεναγμοῦ τῶν πενήτων νῦν ἀναστήσομαι, λέγει κύριος· θήσομαι ἐν σωτηρίῳ, παρρησιάσομαι ἐν αὐτῷ. 
Ταπεινοφρονούντων γάρ ἐστιν ὁ Χριστός οὐκ ἐπαιρομένων ἐπὶ τὸ ποίμνιον αὐτοῦ. τὸ σκῆπτρον τῆς μεγαλωσύνης τοῦ θεοῦ, ὁ κύριος Ἰησοῦς Χριστός, καίπερ δυνάμενος, ἀλλὰ ταπεινοφρονῶν, καθὼς τὸ πνεῦμα τὸ ἅγιον περὶ αὐτοῦ ἐλάλησεν· φησὶν γάρ· Κύριε, τίς ἐπίστευσεν τῇ ἀκοῇ ἡμῶν; καὶ ὁ βραχίων κυρίου τίνι ἀπεκαλύφθη; ἀνηγγείλαμεν ἐναντίον αὐτοῦ, ὡς παιδίον, ὡς ῥίζα ἐν γῇ διψώσῃ· οὐκ ἔστιν αὐτῷ εἶδος οὐδὲ δόξα, καὶ εἴδομεν αὐτόν, καὶ οὐκ εἶχεν εἶδος οὐδὲ κάλλος, ἀλλὰ τὸ εἶδος αὐτοῦ ἄτιμον, ἐκλεῖπον παρὰ τὸ εἰδος τῶν ἀνθρώπων· ἄνθρωπος ἐν πληγῇ ὢν καὶ πόνῳ καὶ εἰδὼς φέρειν μαλακίαν, ὅτι ἀπέστραπται τὸ πρόσωπον αὐτοῦ, ἠτιμάσθη καὶ οὐκ ἐλογίσθη· οὗτος τὰς ἁμαρτίας ἡμῶν φέρει καὶ περὶ ἡμῶν ἀδυνᾶται, καὶ ἡμεῖς ἐλογισάμεθα αὐτὸν εἶναι ἐν πόνῳ καὶ ἐν πληγῇ καὶ ἐν κακώσει· αὐτὸς δὲ ἐτραυματίσθη διὰ τὰς ἁμαρτίας ἡμῶν καὶ μεμαλάκισται διὰ τὰς ἀνομίας ἡμῶν. παιδεία εἰρήνης ἡμῶν ἐπ’ αὐτόν· τῷ μώλωπι αὐτοῦ ἡμεῖς ἰάθημεν. πάντες ὡς πρόβατα ἐπλανήθημεν, ἄνθρωπος τῇ ὁδῳ αὐτοῦ ἐπλανήθη· καὶ κύριος παρέδωκεν αὐτὸν ὑπὲρ τῶν ἁμαρτιῶν ἡμῶν, καὶ αὐτὸς διὰ τὸ κεκακῶσθαι οὐκ ἀνοίγει τὸ στόμα. ὡς πρόβατον ἐπὶ σφαγὴν ἤχθη, καὶ ὡς ἀμνὸς ἐναντίον τοῦ κείραντος ἄφωνος, οὕτως οὐκ ἀνοίγει τὸ στόμα αὐτοῦ. ἐν τῇ ταπεινώσει ἡ κρίσις αὐτοῦ ἤρθη. τὴν γενεὰν αὐτοῦ τίς διηγήσεται; ὅτι αἴρεται ἀπὸ τῆς γῆς ἡ ζωὴ αὐτοῦ. ἀπὸ τῶν ἀνομιῶν τοῦ λαοῦ μου ἥκει εἰς θάνατον. 10. δώσω τοὺς πονηροὺς ἀντὶ τῆς ταφῆς αὐτοῦ καὶ τοὺς πλουσίους ἀντὶ τοῦ θανάτου αὐτοῦ· ὅτι ἀνομίαν οὐκ ἐποίησεν, οὐδὲ εὑρέθη δόλος ἐν τῷ στόματι αὐτοῦ. καὶ κύριος βούλεται καθαρίσαι αὐτὸν τῆς πληγῆς. 11. ἐὰν δῶτε περὶ ἁμαρτίας, ἡ ψυχὴ ὑμῶν ὄψεται σπέρμα μακρόβιον. 12. καὶ κύριος βούλεται ἀφελεῖν ἀπὸ του πόνου τῆς ψυχῆς αὐτοῦ, δεῖξαι αὐτῷ φῶς καὶ πλάσαι τῇ συνέσει, δίκαιον εὖ δουλεύοντα πολλοῖς. καὶ τὰς ἁμαρτίας αὐτῶν αὐτὸς ἀνοίσει. 13. διὰ τοῦτο αὐτὸς κληρονομήσει πολλλοὺς καὶ τῶν ἰσχυρῶν μεριεῖ σκῦλα· ἀνθ’ ὧν παρεδόθη εἰς θάνατον ἡ ψυχὴ αὐτοῦ, καὶ ἐν τοῖς ἀνόμοις ἐλογίσθη. 14. καὶ αὐτὸς ἁμαρτίας πολλῶν ἀνήνεγκεν καὶ διὰ τὰς ἁμαρτίας αὐτῶν παρεδόθη. 15. καὶ´πάλιν αὐτός φησιν· Ἐγὼ δέ εἰμι σκώληξ καὶ οὐκ ἄνθρωπος, ὄνειδος ἀνθρώπων καὶ ἐξουθένημα λαοῦ. 16. πάντες οἱ θεωροῦντές με ἐξεμυκτήρισάν με. ἐλάλησαν ἐν χείλεσιν, ἐκίνησαν κεφαλήν· Ἤλπισεν ἐπὶ κύριον, ῥυσάσθω αὐτόν, σωσάτω αὐτόν, ὅτι θέλει αὐτόν. 17. ὁρᾶτε, ἄνδρες ἀγαπητοί, τίς ὁ ὑπογραμμὸς ὁ δεδομένος ἡμῖν· εἰ γὰρ ὁ κύριος οὕτως ἐταπεινοφρόνησεν, τί ποιήσωμεν ἡμεῖς οἱ ὑπό τὸν ζυγὸν τῆς χάριτος αὐτοῦ δι’ αὐτοῦ ελθόντες;
Μιμηταὶ γενώμεθα κἀκείνων, οἵτινες ἐν δέρμασιν αἰγείοις καὶ μηλωταῖς περιεπάτησαν κηρύσσοντες τὴν ἔλευσιν τοῦ Χριστοῦ· λέγομεν δὲ Ἠλίαν καὶ Ἑλισαιέ, ἔτι δὲ καὶ Ἰεζεκιήλ, τοὺς προφήτας· πρὸς τούτοις καὶ τοὺς μεμαρτυρημένους. ἐμαρτυρήθη μεγάλως Ἀβραὰμ καί φίλος προσηγορεύθη τοῦ θεοῦ, καὶ λέγει ἀτενίζων εἰς τὴν δόξαν τοῦ θεοῦ ταπεινοφρονῶν· Ἐγὼ δέ εἰμι γῆ καὶ σποδός. ἔτι δὲ καὶ περὶ Ἰὼβ οὕτως γέγραπται· Ἰὼβ δὲ ἦν δίκαιος καὶ ἄμεμπτος, ἀληθινός, θεοσεβής, ἀπεχόμενος ἀπὸ παντὸς κακοῦ. ἀλλ’ αὐτὸς ἑαυτοῦ κατηγορεῖ λέγων· Οὐδεὶς καθαρὸς ἀπὸ ῥύπου, οὐδ’ ἂν μιᾶς ἡμέρας ἡ ζωὴ αὐτοῦ. Μωϋσῆς πιστὰς ἐν ὅλῳ τῷ οἰκῳ αὐτοῦ ἐκλήθη, καὶ διὰ τῆς ὑπηρεσίας αὑτοῦ ἔκρινεν ὁ θεὸς Αἴγυπτον διὰ τῶν μαστίγων καὶ τῶν αἰκισμάτων αὐτῶν· ἀλλὰ κἀκεῖνος δοξασθεὶς μεγάλως οὐκ ἐμεγαλορημόνησεν, ἀλλ’ εἶπεν ἐκ τῆς βάτου χρηματισμοῦ αὐτῷ διδομένου· Τίς εἰμι ἐγώ, ὅτι με πέμπεις; Ἐγὼ δέ εἰμι ἰσχνὀφωνος καὶ βραδύγλωσσος. καὶ πάλιν λέγει· Ἐγὼ δέ εἰμι ἀτμὶς ἀπὸ κύθρας. 
Τί δὲ εἴπωμεν ἐπὶ τῷ μεμαρτυρημένῳ Δαυείδ ἐφ’ οὗ εἶπεν ὁ θεός· Εὗρον ἄνδρα κατὰ τὴν καρδίαν μου, Δαυεὶδ τὸν τοῦ Ἰεσσαί, ἐν ἐλέει αἰωνίῳ ἔχρισα αὐτόν. ἀλλὰ καὶ αὐτὸς λέγει πρὸς τὸν θεὸν· Ἐλέησόν με, ὁ θεός, κατὰ το μέγα ἔλεός σου, καὶ κατὰ τὸ πλῆθος τῶν οἰκτιρμῶν σου ἐξάλειψον τὸ ἀνόμημά μου. ἐπὶ πλεῖον πλῦνόν με ἀπὸ τῆς ἀνομίας μου ἐμώπιόν μου ἐστὶν διαπαντός. σοὶ μόνῳ ἥμαρτον, καὶ τὸ πονηρὸν ἐνώπιόν σου ἐποίησα, ὅπως ἂν δικαιωθῇς ἐν τοῖς λόγοις σου. καὶ νικήσῃς ἐν τῷ κρίνεσθαί σε. ἰδοὺ γὰρ ἐν ἀνομίαις συνελήμφθην, καὶ ἐν ἁμαρτίαις ἐκίσσησέν με ἡ μήτηρ μου. ἰδοὺ γὰρ ἀλήθειαν ἠγαπησας· τὰ ἄδηλα καὶ τὰ κρύφια τῆς σοφίας σου ἐδήλωσάς μοι. ῥαντιεῖς με ὑσσώπῳ, καὶ καθαρισθήσομαι· πλυνεῖς με καὶ ὑπὲρ χιόνα λευκανθήσομαι. ἀκουτιεῖς με ἀγαλλίασιν καὶ εὐφροσύνην. ἀγαλλιάσοντα ὀστᾶ τεταπεινωμένα. ἀπόστρεψον τὸ πρόσωπόν σου ἀπὸ τῶν ἁμαρτιῶν μου, καὶ πάσας τὰς ἀνομίας μου ἐξάλειψον. 10. καρδίαν καθαρὰν κτίσον ἐν ἐμοί, ὁ θεός, καὶ πνεῦμα εὐθὲς ἐγκαίνισον ἐν τοῖς ἐγκάτοις μου. 11. μὴ ἀπορίψῃς με ἀπὸ τοῦ προσώπου σου, καὶ τὸ πνεῦμα τὸ ἅγιόν σου μὴ ἀντανέλῃς ἀπ’ ἐμοῦ. 12. ἀπόδος μοι τὴν ἀγαλλίασιν τοῦ σωτηρίου σου, καὶ πνεύματι ἡγεμονικῷ στήρισόν με. 13. διδάξω ἀνόμους τὰς ὁδούς σου, καὶ ἀσεβεῖς ἐπιστρέψουσιν ἐπὶ σέ. 14. ῥῦσαί με ἐξ αἱμάτων, ὁ θεός, ὁ θεὸς τῆς σωτηρίας μου. 15. ἀγαλλιάσεται ἡ γλῶσασά μου τὴν δικαιοσύνην σου. κύριε, τὸ στόμα μου ἀνοίξεις, καὶ τὰ χείλη μου ἀναγγελεῖ τὴν αἴνεσίν σου. 16. ὅτι εἰ ἠθέλησας θυσίαν, ἔδωκα ἂν ὁλοκαυτώματα οὐκ εὐδοκήσεις. 17. θυσία τῷ θεῷ πνεῦμα συντετριμμένον· καρδίαν συντετριμμένην καὶ τεταπεινωμένην ὁ θεὸς οὐκ ἐξουθενώσει.
Τῶν τοσούτων οὖν καὶ τοιούτων οὕτως μεμαρτυρημένων τὸ ταπεινόφρον καὶ τὸ ὑποδεὲς διὰ τῆς ὑπακοῆς οὐ μόνον ἡμᾶς, ἀλλὰ καὶ τὰς πρὸ ἡμῶν γενεὰς βελτίους ἐποίησεν, τούς τε καταδεξαμένους τὰ λόγια αὐτοῦ ἐν φόβῳ καὶ ἀληθείᾳ. πολλῶν οὖν καὶ μεγάλων καὶ ἐνδόξων μετειληφότες πράξεων ἐπαναδράμωμεν ἐπὶ τὸν ἐξ ἀρχῆς παραδεδομένον ἡμῖν τῆς εἰρήνης σκοπόν, καὶ ἀτενίσωμεν εἰς τὸν πατέρα καὶ κτίστην τοῦ σύμπαντος κόσμου καὶ ταῖς μεγαλοπρεπέσι καὶ ὑπερβαλλούσαις αὐτοῦ δωρεαῖς τῆς εἰρήνης εὐεργεσίαις τε κολληθῶμεν. ἴδωμεν αὐτὸν κατὰ διάνοιαν καὶ ἐμβλέψωμεν τοῖς ὄμμασιν τῆς ψυχῆς εἰς τὸ μακρόθυμον αὐτοῦ βούλημα· νοήσωμεν, πῶς ἀόργητος ὑπάρχει πρὸς πᾶσαν τὴν κτίσιν αὐτοῦ.
Οἱ οὐρανοὶ τῇ διοικήσει αὐτοῦ σαλευόμενοι ἐν εἰρήνῃ ὑποτάσσονται αὐτῷ. ἡμέρα τε καὶ νὺξ τὸν τεταγμένον ὑπ’ αὐτοῦ δρόμον διανυουσιν, μηδὲν ἀλλήλοις ἐμποδίζοντα. ἥλιός τε καὶ σελήνη, ἀστέρων τε χοροὶ κατὰ τὴν διαταγὴν αὐτοῦ ἐν ὁμονοίᾳ δίχα πάσης παρεκβάσεως ἐξελίσσουσιν τοὺς ἐπιτεταγμέπνους αὐτοῖς ὁρισμούς. γῆ κυοφοροῦσα κατὰ τὸ θέλημα αὐτοῦ τοῖς ἰδίοις καιροῖς τὴν πανπληθῆ ἀνθρώποις τε καὶ θηρσὶν καὶ πᾶσιν τοῖς οὖσιν ἐπ’ αὐτῆς ζώοις ἀνατέλλει τροφήν, μὴ διχοστατοῦσα μηδὲ ἀλλοιοῦσά τι τῶν δεδογματισμένων ὑπ’ αὐτοῦ. ἀβύσσων τε ἀνεξιχνίαστα καὶ νερτέων ἀνεκδιήγητα κλίματα τοῖς αὐτοῖς συνέχεται προστάγμασιν. τὸ κύτος τῆς ἀπείρου θαλάσσης κατὰ τὴν δημιουργίαν αὐτοῦ συσταθὲν εἰς τὰς συναγωγὰς οὐ παρεκβαίνει τὰ περιτεθειμένα αὐτῇ κλεῖθρα, ἀλλὰ καθὼς δέταξεν αὐτῇ, οὕτως ποιεῖ. εἶπεν γάρ· Ἕως ὧδε ἥξεις, καὶ τὰ κύματά σου ἐν σοὶ συντριβήσεται. ὠκεανὸς ἀπέραντος ἀνθρώποις καὶ οἱ μετ’ αὐτὸν κόσμοι ταῖς αὐταῖς ταγαῖς του δεσπότου διευθύνονται. καιροὶ ἐαρινοὶ καὶ θερινοὶ καὶ μετοπωρινοὶ καὶ ψειμερινοὶ ἐν εἰρήνῃ μεταπαραδιδοασιν ἀλλήλοις. 10. ἀνέμων σταθμοὶ κατὰ τὸν ἴδιον καιρὸν τὴν λειτουργίαν αὐτῶν ἀπροσκόπως ἐπιτελοῦσιν· ἀέναοί τε πηγαί, πρὸς απόλαυσιν καὶ ὐγείαν δημιουργηθεῖσαι, δίχα ἐλλείψεως παρέχονται τοὺς πρὸς ζωῆς ἀνθρωποις μαζούς· τά τε ἐλάχιστα τῶν ζώων τὰς συνελεύσεις αὐτῶν ἐν ὁμονοίᾳ καὶ εἰρήνῃ ποιοῦνται. 11. ταῦτα πάντα ὁ μέγας δημιουργὸς καὶ δεσπότης τῶν ἁπάντων ἐν εἰρήνῃ καὶ ὁμονοίᾳ προσέταξεν εἶναι, εὐεργετῶν τὰ πάντα, ὑπερεκπερισσῶς δὲ ἡμᾶς τοὺς προσπεφευγόντας τοῖς οικτιρμοῖς αὐτοῦ διὰ τοῦ κυρίου ἡμῶν Ἰησοῦ Χριστοῦ, 12. ᾦ ἡ δόξα καὶ ἡ μεγαλωσύνη εἰς τοὺς αἰῶνας τῶν αἰώνων. ἀμήν.
Ὁρᾶτε, ἀγαπητοί, μὴ αἱ εὐεργεσίαι αὐτοῦ αἱ πολλαὶ γένωνται εἰς κρίμα ἡμῖν, ἐὰν μὴ ἀξιως αὐτοῦ πολιτευόμενοι τὰ καλὰ καὶ εὐάρεστα ἐνώπιον αὐτοῦ ποιῶμεν μεθ’ ὁμονοίας. λέγει γάρ που· Πνεῦμα κυρίου λύχνος ἐρευνῶν τὰ ταμιεῖα τῆς γαστρός· ἴδωμεν, πῶς ἐγγύς ἐστιν, καὶ ὅτι οὐδὲν λέληθεν αὐτὸν τῶν ἐννοιῶν ἡμῶν οὐδὲ τῶν διαλογισμῶν ὧν ποιούμεθα· δίκαιον οὖν ἐστὶν μὴ λειποτακτεῖν ἡμᾶς ἀπό τοῦ θελήματος αὐτοῦ. μᾶλλον ἀνθρώποις ἄφροσι καὶ ἀνοήτοις καὶ ἐπαιρομένοις καὶ ἐγκαυχωμένοις ἐν ἀλαζονείᾳ τοῦ λόγου αὐτῶν προσκόψωμεν ἢ τῷ θεῷ. τὸν κύριον Ἰησοῦν Χριστόν, οὗ τὸ αἷμα ὑπὲρ ἡμῶν ἐδόθη, ἐντραπῶμεν τὴν παιδείαν τοῦ φόβου τοῦ θεοῦ, τὰς γυναῖκας ἡμῶν ἐπὶ τὸ ἀγαθὸν διορθωσώμεθα. τὸ ἀκιαγάπητον τῆς ἐπιεικὲς τῆς γλώσσης αὐτῶν διὰ τῆς σιγῆς φανερὸν ποιησάτωσαν, τὴν ἀγάπην αὐτῶν μὴ κατὰ προσκλίσεις, ἀλλὰ πᾶσιν τοῖς φοβουμένοις τὸν θεὸν ὁσίως ἴσην παρεχέτωσαν. τὰ τέκνα ἡμῶν τῆς ἐν Χριστῷ παιδείας μεταλαμβανέτωσαν· μαθέτωσαν, τί ταπεινοφροσύνη παρὰ θεῷ ισχύει, τί ἀγάπη ἁγνὴ παρὰ θεῷ δύνατι, πῶς ὁ φόβος αὐτοῦ καλὸς καὶ μέγας καὶ σώζων πάντας τοὺς ἐν αὐτῷ ὁσίως ἀναστρεφομένους ἐν καθαρᾷ. ἐρευνητὴς γάρ ἐστιν ἐννοιῶν καὶ ἐνθυμήσεων· οὗ ἡ πνοὴ αὐτοῦ ἐν ἡμῖν ἐστίν, καὶ ὅταν θέλῃ, ἀνελεῖ αὐτήν.
Ταῦτα δὲ πάντα βεβαιοῖ ἡ ἐν Χριστῷ πίστις· καὶ γὰρ αὐτὸς διὰ τοῦ πνεύματος τοῦ ἁγίου οὕτως προσκαλεῖται ἡμᾶς· Δεῦτε, τέκνα, ἀκούσατέ μου, φόβον κυρίου διδάξω ὑμᾶς. τίς ἐστιν ἄνθρωπος ὁ θέλων ζωήν, ἀγαπῶν ἡμέρας ἰδεῖν ἀγαθός; παῦσον τὴν γλῶσσάν σου ἀπὸ κακοῦ, καὶ χείλη σου τοῦ μὴ λαλῆσαι δόλον. ἔκκλινον ἀπὸ κακοῦ, καὶ ποίησον ἀγαθόν. ζήτησον εἰρήνην, καὶ δίωξον αὐτήν· ὀφθαλμοὶ κυρίου ἐπὶ δικαίους, καὶ ὦτα αὐτοῦ πρὸς δέησιν αὐτῶν· πρόσωπον δὲ κυρίου ἐπὶ ποιοῦντας κακά, τοῦ ἐξολεθρεῦσαι ἐκ γῆς τὸ μνημόσυνον αὐτῶν. ἐκέκραξεν ὁ δίκαιος, καὶ ὁ κύριος εἰσήκουσεν αὐτοῦ, καὶ ἐκ πασῶν τῶν θλίψεων αὐτοῦ ἐρύσατο αὐτόν. Πολλαὶ αἱ μάστιγες τοῦ ἁμαρτωλοῦ, τοὺς δὲ ἔπίζοντας ἐπὶ κύριον ἔλεος κυκλώσει.Ὁ οἰκτίρμων κατὰ πάντα καὶ εὐεργετικὸς πατὴρ ἔχει σπλάγχνα ἐπὶ τοὺς φοβουμένους αὐτόν, ἠπίως τε καὶ προσηνῶς τὰς χάριτας αὐτοῦ ἀποδιδοῖ τοῖς προσερχομένοις αὐτῷ ἁπλῇ διανοίᾳ. διὸ μὴ διψυχῶμεν, μηδὲ ἰνδαλλέσθω ἡ ψυχὴ ἡμῶν ἐπί ταῖς ὑπερβαλλούσαις καὶ ἐνδόξοις δωρεαῖς αὐτοῦ. πόρρω γενέσθω ἀφ’ ἡμῶν ἡ γραφὴ αὕτη, ὅπου λέγει· Ταλαίπωροί εἰσιν οἱ δίψυχοι, οἱ διστάζοντες τῇ ψυχῇ, οἱ λέγοντες· Ταῦτα ἠκούσαμεν καὶ ἐπὶ τῶν πατέρων ἡμῶν, καὶ ἰδού, γεγηράκαμεν, καὶ οὐδὲν ἡμῖν τούτων συβέβηκεν. ὦ ἀνόητοι, συμβάλετε ἑαυτοὺς ξύλῳ· λάβετε ἄμπελον· πρῶτον μὲν φυλλοροεῖ, εἶτα βλαστὸς γίνεται, εἶτα φύλλον, εἶτα ἄνθος, καὶ μετὰ ταῦτα ὄμφαξ, εἶτα σταφυλὴ παρεστηκυῖα. ὁρᾶτε, ὅτι ἐν καιρῷ ὀλίγῳ εἰς πέπειρον καταντᾷ ὁ καρπὸς τοῦ ξύλου. ἐπ’ ἀληθείας ταχὺ καὶ ἐξαίφνης τελειωθήσεται τὸ βούλημα αὐτοῦ, συνεπιμαρτυρούσης καὶ τῆς γραφῆς, ὅτι ταχὺ ἥξει καὶ οὐ χρονιεῖ, καὶ ἐαίφνης ἥξει ὁ κύριος εἰς τὸν ναὸν αὐτοῦ, καὶ ὁ ἅγιος, ὃν ὑμεῖς προσδοκᾶτε. 
Κατανοήσωμεν, ἀγαπητοί, πῶς ὁ δεσπότης ἐπιδείκνυται διηνεκῶς ἡμῖν τὴν μέλλουσαν ἀνάστασιν ἔσεσθαι, ἧς τὴν ἀπαρχὴν ἐποιήσατο τὸν κύριον Ἰησοῦν Χριστὸν ἐκ νεκρῶν ἀναστήσας. ἴδωμεν, ἀγαπητοί, τὴν κατὰ καιρὸν γινομένην ἀνάστασιν. ἡμέρα καὶ νὺξ ἀνάστασιν ἡμῖν δηλοῦσιν· κοιμᾶται ἡ νὺξ, ἀνίσταται ἡ ἡμέρα· ἡ ἡμέρα ἄπεισιν, νὺξ ἐπέρχεται. λάβωμεν τοὺς καρπούς· ὁ σπόρος πῶς καὶ τίνα τρόπον γίνεται; ἐξῆλθεν ὁ σπείρων καὶ ἔβαλεν εἰς τὴν γῆν ἕκαστον τῶν σπερμάτων, ἅτινα πεσόντα εἰς τὴν γῆν ξηρὰ καὶ γυμνὰ διαλύεται· εἶτ’ ἐκ τῆς διαλύσεως ἡ μεγαλειότης τῆς προνοίας τοῦ δεσπότου ἀνίστησιν αὐτά, καὶ ἐκ τοῦ ἑνὸς πλείονα αὔξει καὶ ἐκφέρει καρπόν. 
Ἴδωμεν τὸ παράδοξον σημεῖον τὸ γινόμενον ἐν τοῖς ἀνατολικοῖς τόποις, τουτέστιν τοῖς περὶ τὴν Ἀραβίαν. ὄρνεον γάρ ἐστιν, ὃ προσονομάζεται φοῖνιξ· τοῦτο μονογενὲς ὑπάρχον ζῇ ἔτη πεντακόσια, γενόμενόν τε ἤδη πρὸς ἀπόλυσιν τοῦ ἀποθανεῖν αὐτό, σηκὸν ἑαυτῷ ποιεῖ ἐκ λιβάνου καὶ σμύρνης καὶ τῶν λοιπῶν ἀρωμάτων, εἰς ὃν πληρωθέντος τοῦ χρόνου εἰσέρχεται καὶ τελευτᾷ. σηπομένης δὲ τῆς σαρκὸς σκώληξ τις γεννᾶται, ὃς ἐκ τῆς ἰκμάδος τοῦ τετελευτηκότος ζώου ἀνατρεφόμενος πτεροφυεῖ· εἶτα γενναῖος γενόμενος αἴρει τὸν σηκὸν ἐκεῖνον, ὅπου τὰ ὀστᾶ τοῦ προγεγονότος ἐστίν, καὶ ταῦτα βαστάζων διανύει ἀπὸ τῆς Ἀραβικῆς χώρας ἕως τῆς Αἰγύπτου εἰς τὴν λεγομένην Ἡλιούπολιν, καὶ ἡμέρας, βλεπόντων πάντων, ἐπιπτὰς ἐπί τὸν τοῦ ἡλίου βωμὸν τίθησιν αὐτὰ καὶ οὕτως εἰς τοὐπίσω ἀφορμᾷ. οἱ οὖν ἱερεῖς ἐπισκέπτονται τὰς ἀναγραφὰς τῶν χρόνων καὶ εὑρίσκουσιν αὐτὸν πεντακοσιοστοῦ ἔτους πεπληρωμένου ἐληλυθέναι.
Μέγα καὶ θαυμαστὸν οὖν νομίζομεν εἶναι, εἰ ὁδημιουργὸς τῶν ἁπάντων ἀνάστασιν ποιήσεται τῶν ὁσίως αὐτῷ δουλευσάντων ἐν πεποιθήσει πίστεως ἀγαθῆς, ὅπου καὶ δι’ ὀρνέου δείκνυσιν ἡμῖν τὸ μεγαλεῖον τῆς ἐπαγγελίας αὐτοῦ; λέγει γάρ που· Καὶ ἐξαναστήσεις με, καὶ ἐξομολογήσομαί σοι, καί· Ἐκοιμήθην καὶ ἐξηγέρθην, ὅτι σὺ μετ’ ἐμοῦ εἶ. ἀναζωπυρησάτω οὖν ἡ πίστις αὐτοῦ ἐν ὑμῖν, καὶ νοήσωμεν ὅτι πάντα ἐγγὺς αὐτῷ ἐστιν. ἐν λόγῳ τῆς μεγαλωσύνης αὐτοῦ συνκαταστρεψαι. Τίς ἐρεῖ αὐτῷ· Τί ἐποίησας; ἢ τίς ἀντιστήσεται τῷ κράτει τῆς ἰσχύος αὐτοῦ; ὅτε θέλει καὶ ὡς θέλει ποιήσει πάντα, καὶ οὐδὲν μὴ παρέλθῃ τῶν δεδγματισμένων ὑπ’ αὐτοῦ. πάντα ἐνώπιον αὐτοῦ εἰσίν, καὶ οὐδὲν λέληθεν τὴν βουλὴν αὐτοῦ, εἰ οἱ οὐρανοὶ διηγοῦνται δόξαν θεοῦ, ποίησιν δὲ χειρῶν αὐτοῦ ἀναγγέλει τὸ στερέωμα· ἡ ἡμέρα τῇ ἡμέρᾳ ἐρεύγεται ῥῆμα, καὶ νὺξ νυκτὶ ἀναγγέλει γνῶσιν· καὶ οὐκ εἰσὶν λόγοι οὐδὲ λαλιαί, ὧν οὐχὶ ἀκούονται αἱ φωναὶ αὐτῶν.
Ταύτῃ οὖν τῇ ἐλπίδε προσδεδέσθωσαν αἱ ψυψαὶ ἡῶν τῷ πιστῷ ἐν ταῖς ἐπαγγελίαις καὶ τῷ δικαίῳ ἐν τοῖς κρίμασσιν. ὁ παραγγείλας μὴ ψεύεσθαι, πολλῷ μᾶλλον αὐτὸς οὐ ψεύσεται· οὐδὲν γὰρ ἀδύνατον παρὰ τῷ θεῷ εἰ μὴ τὸ ψεύσασθαι. ἀναζωπυρησάτω οὖν ἡ πίστις αὐτοῦ ἐν ἡμῖν, καὶ νοήσωμεν ὅτι πάντα ἐγγὺς αὐτῷ ἐν ἡμῖν, καὶ νοήσωμεν ὅτι πάντα ἐγγὺς αὐτῷ ἐστιν. ἐν λόγῳ τῆς μεγαλωσύνης αὐτοῦ συνεστήσατο τὰ πάντα, καὶ ἐν λόγῳ δύναται αὐτὰ καταστρέψαι. Τίς ἐρεῖ αὐτῳ· Τί ἐποίησας; ἢ τίς ἀντιστήσεται τῷ κράτει τῆς ἰσχύος αὐτοῦ; ὅτε θέλει καὶ ὡς θέλει ποιήσει πάντα, καὶ οὐδὲν μὴ παρέλθῃ τῶν δεδεγματισμένων ὑπ’ αὐτοῦ. πάντα ἐνώπιον αὐτοῦ εἰσίν, καὶ οὐδὲν λέληθεν τὴν βουὴν αὐτοῦ, εἰ οἱ οὐρανοὶ διηγοῦνται δόξαν θεοῦ, ποίησιν δὲ χειρῶν αὐτοῦ ἀναγγέλει τὸ στερέωμα· ἡ ἡμέρα τῇ ἡμέρᾳ ἐρεύγεται ῥῆμα, καὶ νὺξ νυκτὶ ἀναγγέλει γνῶσιν· καὶ οὐκ εἰσὶν λόγοι οὐδὲ λαλιαί, ὧν οὐχὶ ἀκούονται αἱ φωναὶ αὐτῶν.
Πάντων οὖν βλεπομένων καὶ ἀκουομένων, φοβηθῶμεν αὐτόν, καὶ ἀπολίπωμεν φαύλων ἔργων μιαρὰς ἐπιθυμίας, ἵνα τῷ ἐλέει αὐτοῦ σκεπασθῶμεν ἀπό τῶν μελλόντων κριμάτων. ποῦ γάρ τις ἡμῶν δύνατια φυγεῖν ἀπό τῆς κραταιᾶς χειρὸς αὐτοῦ; ποῖος δὲ κόμος δέξεταί τινα τῶν αὐτομολούντων ἀπ’ αὐτοῦ; λέγει γάρ που τὸ γραφεῖον· Ποῦ ἀφήξω καὶ ποῦ κρυβήσομαι ἀπὸ τοῦ προσώπου σου; ἐὰν ἀναβῶ εἰς τὸν οὐρανόν, σὺ ἐκεῖ εἶ· ἐὰν ἀπέλθω εἰς τὰ ἔσχατα τῆς ἀβύσσους, ἐκεῖ τὸ πνεῦμά σου. ποῖ οὖν τις ἀπέλθῃ ἢ ποῦ ἀποδράσῃ ἀπὸ τοῦ τὰ πάντα ἐμπεριέχοντος;
Προσέλθωμεν οὖν αὐτῷ ἐν ὁσιότητι ψυχῆς, ἁγνὰς καὶ ἀμιάντους χεῖρας ἄροντες πρὸς αὐτόν, ἀγαπῶντες τὸν ἐπιεικῆ καὶ εὔσπλαγχνον πατέρα ἡμῶν, ὃς ἐκλογῆς μέρος ἡμᾶς ἐποίησεν ἐαυτῷ. οὕτω γὰρ γέγραπται· Ὅτε διεμέριζεν ὁ ὕψιστος ἔθνη, ὡς διέπειρεν υἱοὺς Ἀδάμ, ἔστησεν ὅρια ἐθνῶν κατὰ ἀριθμὸν ἀγγέλων θεοῦ. ἐγενήθη μερὶς κύριος λαμβάνει ἑαυτῷ ἔθνος ἐκ μέσου ἐθνῶν, ὥσπερ λαμβάνει ἄνθρωπος τὴν ἀπαρχὴν αὐτοῦ τῆς ἅλω· καὶ ἐξελεύσεται ἐκ τοῦ ἔθνους ἐκείνου ἅγια ἁγίων. 
Ἁγίου οὖν μερὶς ὑπάρχοντες ποιήσωμεν τὰ τοῦ ἁγιασμοῦ πάντα, φεύγοντες καταλαλιάς, μιαράς τε καὶ ἀνάγνους συμπλοκάς, μέθας τε καὶ νεωτερισμοὺς καὶ βδελυκτὰς ἐπιθυμίας, μυσερὰν μοιχείαν , βδελυκτὴν ὑπερηφανίαν. Θεὸς γάρ, φησίν, ὐπερηφάνοις ἀντιτάσσεται, ταπεινοῖς, οἷς δίδωσιν χάριν. κολληθῶμεν οὖν ἐκείνοις, οἷς ἡ χάρις ἀπὸ τοῦ θεοῦ δέδοται· ἐδυσώμεθα τὴν ὁμόνοιαν ταπεινοφρνοῦντες, ἐγκρατευόμενοι, ἀπὸ παντὸς ψιθυρισμοῦ καὶ καταλαλιᾶς πόρρω ἑαυτοὺς ποιοῦντες, ἔργοις δικαιούμενοι, μὴ λόγοις. λέγει γάρ· Ὁ τὰ πολλὰ λέγων καὶ ἀντακούσεται· ἢ ὁ εὔλαλος οἴεται εἶναι δίκαιος; εὐλογημένος γεννητὸς γυναικὸς ὀλιγόβιος. μὴ πολὺς ἐν ῥήμασιν γίνου. ὁ ἔπαινος ἡμῶν ἔστω ἐν θεῷ καὶ μὴ ἐξ αὐτῶν· αὐτεπαινέτους γὰρ μισεῖ ὁ θεός. ἡ μαρτυρία τῆς ἀγαθῆς πράξεως ἡμῶν διδόσθω ὑπ’ ἄλλων, καθὼς ἐδόθη τοῖς πατράσιν ἡμῶν τοῖς δικαίοις. θράσος καὶ αὐθάδεια καὶ τόλμα τοῖς κατηραμένοις ὑπὸ τοῦ θεοῦ· ἐπιείκεια καὶ ταπεινοφροσύνη καὶ πραΰτης παρὰ τοῖς ηὐλογημένοις ὑπὸ τοῦ θεοῦ.
Κολληθῶμεν οὖν τῇ εὐλογίᾳ αὐτοῦ καὶ ἴδωμεν, τίνες αἱ ὁδοὶ τῆς εὐλογίας. ἀνατυλίξωμεν τὰ ἀπ’ ἀρχῆς γενόμενα. τίνος χάριν ηὐλογήθη ὁ πατὴρ ἡμῶν Ἀβραάμ, οὐχὶ δικαιοσύνην καὶ ἀλήθειαν διὰ πίστεως ποιῆσας; Ἰσαὰκ μετὰ πεποιθήσεως γινώσκων τὸ μέλλον ἡδέως προσήγετο θυσία. Ἰακὼβ μετὰ ταπεινοφροσύνης ἐξεχώρησεν τῆς γῆς αὐτοῦ δι’ ἀδελφὸν καὶ ἐπορεύθη πρὸς Λαβὰν καὶ ἐδούλευσεν, καὶ ἐδόθη αὐτῷ τὸ δωδεκάσκηπτρον τοῦ Ἰσραήλ.
Ὁ ἐὰν τις καθ’ ἓν ἕκατον εἰλικρινῶς κατανοήσῃ, ἐπιγνώσεται μεγαλεῖα τῶν ὑπ’ αὐτοῦ δεδομένων δωρεῶν. ἐξ αὐτοῦ γὰρ ἱερεῖς καὶ Λευῖται πάντες οἱ λειτουργοῦντες τῷ θυσιαστηρίῳ τοῦ θεοῦ· ἐξ αὐτοῦ βασιλεῖς καὶ ἄρχοντες καὶ ἡγούμενοι κατὰ τὸν Ἰούδαν· τὰ δὲ λοιπὰ σκῆπτρα αὐτοῦ οὐκ θεοῦ, ὅτι ἔσται τὸ σπέρμα σου ὡς οἱ ἀστέρες τοῦ οὐρανοῦ. πάντες οὖν ἐδοξάσθησαν καὶ ἐμεγαλύνθησαν οὐ δι’ αὐτῶν ἢ τῶν ἔργων αὐτῶν ἢ τῆς δικαιοπραγίας ἧς κατειργάσαντο, ἀλλὰ διὰ τοῦ θελήματος αὐτοῦ. καὶ ἡμεῖς οὖν, διὰ θελήματος αὐτοῦ ἐν Χριστῷ Ἰησοῦ κληθέντες, οὐ δι’ ἑαυτῶν δικαιούμεθα, οὐδὲ διὰ τῆς ἔργων ὧν κατειργασάμεθα ἐν ὁσιότητι καρδίας, ἀλλὰ διὰ τῆς πίστεως, δι’ ἧς πάντας τοὺς ἀπ’ αἰῶνος ὁ παντοκράτωρ θεὸς ἐδικαίωσεν· ᾧ ἔστω ἡ δόξα εἰς τοὺς αἰῶνας τῶν αἰώνων. ἀμήν.
Τί οὖν ποιήσωμεν, ἀδελφοί; ἀργήσωμεν ἀπὸ τῆς ἀγαθοποιΐας καὶ ἐγκαταλίπωμεν τὴν ἀγάπην; μαθαμῶς τοῦτο ἐάσαι ὁ δεσπότης ἐφ’ ἡμῖν γε γενηθῆναι, ἀλλὰ σπεύςωμεν μετὰ ἐκτενείας καὶ προθυμίας πᾶν ἔργον ἀγαθὸν ἐπιτελεῖν. αὐτὸς γὰρ ὁ δημιουργὸς καὶ δεσπότης τῶν ἁπάντων ἐπὶ τοῖς ἔργοις αὐτοῦ ἀγαλλιᾶται. τῷ γὰρ παμμεγεθεστάτῳ αὐτοῦ συνέσει διεκόσμησεν αὐτούς· γῆ τε διεχώρισεν ἀπὸ τοῦ περιέχοντος αὐτὴν ὕδατος καὶ ἥδρασεν ἐπὶ τὸν ἀσφαλῆ τοῦ ἰδίον βουλήματος θεμέλιον· τά τε ἐν αὐτῃ ζῶα φοιτωντα τῇ ἑαυτοῦ διατάξει ἐκέλευσεν εἶναι· θάλασσαν καὶ τὰ ἐν αὐτῇ ζῶα προετοιμάσας ἐνέκλεισεν τῇ ἑαυτοῦ δυνάμει. ἐπὶ πᾶσι τὸ ἐξοχώτατον καὶ παμμέγεθες κατὰ διάνοιαν, ἄνθρωπον, ταῖς ἱεραῖς καὶ ἀμώμοις χερσὶν ἔπλασεν τῆς ἐαυτοῦ εἰκόνος χαρακτῆρα. οὕτως γάρ φησιν ὁ θεός· Ποιηπσωμεν ἄνθρωπον κατ’ εἰκόνα καὶ καθ’ ὁμοίωσιν ἡμετέραν· καὶ ἐποίησεν ὁ θεὸς τὸν ἄθρωπον, ἄρσεν καὶ θῆλυ ἐποίησεν αὐτούς. ταῦτα οὖν πάντα τελειώσας ἐπῄνεσεν αὐτὰ καὶ ηὐλόγησεν καὶ εἶπεν· Αὐξάνεσθε καὶ πληθύνεσθε. ἴδωμεν, ὅτι ἐν ἔργοις ἀγαθοῖς πάντες ἐκοσμήθησαν οἱ δίκαιοι, καὶ αὐτὸς δὲ ὁ κύριος ἔργοις ἀγαθοῖς ἑαυτὸν κοσμήσας ἐχάρη. ἔχοντες οὖν τοῦτον τὸν ὑπογραμμὸν ἀόκνως προσέλθωμεν τῷ θελήματι αὐτοῦ· ἐξ ὅλης τῆς ἰσχύος ἡμῶν ἐργασώμεθα ἔργον δικαιοσύνης.
Ὁ ἀγαθὸς ἐργάτης μετὰ παρρησίας λαμβάνει τὸν ἄρτον τοῦ ἔργου αὐτοῦ, ὁ νωθρὸς καὶ παρειμένος οὐκ ἀντοφθαλμεῖ τῷ ἐργοπαρέκτῃ αὐτοῦ. δέον οὖν ἐστὶν προθύμους ἡμᾶς εἶναι εἰς ἀγαθοποιΐαν· ἐξ αὐτοῦ γάρ ἐστιν τὰ πάντα. προλέγει γὰρ ἡμῖν· Ἰδοὺ ὁ κύριος, καὶ ὁ μισθὸς αὐτοῦ πρὸ προσώπου αὐτοῦ, ἀποδοῦναι ἕκάστῳ κατὰ τὸ ἔργον αὐτοῦ. προτρέπεται οὖν ἡμᾶς πιστεύοντας ἐξ ὅλης τῆς καρδίας ἐπ’ αὐτῷ μὴ ἀρτοὺς μηδὲ παρειμένους εἶναι ἐπὶ πᾶν ἔργον ἀγαθόν. τὸ καύχημα ἡμῶν καὶ ἡ παρρησία ἔστω ἐν αὐτῷ· ὑποτασσώμεθα τῷ θελήματι αὐτοῦ· κατανοήσωμεν τὸ πᾶν πλῆθος τῶν ἀγγέλων αὐτοῦ, πῶς τῷ θελήματι αὐτοῦ λειτουργοῦσιν παρεστῶτες. λέγει γὰρ ἡ γραφή· Μύιαι μυριάδες παρειστήκεισαν αὐτῷ, καὶ χίλιαι χιλιάδες ἐλειτούργουν αὐτῷ, καὶ ἐκέκραγον, Ἅγιος, ἅγιος, ἅγιος κύριος σαβαώθ, πλήρης πᾶσα ἡ κτίσις τῆς δόξης αὐτοῦ. καὶ ἡμεῖς, οὖν ἐν ὁμονοίᾳ ἐπὶ τὸ αὐτὸ συναχθέντες τῇ συνειδήσει, ὡς ἐξ ἑνὸς στόματος βοήσωμεν πρὸς αὐτὸν ἐκτενῶς εἰς τὸ μετόχους ἡμᾶς γενέσθαι τῶν μεγάλων καὶ ἐνδόξων ἐπαγγελιῶν αὐτοῦ. λέγει γάρ· Ὀφθαλμὸς οὐκ εἶδεν, καὶ οὖς οὐκ ἀνέβη, ὅσα ἡτοίμασεν κύριος τοῖς ὑπομένουσιν αὐτόν.
Ὡς μακάρια καὶ θαυμαστὰ τὰ δῶρα τοῦ θεοῦ, ἀγαπητοί. ζωὴ ἐν ἀθανασίᾳ, λαμπρότης ἐν δικαιοσύνῃ, ἀλήθεια ἐν παρρησίᾳ, πίστις ἐν πεποιθήσει, ἐγκράτεια ἐν ἁγιασμῷ· καὶ ταῦτα ὑπέπιπτεν πάντα ὑπὸ τὴν διάνοιαν ἡμῶν. τίνα οὖν ἄρα ἐστὶν τὰ ἑτοιμαζόμενα τοῖς ὑπομένουσιν; ὁ δημιουργὸς καὶ πατὴρ τῶν αἰώνων ὁ πανάγιος αὐτὸς γινώσκει τὴν ποσότητα καὶ τὴν καλλονὴν αὐτῶν. ἡμεῖς οὖν ἀγωνισώμεθα εὑρεθῆναι ἐν τῷ ἀριθμῷ τῶν ὑπομενόντων, ὅπως μεταλάβωμεν τῶν ἐπηγγελμένων δωρεῶν. πῶς δὲ ἔσται τοῦτο, ἀγαπητοί; ἐὰν ἐστηριγμένη ᾖ ἡ διάνοια ἡμῶν πιστῶς πρὸς τὸν θεόν, ἐὰν ἐπιτελέσωμεν τὰ ἀνήκοντα τῇ ὁδῷ τῆς ἀληθείας, ἀορρίψαντες ἀφ’ ἑαυτῶν πᾶσαν ἀδικίαν καὶ πονηρίαν, πλεονεξίαν, ἔρεις, κακοηθείας τε καὶ δόλους, ψιθυρισμούς τε καὶ καταλαλιάς, θεοστυγίαν, ὑπερηφανίν τε καὶ ἀλαζονείαν, κενοδοξίαν τε καὶ ἀφιλοξενίαν. ταῦτα γὰρ οἱ πράσσοντες στυγητοὶ τῷ θεῷ ὑπάρχουσιν· οὐ μόνον δὲ οἱ πράσσοντες αὐτά, ἀλλὰ καὶ οἱ συνευδοκοῦντες αὐτοῖς. λέγει γὰρ ἡ γραφή· Τῷ δὲ ἁμαρτωλῷ εἶπεν ὁ θεός· Ἱναντί σὺ διηγῇ τὰ δικαιώματά μου, καὶ ἀναλαμβάνεις τὴν διαθήκην μου ἐπὶ στόματός σου; σὺ δὲ ἐμίσησας παιδείαν καὶ ἐξέβαλες τοὺς λόγους μου εἰς τὰ ὀπίσω. εἰ ἐθεώρεις κλέπτην, συνέτερχες αὐτῷ καὶ μετὰ μοιχῶν τὴν μερίδα σου ἐτίθεις. τὸ στόμα σου ἐπλεόνασεν κακίαν, καὶ ἡ γλῶσσα σου περιέπλεκεν δολιότητα. ταῦτα ἐποίησας, καὶ ἐσίγησα· ὑπέλαβες, ἄνομε, ὅτι ἔσομαί σοι ὅμοιος. 10. ἐλέγξω σε καὶ παραστήσω σε κατὰ πρόσωπόν σου. 11. σύνετε δὴ ταῦτα, οἱ ἐπιλανθανόμενοι τοῦ θεοῦ, μήποτε ἁρπάσῃ ὡς λέων, καὶ μὴ ᾖ ὁ ῥυόμενος. 12. θυσία αἰνέσεως δοξάσει με, καὶ ἐκεῖ ὁδός, ᾗ δείξω αὐτῷ τὸ σωτήριον τοῦ θεοῦ.
Αὕτη ἡ ὁδός, ἀγαπητοί, ἐν ᾗ εὕρομεν τὸ σωτήριον ἡμῶν, Ἰησοῦν Χριστόν, τὸν ἀρχιερέα τῶν προσφορῶν ἡμῶν, τὸν προστάτην καὶ βοηθὸν τῆς ἀσθενείας ἡμῶν. διὰ τούτου ἀτενίζομεν εἰς τὰ ὕψη τῶν οὐρανῶν, διὰ τούτου ἐνοπτριζόμεθα τὴν ἄμωμον καὶ καὶ ὑπερτάτην ὄψιν αὐτοῦ, διὰ τούτου ἠνεῴχθησαν ἡμῶν οἱ ὀφθαλμοὶ τῆς καρδίας, διὰ τούτου ἡ ἀσύνετος καὶ ἐσκοτωμένη διάνοια ἡμῶν ἀναθάλλει εἰς τὸ φῶς, διὰ τούτου ἠθέλησεν ὁ δεσπότης τῆς ἀθανάτου γνώσεως ἡμᾶς γευσασθαι, ὃς ὢν ἀπαύγασμα τῆς μεγαλωσύνης αὐτοῦ, τοσούτῳ μείζων ἐστὶν ἀγγέλων, ὅσῳ διαφορώτερον ὄνομα κεκληρονόμηκεν. γέγραπται γὰρ οὕτως· Ὁ ποιῶν τοὺς ἀγγέλους αὐτοῦ πνεύματα καὶ τοὺς λειτουργοὺς αὐτοῦ πυρὸς φλόγα. ἐπὶ δὲ τῷ υἱῷ αὐτοῦ οὕτως εἶπεν ὁ δεσπότης. Υἱός μου εἶ σύ, ἐγὼ σήμερον γεγέννηκά σε· αἴτησαι παρ’ ἐμοῦ, καὶ δώσω σοι ἔθνη τὴν κληρονομίαν σου καὶ τὴν κατάσχεσίν σου τὰ πέρατα τῆς τῆς. καὶ πάλιν λέγει πρὸς αὐτόν· Κάθου ἐκ δεξιῶν μου, ἕως ἂν θῶ τίνες οὖν οἱ ἐχθροί; οἱ φαῦλοι καὶ ἀντιτασόμενοι τῷ θελήματι αὐτοῦ.
Στρατευσώμεθα οὖν, ἄνδρες αδελφοί, μετὰ πάσης ἐκτενείας ἐν τοῖς ἀμώμοις προστάγμασιν αὐτοῦ. κατανοήσωμεν τοὺς στρατευομένους τοῖς ἡγουμένοις ἡμῶν, πῶς εὐτάκτως, πῶς ἑκτικῶς, πῶς ὑποτεταγμένως ἐπιτελοῦσιν τὰ διατασσόμενα. οὐ πάντες εἰσὶν ἔπαρχοι οὐδὲ χιλίαρχοι οὐδὲ ἑκατόνταρχαι οὐδὲ πεντηκόνταρχοι οὐδὲ τὸ καθεξεῆς, ἀλλ’ ἕκαστος ἐν τῷ ἰδίῳ τάγματι τὰ ἐπιτασσόμενα ὑπὸ τοῦ βασιλέως καὶ τῶν ἡγουμένων ἐπιτελεῖ. οἱ μικροὶ δίχα τῶν μεγάλων· σύγκρασίς. λάβωμεν τὸ σῶμα ἡμῶν· ἡ κεφαλὴ δίχα τῶν ποδῶν οὐδέν ἐστίν, οὕτως οὐδὲ οἱ πόδες δίχα τῆς κεφαλῆς· τὰ δὲ εὔχρηστά εἰσιν ὅλῳ τῷ σώμᾳτι· ἀλλὰ πάντα συνπνεῖ καὶ ὑποταγῇ μιᾷ χρῆται εἰς τὸ σώζεσθαι ὅλον τὸ σῶμα.
Χωζέσθω οὖν ἡμῶν ὅλον τὸ σῶμα ἐν Χριστῷ Ἰηςοῦ, καὶ ὑποτασσέσθω ἕκαστος τῷ πλησίον αὐτοῦ, καθὼς ἐτέθη ἐν τῷ χαρίσματι αὐτοῦ. ὁ ἰσχυρὸς τημελείτω τὸν ἀσθενῆ, ὁ δὲ ἀσθενὴς ἐντρεπέσθω τὸν ἰσχυρόν· ὁ πλούσιος ἐπιχορηγείτω τῷ πτωχῷ, ὁ δὲ πτωχὸς εὐχαριστείτω τῷ θεῷ, ὅτι ἔδωκεν αὐτῷ δι’ οὗ ἀναπληρωθῇ αὐτοῦ τὸ ὑστέρημα· ὁ σοφὸς ἐνδεικνύσθω τὴν σοφίαν αὐτοῦ μὴ ἐν λόγοις, ἀλλ’ ἐν ἔργοις ἀγαθοῖς· ὁ ταπεινοφρονῶν μὴ ἐαυτῷ μαρτυρείτω, ἀλλ’ ἐάτω ὑφ’ ἑτέρου ἑαυτὸν ναρτυρεῖσθαι· ὁ ἁγνὸς τῇ σαρκὶ μὴ ἀλαζονευέσθω, γινώσκων ὅτι ἕτερός ἐστιν ὁ ἐπιχορηγῶν αὐτῷ τὴν ἐγκράτειαν. ἀναλογισώμεθα οὖν, ἀδελφοί, ἐκ ποίας ὕλης ἐγενήθημεν, ποῖου τάφου καὶ σκότους ὁ πλάσας ἡμᾶς καὶ δημιουργήσας εἰσηπγαγεν εἰς τὸν κόσμον, αὐτοῦ, προετοιμάσας τὰς εὐεργεσίας αὐτοῦ, πρὶν ἡμᾶς γεννηθῆναι. ταῦτα οὖν πάντα ἐξ αὐτοῦ ἔχοντες ὀφείλομεν κατὰ πάντα εὐχαριστεῖν αὐτῷ· ᾧ ἡ δόξα εἰς τοὺς αἰῶνας τῶν αἰώνων. ἀμήν.
Ἄφρονες καὶ ἀσύνετοι καὶ μωροὶ καὶ ἀπαίδευτοι χλευάζουσιν ἡμᾶς καὶ μυκτηρίζουσιν, ἑαυτοὺς βουλόμενοι ἐπαίρεσθαι ταῖς διανοίαις αὐτῶν. τί γὰρ δύναται θνητός; ἢ τίς ἰσχὺς γηγενοῦς; γέγραπται γάρ· Οὐκ ἦν μορπὴ πρὸ ὀφθαλμῶν μου, ἀλλ’ ἢ αὔραν καὶ φωὴν ἤκουον· ‘Τί γάρ; μὴ καθαρὸς ἔσται βροτὸς ἔναντι κυρίου; ἢ ἀπὸ τῶν ἐργων αὐτοῦ ἄμεμπτος ἀνήρ, ἐι κατὰ παίδων αὐτοῦ οὐ πιστεύει, κατὰ δὲ ἀγγέλων αὐτοῦ σκολιόν τι ἐπενόησεν; οὐρανὸς δὲ οὐ καθαρὸς ἐνώπιον αὐτοῦ· ἔα δέ, οἱ κατοικοῦντες οἰκίας πηλίνας, ἐξ ὧν καὶ αὐτοὶ ἐκ τοῦ αὐτοῦ πηλοῦ ἐσμέν· ἔπαισεν αὐτοὺς σητὸς τρόπον καὶ ἀπὸ πρωΐθεν ἕως ἑσπέρας οὐκ ἔτι εἰσίν· παρὰ τὸ μὴ δύνασθαι αὐτοὺς ἑαυτοῖς βοηθῆσαι ἀπώλοντο. ἐνεφύσησεν αὐτοῖς, καὶ ἐτελεύτησαν παρὰ τὸ μὴ ἔχειν αὐτοὺς σοφίαν. ἐπικάλεσαι δὲ, εἴ τίς σοι ὑπακούσεται, ἢ εἴ τινα ἁγίων ἀγγέλων ὄψῃ· καὶ γὰρ ἄφρονα ἀναιρεῖ ὀργή, πεπλανημένον δὲ θανατοῖ ζῆλος. ἐγὼ δὲ ἑώρακα ἄφρονας ῥίζας βάλλοντας, ἀλλ’ εὐθέως ἐβρώθη αὐτῶν ἡ δίαιτα. πόρρω γένοιντο οἱ υἱοὶ αὐτῶν ἀπὸ σωτηρίας· κολαβρισθείησαν ἐπὶ θύαις ἡσσόνων, καὶ οὐκ ἔσται ὁ ἐξαιρούμενος· ἃ γὰρ ἐκείνοις ἡτοίμασται, δίκαιοι ἔδονται, αὐτοὶ δὲ ἐκ κακῶν οὐκ ἐξαίρετοι ἔσονται.
Προδήλων οὖν ἡμῖν ὄντων τούτων, καὶ ἐγκεκυφότες εἰς τὰ βάθη τῆς θείας γνώσεως, πάντα τάξει ποιεῖν ὀφείλομεν, ὅσα ὁ δεσπότης ἐπιτελεῖν ἐκέλευσεν κατὰ καιροὺς τεταγμένους. τάς τε προσφορὰς καὶ λειτουργίας ἐπιτελεῖσθαι, καὶ οὐκ εἰκῆ ἢ ἀτάκτως ἐκέλευσεν γίνεσθαι, ἀλλ’ ὡρισμένοις καιροῖς καὶ ὥραις. ποῦ τε καὶ διὰ τίνων ἐπιτελεῖσθαι θέλει, αὐτὸς ὥρισεν τῇ ὑπερτάτω αὐτοῦ βουλήσει, ἵν’ ὁσίως πάντα γινόμενα ἐν εὐδοκήσει εὐπρόσδεκτα εἴη τῷ θελήματι αὐτοῦ. οἱ οὖν τοῖς προστεταγμένοις καιροῖς ποιοῦντες τὰς προσφορὰς αὐτῶν εὐπρόσδεκτοί τε καὶ μακάριοι· τοῖς γὰρ νομίμοις τοῦ δεσπότου ἀκολουθοῦντες οὐ διαμαρτάνουσιν. τῷ γὰρ ἀρχιερεῖ ἴδιαι λειτουργίαι δεδομέναι εἰσίν, καὶ τοῖς ἱερεῦσιν ἴδιος ὁ τόπος προστέτακται, καὶ Λευΐταις ἴδιαι διακονίαι ἐπίκεινται· ὁ λαϊκὸς ἄνθρωπος τοῖς λαϊκοῖς προστάγμασιν δέδεται.
Ἕκαστος ἡμῶν, ἀδελφοί ἐν τῷ ἰδίῳ τάγματι εὐαριστείτω τῷ θεῷ ἐν ἀγαθῇ συνειδήσει ὑπάρχων, μὴ παρεκβαίνων τὸν ὡρισμένον τῆς λειτουργίας αὐτοῦ κανόνα, ἐν σεμνότητι. οὐπανταχοῦ, ἀδελφοί, προσφέρονται θυσίαι ἐνδελεχισμοῦ ἢ εὐχῶν ἢ περὶ ἁμαρτίας καὶ πλημμελεία. ἀλλ’ ἢ ἐν Ἱερουσαλὴμ μόνῃ· κἀκεῖ δὲ οὐκ ἐν παντὶ τόπῳ προσφέρεται, ἀλλ’ ἔμπροσθεν τοῦ ναοῦ πρὸς τὸ θυσιαστήριον, μωμοσκοπηθὲν τὸ προσφερόνενον διὰ τοῦ ἀρχιερέως καὶ τῶν προειρημένων λειτουργῶν. οἱ οὖν παρὰ τὸ καθῆκον τῆς βουλήσεως αὐτοῦ ποιοῦντές τι θάνατον τὸ πρόςτιμον ἔχουσιν. ὁρᾶτε, ἀδελφοί· ὅσῳ πλείονος κατηξιώθημεν γνώσεως, τοσούτῳ μᾶλλον ὑποκείμεθα κινδύνῳ.
Οἱ ἀπόστολοι ἡμῖν εὐηγγελίσθησαν ἀπὸ τοῦ κυρίου Ἰησοῦ Χριστοῦ, Ἰησοῦς ὁ Χριστὸς ἀπὸ τοῦ θεοῦ ἐξεπέμφθη. ὁ Χριστὸς οὖν ἀπὸ τοῦ θεοῦ καὶ οἱ ἀπότολοι ἀπὸ τοῦ Χριστοῦ· ἐγένοντο οὖν ἀμφότερα εὐτάκως ἐκ θελήματος θεοῦ. παραγγελίας οὖν λαβόντες καὶ πληροφορηθέντες διὰ τῆς ἀναστάσεως τοῦ κυρίου ἡμῶν Ἰησοῦ Χριστοῦ καὶ πιστωθέντες ἐν τῷ λόγῳ τοῦ θεοῦ, μετὰ πληροφορίας πνεύματος ἁγίου ἐξῆλθον εὐαγγελιζόμενοι, τὴν βασιλείαν τοῦ θεοῦ μέλλειν ἔρχεσθαι. κατὰ χώρας οὖν καὶ πόλεις κηρύσσοντες καθίστανον τὰς ἀπαρχὰς αὐτῶν, κοκιμάσαντες τῷ πνεύματι, εἰς ἐπισκόπους καὶ διακόνους τῶν μελλόντων πιστεύειν. καὶ τοῦτο οὐ καινῶς· ἐκ γὰρ δὴ πολλῶν χρόνων ἐγέγραπτο περὶ ἐπισκόπων καὶ διακόνων. οὕτως γάρ που λέγει ἡ γραφή· Καταστήσω τοὺς ἐπισκόπους αὐτῶν ἐν δικαιοσύνῃ καὶ τοὺς διακόνους αὐτῶν ἐν πίστει.
Καὶ τί θαυμαστόν, εἰ οἱ ἐν Χριστῷ πιστευθέντες παρὰ θεοῦ ἔργον τοιοῦτο κατέστησαν τοὺς προειρημένους; ὅπου καὶ ὁ μακάριος πιστὸς θεράπων ἐν ὅλῳ τῷ οἴκῳ Μωϋσῆς τὰ διατεταγμένα αὐτῷ πάντα ἐσημειώσατο ἐν ταῖς ἱεραῖς βίβλοις, ᾧ καὶ ἐπηκολούθησαν οἱ λοιποὶ προφῆται, συνεπιμαρτυροῦντες τοῖς ὑπ’ αὐτοῦ νενομοθετημένοις. ἐκεῖνος γάρ, ζήλου ἐπεσον́τος περὶ τῆς ἱερωσύνης καὶ στασιαζουσῶν τῶν φυλῶν, ὁποία αὐτῶν εἴη τῷ ἐνδόξῳ ὀνόματι κεκοσμημένη, ἐκέλευσεν τοὺς δώδεκα φυλάρχους προσενεγκεῖν αὐτῳ ῥάβδους ἐπιγεγραμμένας ἑκάστης φυλῆς κατ’ ὄνομα· καὶ λαβὼν αὐτὰς ἔδησεν καὶ ἐσφράγισεν τοῖς δακτυλίοις τῶν φυλάρχων, καὶ ἀπέθετο αὐτὰς εἰς τὴν σκηνὴν τοῦ μαρτυρίου ἐπὶ τὴν τράπεζαν τοῦ θεοῦ. καὶ κλείσας τὴν σκηνὴν ἐσφράγισεν τὰς κλεῖδας ὡσαύτως καὶ τὰς ῥάβδους, καὶ εἶπεν αὐτοῖς· Ἄδρες ἀδελφοί, ἧς ἂν φυλῆς ἡ ῥάβδος βλαστήσῃ, ταύτην ἐκλέλεκται ὁ θεὸς εἰς τὸ ἱερατεύειν καὶ λειτοργεῖν αὐτῷ. πρωΐας δὲ γενομένης συνεκάλεσεν πάντα τὸν Ἰσραήλ, τὰς ἑξακοσίας χιλιάδας τῶν ἀνδρῶν, καὶ ἐπεδείξατο τοῖς φυλάρχοις τὰς σφραγῖδας, καὶ ἤνοιξεν τὴν σκηνὴν τοῦ μαρτυρίου καὶ προεῖλεν τὰς ῥάβδους· καὶ εὑρέθη ἡ ῥάβδος Ἀαρὼν οὐ μόνον βεβλαστηκυῖα, ἀλλὰ καὶ κρπὸν ἔχουσα. τί δοκεῖτε, ἀγαπητοί; οὐ προῄδει Μωϋσῆς τοῦτο μέλλειν ἔσεσθαι; μάλιστα ᾔδει· ἀλλ’ ἵνα μὴ ἀκαταστασία γένηται ἐν τῷ Ἰσραήλ, οὕτως ἐποίησεν, εἰς τὸ δοξασθῆναι τὸ ὄνομα τοῦ ἀληθινοῦ καὶ μόνου θεοῦ· ᾧ ἡ δόξα εἰς τοὺς αἰῶνας τῶν αἰώνων. ἀμήν.
Καὶ οἱ ἀπόστολοι ἡμῶν ἔγνωσαν διὰ τοῦ κυρίου ἡμῶν Ἰησοῦ Χριστοῦ, ὅτι ἔρις ἔσται ἐπὶ τοῦ ὀνόματος τῆς ἐπισκοῆς. διὰ ταύτην οὖν τὴν αἰτίαν πρόγνωσιν εἰληφότες τελείαν κατέστησαν τοὺς προειρημένους, καὶ μεταξὺ ἐπινομὴν δεδώκασιν, ὅπως, ἐὰν κοιμηθῶσιν, διαδέξωνται ἕτεροι δεδοκιμασμένοι ἄντρες τὴν λειτουργίαν αὐτῶν. τοὺς οὖν κατασταθέντας ὑπ’ ἐκείνων ἢ μεταξὺ ὑφ’ ἑτέρων ἐλλογίμων ἀνδρῶν συνευδοκησάσης τῆς ἐκκλησίας πάσης, καὶ λειτουργήσαντας ἀμέμπτως τῷ ποιμνίῳ τοῦ Χριστοῦ μετὰ ταπεινοφροσύνης, ἡσύχως καὶ ἀβαναύσως, μεμαρτυρημένους τε πολλοῖς χρόνοις ὑπὸ πάντων, τούτους οὐ δικαίως νομίζομεν ἀποβάλλεσθαι τῆς λειτουργίας. ἁμαρτία γὰρ οὐ μικρὰ ἡμῖν ἔσται, ἐὰν τοὺς ἀμέμτως καὶ ὁσίως προσενεγκόντας τὰ δῶρα τῆς ἐπισκοπῆς ἀποβάλωμεν. μακάριοι οἱ προοδοιπορήσαντες πρεσβύτεροι, οἵτινες ἔγκαρπον καὶ τελείαν ἔσχον τὴν ἀνάλυσιν· οὐ γὰρ εὐλαβοῦνται μὴ τις αὐτοὺς μεταστήσῃ ἀπὸ τοῦ ἱδρυμένου αὐτοῖς τόπου. ὁρῶμεν γάρ, ὅτι ἐνίους ὑμεῖς μετηγάγετε καλῶς πολιτευομένους ἐκ τῆς ἀμέμπτως αὐτοῖς τετειμημένης λειτουργίας.
Φιλόνεικοι ἔστε, ἀδελφοί, καὶ ζηλωταὶ περὶ τῶν ἀνηκόντων εἰς σωτηρίαν. ἐγκεκύφατε εἰς τὰς ἱερὰς γραφάς, τὰς ἀληθεῖς, τὰς διὰ τοῦ πνεύματος τοῦ ἁγίου. ἐπίστασθε, ὅτι οὐδὲν ἄδικον οὐδὲ παραπεποιημένον γέγραπται ἐν αὐταῖς. οὐχ εὑρήσετε δικαίους ἀποβεβλημένους ἀπὸ ὁσίων ἀνδρῶν. ἐδιώχθησαν δίκαιοι, ἀλλ’ ὑπό ἀνόμων· ἐφυλακίσθησαν, ἀλλ’ ὑπὸ ἀνοσίων· ἐλιθάσθησαν ὑπὸ παρανόμων· ἀπεκτάνθησαν ὑπὸ τῶν μιαρὸν και ἄδικον ζῆλον ἀνειληφότων. ταῦτα πάσχοντες εὐκλεῶς ἤνγκαν. τί γὰρ ἔπωμεν, ἀδελφοί; Δανιὴλ ὑπὸ τῶν φοβουμένων τὸν θεὸν ἐβλήθη εἰς λάκκον λεόντων; ἡ Ἀνανίας καὶ Ἀζαρίας καὶ Μιαὴλ ὑπὸ τῶν θρησκευόντων τὴν μεγαλοπρεπῆ καὶ ἔνδοξον θρησείαν τοῦ ὑψίστου κατείρχθησαν εἰς κάμινον πυρός; μηθαμῶς τοῦτο γένοιτο. τίνος οὖν οἱ ταῦτα δράσαντες; οἱ στυγητοὶ καὶ πάσης κακίας πλήρεις εἰς τοσοῦτο ἐξήρισαν θυμοῦ, ὥτε τοὺς εν ὁσίᾳ καὶ ἀμώμῳ προθέσει δουλεύοντας τῷ θεῷ εἰς αἰκίαν περιβαλεῖν, μὴ εἰδότες ὅτι ὁ ὕψιστος ὑπέρμαχος καὶ ὑπὲρμαχος καὶ ὑπερασπιστής ἐστιν τῶν ἐν καθαρᾷ συνειδήσει λατρευόντων τῷ παναρέτῳ ὀνόματι αὐτοῦ· ᾧ ἡ δόξα εἰς τοὺς αἰῶνας τῶν αἰώνων. ἀμήν. οἱ δὲ ὑπομένοντες ἐν πεποιθήσει δόξαν καὶ τιμὴν ἐκληρονόμησαν, ἐπήρθησάν τε καὶ ἔγγραφοι ἐγένοντο ἀπό τοῦ θεοῦ ἐν τῷ μνημοσύνῳ αὐτοῦ εἰς τοὺς αἰῶνας τῶν αἰώνων. ἀμήν. 
Τοιούτοις οὖν ὑποδείγμασιν κολληθῆναι καὶ ἡμᾶς δεῖ, ἀδελφοί. γέγραπται γάρ· Κολλᾶσθε τοῖς ἁγίοις, ὅτι οἱ κολλώμενοι αὐτοῖς ἁγιασθήσονται. καὶ πάλιν ἐν ἑτέρῳ τύπῳ λέγει· Μετὰ ἀνδρὸς ἀθῴμου ἀθῷος ἔσῃ καὶ μετὰ ἐκλεκτοῦ ἐκλεκτὸς ἔσῃ, καὶ μετὰ στρεβλοῦ διαστρέψεις. κολληθῶμεν οὖν τοῖς ἀθῷοις καὶ δικαίοις· εἰσὶν δὲ οὗτοι ἐκλεκτοὶ τοῦ θεοῦ. ἱναντί ἔρεις καὶ θυμοὶ καὶ διχοσατσίαι καὶ σχίσματα πόλεμός τε ἐν ὑμῖν; ἢ οὐχὶ ἕνα θεὸν ἔχομεν καὶ ἕνα Χριστὸν καὶ ἓν πνεῦμα τῆς χάριτος τὸ ἐχυθὲν ἐφ’ ἡμᾶς; καὶ μία κλῆσις ἐν Χριστῷ; ἱναντί διέλκομεν καὶ διασπῶμεν τὰ μέλη τοῦ Χριστοῦ καὶ στασιάζομεν πρὸς τὸ σῶμα τὸ ἴδιον, καὶ εἰς τοσαύτην ἀπόνοιαν ἐρχόμεθα, ὥστε ἐπιλαθέσθαι ἡμᾶς, ὅτι μέλη ἐσμὲν ἀλλήλων; μνήσθητε τῶν λόγων τοῦ κυρίου Ἰησοῦ. εἶπεν γάρ· Οὐαὶ τῷ ἀνθρώπῳ ἐκείνῳ· καλὸν ἧν αὐτῷ, εἰ οὐκ ἐγεννήθη, ἢ ἕνα τῶν ἐκλεκτῶν μου σκανδαλίσαι· κρεῖττον ἦν αὐτῷ περιτεθῆναι μύλον καὶ καταποντισθῆναι εἰς τὴν θάλασσαν, ἢ ἕνα τῶν ἐκλεκτῶν μου διαστρέψαι. τὸ σχίσμα ὑμῶν πολλοὺς διέστρεψεν, πολλούς εἰς ἀθυμίαν ἔβαλεν, πολλοὺς εἰς δισταγμόν, τοὺς πάντας ἡμᾶς εἰς λύπην· καὶ ἐπίμονς, ὑμῶν ἐστιν ἡ στάσις.
Ἀναλάβετε τὴν ἐπιστολὴν τοῦ μακαρίου Παύλου τοῦ ἀποστόλου. τί πρῶτον ὑμῖν ἐν ἀρχῇ τοῦ εὐαγγελίου ἔγραψεν; ἐπ’ ἀληθείας πνευματιῶς ἐπέστειλεν ὑμῖν περὶ ἑαυτοῦ τε καὶ Κηφᾶ τε καὶ Ἀπολλώ, διὰ τὸ καὶ τότε προσκλίσεις ὑμᾶς πεποιῆσθαι. ἀλλ’ ἡ πρόςκλισις ἐκείνη ἥττονα ἁμαρτίαν ὑμῖν προσήνεγκεν· προσεκλίθητε γὰρ ἀποστόλοις μεμαρτυρημένοις καὶ ἀνδρὶ δεδοκιμασμένῳ παρ’ αὐτοῖς. νυνὶ δὲ κατανοήσατε, τίνες ὑμᾶς διέστρεψαν καὶ τὸ σεμνὸν τῆς περιβοήτου φιλαδεφίας ὑμῶν ἐμείνωσαν. αἰσψρά, ἀγαπητοί, καὶ λίαν αἰσψρά, καὶ ἀνάξια τῆς ἐν Χριστῷ ἀγωγῆς ἀκούεσθαι, τὴν βεβαιοτάτην καὶ ἀρχαίαν Κορινθίων ἐκκλησίαν δι’ ἓν ἢ δύο πρόσωπα στασιάζειν πρὸς τοὺς πρεσβυτέρους· καὶ αὕτη ἡ ἀκοὴ οὐ μόνον εἰς ἡμᾶς ἐχώρησεν, ἀλλὰ καὶ εἰς τοὺς ἑτεροκλινεῖς ὑπάρχοντας ἀφ’ ἡμῶν, ὥστε καὶ βλασφημίας ἐπιφέρεσθαι τῷ ὀνόματι κυρίου διὰ τὴν ὑμετέραν ἀφροσύνην, ἑαυτοῖς δὲ κίνδυνον ἐπεξεργάζεσθαι.
Ἐξάρωμεν οὖν ἐν τάχει καὶ προσπέσωμεν τῷ δεσπότῃ καὶ κλαύσωμεν ἱκετευοντεσ αὐτόν, ὅπως ἵλεως γενόμενος ἐπικαταλλαγῇ ἡμῖν καὶ ἐπὶ τὴν σεμνὴν τῆς φιλαδελφίας ἡμῶν ἁγνὴν ἀγωγὴν ἀποκαταστήσῃ ἡμᾶς. πύλη γὰρ δικαιοσύνης ἀνεῳγγυῖα εἰς ζωὴν αὕτη, καθὼς γέγραπται· Ἀνοίξατέ μοι πύλας δικαιοσύνης, ἵνα εἰςελθὼν ἐν αὐταῖς ἐξομολογήσωμαι τῷ κυρίῳ. αὕτη ἡ πύλη τοῦ κυρίου· δίκαιοι εισελεύσονται ἐν αὐτῇ. πολλῶν οὖν πυλῶν ἀνεῳγιῶν ἡ ἐν δικαιοσύνῃ αὕτη ἐστὶν ἡ ἐν Χριστῷ, ἐν ᾗ μακάριοι πάντες οἱ εἰσελθόντες καὶ κατευθύνοντες τὴν πορείαν αὐτῶν ἐν ὁσιότητι καὶ δικαιοσύνῃ, ἀταράχως πάντα ἐπιτεοῦντες. ἤτω σοφὸς ἐν διακρίσει λόγων, ἤτω ἁγνὸς ἐν ἔργοις. τοσούτῳ γὰρ μᾶλλον ταπεινοφρονεῖν ὀφείλει, ὅσῳ δοκεῖ μᾶλλον μείζων εἶναι, καὶ ζητεῖν τὸ κοινωφελὲς πᾶσιν, καὶ μὴ τὸ ἑαυτοῦ.
Ὁ ἔχων ἀγάπην ἐν Χριστῷ ποιησάτω τὰ τοῦ Χριστοῦ παραγγέλματα. τὸν δεσμὸν τῆς ἀγάπης τοῦ θεοῦ τίς δύνατια ἐξηγήσασθαι; τὸ μεγαλεῖον τῆς καλλονῆς αὐτου τίς ἀρκετὸς ἐξειπεῖν; τὸ ὕψος, εἰς ὃ ἀνάγε ἡ ἀγάπη, ἀνεκδιήγητόν ἐστιν. ἀγάπη κολλᾷ ἡμᾶς τῷ θεῷ, ἀγάπη καλύπτε πλῆθος ἁμαρτιῶν, ἀγάπη πάντα ἀνέχεται, πάντα μακροθυμεῖ· οὐδὲν βάναυσον ἐν ἀγάπῃ, οὐδὲν ὑπερήφανον· ἀγάπη σχίσμα οὐκ ἔχει, ἀγάπη οὐ στασιάζει, ἀγάπη πάντα ποιεῖ ἐν ὁμονοίᾳ· ἐν τῇ ἀγάπῃ ἐτελειώθησαν πάντες οἱ ἐκλεκτοὶ τοῦ θεοῦ, δίχα ἀγάπης οὐδὲν εὐάρεστόν ἐστιν τῷ θεῷ. ἐν ἀγάπῃ προσελάβετο ἡμᾶς ὁ δεσπότης· διὰ τὴν ἀγάπην, ἣν ἔσχεν πρὸς ἡμᾶς, τὸ αἷμα αὐτοῦ ἔδωκεν ὑπὲρ ἡμῶν Ἰησοῦς Χριστὸς ὁ κύριος ἡμῶν ἐν θελήματι θεοῦ, καὶ τὴν σάρκα ὑπὲρ τῆς σαρκὸς ἡμῶν καὶ τὴν ψυχὴν ὑπὲρ τῶν ψυχῶν ἡμῶν.
Ὁρᾶτε ἀγαπητοί, πῶς μέγα καὶ θαυμαστόν ἐστιν ἡ ἀγάπη, καὶ τῆς τελειότητος αὐτῆς οὐκ ἔστιν ἐξήγησις. τίς ἱκανὸς ἐν αὐτῇ εὑρεθῆναι, εἰ μὴ οὓς ἂν καταξιώσῃ ὁ θεός; δεώμεθα οὖν καὶ αἰτώμεθα δίχα προσκλίσεως ἀνθρωπίνης, ἄμωμοι. αἱ γενεαὶ´πᾶσαι ἀπὸ Ἀδὰμ ἕως τῆσδε τῆς ἡμέρας παρῆλθον, ἀλλ’ οἱ ἐν ἀγάπῃ τελειωθέντες κατὰ τὴν τοῦ θεοῦ χάριν ἔχουσιν χῶρον εὐσεβῶν, οἳ φανερωθήσονται ἐν τῇ ἐπισκοπῇ τῆς βασιλείας τοῦ Χριστοῦ. γέγραπται γάρ· Εἰσέλθετε εἰς τὰ ταμεῖα μικρὸν ὅον ὅσον, ἕως οὗ παρέλθῃ ἡ ὀργὴ καὶ ὁ θυμός μου, καὶ μνησθήσομαι ἡμέρας ἀγαθῆς, καὶ ἀναστήσω ὑμᾶς ἐκ τῶν θηκῶν ὑμῶν. μακάριοί ἐσμεν, ἀγαπητοί, εἰ τὰ προτάγματα τοῦ θεοῦ ἐποιοῦμεν ἐν ὁμονοίᾳ ἀγάπης, εἰς τὸ ἀφεθῆναι ἡμῖν δι’ ἀγάπης τὰς ἁμαρτίας. γέγραπται γάρ· Μακάριοι, ὧν ἀφέθησαν αἱ ἀνομίαι καὶ ὧν ἐπεκαλύφθησαν αἱ ἁματίαν, οὐδέ ἐστιν ἐν τῷ στόματι αὐτου δόλος· οὗτος ὁ μακαρισμὸς ἐγένετο ἐπὶ τοὺς ἐκλελεγμένους ὑπὸ τοῦ θεοῦ διὰ Ἰησοῦ Χριστοῦ τοῦ κυρίου ἡμῶν, ᾧ ἡ δόξα εἰς τοὺς αἰῶνας τῶν αἰώνων. ἀμήν.
Ὅσα οὖν παρεπέσαμεν καὶ ἐποιήσαμεν διά τινας παρεμπτώσεις τοῦ ἀντικειμένου, ἀξιώσωμεν ἀφεθῆναι ἡμῖν. καὶ ἐκεῖνοι δέ, οἵτινες ἀρχηγοὶ στάσεως καὶ διχοστασίας ἐγενήθησαν, οφείουσιν τὸ κοινὸν τῆς ἐλπίδος σκοπεῖν. οἱ γάρ μετὰ φόβου καὶ ἀγάπης πολιτευόμενοι ἑαυτοὺς θέλουσιν μᾶλλον αἰκίαις περιπίπτειν φέρουσιν ἢ τῆς παρεδεδομένης ἡμῖν καλῶς καὶ δικαίως ὁμοφωνίας, καλὸν γὰρ ἀνθρώπῳ ἐξομολογεῖσθαι περὶ τῶν παραπτωμάτων ἢ σκληρῦναι τὴν καρδίαν αὐτοῦ, καθῶς ἐσκληρύνθη ἡ καρδία τῶν στασιαζόντων πρὸς τὸν θεράποντα τοῦ θεοῦ Μωϋσῆν, ὧν τὸ κρίμα πρόδηλον ἐγενήθη, κατέβησαν γὰρ εἰς ᾅδου ζῶντες, καὶ θάνατος ποιμανεῖ αυτούς. Φαραὼ καὶ ἡ στρατιὰ αὐτοῦ καὶ πάντες οἱ ἡγούμενοι Αἰγύπτου, τά τε ἅρματα καὶ οἱ ἀνάβαται αὐτῶν οὐ δι’ ἄλλην τινὰ αἰτίαν ἐβυθηισθησαν εἰς θάλασσαν ἐρυθρὰν καὶ ἀπώλοντο, ἀλλὰ διὰ τὸ σκληρυνθῆναι αὐτῶν τὰς ἀσυνέτους καρδίας μετὰ τὸ γενέσθαι τὰ σημεῖα καὶ τὰ τέρατα ἐν γῇ Αἰγύπτου διὰ τοῦ θεράποντος τοῦ θεοῦ Μωϋσέως.
Ἀπροσδεής, ἀδελφοί, ὁ δεσπότης ὑπάρχει τῶν ἁπάντων· οὐδὲν οὐδενὸς χρῄζει εἰ μὴ τὸ ἐξομολογεῖσθαι αὐτῷ. φησὶν γὰρ ὁ ἐκλεκτὸς Δαυείδ· Ἐξομολογήσομαι τῷ κυρίῳ, καὶ ἀρέσει αὐτῷ ὑπέρ μόχον νέον κέρατα ἐκφέροντα καὶ ὁπλάς· ἰδέτωσαν πτωχοὶ καὶ εὐφρανθήτωσαν. καὶ πάλιν λέγει· Θῦσον τῷ θεῷ. θυσίαν αἰνέσεως καὶ ἀπόδος τῷ ὑψίστῳ τὰς εὐχάς σου· καὶ ἐπικάλεσαί με ἐν ἡμέρᾳ θλίψεώς σου, καὶ ἐξελοῦμαι σε, καὶ δοξάσεις με. θυσία γὰρ τῷ θεῷ πνεῦμα συντετριμμένον.
Ἐπίστασθε γάρ καὶ καλῶς ἐπίσταθε τὰς ἱερὰς γραφάς, ἀγαπητοί, καὶ ἐγκεκύφατε εἰς τὰ λόγια τοῦ θεοῦ. πρὸς ἀνάμνησιν οὖν ταῦτα γράφομεν. Μωϋσέως γὰρ ἀναβάντος εἰς τὸ ὄρος καὶ ποιήσαντος τεσσαράκοντα ἡμέρας καὶ τεσσαράκοντα νύκτας ἐν νηστείᾳ καὶ ταπεινώσει, εἶπεν πρὸς αὐτὸν ὁ θεός· Κατάβηθι τὸ τάχος ἐντεῦθεν, ὅτι ἠνόμησεν ὁ λαός σου, οὓς εξήγαες ἐκ γῆς Αἰγύπτου· παρέβησαν ταχὺ ἐκ τῆς ὁδοῦ ἧς ἐντείλω αὐτοῖς, ἐποίησαν ἑαυτοῖς χωνεύματα. καὶ εἶπεν κύριος πρὸς αὐτόν· Ἑώρακα τὸν λαὸν τοῦτον, καὶ ἰδού ἐστιν σκληροτράχηλος· ἔασόν με ἐξολεθρεῦσαι αυτούς, καὶ ἐξαλείψω τὸ ὄνομα αὐτῶν ὑποκάτωθεν τοῦ οὐρανοῦ, καὶ ποιήσω σε εἰς ἔθνος μέγα καὶ θαυμαστὸν καὶ πολὺ μᾶλλον ἢ τοῦτο. καὶ εἶπεν Μωϋσῆς· Μηδαμῶς, κύριε· ἄφες τὴν ἁμαρτίαν τῷ λαῷ τούτῳ ἢ κἀμὲ ἐξάλειψον ἐκ βίβλου ζώτων. ὢ μεγάλης ἀγάπης, ὢ τελειότητος ἀνυπερβλήτου. παρρησιάζεται θεράπων πρὸς κύριον, αἰτεῖται ἄφεσιν τῷ πλήθει, ἢ καὶ ἑαυτὸν ἐξαλειφθῆναι μετ’ αὐτῶν ἀξιοῖ.
Τίς οὖν ἐν ὑμῖν γενναῖοσ, τίς εὔσπλαγχνος, τίς πεπληροφορημένος ἀγάπης; εἰπάτω· Εἰ δι’ ἐμὲ στάσις καὶ ἔρις καὶ σχίσματα, ἐκχωρῶ, ἄπειμι, οὗ ἐὰν βούλησθε, καὶ ποιῶ τὰ ποίμνιον τοῦ Χριστοῦ εἰρηνευέτω μετὰ τῶν καθεσταμένων πρεσβυτέρων. τοῦτο ὁ ποιήσας ἑαυτῷ μέγα κλέος ἐν Χριστῷ περιποιήσεται, καὶ πᾶς τόπος δέξεται αὐτόν, τοῦ γὰρ κυρίου ἡ γῆ καὶ τὸ πλήρωμα αὐτῆς. ταῦτα οἱ πολιτευόμενοι τὴν ἀμεταμέλητον πολιτείαν τοῦ θεοῦ ἐποίησαν καὶ ποιήσουσιν.
Ἵνα δὲ καὶ ὑποδείγματα ἐθνῶν ἐνέγκωμεν. πολλοὶ βασιλεῖς καὶ ἡγούμενοι, λοιμικοῦ τινος ἑαυτοὺς εἰς θάνατον, ἵνα ῥύσωνται διὰ τοῦ ἑαυτῶν αἵματος τοὺς πολίτας· πολλοὶ ἐξεχώρησαν ἰδίων πόλεων, ἵνα μὴ στασιάζωσιν ἐπὶ πλεῖον. ἐπιστάμεθα πολλοὺς ἐν ἡμῖν παραδεδωκότας ἑαυτοὺς εἰς δεσμά, ὅπως ἑτέρους λυτρώσονται· πολλοὶ ἑαυτοὺς παρέδωκαν εἰς δουλείαν, καὶ λαβόντες τὰς τιμὰς αὐτῶν ἑτέρους ἐψώμισαν. πολλαὶ γυναῖκες ἐνδυναμωθεῖσαι διὰ τῆς χάριτος τοῦ θεοῦ ἐπετελέσαντο πολλὰ ἀνδρεῖα. Ἰουδὶθ ἡ μακαρία, ἐν συγκλεισμῷ οὔσης τῆς πόλεως, ᾐτήσατο παρὰ τῶν πρεσβυτέρων ἐαθῆναι αὐτὴν ἐξελθεῖν εἰς τὴ παρεμβολὴν τῶν ἀλλοφύλων. παραδοῦσα οὖν ἑαυτὴν τῷ κινδύνῳ ἐξῆλθεν δι’ ἀγάπην τῆς πατρίδος καὶ τοῦ λαοῦ τοῦ ὄντος ἐν συγκεισμῷ, καὶ παρέδωκεν κύριος Ὀλοφερνην ἐν χειρὶ θηλείας. οὐχ ἧττον καὶ ἡ τελεία κατὰ πίστιν Ἐσθὴρ κινδυνῳ ἑαυτὴν παρέβαλεν, ἵνα τὸ ἔθνος τοῦ Ἰσραὴλ μέλλον ἀπολέσθαι ῥύσηται· διὰ γὰρ τῆς νηστείας καὶ τῆς ταπεινώσεως αὐτῆς ἠξίωσεν τὸν ταπεινὸν τῆς ψυχῆς αὐτῆς ἐρύσατο τὸν λαόν ὧν χάριν ἐκινδύνευσεν.
Καὶ ἡμεῖς οὖν ἐντύχωμεν περὶ τῶν ἔν τινι παραπτώματι ὑπαρχόντων, ὅπως δοθῇ αὐτοῖς ἐπιείκεια καὶ ταπεινοφροσύνη εἰς τὸ εἶξαι αὐτοὺς μὴ ἡμῖν ἀλλὰ τῷ θελήματι τοῦ θεοῦ· οὕτως γὰρ ἔσται αὐτοῖς ἔγκαρπος καὶ τελεία ἡ πρὸς τὸν θεὸν καὶ τοὺς ἁγίους μετ’ οἰκτιρμῶν μνεία. ἀναλάβωμεν παιδείαν, ἐφ’ ᾗ οὐδεὶς ἀγανακτεῖν, ἀγαπητοί. ἡ νουθέτησις, ἣν ποιούμεθα εἰς ἀλλήλους, καλή ἐστιν καὶ ὑπεράγαν ὠφέλιμος· κολλᾷ γὰρ ἡμᾶς τῷ θελήματι τοῦ θεοῦ. οὕτως γάρ φησιν ὁ ἅγιος λόγος· Παιδεύων ἐπαίδευσέν με ὁ κύριος, καὶ τῷ θανάτῳ οὐ παρέδωκέν με· ὃν γὰρ ἀγαπᾷ κύριος παιδεύει, μαστιγοῖ δὲ πάντα υἱὸν ὃν παραδέχεται. Παιδεύσει με γάρ, φησίν, δίκαιος ἐν ἐλέει καὶ ἐλέγξει με, ἔλαιον δὲ ἁμαρτωλῶν μὴ λιπανάτω τὴν κεφαλήν μου. καὶ πάλιν λέγει· Μακάριος ἄνθρωπος, ὃν ἤλεγξεν ὁ κύριος· νουθέτημα δὲ παντοκράτορος μὴ ἀπαναίνου· αὐτὸς γὰρ ἀλγεῖν ποιεῖ, καὶ πάλιν ἀποκαθίτησιν· ἔπαισεν, καὶ αἱ χεῖρες αὐτοῦ ἰάσαντο. ἑξάκις ἐξ ἀναγκῶν ἐξελεῖταί σε, ἐν δὲ τῷ ἑβδόμῳ οὐχ ἅψεταί σου κακόν. ἐν λιμῷ ῥύσεταί σε ἐκ θανάτου, ἐν πολέμῳ δὲ ἐκ χειρὸς σιδήρου λύσει σε· 10. καὶ ἀπὸ μάστιγος γλώσσης σε κρύψει, καὶ οὐ φοβηθήσῃ κακῶν ἐπερχομένων. 11. ἀδίκων καὶ ἀνόμων καταγελάσῃ, ἀπὸ δὲ θηρίων ἀγρίων οὐ μὴ φοβηθῇς· 12. θῆρες γὰρ δὲ θηρίων ἄγριοι εἰρηνεύσουσίν σοι. 13. εἶτα γνώσῃ, ὄτι εἰρηνεύσει σου ὁ οἶκος, ἡ δὲ δίαιτα τῆς σκηνῆς σου οὐ μὴ ἁμάρτῃ. 14. γνώσῃ δέ, ὅτι πολὺ τὸ σπέρμα σου, τὰ δὲ τέκνα σου ὥσπερ τὸ παμβότανον τοῦ ἀγροῦ. 15. ἐλεύσῃ δὲ ἐν τάφῳ ὥσπερ σῖτος ὥριμος κατὰ καιρὸν θεριζομενος, ἢ ὥσπερ θημωνιὰ ἅλωνος καθ’ ὥραν συγκομισθεῖσα. 16. βλέπετε, ἀγαπητοί, πόσος ὑπερασπισμός ἐστιν τοῖς παιδευομένοις ὑπὸ τοῦ δεπότου· πατὴρ γὰρ ἀγαθὸς ὢν παιδεύει εἰς τὸ ἐλεηθῆναι ἡμᾶς διὰ τῆς ὁσίας παιδείας αὐτοῦ.
Ὑμεῖς οὖν οἱ τὴν καταβολὴν τῆς στάσεως ποιήσαντες ὑποτάγητε τοῖς πρεσβυτέροις καὶ παιδεύθητε εἰς μετάνοιαν, κάμψαντες τὰ γόνατα τῆς καρδίας ὑμῶν. μάθετε ὑποτάσσεσθαι, ἀποθέμενοι τὴν ἀλαζόνα καὶ ὑπερήφανον τῆς γλώσσης ὑμῶν αὐθάδειαν· ἄμεινον γάρ ἐστιν ὑμῖν, ἐν τῷ ποιμνίῳ τοῦ Χριστοῦ μικροὺς καὶ ἐλλογιμους εὑρεθῆναι, ἢ καθ’ ὑπεροχὴν δοκοῦντας ἐκριφῆναι ἐκ τῆς ἐλπίδος αὐτοῦ. οὕτως γὰρ λέγει ἡ πανάρετος σοφία· Ἰδού, προήσομαι ὑμῖν ἐμῆς πνοῆς ῥῆσιν, διδάξω δὲ ὑμᾶς τὸν ἐμὸν λόγον. ἐπειδὴ ἐκάλουν καὶ οὐχ ὑπηκούσατε, καὶ ἐξέτεινον λόγους καὶ οὐ προσείχετε, ἀλλὰ ἀκύους ἐποιεῖτε τὰς ἐμὰς βουλάς, τοῖς δὲ ἐμοῖς ἐλέγχοις ἠπειθήσατε· τοιγαροῦν κἀγὼ τῇ ὑμετέρᾳ ἀπωλέᾳ ἐπιγελάσομαι, καταχαροῦμαι δὲ ἡνίκα ἂν ἔρχηται ὑμῖν ὄλεθρος καὶ ὡς ἂν ἀφίκηται ὑμῖν ἄφνω θόρυβος, ἡ δὲ καταστροφὴ ὁμοια καταιγίδι παρῇ, ἢ ὅταν ἔρχηται ὑμῖν θλίψις καὶ πολιορκία. ἔσται γὰρ ὅταν ἐπικαλέσησθέ με, ἐγὼ δὲ οὐκ εἰσακούσομαι ὑμῶν· ζητήσουσίν με κακοί, καὶ οὐχ εὑρήσουσιν. ἐμίσησαν γὰρ σοφίαν, τὸν δὲ φόβον τοῦ κυρίου οὐ προείλαντο, οὐδὲ ἤθελον ἐμαῖς προσέχειν βουλαῖς, ἐμυκτήριζον δὲ ἐμοὺς ἐλέγχους. τοιγαροῦν ἔδουνται τῆς ἑαυτῶν ὁδοῦ τοὺς καρπούς, καὶ τῆς ἑαυτῶν ἀσεβείας πλαησθήσονται· ἀνθ’ ὧν γὰρ ἠδίδουν νηπίους φονευθήσονται, καὶ ἐξετασμὸς ἀσεβεῖς ὀλεῖ· ὁ δὲ ἐμοῦ ἀκούων κατασκηνώσει ἐπ’ ἐλπίδι πεποιθὼς καὶ ἡσυχάσει ἀφόβως ἀπὸ παντὸς κακοῦ. 
Ὑπακούσωμεν οὖν τῷ παναγίῳ καὶ ἐνδόξῳ ὀνόματι αὐτοῦ φυγόντες τὰς προειρημένας διὰ τῆς σοφίας τοῖς ἀπειθουσιν ἀπειλάς, ἵνα κατασκηνώσωμεν πεποιθότες ἐπὶ το ὁσιώτατον τῆς μεγαλωσύνης αὐτοῦ ὄνομα. δέξασθε τὴν συμβουλὴν ἡμῶν, καὶ ἔσται ἀμεταμέλητα ὑμῖν. ζῇ γὰρ ὁ θεὸς καὶ ζῇ ὁ κύριος Ἰησοῦς Χριστὸς καὶ τὸ πνεῦμα το ἅγιον, ἥ τε πίστις καὶ ἡ ἐλπὶς τῶν ἐκλεκτῶν, ὅτι ὁ ποιήσας ἐν ταπεινοφροσύνῃ μετ’ ἐκτενοῦς ἐπιεικείας ἀμεταμελήτως τὰ ὑπὸ τοῦ θεοῦ δεδομένα δικαιώματα καὶ προστάγματα, οὗτος ἐντεταγμένος καὶ ἐλλόγιμος ἔσται εἰς τὸν ἀριθμὸν τῶν σωζομένων διὰ Ἰησοῦ Χριστοῦ, δι’ οὗ ἐστὶν αὐτῳ ἡ δόξα εἰς τοὺς αἰῶνας τῶν αἰώνων. ἀμήν.
Ἐὰν δέ τινες ἀπειθήσωσιν τοῖς ὑπ’ αὐτοῦ δι’ ἡμῶν εἰρημένοις, γινωσέτωσαν ὅτι παραπτώσει καὶ κινδύνῳ οὐ μικρῷ ἑαυτοὺς ἐνδήσουσιν. ἡμεῖς δὲ ἀθῷοι ἐσόμεθα ἀπὸ ταύτης τῆς ἁμαρτίας καὶ αἰτησόμεθα ἐκτενῆ τὴν δέησιν καὶ ἱκεσίαν ποιούμενοι, ὅπως τὸν ἀριθμὸν τὸν κατηριθμημένον τῶν ἐκλεκτῶν αὐτοῦ ἐν ὅλῳ τῷ κόσμῳ διαφυλάξῃ ἄθραυστον ὁ δημιουργὸς τῶν ἁπάντων διὰ τοῦ ἠγαπημένου παιδὸς αὐτοῦ Ἰησοῦ Χριστοῦ, δι’ οὑ ἐκάλεσεν ἡμᾶς ἀπὸ σκότους εἰς φῶς, ἀπὸ ἀγνωσίας εἰς ἐπίγνωσιν δόξης ὀνόματος αὐτοῦ, . . .* ἐλπίζειν ἐπι τὸ ἀρχεγόνον πάσης κτίσεως ὄνομά σου, ἀνοίξας τοὺς ὀφθαλμοὺς τῆς καρδίας ἡμῶν εἰς τὸ γινώσκειν σε τὸν μόνον ὕψιστον ἐν ὑψίστοοις, ἅγιον ἐν ἁγίοις ἀναπαυόμενον. τὸν ταπεινοῦντα ὕβριν ὑεπερηφάνων, τὸν διαλύοντα λογισμοὺς ἐθνῶν, τὸν ποιοῦντα ταπεινοὺς εἰς ὕψος καὶ τοὺς ὑψηοὺς ταπεινοῦντα, τὸν πλουτίζοντα καὶ πτωχίζοντα, τὸν ἀποκτείνοντα καὶ ζῆν ποιοῦντα, μόνον εὑρέτην πνευμάτων καὶ θεὸν πάσης σαρκός· τὸν ἐπιβλέποντα ἐν τοῖς ἀβύσσοις, τὸν ἐπόπτην ἀνθρωπίνων ἔργων, τὸν τῶν κινδυνευόντων βοηθόν, τὸν τῶν ἀπηλπισμένων σωτῆρα, τὸν παντὸς πνεύματος κτίστην καὶ ἐπίσκοπον· τὸν πληθύνοντα ἔθνη ἐπὶ γῆς καὶ ἐκ πάντων ἐκλεξάμενον τοὺς ἀγαπῶντάς σε διὰ Ἰησοῦ Χριστοῦ τοῦ ἠγαπημένου παιδός σου, δι’ οὗ ἡμᾶς ἐπαιδευσας, ἡγίασας, ἐτίμησας· ἀξιοῦμέν σε, δέσποτα, βοηθὸν γενέσθαι καὶ ἀντιλήπτορα ἡμῶν. τοὺς ἐν θλίψει ἡμῶν σῶσον, τοὺς ταπεινοὺς ἐλέησον, τοὺς πεπτωκότας ἔγειρον, τοῖς δεομένοις ἐπιφάνηθι, τοὺς ἀσθενεῖς ἴασαι, τοὺς πλανωμένους τοῦ λαοῦ σου ἐπίστρεψον· χόρτασον τοὺς πεινῶντας, λύτρωσαι τοὺς δεσμίους ἡμῶν, ἐξανάστησον τοὺς ἀσθενοῦντας, παρακάλεσον τοὺς ὀλιγοψυχοῦντας· γνώτωσάν σε ἅπαντα τὰ ἔθνη, ὅτι σὺ εἶ ὁ θεὸς μόνος καὶ Ἰησοῦς Χριστὸς ὁ παῖς σου καὶ ἡμεῖς λαός σου καὶ πρόβατα τῆς νομῆς σου.
Σὺ γὰρ τὴν ἀέναον τοῦ κόσμου σύστασιν διὰ τῶν ἐνεργουμένων ἐφανεροποίησας· σύ, κύριε, τὴν οἰκουμένην ἔκτισας, ὁ πιστὸς ἐν πάσαις ταῖς γενεαῖς, δίκαιος ἐν τοῖς κρίμασιν, θαυμαστὸς ἐν ἰσχύϊ καὶ συνετὸς ἐν τῷ τὰ γενόμενα ἑδράσαι, ὁ ἀγαθὸς ἐν τοῖς ὁρωμένοις καὶ χρηστὸς ἐν τοῖς πεποιθόσιν ἐπὶ σέ, ἐλεῆμον καὶ οἰκτίρμον, ἄφες ἡμῖν τὰς ἀνομίας ἡμῶν καὶ πλημμελείας. μὴ λογίσῃ πᾶσαν ἁμαρτίαν δούλων σου καὶ παιδισκῶν, ἀλλὰ καθάρισον ἡμᾶς τὸν καθαρισμὸν τῆς σῆς ἀληθείας, καὶ κατεύθυνον τὰ διαβήματα ἡμῶν ἐν ὁσιότητι καρδίας προεύεσθαι καὶ ποιεῖν τὰ καλὰ καὶ εὐάρεστα ἐνώπιόν σου καὶ ἐνώπιον τῶν ἀρχόντων ἡμῶν. ναί, δέσποτα, ἐπίφανον τὸ πρόσωπόν σου ἐφ’ ἡμᾶς εἰς ἀγαθὰ ἐν εἰρήνῃ, εἰς τὸ σκεπασθῆναι ἡμᾶς τῇ χειρί σου τῇ κραταιᾷ καὶ ῥυσθῆναι ἀπὸ πάσης ἁμαρτίας τῷ βραχίονί σου τῷ ὑψηλῷ, καὶ ῥῦσαι ἡμᾶς ἀπὸ τῶν μισούντων ἡμᾶς ἀδίκως. δὸς ὁμόνοιαν καὶ εἰρήνην ἡμῖν τε καὶ πᾶσιν τοῖς κατοικοῦσιν τὴν γῆν, καθὼς ἔδωκας τοις πατράσιν ἡμῶν, ἐπικαλουμένων σε αὐτῶν ὁσίως ἐν πίστει καὶ ἀληθείᾳ, ὑπηκόους γινομένους τῷ παντοκράτορι καὶ ἐνδόξῳ ὀνόματί σου, τοῖς τε ἄρχουσιν καὶ ἡγουμένοις ἡμῶν ἐπὶ τῆς γῆς.
Σύ, δεσποτα, ἔδωκας τὴν ἐξουσίαν τῆς βασιλείας αὐτοῖς διὰ του μεγαλοπρεποῦς καὶ ἀνεκδιηγήτου κράτους σου, εἰς τὸ γινώσκοντας ἡμᾶς τὴν ὑπὸ σοῦ αὐτοῖς δεδομένην δόξαν καὶ τιμὴν ὑποτάσσεσθαι αὐτοῖς, μηδὲν ἐναντιουμένους τῷ θελήματί σου· οἷς δός, κύριε, ὑγίειαν, εἰρήνην, ὁμόνοιαν, εὐστάθειαν, εἰς τὸ διέπειν αὐτοὺς τὴν ὑπὸ σοῦ δεδομένην αὐτοῖς ἡγεμονίαν ἀπροσκόπως. σὺ γάρ, δέσποτα ἐουράνιε, βασιλεῦ τῶν αἰώνων, δίδως τοῖς υἱοῖς τῶν ἀνθρώπων δόξαν καὶ τιμὴν καὶ ἐξουσίαν τῶν ἐπὶ τῆς γῆς ὑπαρχόντων· σύ, κύριε, διεύθυνον τὴν βουλὴν αὐτῶν κατὰ τὸ καλὸν καὶ εὐάρεστον ἐνώπιόν σου, ὅπως διέποντες ἐν εἰρήνῃ καὶ πραΰτητι εὐσεβῶς τὴν ὑπὸ σοῦ αὐτοῖς δεδομένην ἐξουσίαν ἵλεώ σου τυγχάνωσιν. ὁ μόνος δυνατὸς ποιῆσαι ταῦτα καὶ περισσότερα ἀγαθὰ μεθ’ ἡμῶν, σοὶ ἐξομολογούμεθα διὰ τοῦ ἀρσιερέως καὶ προστάτου τῶν ψυχῶν Ἰησοῦ Χριστοῦ, δι’ οὗ σοι ἡ δόξα καὶ ἡ μεαλωσύνη καὶ νῦν καὶ εἰς γενεὰν γενεῶν καὶ εἰς τοὺς αἰῶνας τῶν αἰώνων. ἀμήν.
Περὶ μὲν τῶν ἀνηκόντων τῇ θρησκείᾳ ἡμῶν καὶ τῶν ὠφελιμωτάτων εἰς ἐνάρετον βίον τοῖς θέλουσιν εὐσεβῶς καὶ δικαίως διευθύνειν, ἱκανῶς ἐπεστείλαμεν ὑμῖν, ἄνδρες ἀδελφοί. περὶ γὰρ πίστεως καὶ μεταμοίας καὶ γνησίας ἀγαπης καὶ ἐγκρατείας καὶ σωφροσύνης καὶ ὑπομοῆς πάντα τόπον ἐψηλαφήσύνῃ καὶ ἀληθείᾳ καὶ μακροθυμίᾳ τῷ παντοκράτορι θεῷ ὁσίως εὐαρεστεῖν, ὁμονοοῦντας ἀμνησικακως ἐν ἀγάπῃ καὶ εἰρήνῃ μετὰ ἐκτενοῦς ἐπιεικείας, καθὼς καὶ οἱ προδεδηλωμένοι πατέρες ἡμῶν εὐηρέστησαν ταπεινοφρονοῦντες τὰ πρὸς τὸν πατέρα καὶ κτίστην θεὸν· καὶ πάντας ἀνθρώπους. καὶ ταῦτα τοσούτῳ ἥδιον ὑπεμνήσαμεν, ἐπειδὴ σαφῶς ᾔδειμεν γράφειν ἡμᾶς ἀνδράσιν πιστοῖς καὶ ἐλλογιμωτάτοις καὶ ἐγκεκυφόσιν εἰς τὰ λόγια τῆς παιδείας τοῦ θεοῦ.
Θεμιτὸν οὖν ἐστιν τοῖς τοιούτοις καί τοσούτοις ὑποδιέγμασιν προελθόντας ὑποθεϊναι τὸν τράχηλον καὶ τὸν τῆς ὑπακοῆς τόπον ἀναπληρῶσαι, ὅπως ἡσυχάσαντες τῆς ματαίας στάσεως ἐπὶ τὸν προκείμενον ἡμῖν ἐν ἀληθείᾳ σκοπὸν δίχα παντὸς μώμου καταντήσωμεν. χαρὰν γὰρ καὶ ἀγαλλίασιν ἡμῖν παρέξετε, ἐὰν ὑπήοοι γενόμενοι τοῖς ὑφ’ ἡμῶν γεγραμμένοις διὰ τοῦ ἁγίου πνεύματος ἐκκόψητε τὴν ἀθέμιτον τοῦ ζήλους ὑμῶν ὀργὴν κατὰ τὴν ἔντευξιν, ἣν ἐποιησάμεθα περὶ εἰρήνης καὶ ὁμονοίας ἐν τῇδε τῇ ἐπιστολῇ. ἐπεμψσαμεν δὲ ἄνδρας πιστοὺς καὶ σώφρονας ἀπὸ νεότητος ἀναστραφέντας ἕως γήρου ἀμέμπτως ἐν ἡμῖν, οἵτινες καὶ μάρτθρες ἔσονται μεταξὺ ὑμῶν καὶ ἡμῶν. τοῦτο δὲ ἐποιήσαμεν, ἵνα εἰδῆτε, ὅτι πᾶσα ἡμῖν φροντὶς καὶ γέγονεν καὶ ἔστιν εἰς τὸ ἐ τάχει ὑμᾶς εἰρηνεῦσαι.
Λοιπὸν ὁ παντεπόπτης θεὸς καὶ δεσπότης τῶν πνευμάτων καὶ κυριος πασης σαρκός, ὁ ἐκλεξάμενος τὸν κύριον Ἰησοῦν Χριστὸν καὶ ἡμᾶς δι’ αὐτοῦ εἰς λαὸν περούσιον, δῴη πάσῃ ψυχῇ ἐπικεκλημένῃ τὸ μεγαλοπρεπὲς καὶ ἅγιον ὄνομα αὐτοῦ πίστιν, φόβον, εἰρηνην, ὑπομονὴν καὶ μακροθυμίαν, ἐγκράτειαν ἁγνείαν, σωφροσύνην, εἰς εὐαρέστησιν τῷ ὀνόματι αὐτοῦ διὰ τοῦ ἀρχιερέως καὶ προστάτου ἡμῶν Ἰησοῦ Χριστου, δι’ οὗ αὐτῷ δόξα καὶ μεγαλωσύνη, κράτος καὶ τιμή, καὶ νῦν καὶ εἰς πάντας τοῦς αἰῶνας τῶν αἰώνων. ἀμήν.
Τοὺς δὲ ἀπεσταλμένους ἀφ’ ἡμῶν Κλαύδιον Ἔφηβον καὶ Οὐαλέριον Βίτωνα σὺν καὶ Φορτουνάτῳ ἐν εἰρήνῃ μετὰ χαρᾶς ἐν τάχει ἀναπέμψατε πρὸς ἡμᾶς, ὅπως θᾶττον τὴν εὐκταίαν καὶ ἐπιποθητην ἡμῖν εἰρήνην καὶ ὁμόνοιαν ἀπαγγέλωσιν, εἰς τὸ τάχιον καὶ ἡμᾶς χαρῆναι περὶ τῆς εὐσταθείας ὑμῶν.
Ἡ χάρις τοῦ κυρίου ἡμῶν Ἰησοῦ Χριστοῦ μεθ’ ὑμῶν καὶ μετὰ πάντων πανταχη τῶν κεκλημένων ὑπὸ τοῦ θεοῦ δι’ αὐτοῦ, δι’ οὗ αὐτῷ δόξα, τιμή, κράτος καί μεγαλωσύνη, θρόνος αἰώνιος, ἀπὸ τῶν αἰώνων εἰς τοὺς αἰῶνας τῶν αιώνων. ἀμήν.Ἐπιστολὴ τῶν ῾Ρωμαίων πρὸς τοὺς Κορινθίους.
\section{ΚΛΗΜΕΝΤΟΣ ΠΡΟΣ ΚΟΡΙΝΘΙΟΥΣ Β}
Εὐφράνθητι, στεῖρα ἡ οὐ τίκτουσα, ῥῆξον καὶ βόησον, ἡ οὐκ ὠδίνουσα, ὅτι πολλὰ τὰ τέκνα τῆς ἐρήμου μᾶλλον ἢ τῆς ἐχούσης τὸν ἄνδρα. ὃ εἶπεν· Εὐφράνθητι, στεῖρα ἡ οὐ τίκτουσα, ἡμᾶς εἶπεν· στεῖρα γὰρ ἦν ἡ ἐκκλησία ἡμῶν πρὸ τοῦ δοθῆναι αὐτῇ τέκνα. ὃ δὲ εἶπεν· Βόησον, ἡ οὐκ ὠδίνουσα, τοῦτο λέγει· τας προσευχὰς ἡμῶν ἁπλῶς ἀναφέρειν πρὸς τὸν θεόν, μὴ ὡς αἱ ὠδίνουσαι ἐγκακῶμεν, ὃ δὲ εἶπεν· Ὅτι πολλὰ τὰ τέκνα τῆς ἐρήμου μᾶλλον ἢ τῆς ἐχούσης τὸν ἄνδρα· ἐπεὶ ἔρήμος ἐδόκει εἶναι ἀπὸ τοῦ θεοῦ ὁ λαὸς ἡμῶν, νυνὶ δὲ πιστεύσαντες πλείονες ἐγενόμεθα τῶν δοκούντων ἔχειν θεόν. καὶ ἑτέρα δὲ γραφὴ λέγει, ὅτι οὐκ ἦλθον καλέσαι δικαίους, ἀλλὰ ἁμαρτωλούς· τοῦτο λέγει, ὅτι δεῖ τοὺς ἀπολλυμένους σώζειν. ἐκεῖνο γάρ ἐστιν μέγα καὶ θαυμαστὸν οὐ τὰ ἑστῶτα στηρίζειν, ἀλλὰ τὰ πίπτοντα. οὕτως καὶ ὁ Χριστὸς ἠθέλησεν σῶσαι τὰ ἀπολλύμενα, καὶ ἔσωσεν πολλούς ἐλθὼν καὶ καλέσας ἡμᾶς ἤδη ἀπολλυμένους.
Τοσοῦτον οὖν ἔλεος ποιήσαντος αὐτοῦ εἰς ἡμᾶς, πρῶτον μέν, ὅτι ἡμεῖς οἱ ζῶντες τοῖς νεκροῖς θεοῖς οὐ θύομεν καὶ οὐ προσκυνοῦμεν αὐτοῖς, ἀλλὰ ἔγνωμεν δι’ αὐτοῦ τὸν πατέρα τῆς ἀληθείας· τίς ἡ γνῶσις ἡ πρὸς αὐτοῦ τὸν πατέρα τῆς ἀληθείας· τίς ἡ γνῶσις ἡ πρὸς αὐτόν, ἢ τὸ μὴ ἀρνεῖσθαι δι’ οὗ ἔγνωμεν αὐτόν; λέγει δὲ καὶ αὐτός· Τὸν ὁμολογήσαντά με ἐνώπιον τῶν ἀνθρώπων, ὁμολογήσω αὐτὸν ἐνώπιον τοῦ πατρός μου. οὗτος οὖν ἐστὶν ὁ μισθὸς ἡμῶν, ἐὰν οὖν ὁμολογήσωμεν δι’ οὗ ἐσώμεν. ἐν τίνι δὲ αὐτὸν ὁμολογοῦμεν; ἐν τῷ ποιεῖν ἃ λέγει καὶ μὴ παρακούειν αὐτοῦ τῶν ἐντολῶν, καὶ μὴ μόνον χείλεσιν αὐτὸν τιμᾶν, ἀλλὰ ἐξ ὅλης καρδίας καὶ ἐξ ὅλης τῆς διανοίας. λέγει δὲ καὶ ἐν τῷ Ἡσαΐα· Ὁ λαὸς οὗτος τοῖς χείλεσίν με τιμᾷ, ἡ δὲ καρία αὐτῶν πόρρω ἄπεστιν ἀπ’ ἐμοῦ.
Μὴ μόνον οὖν αὐτὸν καλῶμεν κύριον· οὐ γὰρ τοῦτο σώσει ἡμᾶς. λέγει γάρ· Οὐ πᾶς ὁ λέγων μοι· Κύριε, κύριε, σωθήσεται, ἀλλ’ ὁ ποιῶν τὴν κικαιοσύνην. ὥστε οὖν, ἀδελφοί, ἐν τοῖς ἔργοις αὐτὸν ὁμολογῶμεν, ἐν τῷ ἀγαπᾶν ἐαυτούς, ἐν τῷ μὴ μοιχᾶσθαι μηδὲ καταλαλεῖν ἀλλήλων μηδὲ ζηλοῦν, ἀλλ’ ἐγκρατεῖς εἶναι, ἐλεήμονας, ἀγαθούς· καὶ συμπάσχειν ἀλλήλοις ὀφείλομεν, καὶ μὴ φιλαργυρεῖν. ἐν τούτοις τοῖς ἔργοις ὁμολογῶμεν αὐτὸν καὶ μὴ ἐν τοῖς ἐναντίοις· καὶ οὐ δεῖ ἡμᾶς φοβεῖσθαι τοὺς ἀνθρώπους μᾶλλον, ἀλλὰ τὸν θεόν. διὰ τοῦτο, ταῦτα ὑμῶν πρασσόντων, εἶπεν ὁ κύριος· Ἐὰν ἦτε μετ’ ἐμοῦ συνηγμένοι ἐν τῷ κόλπῳ μου καὶ μὴ ποιῆτε τὰς ἐντολάς μου, ἀποβαλῶ ὑμᾶς καὶ ἐρῶ ὑμῖν· Ὑπάγετε ἀπ’ ἐμου, οὐκ οἶδα ὑμᾶς, πόθεν ἐστέ, ἐργάται ἀνομίας.
Ὅθεν, ἀδελφοί, καταλείψαντες τὴν παροικίαν τοῦ κόσμου τούτου ποιησωμεν τὸ θέλημα τοῦ καλέσαντος ἡμᾶς, καὶ μὴ φοβηθῶμεν ἐξελθεῖν ἐκ τοῦ κόσμου τούτου. λέγει γὰρ ὁ κύριος Ἔσεσθε ὡς ἀρνία ἐν μέσῳ λύκων. ἀποκριθεὶς δὲ ὁ Πέτρος αὐτῷ λέγει· Ἐὰν οὖν διασπαράξωσιν οἱ λύκοι τὰ ἀρνία; εἶπεν ὁ Ἰησοῦς τῷ Πέτρῳ· Μῃ φοβείθωσαν τὰ ἀρνία τοὺς λύκους μετὰ τὸ ἀποθανεῖν αὐτά· καὶ ὑμεῖς μὴ φοβεῖσθε τοὺς ἀποκτέννοντας ὑμᾶς καὶ μηδὲν ὑμῖν δυναμένους ποιεῖν, ἀλλὰ φοβεῖσθε τὸν μετὰ το ἀποθανεῖν ὑμᾶς ἔχοντα ἐξουσίαν ψυχῆς καὶ σώματος τοῦ βαλεῖν εἰς γέενναν πυρός. καὶ γινώσκετε, ἀδελφοί, ὅτι ἡ ἐπιδημία ἡ ἐν τῷ κόσμῳ τούτῳ τῆς σαρκὸς ταύτης μικρά ἐστιν καὶ ὀλιγοχρόνιος, ἡ δὲ ἐαπγγελία τοῦ Χριστοῦ μεγάλη καὶ θαυμαστή ἐστιν, καὶ ἀνάπαυσις τῆς μελλούσης βασιλείας καὶ ζωῆς αἰωνίου. τί οὖν ἐστὶν ποιήσαντας ἐπιτυχεῖν αὐτῶν, εἰ μὴ τὸ ὁσίως καὶ δικαίως ἀναστρέφεσθαι καὶ τὰ κοσμικὰ ταῦτα ὡς ἀλλότρια ἡγεῖσθαι καὶ μὴ ἐπιθυμεῖν αὐτῶν; ἐν γὰρ τῷ ἐπιθυμεῖν ἡμᾶς κτήσασθαι ταῦτα ἀποπίπτομεν τῆς ὁδοῦ τῆς δικαίας.
Λέγει δὲ ὁ κύριος· Οὐδεὶς οἰκέτης δύναται δυσὶ κυρίοις δουλεύειν. ἐὰν ἡμεῖς θέλωμεν καὶ θεῷ δουλεύειν καὶ μαμωνᾷ, ἀσύμφορον ἡμῖν ἐστίν. τί γὰρ τὸ ὄφελος, ἐάν τις τὸν κόσμον ὅλον κερδήσῃ, τὴν δὲ ψυχὴν ζημιωθῇ; ἔστιν δὲ οὗτος ὁ αἰὼν καὶ ὁ μέλλων δύο ἐχθροί. οὗτος λέγει μοιχείαν καὶ φθορὰν καὶ φιλαργυρίαν καὶ ἀπάτην, ἐκεῖνος δὲ τούτοις ἀποτάσσεται. οὐ δυνάμεθα οὖν τῶν δύο φίλοι εἶναι· δεῖ δὲ ἡμᾶς τούτῳ ἀποταξαμένους ἐκείνῳ χρᾶσθαι. οἰόμεθα, ὅτι βέλτιόν ἐστιν τὰ ἐνθάδε μισῆσαι, ὅτι μικρὰ καὶ ὀλιγοχρόνια καὶ φθαρτά, ἐκεῖνα δὲ ἀγαπῆσαι, τὰ ἀγαθὰ τὰ ἄφθαρτα. ποιοῦντες γὰρ τὸ θέλημα τοῦ Χριστοῦ εὑρήσομεν ἀνάπαυσιν· εἰ δὲ μήγε, οὐδὲν ἡμᾶς ῥύσεται ἐκ τῆς αἰωνίου κολάσεως,ἐὰν παρακούσωμεν τῶν ἐντολῶν αὐτοῦ. λέγει δὲ καὶ ἡ γραφὴ ἐν τῷ Ἰεζεκιήλ, ὅτι ἐὰν ἀναστῇ Νῶε καὶ Ἰωβ καὶ Δανιήλ, οὐ ῥύσονται τὰ τέκνα αὐτῶν ἐν τῇ αἰχμαλωσίᾳ. εἰ δὲ καὶ οἱ τοιοῦτοι δίκαιοι οὐ δύνανται ταῖς ἑαυτῶν δικαιοσύναις ῥύσασθαι τὰ τέκνα αὐτῶν, ἡμεῖς, ἐὰν μὴ τηρήσωμεν τὸ βάπτισμα ἁγνὸν καὶ ἀμίαντον, ποίᾳ πεποιθήσει εἰσελευσόμεθα εἰς τὸ βασίλειον τοῦ θεοῦ; ἢ τίς ἡμῶν παράκλητος ἔσται, ἐὰν μὴ εὑρεθῶμεν ἔργα ἔχοντες ὅσια καὶ δίκαια;
Ὥστε οὖν, ἀδελφοί μου, ἀγωισώμεθα εἰδότες, ὅτι ἐν χερσὶν ὁ ἀγὼν καὶ ὅτι εἰς τοὺς φθαρτοὺς ἀγῶνας καταπλέουσιν πολλοί, ἀλλ’ οὐ πάντες στεφανοῦνται, εἰ μὴ οἱ πολλὰ κοπιάσαντες καὶ καλῶς ἀγωνισάμενοι. ἡμεῖς οὖν ἀγωνισώμεθα, ἵνα πάνατες στεφανωθῶμεν. ὥστε θέωμεν τὴν ὁδὸν τὴν εὐθείαν, ἀγῶνα τὸν ἄφθαρτον, καὶ πολλοὶ εἰς στεφανωθῶμεν· καὶ εἰ μὴ δυνάμεθα πάντες στεφανωθῆναι, κἂν ἐγγὺς τοῦ στεφαν́νου γενώμεθα. εἰδέναι ἡμᾶς δεῖ, ὅτι ὁ τὸν φθαρτὸν ἀγῶνα ἀγωνιζόμενος, ἐὰν εὑρεθῇ φθείρων, μαστιγωθεὶς αἴρεται καὶ ἔξω βάλλεται τοῦ σταδίου. τί δοκεῖτε; ὁ τὸν τῆς ἀφθαρσίας ἀγῶνα φθείρας τί παθεῖται; τῶν γὰρ μὴ τηρησάντων, φησίν, τὴν σφραγῖδα ὁ σκώληξ αὐτῶν οὐ τελευτήσει καὶ τὸ πῦρ αὐτῶν οὐ σβεσθήσεται, καὶ ἔσονται εἰς ὅρασιν πάσῃ σαρκί.
Ὡς οὖν ἐσμὲν ἐπὶ γῆς, μετανοήσωμεν. πηλὸς γάρ ἐσμεν εἰς τὴν χεῖρα τοῦ τεχνίτου· ὃν τρόπον γὰρ ὁ κεραμεύς, ἐὰν ποιῇ σκεῦος καὶ ἐν ταῖς χερσὶν αὐτοῦ διαστραφῇ ἢ συντριβῇ, πάλιν αὐτὸ ἀναπλάσσει, ἐὰν δὲ προφθάσῃ εἰς τὴν κάμινον τοῦ πυρὸς αὐτὸ βαλεῖν, οὐκέτι βοηθήσει αὐτῷ· οὕτως καὶ ἡμεῖς, ἕως ἐσμὲν ἐν τούτῳ τῷ κόσμῳ, ἐν τῇ σαρκὶ ἃ ἐπράξαμεν πονηρὰ μετανοήσωμεν ἐξ ὅλης τῆς καρδίας, ἵνα σωθῶμεν ὑπὸ τοῦ κυρίου, ἕως ἔχομεν καιρὸν μετανοίας. μετὰ γὰρ τὸ ἐξελθεῖν ἡμᾶς ἐκ τοῦ κόσμου οὐκέτι δυνάμεθα ἐκεῖ ἐξομολογήσασθαι ἢ μεταμοεῖν ἔτι. ὥστε, ἀδελφοί, ποιήσαντες καὶ τὰς ἐντολὰς τοῦ κυρίου φυλάξαντες ληψόμεθα ζωὴν αἰώνιον. λέγει γὰρ ὁ κύριος ἐν τῷ εὐαγγελίῳ· Εἰ τὸ μικρὸν οὐκ ἐτηρήσατε, τὸ μέγα τίς ὑμῖν δώσει; λέγω γὰρ ὑμῖν, ὅτι ὁ πιστὸς ἐν ἐλαχίστῳ καὶ ἐν πολλῷ πιστός ἐστιν. ἆρα οὖν τοῦτο λέγει· τηρήσατε τὴν σάρκα ἁγνὴν καὶ τὴν σφραγῖδα ἄσπιλον, ἵνα τὴν αἰώνιον ζωὴν ἀπολάβωμεν.
Καὶ μὴ λεγέτω τις ὑμῶν, ὅτι αὕτη ἡ σὰρξ οὖ κρίνεται οὐδὲ ἀνίταται. γνῶτε· ἐν τίνι ἐσώθητε, ἐν τίνι ἀνελέψατε, εἰ μὴ ἐν τῇ σαρκὶ ταύτῃ ὄντες; δεῖ οὖν ἡμᾶς ὡς ναὸν θεοῦ φυλάσσειν τὴν σαπρκα· ὃν τρόπον γὰρ ἐν τῇ σαρκὶ ἐκλήθητε, καὶ ἐν τῇ σαρκὶ ἐλεύσεσθε. εἰ Χριστός, ὁ κύριος ὁ σώσας ἡμᾶς, ὢν μὲν τὸ πρῶτον πνεῦμα, ἐγένετο σὰρξ καὶ οὕτως ἡμᾶς ἐκάλεσεν· οὕτως καὶ ἡμεῖς ἐ ταύτῃ τῇ σαρκὶ ἀποληψόμεθα τὸν μισθόν. ἀγαπῶμεν οὖν ἀλλήλους, ὅπως ἔλθωμεν´πάντες εἰς τὴν βασιλείαν τοῦ θεοῦ. ὡς ἔχομεν καιρὸν τοῦ ἰαθῆναι, ἐπιδῶμεν ἑαυτοὺς τῷ θεραπεύοντι θεῷ, ἀντιμισθίαν αὐτῷ διδόντες. ποίαν; τὸ μετανοῆσαι ἐξ εἰλικρινοῦς καρδίας. προγνώστης γάρ ἐστιν τῶν πάντων καὶ εἰδὼς ἡμῶν τὰ ἐν καρδίᾳ. 10. δῶμεν οὖν αὐτῷ αἶνον, μὴ ἀπὸ στόματος μόνον, ἀλλὰ καὶ ἀπὸ καρδίας, ἵνα ἡμᾶς προσδέξηται ὡς υἱούς. 11. καὶ γὰρ εἶπεν ὁ κύριος· Ἀδελφοί μου οὗτοί εἰσιν οἱ ποιοῦντες τὸ θέλημα τοῦ πατρός μου.
Ὥστε, ἀδελφοί μου, ποιήσωμεν τὸ θέλημα τοῦ πατρὸς τοῦ καλέσαντος ἡμᾶς, ἵνα ζήσωμεν, καὶ διώξωμεν μᾶλλον τὴν ἀρετήν, τὴν δὲ κακίαν καταλείψωμεν ὡς προοδοιπόρον τῶν ἁμαρτιῶν ἡμῶν, καὶ φύγωμεν τὴν ἀσέβειαν, μὴ ἡμᾶς καταλάβῃ κακά. ἐὰν γὰρ σπουδάσωμεν ἀγαθοποιεῖν, διώξεται ἡμᾶς εἰρήνη. διὰ ταύτην γὰρ τὴν αἰτίαν οὐκ ἔστιν εὑρεῖν ἄνθρωπον, οἵτινες παράγουσι φόβους ἀνθρωπίνους, προῃρημένοι μᾶλλον τὴν ἐνθάδε ἀπόλαυσιν ἢ τὴν μέλλουσαν ἐπαγγελίαν. ἀγνοοῦσιν γὰρ ἡλίκην ἔχει βάσανον ἡ ἐνθάδε ἀπόλαυσις, καὶ οἵαν τρυφὴν ἔχει ἡ μέλλουσα ἐπαγγελία. καὶ εἰ μὲν αὐτοὶ μόνοι ταῦτα ἔπρασσον, ἀνεκτὸν ἦν· νῦν δὲ ἐπιμένουσιν κακοδιδασκαλοῦντες τὰς ἀναιτίους ψυχάς, οὐκ εἰδότες, ὅτι δισσὴν ἕξουσιν τὴν κρίσιν, αὐτοί τε καὶ οἱ ἀκούοντες αὐτῶν.
Ἡμεῖς οὖν ἐν καθαρᾷ καρδίᾳ δουλεύσωμεν τῷ θεῷ, καὶ ἐσόμεθα δίκαιοι· ἐὰν δὲ μὴ δουλεύσωμεν διὰ τὸ μὴ πιστεύειν ἡμᾶς τῇ ἐπαγγελίᾳ τοῦ θεοῦ, ταλαίπωροι ἐσόμεθα. λέγει γὰρ καὶ ὁ προφητικὸς λόγος· Ταλαίπωροί εἰσιν οἱ δίψυχοι, οἱ διστάζοντες τῇ καρδίᾳ, οἱ λέγοντες· Ταῦτα πάλαι ἠκούσαμεν καὶ ἐπὶ τῶν πατέρων ἡμῶν, ἡμεῖς δὲ ἡμέραν ἐξ ἡμέρας προσδεχόμενοι οὐδὲν τούτων ἑωράκαμεν. ἀνόητοι, συμβάλετε ἑαυτοὺς ξύλῳ· λάβετε ἄμπελον· πρῶτον μὲν φυλλοροεῖ, εἶτα βλαστὸς γίνεται, μετὰ ταῦτα ὄμφαξ, εἶτα σταφυλὴ παρεστηκυῖα. οὕτως καὶ ὁ λαός μου ἀκαταστασίας καὶ θλίψεις ἔσχεν· ἔπειτα ἀπολήψεται τὰ ἀγαθά. ὥστε, ἀδελφοί μου, μὴ διψυχῶμεν, ἀλλὰ ἐλπίσαντες ὑπομείνωμεν, ἵνα καὶ τὸν μισθὸν κομισώμεθα. πιστὸς γάρ ἐστιν ὁ ἐπαγγειλάμενος τὰς ἀντιμισθιας ἀποδιδόναι ἑκάστῳ τῶν ἔργων αὐτοῦ. ἐὰν οὖν ποιήσωμεν τὴν δικαιοσύνην ἐναντίον τοῦ θεοῦ, εἰσήξομεν εἰς τὴν βασιλείαν αὐτοῦ καὶ ληψόμεθα τὰς ἐπαγγελίας, ἃς οὖς οὐκ ἤκουσεν οὐδὲ ὀφθαλμὸς εἶδεν, οὐδὲ ἐπὶ καρδίαν ἀνθρώπου ἀνέβη. 
Ἐκδεχόμεθα οὖν καθ’ ὥραν τὴν βασιλείαν τοῦ θεοῦ ἐν ἀγάπῃ καὶ δικαιοσύνῃ, ἐπειδὴ οὐκ οἴδαμεν τὴν ἡμέραν τῆς ἐπιφανείας τοῦ θεου. ἐπερωτηθεὶς γὰρ αὐτὸς ὁ κύριος ὑπό τινος, πότε ἥξει αὐτοῦ ἡ βασιλεία, εἶπεν· Ὅταν ἔσται τὰ δύο ἕν, καὶ τὸ ἔξω ὡς τὸ ἔσω, καὶ τὸ ἄρσεν μετὰ τῆς θηλείας οὔτε ἄρσεν οὔτε θῆλυ. τὰ δύο δὲ ἕν ἐστιν, ὅταν λαλῶμεν ἑαυτοῖς ἀλήθειαν καὶ ἐν δυσὶ σώμασιν ἀνυποκρίτως εἴη μία ψυχή, καὶ τὸ ἔξω ὡς τὸ ἔσω, τοῦτο λέγει· τὴν ψυχὴν λέγει τὸ ἔσω, τὸ δὲ ἔξω τὸ σώμα λέγει· ὃν τρόπον οὖν σου τὸ σῶμα φαίνεται, οὕτως καὶ ἡ ψυχή σου δῆλος ἔστω ἐν τοῖς καλοῖς ἔργοις.  καὶ τὸ ἄρσεν μετὰ τῆς θηλείας, οὔτε ἄρσεν οὔτε θῆλυ, ταῦτο λέγει· ἵνα ἀδελφὸς ἰδὼν ἀδελφὴν οὐδὲν φρονῇ περὶ αὐτῆς θηλυκόν, μηδ̀ φρονῇ τι περὶ αὐτοῦ ἀρσενικόν. ταῦτα ὑμῶν ποιούτων, φησίν, ἐλεύσεται ἡ βασιλεία τοῦ πατρός μου.
Ἀδελφοὶ οὖν, ἤδη ποτὲ μετανήσωμεν, νήψωμεν ἐπὶ τὸ ἀγαθόν· μεστοὶ γάρ ἐσμεν πολλῆς ἀνοίας καὶ´πονηρίας. ἐξαλείψωμεν ἀφ’ ἡμῶν τὰ πρότερα ἁμαρτήματα καὶ μετανοήσαντες ἐκ ψυχῆς σωθῶμεν, καὶ μὴ γινώμεθα ἀνθρωπάρεσκοι μηδὲ θέλωμεν μόνον ἑαυτοῖς ἀρέσκειν, ἀλλὰ καὶ τοῖς ἔξω ἀνθρώποις ἐπὶ τῇ δικαιοσύνῃ, ἵνα τὸ ὄνομα δι’ ἡμᾶς μὴ βλασφημῆται. λέγει γὰρ ὁ κύριος· Διὰ παντὸς τὸ ὀνομά μου βλασφημεῖται ἐν πᾶσιν τοῖς ἔθνεσιν, καὶ πάλιν· Οὐαὶ δι’ ὃν βλασφημεῖται τὸ ὀνομά μου. ἐν τίνι βλασφημεῖται; ἐν τῷ μὴ ποιεῖν ὑμᾶς ἃ βούλομαι. τὰ ἔθνη γὰρ ἀκούοντα ἐκ τοῦ στόματος ἡμῶν τὰ λόγια τοῦ θεοῦ ὡς καλὰ καὶ μεγάλα θαυμάζει· ἔπειτα καταματων ὧν λέγομεν, ἔνθεν εἰς βλασφημίαν τρέπονται, λέγοντες εἶναι μῦθόν τινα καὶ πλάνην. ὅταν ἀλλὰ χάρις ὑμῖν, εἰ ἀγαπᾶτε τοὺς ἐχθροὺς καὶ τοὺς μισοῦντας ὑμᾶς· ταῦτα ὅταν ἀκούσωσιν, θαυμάζουσιν τὴν ὑπερβολὴν τῆς ἀγαθότητος· ὅταν δὲ ἴδωσιν, ὅτι οὐ μόνον τοὺς μισοῦντας οὐκ ἀγαπῶμεν, ἀλλ’ ὅτι οὐδὲ τοὺς ἀγαπῶντας, καταγελῶσιν ἡμῶν, καὶ βλασφημεῖται τὸ ὄνομα.Ὥστε, ἀδελφοί, ποιοῦντες τὸ θέλημα τοῦ πατρὸς ἡμῶν θεοῦ ἐσόμεθα ἐκ τῆς ἐκκλησίας τῆς πρώτης, τῆς πνευματικῆς, τῆς πρὸ ἡλίου καὶ σελήνης ἐκτισμένης. ἐὰν δὲ μὴ ποιήσωμεν τὸ θέλημα κυρίου, ἐσόμεθα ἐκ τῆς γραφῆς τῆς λεγούσης· Ἐγενήθη ὁ οἶκός μου σπήλαιον λῃστῶν. ὥστε οὖν αἱρετισώμεθα ἀπὸ τῆς ἐκκλησίας τῆς ζωῆς εἶναι, ἵνα σωθῶμεν. οὖκ οἴομαι δὲ ὑμᾶς ἀγνοεῖν, ὅτι ἐκκλησία ζῶσα σῶμά ἐστιν Χριστοῦ· λέγει γὰρ ἡ γραφή· Ἐποίησεν ὁ θεὸς τὸν ἄνθρωπον ἄρσεν καὶ θῆλυ· τὸ ἄρσεν ἐστὶν ὁ Χριστός, τὸ θῆλυ ἡ ἐκκλησία· καὶ ἔτι τὰ βιβλία καὶ οἱ ἀπόστολοι τὴν ἐκκλησίαν οὐ νῦν εἶνι λέγουσιν ἀλλὰ ἄνωθεν. ἦν γὰρ πνευματική, ὡς καὶ ὁ Ἰησοῦς ἡμῶν, ἐφανερώθη δὲ ἐπ’ ἐσχάτων τῶν ἡμέρῶν, ἵνα ἡμᾶς σώσῃ. ἡ ἐκκλησία δὲ πνευματικὴ οὖσα ἐφανερώθη ἐν τῇ σαρκὶ Χριστοῦ, δηλοῦσα ἡμῖν φθείρῃ, ἀπολήψεται αὐτὴν ἐν τῷ πνεύματι τῷ ἁγίῳ· ἡ γὰρ σὰρξ αὕτη ἀντίτυπός ἐστιν τοῦ πνεύματος· οὐδεὶς οὖν τὸ ἀντίτυπον φθείρας τὸ αὐθεντικὸν μεταλήψεται. ἄρα οὖν τοῦτο λέγει, ἀδελφοί· τηρήσατε τὴν σάρκα, ἵνα τοῦ πνεύματος μεταλάβητε. εἰ δὲ λέγομεν εἶναι τὴν σάρκα τὴν ἐκκλησίαν καὶ τὸ πνεῦμα Χριστόν, ἄρα οὖν ὁ ὑβρίσας τὴν σάρκα ὕβρισεν τὴν ἐκκλησίαν. ὁ τοιοῦτος οὖν οὐ μεταλήψεται τοῦ πνεύματος, ὁ ἐστιν ὁ Χριστός. τοσαύτην δύναται ἡ σὰρξ αὕτη μεταλαβεῖν ζωὴν καὶ ἀφθαρίαν κολληθέντος αὐτῇ τοῦ πνεύματος τοῦ ἁγίου, οὐτε, ἐξειπεῖν τις δύναται οὔτε λαλῆσαι ἃ ἡτοίμασεν ὁ κύριος τοῖς ἐκλεκτοῖς αὐτοῦ. 
Οὐκ οἴομαι δέ, ὅτι μικρὰν συμβουλίαν ἐποιησάμην περὶ ἐγκρατείας, ἣν ποιήσας τις οὐ μετανοήσει, ἀλλὰ καὶ ἑαυτὸν σώσει κἀμὲ τὸν συμβουλεύσαντα. μισθὸς γὰρ οὔκ ἐστιν μικρὸς πλανωμένην ψυχὴν καὶ ἀπολλυμένην ἀποστρέψαι εἰς τὸ σωθῆναι. ταύτην γὰρ ἔχομεν τὴν ἀντιμισθίαν ἀποδοῦναι τῷ θεῷ τῷ κτίσαντι ἡμᾶς, ἐὰν ὁ λέγων καὶ ἀκούων μετὰ πίστεως καὶ ἀγάπης καὶ λέγῃ καὶ ἀκούῃ. ἐμμείνωμεν οὖν ἐφ’ οἷς ἐπιστεύσαμεν δίκαιοι καὶ ὅσιοι, ἵνα μετὰ παρρησίας αἰτῶμεν τὸν θεὸν τὸν λέγοντα· Ἔτι λαλοῦντός σου ἐρῶ· ἰδοὺ πάρειμι. τοῦτο γὰρ τὸ ῥῆμα μεγάλης ἐστιν ἐπαγγελίας σημεῖον· ἑτοιμότερον γὰρ ἑαυτὸν λέγει ὁ κύριος εἰς τὸ διδόναι τοῦ αἰτοῦντος. τοσαύτης οὖν χρηστότητος μεταλαμβάνοντες μὴ φθονήσωμεν ἑαυτοῖς τυχεῖν τοσούτων ἀγαθῶν. ὅσην γὰρ ἡδονὴν ἔχει τὰ ῥήματα ταῦτα τοῖς ποιήσασιν αὐτά, τοσαύτην κατάκρισιν ἔχει τοῖς παρακούσασιν.
Ὥστε, ἀδελφοί, ἀφορμὴν λαβόντες οὐ μιδρὰν εἰς τὸ μεταμοῆσαι, καιρὸν ἔχοντες ἐπιστρέψωμεν ἐπὶ τὸν καλέσαι ἡμᾶς θεόν, ἕως ἔτι ἔχομεν τὸν παραδεχόμενον ἡμᾶς. ἐὰν γὰρ ταῖς ἡδυπαθείαις ταύταις ἀποταξώμεθα καὶ τὴν ψυχὴν ἡμῶν νικήσωμεν ἐν τῷ μὴ ποιεῖν τὰς ἐπιθυμίας αὐτῆς τὰς πονηράς, μεταληψόμεθα τοῦ ἐλέους Ἰησοῦ. γινώσκετε δέ, ὅτι ἔρχεται ἤδη ἡ ἡμέρα τῆς κρίσεως ὡς κλίβανος καιόμενος, καὶ τακήσονταί τινες τῶν οὐρανῶν καὶ πᾶσα ἡ γῆ ὡς μόλιβος ἐπὶ πυρὶ τηκόμενος· καὶ τότε φανήσεται τὰ κρύφια καὶ φανερὰ ἔργα τῶν ἀνθρώπων. καλὸν οὖν ἐλεημοσύνη ὡς μετάνοια ἁμαρτίας· κρείσσων νηστεία προσευχῆς, ἐλεημοσύνη δὲ ἀμφοτέρων· ἀγάπη δὲ καλύπτει πλῆθος ἁμαρτιῶν, προσευχὴ δὲ ἐκ καλῆς συνειδήσεως ἐκ θανάτου ῥύεται. μακάριος πᾶς ὁ εὑρεθεὶς ἐν τούτοις πλήρης· ἐλεημοσύνη γὰρ κούφισμα ἁμαρτίας γίνεται. 
Μετανοήσωμεν οὖν ἐξ ὅλης καρδίας, ἵνα μή τις ἡμῶν παραπόληται. εἰ γὰρ ἐντολὰς ἔχομεν, ἵνα καὶ τοῦτο πράσσωμεν, ἀπὸ τῶν εἰδώλων ἀποσπᾶν καὶ κατηχεῖν, πόσῳ μᾶλλον ψυχὴν ἤδη γινώσκουσαν τὸν θεὸν οὐ δεῖ ἀπόλλυσθαι; συλλάβωμεν οὖν ἑαυτοῖς καὶ τοὺς ἀσθενοῦντας ἀνάγειν περὶ τὸ ἀγαθόν, ὅπως σωθῶμεν ἅπαντες καὶ ἐπιστρέψωμεν ἀλλήλους καὶ νουθετήσωμεν. καὶ μὴ μόνον ἄρτι δοκῶμεν πιστεύειν καὶ προσέχειν ἐν τῷ νουθετεῖσθαι ἡμᾶς ὑπὸ τῶν πρεσβυτέρων, ἀλλὰ καὶ ὅταν εἰς οἶκον ἀπαλλαγῶμεν, μνημονηεύωμεν τῶν τοῦ κυρίου ἐνταλμάτων καὶ μὴ ἀντιπαρελκώμεθα ἀπὸ τῶν κοσμικῶν ἐπιθυμιῶν, ἀλλὰ πυκνότερον προσερχόμενοι πειρώμεθα προκόπτειν ἐν ταῖς ἐντολαῖς τοῦ κυρίου, ἵνα´πάντες τὸ αὐτὸ φρονοῦντες συνηγμένοι ὦμεν ἐπὶ τὴν ζωήν· εἶπεν γὰρ ὁ κύριος· Ἔρχομαι συναγαγεῖν πάντα τὰ ἔθθνη, φυλὰς καὶ γλώσσας· τοῦτο δὲ λέγει τὴν ἡμέραν τῆς ἐπιφανείας αὐτοῦ, ὅτε ἐλθὼν λυτρώσεται ἡμᾶς, ἕκαστον κατὰ τὰ ἔργα αὐτοῦ. καὶ ὄψονται τὴν δόξαν αὐτοῦ καὶ τὸ κράτος οἱ ἄπιστοι, και ξενισθήσονται ἰδόντες τὸ βασίλειον τοῦ κόσμου ἐν τῷ Ἰησοῦ, λέγοντες· Οὐαὶ ἡμῖν, ὅτι σὺ ἦς, καὶ οὐκ ᾔδειμεν καὶ οὐκ ἐπιστεύομεν καὶ οὐκ ἐπειθόμεθα τοῖς πρεσβυτέροις τοῖς ἀναγγέλλουσιν ἡμῖν περὶ τῆς σωτηρίας ἡμῶν. καὶ ὁ σκώληξ αὐτῶν οὐ τελευτήσει καὶ τὸ πῦρ αὐτῶν οὐ σβεσθήσεται, καὶ ἔσονται εἰς ὅρασιν πάσῃ σαρκί. τὴν ἡμέραν ἐκείνην λέγει τῆς κρίσεως, ὅταν ὄψονται τοὺς ἐν ἡμῖν ἀσεβήσαντας καὶ παραλογισαμένους τὰς ἐντολὰς Ἰησοῦ Χριστοῦ. οἱ δὲ δίκαιοι εὐπραγήσαντες καὶ ὑπομείναντες τὰς βασάνους καὶ μισήσαντες τὰς ὑδυπαθείας τῆς ψυχῆς, ὅταν θεάσωνται τοὺς ἀστοχήσαντας καὶ ἀρνησαμένους διὰ τῶν λόγων ἢ διὰ τῶν ἔργων τὸν Ἰησοῦν, ὅπως κολάζονται δειναῖς βασάνοις πυρὶ ἀσβέστῳ ἔσονται δόξαν διδόντες τῷ θεῷ αὐτῶν λέγοντες, ὅτι ἔστα ἐλπὶς τῷ δεδουλευκότι θεῷ ἐξ ὅλης καρδίας.
Καὶ ἡμεῖς οὖν γενώμεθα ἐκ τῶν εὐχαρετούντων, δεδουλευκότων τῷ θεῷ, καὶ μὴ ἐκ τῶν κρινομένων ἀσεβῶν. καὶ γὰρ αὐτὸς πανθαμαρτωλὸς ὢν καὶ μήπω φυγὼν τὸν πειρασμόν, ἀλλ’ ἔτι ὢν ἐν μέσοις τοῖς ὀργάνοις τοῦ διαβόλου σπουδάζω τὴν δικαιοωύνην διώκειν, ὅπως ἰσχύσω κἂν ἐγγὺς αὐτῆς γενέσθαι, φοβούμενος τὴν κρίσιν τὴν μέλλουσαν.
Ὥστε, ἀδελφοὶ καὶ ἀδελφαί, μετὰ τὸν θεὸν τῆς ἀληθείας ἀναγινώσκω ὑμῖν ἔντευξιν εἰς τὸ προσέχειν τοῖς γεγραμμένοις, ἵνα καὶ ἑαυτοὺς σώσητε καὶ τὸν ἀναγινώσκοντα ἐν ὑμῖν. μισθὸν γὰρ αἰτῶ ὑμᾶς τὸ μετανοῆσαι ἐξ ὅλης καρδίας, σωτηρίαν ἑαυτοῖς καὶ ζωὴν διδόντας. τοῦτο γὰρ ποιήσαντες σκοπὸν πᾶσιν τοῖς νέοις θήσομεν, τοῖς βουλομένοις περὶ τὴν εὐσέβειαν καὶ τὴν χρηστότητα τοῦ θεοῦ φιλοπονεῖν. καὶ μὴ ἀηδῶς ἔχωμεν καὶ ἀγανακτῶμεν οἱ ἄσοφοι, ὅταν τις ἡμᾶς νουθετῇ καὶ ἐπιστρέφῃ ἀπὸ τῆς ἀδικίας εἰς τὴν δικαιοσύνην. ἐνίοτε γὰρ πονηρὰ πράσσοντες οὐ γινώσκομεν διὰ τὴν διψυχίαν καὶ ἀπιστίαν τὴν ἐνοῦσαν ἐν τοῖς στήθεσιν ἡμῶν, καὶ ἐσκοτίσμεθα τὴν διάνοιαν ὑπὸ τῶν ἐπιθυμιῶν τῶν ματαίων. πράξωμεν οὖν τὴν δικαιοφύνην, ἵνα εἰς τέλος σωθῶμεν. μακάριοι οἱ τούτοις ὑπακούοντες τοῖς προστάγμασιν· κἂν ὀλίγον χρόνον κακοπαθήσωσιν ἐν τῷ κόσμῳ τούτῳ, τὸν ἀθάνατον τῆς ἀναστάσεως καρπὸν τρυγήσουσιν. μὴ οὖν λυπείσθω ὁ εὐσεβής, ἐὰν ἐπὶ τοῖς νῦν χρόνος· ἐκεῖνος ἄνω μετὰ τῶν πατέρων ἀναβιώσας εὐφρανθήσεται εἰς τὸν ἀλύπητον αἰῶνα. 
Ἀλλὰ μηδὲ ἐκεῖνο τὴν διάνοιαν ὑμῶν ταρασσέτω, ὅτι βλέπομεν τοὺς ἀδίκους πλουτοῦντας καὶ στενοχωρουμένους τοὺς τοῦ θεοῦ δούλους. πιστεύωμεν οὖν, ἀδελφοὶ καὶ ἀδελφαί· θεοῦ ζῶντος πεῖραν ἀθλοῦμεν καὶ γυμναζόμεθα τῷ νῦν βίῳ, ἵνα τῷ μέλλοντι στεφανωθῶμεν. οὐδεὶς τῶν δικαίων ταχὺν καρπὸν ἔλαβεν, ἀλλ’ ἐκδέχεται αυτόν. εἰ γὰρ τὸν μισθὸν τῶν δικαίων ὁ θεὸς συντόμως ἀπεδίδου, εὐθέως ἐμπορίαν ἠσκοῦμεν καὶ οὐ θεοσέβειαν· ἐδοκοῦμεν γὰρ εἶναι δίκαιοι, οὐ τὸ εὐσεβές, ἀλλὰ τὸ κερδαλέον διώκοντες. καὶ διὰ τοῦτο θεία κρίσις ἔβλαψεν πνεῦμα μὴ ὂν δίκαιον, καὶ ἐβάρυνεν δεσμοῖς.Τῷ μόνῳ θεῷ ἀοράτῳ, πατρὶ τῆς ἀληθείας, τῷ ἐξαποστείλαντι ἡμῖν τὸν σωτῆρα καὶ ἀρχηγὸν τῆς ἀφθαρσίας, δι’ οὗ καὶ ἐφανέρωσεν ἡμῖν τὴν ἀλήθειαν καὶ τὴν ἐπουράνιον ζωήν, αὐτῷ ἡ δόξα εἰς τοὺς αἰῶνας τῶν αἰώνων. ἀμήν.Κλήμεντος πρὸς Κορινθίους ἐπιστολὴ β̅.
\section{ΠΟΙΜΗΝ}
Ὅρασις α’
Ὁ θρέψας με πεπρακέν με ῾Ρόδῃ τινὶ εἰς ῾Ρώμηνμετὰ πολλὰ ἔτη ταύτην ἀνεγνωρισάμην καὶ ἠρξάμην αὐτὴν ἀγαπᾶν ὡς ἀδελφήνμετὰ χρόνον τινὰ λουομένην εἰς τὸν ποταμὸν τὸν Τίβεριν εἶδον καὶ ἐπέδωκα αὐτῇ τὴν χεῖρα καὶ ἐξήγαγον αὐτὴν ἐκ τοῦ ποταμοῦταύτης οὖν ἰδὼν τὸ κάλλος διελογιζόμην ἐν τῇ καρδίᾳ μου λέγων· Μακάριος ἤμην, εἰ τοιαύτην γυναῖκα εἶχον καὶ τῷ κάλλει καὶ τῷ τρόπῳμόνον τοῦτο ἐβουλευσάμην, ἕτερον δὲ οὐδὲ ἕνμετὰ χρόνον τινα πορευομένου μου εἰς Κώμας καὶ δοξάζοντος τὰς κτίσεις τοῦ θεοῦ, ὡς μεγάλαι καὶ ἐκπρεπεῖς καὶ δυναταί εἰσιν, περιπατῶν ἀφύπνωσακαὶ πνεῦμά με ἔλαβεν καὶ ἀπήνεγκέ με δι’ ἀνοδίας τινός, δι’ ἧς ἄνθρωπος οὐκ ἐδύνατο ὁδεῦσαι· ἧν δὲ ὁ τόπος κρημνώδης καὶ ἀπερρηγὼς ἀπὸ τῶν ὑδάτωνδιαβὰς οὖν τὸν ποταμὸν ἐκεῖνον ἦλθον εἰς τὰ ὁμαλὰ καὶ τιθῶ τὰ γόνατα καὶ ἠρξάμην προσεύχεσθαι τῷ κυρίῳ καὶ ἐξομολογεῖσθαί μου τὰς ἁμαρτίαςπροσευχομένου δέ μου ἠνοίγη ὁ οὐρανός, καὶ βλέπω τὴν γυναῖκα ἐκείνην, ἣν ἐπεθύμησα, ἀσπαζομένην με ἐκ τοῦ οὐρανοῦ, λέγουσαν· Ἑρμᾶ χαῖρεβλέψας δὲ εἰς αὐτὴν λέγω αὐτῇ· Κυρία, τί σὺ ὧδε ποιεῖς; ἡ δὲ ἀπεκρίθη μοι· Ἀνελήμωθην, ἵνα σοῦ τὰς ἁμαρτίας ἐλεγχος πρὸς τὸν κύριονλέγω αὐτῇ· Νῦν σύ μου ἔλεγχος εἶ; Οὔ, φησίν, ἀλλὰ ἄκουσον τὰ ῥήματα, ἅ σοι μέλλω λέγεινὁ θεὸς ὁ ἐν τοῖς οὐρανοῖς κατοικῶν καὶ κτίσας ἐκ τοῦ μὴ ὄντος τὰ ὄντα καὶ πληθύνας καὶ αὐξήσας ἕνεκεν τῆς ἁγίας ἐκκλησίας αὐτοῦ ὀργίζεταί σοι, ὅτι ἥμαρτες εἰς ἐμέἀποκριθεὶς αὐτῇ λέγω· Εἰς σὲ ἥμαρτον; ποίῳ τόπῳ ἢ πότε σοι αἰσχρὸν ῥῆμα ἐλάλησα; οὐ πάντοτέ σε ὡς θεὰν ἡγησάμην; οὐ πάντοτέ σε ἐνετράπην ὡς ἀδελφήν; τί μου καταψεύδῃ, ὦ γύναι, τὰ πονηρὰ ταῦτα καὶ ἀκάθαρτα; γελάσασά μοι λέγει· Ἐπὶ τὴν καρδίαν σου ἀνέβη ἡ ἐπιθυμία τῆς πονηρίαςἢ οὐ δοκεῖ σοι ἀνδρὶ δικαίῳ πονηρὸν πρᾶγμα εἶναι, ἐὰν ἀναβῇ αὐτοῦ ἐπὶ τὴν καρδίαν ἡ πονηρὰ ἐπιθυμία; ἁμαρτία γέ ἐστιν, καὶ μεγάλη, φησίνὁ γὰρ δίκαιος ἀνὴρ δίκαια βουλεύεταιἐν τῷ οὖν δίκαια βουλεύεσθαι αὐτὸν κατορθοῦνται ἡ δόξα αὐτοῦ ἐν τοῖς οὐρανοῖς καὶ εὐκατάλλακτον ἔχει τὸν κύριον ἐν παντὶ πράγματι αὐτοῦ· οἱ δὲ πονηρὰ βουλευόμενοι ἐν ταῖς καρδίαις αὐτῶν θάνατον καὶ αἰχμαλωτισμὸν ἑαυτοῖς ἐπισπῶνται, μάλιστα οἱ τὸν αἰῶνα τοῦτον περιποιούμενοι καὶ γαυριῶντες ἐν τῷ πλούτῳ αὐτῶν καὶ μὴ ἀντεχόμενοι τῶν ἀγαθῶν τῶν μελλόντωνμετανοήσουσιν αἱ ψυχαὶ αὐτῶν, οἵτινες οὐκ ἔχουσιν ἐλπίδα, ἀλλὰ ἑαυτοὺς ἀπεγνώκασιν καὶ τὴν ζωὴν αὐτῶνἀλλὰ σὺ προσεύχου πρὸς τὸν θεόν, καὶ ἰάσεται τὰ ἁμαρτήματά σου καὶ ὅου τοῦ οἴκου σου καὶ πάτων τῶν ἁγίων.
Μετὰ τὸ λαλῆσαι αὐτὴν τὰ ῥήματα ταῦτα ἐκλείσθησαν οἱ οὐρανοί· κἀγῲ ὅλος ἤμην πεφρικὼς καὶ λυπούμενοςἔλεγον δὲ ἐν ἐμαυτῷ· Εἰ αὕτη μοι ἡ ἁμαρτία ἀναγράφεται, πῶς δυνήσομαι σωθῆναι; ἢ πῶς ἐξιλάσομαι τὸν θεὸν περὶ τῶν ἁμαρτιῶν μου τῶν τελείςν; ἢ ποίοις ῥήμασιν ἐρωτήσω τὸν κύριον, ἵνα ἱλατεύσηταί μοι; ταῦτά μου συμβουλευομένου καὶ διακρίνοντος ἐν τῇ κάρδίᾳ μου, βλέπω κατέναντί μου καθέδραν λευκὴν ἐξ ἐρίων χιονίνων γεγονυῖαν μεγάλην· καὶ ἦλθεν γυνὴ πρεσβῦτις ἐν ἱματισμῷ λαμπροτάτῳ, ἔχουσα βιβλίον εἰς τὰς χεῖρας, καὶ ἐκάθισεν μόνη καὶ ἀσπάζεταί με· Ἑρμᾶ, χαῖρε, κἀγὼ λυπούμενος καὶ κλαίων εἶπον· Κυρία, χαῖρεκαὶ εἶπέν μοι· Τί στυγνός, Ἑρμᾶ; ὁ μακρόθυμος καὶ ἀστομάχητος, ὁ πάντοτε γελῶν τί οὕτω κατηφὴς τῇ ἰδέᾳ καὶ οὐχ ἱλαρός; κἀγὼ εἶπον αὐτῇ· Ὑπὸ γυναικὸς ἀγαθωτάτης λεγούσης, ὅτι ἥμαρτον εἰς αὐτήνἡ δὲ ἔφη· Μηδαμῶς ἐπὶ τὸν δοῦλον τοῦ θεοῦ τὸ πρᾶγμα τοῦτοἀλλὰ πάντως ἐπὶ τὴν καρδίαν σου ἀνέβη περὶ αὐτῆςἔστιν μὲν τοῖς δούλοις τοῦ θεοῦ ἡ τοιαύτη βουλὴ ἁμαρτίαν ἐπιφέρουσα· πονηρὰ γὰρ βουλὴ καὶ ἔκπληκτος εἰς πάνσεμνον πνεῦμα καὶ ἤδη δεδοκιμασμένον, ἐὰν ἐπιθυμήσῃ πονηρὸν ἔργον, καὶ μάλιστα Ἑρμᾶς ὁ ἐγκρατής, ὁ ἀπεχόμενος´πάσης ἐπιθυμίας πονηρᾶς καὶ πλήρης πάσης ἁπλότητος καὶ ἀκακίας μεγάλης.
Ἀλλ’ οὐχ ἕνεκα τούτου ὀργίζεταί σοι ὁ θεός, ἀλλ’ ἵνα τὸν οἶκόν σου τὸν ἀνομήσαντα εἰς τὸν κύριον καὶ εἰς ὑμᾶς τοὺς γονεῖς αὐτῶν ἐπιστρέψῃςἀλλὰ φιλότεκνος ὢν οὐκ ἐνουθέτεις σου τὸν οἶκον, ἀλλὰ ἀφῆκες αὐτὸν καταφθαρῆναι, διὰ τοῦτό σοι οργίζεται ὁ κύριος· ἀλλὰ ἰάσεταί σου πάντα τὰ προγεγονότα πονηρὰ ἐν τῷ οἴκῳ σου· διὰ γὰρ τὰς ἐκείνων ἁμαρτίας καὶ ἀνομήματα σὺ κατεφθάρης ἀπὸ τῶν βιωτικῶν πράξεωνἀλλ’ ἡ πολυσπλαγχνία τοῦ κυρίου ἠλέησέν σε καὶ τὸν οἶκόν σου καὶ ἰσχυροποιήσει σε καὶ θεμελιώσει σε ἐν τῇ δόξῃ αὐτοῦσὺ μόνον μὴ ῥᾳθυμήσῃς, ἀλλὰ εὐψύχει καὶ ἰσχυροποίει σου τὸν οἶκονὡς γὰρ ὁ χαλκεὺς σφυροκοῶν τὸ ἔργον αὐτοῦ περιγίνεται τοῦ πράγματος οὗ θέλει, οὕτω καὶ ὁ λόγος ὁ καθημερινὸς ὁ δίκαιος περιγίνεται πάσης πονηρίαςμὴ διαλίπῃς οὖν νουθετῶν σου τὰ τέκναοἶδα γάρ, ὅτι, ἐὰν μετανοήσουσιν ἐξ ὅλης καρδίας αὐτῶν, ἐνγραφήσονται εἰς τὰς βίβλους τῆς ζωῆς μετὰ τῶν ἁγίωνμετὰ τὸ παῆναι αὐτῆς τὰ ῥήματα ταῦτα λέγει μοι· Θέλεις ἀκοῦσαί μου ἀναγινωσκούσης; λέγω κἀγώ· Θέλω, κυρίαλέγει μοι· Γενοῦ ἀκροατὴς καὶ ἄκουε τὰς δόξας τοῦ θεοῦἤκουσα μεγάλως καὶ θαυμαστῶς, ὃ οὐκ ἴσχυσα μνημονεῦσαι· πάντα γὰρ τὰ ῥήματα ἔκφρικτα, ἃ οὐ δύναται ἄνθρωπος βαστάσαιτὰ οὖν ἔσχατα ῥήματα ἐμνημόνευσα· ἦν γὰρ ἡμῖν σύμφορα καὶ ἥμερα· Ἰδού, ὁ θεὸς τῶν δυνάμεων, ὃ ἀγαπῶ, δυνάμει κραταιᾷ καὶ τῇ μεγάλῃ συνέσει αὐτοῦ κτίσας τὸν κόσνον καὶ τῇ ἐνδόξῳ βουλῇ περιθεὶς τὴν εὐπρέπειαν τῇ κτίσει αὐτοῦ καὶ τῷ ἰσχυρῷ ῥήματι πήξας τὸν οὐρανὸν καὶ θεμελώσας τὴν γῆν ἐπὶ ὑδάτων καὶ τῇ ἰδίᾳ σοφίᾳ καὶ προνοίᾳ κτίσας τὴν ἁγίαν ἐκκλησίαν αὐτοῦ, ἣν καὶ ηὐλόγησεν, ἰδού, μεθιστάνει τοὺς οὐρανούς, καὶ τὰ ὄρη καὶ τοὺς βουνοὺς καὶ τὰς θαλ́σσας, καὶ πάντα ὁμαλὰ γίνεται τοῖς ἐκλεκτοῖς αὐτοῦ, ἵνα ἀποδῷ αὐτοῖς τὴν ἐπαγγελίαν, ἣν ἐπηγγείλατο μετὰ πολλῆς δόξης καὶ χαρᾶς, ἐὰν τηρήσωσιν τὰ νόμιμα τοῦ θεοῦ, ἃ παρέλαβον ἐν μεγάλῃ πίστει.
Ὅτε οὖν ἐτέλεσεν ἀναγινώσκουσα καὶ ἠγέρθη ἀπὸ τῆς καθέδρας, ἦλθαν τέσσαρες νεανίαι καὶ ἦραν τὴν καθέδραν καὶ ἀπῆλθον πρὸς τὴν ἀνατολήνπροσκαλεῖται δέ με καὶ ἥψατο τοῦ στήθους μου καὶ λέγει μοι· Ἤρεσέν σοι ἡ ἀνάγνωσίς μου; καὶ λέγω αὐτῇ· Κυρία, ταῦτά μοι τὰ ἔσχατα ἀρέσκει, τὰ δὲ πρῶτα χαλεπὰ καὶ σκληράἡ δὲ ἔφη μοι λέγουσα· Ταῦτα τὰ ἔσχατα τοῖς ἀποστάταιςλαλούσης αὐτῆς μετ’ ἐμοῦ δύο τινὲς ἄνδρες ἐφάνησαν καὶ ἦραν αὐτῆς μετ’ ἐμοῦ δύο τινὲς ἄνδρες ἐφάνησαν καὶ ἦραν αὐτὴν τῶν ἀγκώνων καὶ ἀπῆλθαν, ὅου ἡ καθέδρα, πρὸς τὴν ἀνατολήνἱλαρὰ δὲ ἀπῆλθεν καὶ ὑπάγουσα λέγει μοι· Ἀνδρίζου, Ἑρμᾶ.

Ὅρασις β’
Πορευομένου μου εἰς Κώμας κατὰ τὸν καιρόν, ὃν καὶ πέρυσι, περιπατῶν ἀνεμνήσθην τῆς περυσινῆς ὁράσεως, καὶ πάλιν με αἵρει πνεῦμα καὶ ἀποφέρει εἰς τὸν αὐτὸν τόπον, ὅπου καὶ πέρυσιἐλθὼν οὖν εἰς τὸν τόπον τιθῶ τὰ γόνατα καὶ ἠρξάμην προσεύχεσθαι τῷ κυρίῳ καὶ δοξάζειν αὐτοῦ τὸ ὄνομα, ὅτι με ἄξιον ἡγήσατο καὶ ἐγνώρισέν μοι τὰς ἁμαρτίας μου τὰς πρότερονμετὰ δὲ τὸ ἐγερθῆναί με ἀπὸ τῆς προσευχῆς βλέπω ἀπέναντί μου τὴν πρεσβυτέραν, ἣν καὶ πέρυσεν ἑωράκειν, περιπατοῦσαν καὶ ἀναγινώσουσαν βιβλαρίδιον, καὶ λέγει μοι· Δύνῃ ταῦτα τοῖς ἐκλεκτοῖς τοῦ θεοῦ ἀναγγεῖλαι; λέγω αὐτῇ· Κυρία, τοςαῦτα μνημονεῦσαι οὐ δύναμαι· δὸς δέ μοι τὸ βλβλίδιον, ἵνα μεταγράψωμαι αὐτόΛάβε, φησίν, καὶ ἀποδώσεις μοιἔλαβον ἐγώ, καὶ εἴς τινα τόπον τοῦ ἀγροῦ ἀναχωρήσας μετεγραψάμην πάντα πρὸς γράμμα· οὐχ ηὕρισκον γὰρ τὰς συλλαβάςτελέσαντος οὖν τὰ γράμματα τοῦ βιβλιδίου ἐξαίφνης ἡρπάγη μου ἐκ τῆς χειρὸς τὸ βιβλίδιον· ὑπὸ τίνος δὲ οὐκ εἶδον.
Μετὰ δὲ δέκα καὶ πέντε ἡμέρας νηστεύσαντός μου καὶ πολλὰ ἐρωτήσαντος τὸν κύριον ἀπεκαλύφθη μοι ἡ γνῶσις τῆς γραφῆςἦν δὲ γεγραμμένα ταῦτα· Τὸ σπέρμα σου, Ἑρμᾶ, ἠθέτησαν εἰς τὸν θεὸν καὶ ἐβλασφήμησαν εἰς τὸν κύριον καὶ προέδωκαν τοὺς γονεῖς αὐτῶν ἐν πονηρίᾳ μεγάλῃ καὶ ἤκουσαν προδόται γονέων καὶ προδόντες οὐκ ὠφελήθησαν, ἀλλὰ ἔτι προσέθηκαν ταῖς ἁμαρτίαις αὐτῶν τὰς ἀσελγείας καὶ συμφυρμοὺς πονηρίας, καὶ οὕτως ἐπλήσθησαν αἱ ἀνομίαι αὐτῶνἀλλὰ γνώρισον ταῦτα τὰ ῥήματα τοῖς τέκνοις σου πᾶσιν καὶ τῇ συμβίῳ σου τῇ μελλούσῃ ἀδελφῇ· καὶ γὰρ αὕτη οὐκ ἀπέχεται τῆς γλώσσης, ἐν ᾗ πονηρεύεται· ἀλλὰ ἀκούσασα τὰ ῥήματα ταῦτα ἀφέξεται καὶ ἕξει ἔλεοςμετὰ τὸ γνωρίσαι σε ταῦτα τὰ ῥήματα αὐτοῖς, ἃ ἐντείλατό μοι ὁ δεσπότης ἵνα σοι ἀποκαλυφθῇ, τότε ἀφίενται αὐτοῖς αἱ ἁμαρτίαι πᾶσαι, ἃς πρότερον ἥμαρτον, καὶ πᾶσιν τοῖς ἁγίοις τοῖς ἁμαρτήσασιν μέχρι ταύτης τῆς ἡμέρας, ἐὰν ἐξ ὅλης τῆς καρδίας μετανοήσωσιν καὶ ἄρωςιν ἀπὸ τῆς καρδίας αὐτῶν τὰς διψυχίαςὤμοσεν γὰρ ὁ δεσπότης κατὰ τῆς δόξης αὐτοῦ ἐπὶ τοὺς ἐκλεκτοὺς αὐτοῦ· ἐὰν ὡρισμένης τῆς ἡμέρας ταύτης ἔτι ἁμάρτησις γένηται, μὴ ἔχειν αὐτοὺς σωτηρίαν· ἡ γὰρ μετάνοια τοῖς δικαίοις ἔχει τέλος· πεπλήρωνται αἱ ἡμέραι μενανοίας πᾶσιν τοῖς ἁγίοις· καὶ τοῖς δὲ ἔθνεσιν μετάνοιά ἐστιν ἕως ἐσχάτης ἡμέραςἐρεῖς οὖν τοῖς προηγουμένοις τῆς ἐκκλησίας, ἵνα κατορθώσωνται τὰς ὁδοὺς αὐτῶν ἐν δικαιοσύνῃ, ἵνα ἀπολάβωσιν ἐκ πλήρους τὰς ἐπαγγελίας μετὰ πολλῆς δόξηςἐμμείνατε οὖν οἱ ἐργαζόμενοι τὴν δικαιοσύνην καὶ μὴ διψυχήσητε, ἵνα γένηται ὑμῶν ἡ παροδος μετὰ τῶν ἀγγέλων τῶν ἁγίωνμακάριοοι ὑμεῖς, ὅσοι ὑπομένετε τὴν θλῖψιν τὴν ἐρχομένην τὴν μεγάλην καὶ ὅσοι οὐκ ἀρνήσονται τὴν ζωὴν αὐτῶνὤμοσεν γὰρ κύριος κατὰ τοῦ υἱοῦ ζωὴν αὐτοῦ, τοὺς ἀρνησαμένους τὸν Χριστὸν αὐτῶν ἀπεγνωρίσθαι ἀπὸ τῆς ζωῆς αὐτῶν, τοὺς νῦν μέλλοντας ἀρνεῖσθαι ταῖς ἐρχομέναις ἡμέραις· τοῖς δὲ πρότερον ἀρνησαμένοις, διὰ τὴν πολυσπλαγχνίαν ἵλεως ἐγένετο αὐτοῖς.
Σὺ δέ, Ἑρμᾶ, μηκέτι μνησικακήσῃς τοῖς τέκνοις σου μηδὲ τὴν ἀδελφήν σου ἐάσῃς, ἵνα καθαρισθῶσιν ἀπὸ τῶν προτέρων ἁμαρτιῶν αὐτῶνπαιδευθήσονται γὰρ παιδείᾳ δικαίᾳ, ἐὰν σὺ μὴ μνησικακήσῃς αὐτοῖςμνησικακία θάνατον κατεργάζεταισὺ δέ, Ἑρμᾶ, μεγάλας θλίψεις ἔσχες ἰδιωτικὰς διὰ τὰς παραβάσεις τοῦ οἴκου σου, ὅτι οὐκ ἐμέλησέν σοι περὶ αὐτῶν· ἀλλὰ παρενεθυμήθης καὶ ταῖς πραγματείαις σου συνανεφύρης ταῖς πονηραῖς· ἀλλὰ σώζει σε τὸ μὴ ἀποστῆναί σε ἀπὸ θεοῦ ζῶντος καὶ ἡ ἁπλοτης σου καὶ ἡ πολλὴ ἐγκράτεια· ταῦτα σέσωκέν σε, ἐὰν ἐμμείνῃς, καὶ πάντας σώζει τοὺς τὰ τοιαῦτα ἐργαζομένους καὶ πορευομένους ἐν ἀκακίᾳ καὶ ἁπλότητιοὗτοι κατισχύσουσιν πάσης πονηρίας καὶ παραμενοῦσιν εἰς ζωὴν αἰώνιονμακάριοι πάντες οἱ ἐργαζομενοι τὴν δικαιοσύνηνοὐ διαφθαρήσονται ἕως αἰῶνοςἐρεῖς δὲ Μαξίμῳ· ἄρνησαιἘγγὺς κύριος τοῖς ἐπιστρεφομένοις, ὡς γέγραπται ἐν τῷ Ἐλδὰδ καὶ Μωδάτ, τοῖς προφητεύσασιν ἐν τῇ ἐρήμῳ τῷ λαῷ.
Ἀπεκαλύφθη δέ μοι, ἀδελφοί, κοιμωμένῳ ὑπὸ νεανίσκου εὐειδεστάτου λέγοντός μοι· Τὴν πρεσβυτέραν, παρ’ ἧς ἔλαβες τὸ βιβλίδιον, τίνα δοκεῖς εἶναι; ἐγώ φημι· Τὴν ΣίβυλλανΠλανᾶσαι, φησίν, οὐκ ἔστινΤίς οὖν ἐστιν; φημίἩ Ἐκκλησία, φησίνεἶπον αὐτῷ· Διατί οὖν πρεσβυτέρα· καὶ διὰ ταύτην ὁ κόσμος κατηρτίσθημετέπειτα δὲ ὅρασιν εἶδον ἐν τῷ οἴκῳ μουἦλθεν ἡ πρεσβυτέρα καὶ ἠρώτησέν με, εἰ ἤδη τὸ βιβλίον δέδωκα τοῖς πρεσβυτέροιςἠρνησάμην δεδωκέναιΚαλῶς, φησίν, πεποίηκας· ἔχω γὰρ ῥήματα προσθεῖναιὅταν οὖν ἀποτελέσω τὰ ῥήματα πάντα, διὰ σοῦ γνωρισθήσεται τοῖς ἐκλεκτοῖς πᾶσινγράψεις οὖν δύο βιβλαρίδια καὶ πέμψεις ἓν Κλήμεντι καὶ ἓν Γραπτῇπέμψει οὖν Κλήμης εἰς τὰς ἔξω πόλεις, ἐκείνῳ γὰρ ἐπιτέτραπται· Γραπτὴ δὲ νουθετήσει τὰς χήρας καὶ τοὺς ὀρφανούςσὺ δὲ ἀναγνώσῃ εἰς ταύτην τὴν πόλιν μετὰ τὴν πόλιν μετὰ τῶν πρεσβυτέρων τῶν προϊταμένων τῆς ἐκκλησίας.

Ὅρασις γ’
Ἣν εἶδον, ἀδελφοί, τοιαύτηνηστεύσας πολλάκις καὶ δεηθεὶς τοῦ κυρίου, ἵνα μοι φανερώσῃ τὴν ἀποκάλυψιν, ἣν μοι ἐπηγγείλατο δεῖξαι διὰ τῆς πρεσβυτέρας ἐκείνης, αὐτῇ τῇ νυκτί μοι ὦπται ἡ πρεσβυτέρα καὶ εἶπέν μοι· Ἐπεὶ οὕτως ἐνδεὴς εἶ καὶ σπουδαῖος εἰς τὸ γνῶναι´πάτα, ἐλθὲ εἰς τὸν ἀγρόν, ὅου χονδρίζεις, καὶ περὶ ὥραν πέμπτην ἐμφανισθήσομαί σοι καὶ δείξω σοι, ἃ δεῖ σε ἰδεῖνἠρώτησα αὐτὴν λέγων· Κυρία, εἰς ποῖον τόπον τοῦ ἀγροῦ; Ὅπου, φησίν, θέλειςἐξελεξάμην τόπον καλὸν ἀνακεχωρηκόταπρὶν δὲ λαλῆσαι αὐτῇ καὶ εἰπεῖν τὸν τόπον, λέγει μοι· Ἥξω ἐκεῖ, ὅπου θέλεις, ἐγενόμην οὖν, ἀδελφοί, εἰς τὸν ἀγρὸν καὶ συεψήφισα τὰς ὥρας καὶ ἦλθον εἰς τὸν τόπον, ὅπου διεταξάμην αὐτῇ ἐλθεῖν, καὶ βλέπω συμψέλιον κείμενον ἐλεφάντινον, καὶ ἐπὶ τοῦ συμψελίου ἔκειτο κερβικάριον λινοῦν καὶ ἐπάνω λέντιον ἐξηπλωμένον λινοῦν καρπάσιονἰδὼν ταῦτα κείμενα καὶ μηδένα ὄντα ἐν τῷ τόπῳ ἔκθαμβος ἐγενόμην, καὶ ὡσεὶ τρόμος με ἔλαβεν καὶ αἱ τρίχες μου ὀρθαί· καὶ ὡσεὶ φρίκη μοι προσῆλθεν μόνου μου ὄντοςἐν ἐμαυτῷ οὖν γενόμενος καὶ μνησθεὶς τῆς δόξης τοῦ θεοῦ καὶ λαβὼν θάρσος, θεὶς τὰ γόνατα ἐξωμολογούμην τῷ κυρίῳ πάλιν τὰς ἁμαρτίας μου ὡς καὶ πρότερονἡ δὲ ἦλθεν μετὰ νεανίσκων ἕξ, οὓς καὶ πρότερον ἑωράκειν, καὶ ἐστάθη μοι καὶ κατηκροᾶτο προσευχομένου καὶ ἐξομολογουμένου τῷ κυρίῳ τὰς ἁμαρτίας μουκαὶ ἁψαμένη μου λέγει· Ἑρμᾶ, παῦσαι περὶ τῶν ἁμαρτιῶν σου πάντα ἐρωτῶν· ἐρώτα καὶ περὶ δικαιοσύνης, ἵνα λάβῃς μέρος τι ἐξ αὐτῆς εἰς τὸν οἶκον σουκαὶ ἐξεγείρει με τῆς χειρὸς καὶ ἄγει με πρὸς τὸ συμψέλιον καὶ λέγει τοῖς νεανίσκοις· Ὑπάγετε καὶ οἰκοδομεῖτεκαὶ κετὰ τὸ ἀναχωρῆσαι τοὺς νεανίσκους καὶ μόνων ἡμῶν γεγονότων λέγει μοι· Κάθισον ὧδελέγω αὐτῇ· Κυρία, ἄφες τοὺς πρεβυτέρους πρῶτον καθίσαιὍ σοι λέγω, φησίν, κάθισονθέλοντος οὖν μου καθίσαι εἰς τὰ δεξιὰ μέρη οὐκ εἴασέ με, ἀλλ’ ἐννεύει μοι τῇ χειρί, ἵνα εἰς τὰ ἀριστερὰ μέρη καθίσωδιαλογιζομένου μου οὖν καὶ λυπουμένου, ὅτι οὐκ εἴασέ Ἑρμᾶ; ὁ εἰς τὰ δεξιὰ μέρη τόπος ἄλλων ἐστίν, τῶν ἤδη εὐαρεστηκότων τῷ θεῷ καὶ παθόντων εἵνεκα τοῦ ὀνόματος· σοὶ δὲ πολλὰ λείπει ἵνα μετ’ αὐτῶν καθίσῃς· ἀλλὰ σὸς μένεις τῇ ἁπλότητί σου, μεῖνον, καὶ καθιῇ μετ’ αὐτῶν καὶ ὅσοι ἐὰν ἐργάσωνται τὰ ἐκείνων ἔργα καὶ ὑενέγκωσιν, ἃ καὶ ἐκεῖνοι ὑπήνεγκαν
Τί, φημί, ὐπήνεγκαν; Ἄκουε, φησίν· μάστιγας, φυλακάς, θλίψεις μεγάλας, σταυρούς, θηρία εἵνεκεν τοῦ ὀνόματος· διὰ τοῦτο ἐκείνων ἐστὶν τὰ δεξιὰ μέρη τοῦ ἁγιάσματος καὶ ὃς ἐὰν πάθῃ διὰ τὸ ὄνομα· τῶν δὲ λοιπῶν τὰ ἀριστερὰ μέρη ἐστίνἀλλὰ ἀμφοτέρων, καὶ τῶν ἐκ δεξιῶν καὶ τῶν ἀριστερῶν καθημένων, τὰ αὐτὰ δῶρα καὶ αἱ αὐταὶ ἐπαγγελίαι· μόνον ἐκεῖνοι ἐκ δεξιῶν κάθηνται καὶ ἔχουσιν δόξαν τινάσὺ δὲ κατεπιθυμεῖς καθίσαι ἐκ δεξιῶν μετ’ αὐτῶν, ἀλλὰ τὰ ὑστερήματά σου πολλάκαθαρισθήσῃ δὲ ἀπὸ τῶν ὑστερημάτων σου· καὶ πάντες οἱ μὴ διψυχοῦντες καθαρισθήσονται ἀπὸ πάντων τῶν ἁμαρτημάτων εἰς ταύτην τὴν ἡμέρανταῦτα εἴπασα ἤθελεν ἀπελθεῖν· πεσὼν δὲ αὐτῆς πρὸς τοὺς πόδας ἠρώτησα αὐτὴν κατὰ τοῦ κυρίου, ἵνα μοι ἐπιδείξῃ ὃ ἐπηγγείλατο ὅραμαἡ δὲ πάλιν ἐπελάβετό μου τῆς χειρὸς καὶ ἐγείρει με καὶ καθίζει ἐπὶ τὸ συμψέλιον ἐξ εὐωνύμων· ἐκαθέζετο δὲ καὶ αὐτὴ ἐκ δεξιῶνκαὶ ἐπάρασα ῥάβψον τινὰ λαμπρὰν λέγει μοι· Βλέπεις μέγα πρᾶγμα; λέγω αὐτῇ· Κυρία οὐδὲν βλέπωλέγει μοι· Σύ, ἰδού, οὐχ ὁρᾷς κατέναντί σου πύργον μέγαν οἰκοδομούμενον ἐπὶ ὑδάτων λίθοις τετραγώνοις λαμπροῖς; ἐν τετραγωονῳ δὲ ᾠκοδομεῖτο ὁ πύργος ὑπὸ τῶν ἓξ νεανίσκων τῶν ἐληλυθότων μετ’ αὐτῆς· ἄλλαι δὲ μυριάδες ἀνδρῶν παρέφερον λίθους, οἱ μὲν ἐκ τοῦ βυθοῦ, οἱ δὲ ἐκ τῆς γῆς, καὶ ἐπεδίδουν τοῖς ἓξ νεανίσκοις· ἐκεῖνοι δὲ ἐλάμβανον καὶ ᾠκοδόμουντοὺς μὲν ἐκ τοῦ βυθοῦ λίθους ἐλκομένους πάντας οὕτως ἐτίθεσαν εἰς τὴν οἰκοδομήν· ἡρμοσμένοι γὰρ ἦσαν καὶ συνεφώνουν τῇ ἁρμογῇ μετὰ τῶν ἑτέρων· καὶ οὕτως ἐκολλῶντο ἀλλήλοις, ὥστε τὴν ἁρμογὴν αὐτῶν μὴ φαίνεσθαιἐφαίνετο δὲ ἡ οἰκοδομὴ τοῦ πύργου ὡς ἐξ ἑνὸς λίθου ᾠκοδομημένητοὺς δὲ ἑτέρους λίθους τοὺς φερομένους ἀπό τῆς ξηρᾶς τοὺς μὲν ἀπέβαλλον, τοὺς δὲ ἐτίθουν εἰς τὴν οἰκοδομήν· ἄλλους δὲ κατέκοπτον καὶ ἔρριπτον μακρὰν ἀπὸ τοῦ πύργουἄλλοι δὲ λίθοι πολλοὶ κύκῳ τοῦ πύργου ἔκειντο, καὶ οὐκ ἐχρῶντο αὐτοῖς ἐπὶ τὴν οἰκοδομήν· ἦσαν γάρ τινες ἐξ αὐτῶν ἐψωριακότες, ἕτεροι δὲ σχισμὰς ἔχοντες, ἄλλοι δὲ κεκολοβωμένοι, ἄλλοι δὲ λευκοὶ καὶ στρογγύλοι, μὴ ἁρμόζοντες εἰς τὴν οἰκοδομήνἔβλεπον δὲ ἑτέρους λίθους ῥιπτομένους μακρὰν ἀπὸ τοῦ πύργου καὶ ἐρχομένους εἰς τὴν ὁδὸν καὶ μὴ μένοντας ἐν τῇ ὁδῷ, ἀλλὰ κυλιομένους ἐκ τῆς ὁδοῦ εἰς τὴν ἀνοδίαν· ἑτέρους δὲ πίπτοντας ἐγγὺς ὑδάτων καὶ μὴ δυναμένους κυλισθῆναι εἰς τὸ ὕδωρ, καίπερ θελόντων κυλισθῆναι καὶ ἐλθεῖν εἰς τὸ ὕδωρ.
Δείξασά μοι ταῦτα ἤθελεν ἀποτρέχεινλέγω αὐτῇ· Κυρία, τί μοι ὄφελος ταῦτα ἑωρακότι καὶ μὴ γινώσκοντι, τί ἐστιν τὰ πράγματα; ἀποκριθεῖσά μοι λέγει· Πανοῦργος εἶ ἄνθρωπος, θέλων γινώσκειν τὰ περὶ τὸν πύργονΝαί, φημί, κυρία, ἵνα τοῖς ἀδελφοῖς ἀναγγείλω καὶ ἱλαρώτεροι γένωνται καὶ ταῦτα ἀκούσαντες γινώσκωσιν τὸν κύριον ἐν πολλῇ δόξῃἡ δὲ ἔφη· Ἀκούσονται μὲν πολλοί· ἀκουσαντες δέ τινες ἐξ αὐτῶν χαρήσονται, τινὲς δὲ κλαύσονται· ἀλλὰ καὶ οὗτοι, ἐὰν ἀκούσωσιν καὶ μετανοήσωσιν, καὶ αὐτοὶ χαρήσονταιἄκουε οὖν τὰς παραβολὰς τοῦ πύρου· ἀποκαλύψω γάρ σοι πάντακαὶ μηκέτι μοι κόπους πάρεχε περὶ ἀποκαλύψεως· αἱ γὰρ ἀποκαλύψεις αὗται τέλος ἔχουσιν· πεπληρωμέναι γάρ εἰσινἀλλ’ οὐ παύσῃ αἰτούμενος ἀποκαλύψεις· ἀναιδὴς γὰρ εἶὁ μὲν πύργος, ὃν βλέπεις οἰκοδομούμενον, ἐγώ εἰμι ἡ Ἐκκλησία, ἡ ὀφθεῖσά σοι καὶ νῦν καὶ τὸ πρότερον· ὃ ἂν οὖν θελήσῃς, ἐπερώτα περὶ τοῦ πύργου, καὶ ἀποκαλύψω σοι, ἵνα χαρῇς μετὰ τῶν ἁγίωνλέγω αὐτῇ· Κυρία, ἐπεὶ ἅπαξ ἄξιόν με ἡγήσω τοῦ πάντα μοι ἀποκαλύψαι, ἀποκάλυψονἡ δὲ λέγει μοι· Ὃ ἐὰν ἐνδέχητά σοι ἀποκαλυφθῆναι, ἀποκαλυφθήσεταιμόνον ἡ καρδία σου πρὸς τὸν θεὸν ἤτω καὶ μὴ διψυχήσεις, ὁ πύργος ἐπὶ ὑδάτων ᾠκοδόμηται, κυρία; Εἶπά σοι, φησίν, καὶ τὸ πρότερον, καὶ ἐκζητεῖς ἐπιμελῶς· ἐκζητῶν οὖν εὑρίσκεις τὴν ἀλήθειανδιατί οὖν ἐπὶ ὑδάτων ᾠκοδόμηται ὁ πύργος, ἄκουε· ὅτι ἡ ζωὴ ὑμῶν διὰ ὕδατος ἐσώθη καὶ σωθήσεταιτεθεμελίωται δὲ ὁ πύργος τῷ ῥήματι τοῦ παντοκράτορος καὶ ἐνδόξου ὀνόματος, κρατεῖται δὲ ὑπὸ τῆς ἀοράτου δυνάμεως τοῦ δεσπότου.
Ἀποκριθεὶς λέγω αὐτῇ· Κυρία, μεγάλως καὶ θαυμαστῶς ἔχει τὸ πρᾶγμα τοῦτο· οἱ δὲ νεανίσκοι οἱ ἓξ οἱ οἰκοδομοῦντες, τίνες εἰσίν, κυρία; Οὗτοι εἰσιν οἱ ἅγιοι ἄγγελοι τοῦ θεοῦ οἱ πρῶτοι κτισθέντες, οἷς παρέδωκεν ὁ κύριος πᾶσαν τὴν κτίσιν αὐτοῦ αὔξειν καὶ οἰκοδομεῖν καὶ δεσπόζειν τῆς κτίσεως πάσης· διὰ τούτων οὖν τελεσθήσεται ἡ οἰκοδομὴ τοῦ πύργουΟἱ δὲ ἕτεροι οἱ παραφέροντες τοὺς λίθους, τίνες εἰσίν; Καὶ αὐτοὶ ἅγιοι ἄγγελοι τοῦ θεοῦ· οὗτοι δὲ οἱ ἓξ ὑπερέχοντες αὐτούς εἰσιν· συντελεσθήσεται οὖν ἡ οἰκοδομὴ τοῦ πύργου, καὶ πάντες ὁμοῦ εὐφρανθήσονται κύκῳ τοῦ πύργου καὶ δοξάσουσιν τὸν θεόν, ὅτι ἐτελέσθη ἡ οἰκομὴ τοῦ πύργουἐπηρώτησα αὐτὴν λέγων· Κυρία, ἤθελον γνῶναι ποταπή ἐστινἀποκριθεῖσά μοι λέγει· Οὐχ ὅτι σὺ ἐκ´πάντων ἀξιώτερος εἶ, ἵνα σοι ἀποκαλυφθῇἄλλοι γάρ σου πρότεροί εἰσιν καὶ βελτίονές σου, οἷς ἔδει ἀποκαλυφθῆναι τὰ ὁράματα ταῦτα· ἀλλ’ ἵνα δοξασθῇ τὸ ὀνομα τοῦ θεοῦ, σοὶ ἀπεκαλύφθη καὶ ἀποκαλυφθήσετα διὰ τοὺσ διψύχους, τοὺς διαλογιζομένους ἐν ταῖς καρδίαις αὐτῶν, εἰ ἄρα ἔστιν ταῦτα ἢ οὐκ ἔστινλέγε αὐτοῖς, ὅτι ταῦτα πάντα ἐστὶν ἀληθῆ καὶ οὐθὲν ἔξωθέν ἐστιν τῆς ἀληθείας, ἀλλὰ πάντα ισχυρὰ καὶ βέβαια καὶ τεθεμελιωμένα ἐστίν.
Ἄκουε νῦν περὶ τῶν λίθων τῶν ὑπαγόντων εἰς τὴν οἰκοδομήνοἱ μὲν οὖν λίθοι οἱ τετράγωνοι καὶ λευκοὶ καὶ συμφωνοῦντες ταῖς ἁρμογαῖς αὐτῶν, οὗτοι εἰσιν οἱ ἀπόστολοι καὶ ἐπίσκοποι καὶ διδάσκαλοι καὶ διάκονοι οἱ πορευθέντες κατὰ τὴν σεμνοτητα του θεοῦ καὶ ἐπισκοπήσαντες καὶ διδάξαντες καὶ διακονήσαντες ἁγνῶς καὶ σεμνῶς τοῖς ἐκλεκτοῖς τοῦ θεοῦ, οἱ μὲν κεκοιμημένοι, οἱ δὲ ἔτι ὄντες· καὶ πάντοτε ἑαυτοῖς συνεφώνησαν καὶ ἐν ἑαυτοῖς εἰρήνην ἔσχον καὶ ἀλλήλων ἤκουον· διὰ τοῦτο ἐν τῇ οἰκοδομῇ τοῦ πύργου συμφωνοῦσιν αἱ ἁρμογαὶ αὐτῶνΟἱ δὲ ἐκ τοῦ βυθοῦ ἐλκόμενοι καὶ ἐπιτιθέμενοι εἰς τὴν οἰκοδομὴν καὶ συμφωνοῦντες ταῖς ἁρμογαῖς αὐτῶν μετὰ τῶν ἑτέρων λίθων τῶν ἤδη ᾠκοδομημένων, τίνες εἰσίν; Οὗτοί εἰσιν οἱ παθόντες ἕνεκεν τοῦ ὀνόματος τοῦ κυρίουΤοὺς δὲ ἑτέρους λίθους τοὺς φερόμένους ἀπὸ τῆς ξηρᾶς θέλω γνῶναι, τίνες εἰσίν, κυρίαἔφη· Τοὺς μὲν εἰς τὴν οἰκοδομὴν ὑπάγοντας καὶ μὴ λατομουμένους, τούτους ὁ κύριος ἐδοκίμασεν, ὅτι ἐπορεύθησαν ἐν τῇ εὐθύτητι τοῦ κυρίου καὶ καταωρθώσαντο τὰς ἐντολὰς αὐτοῦΟἱ δὲ ἀγόμενοι καὶ τιθέμενοι εἰς τὴν οἰκοδομήν, τίνες εἰσίν; Νέοι εἰσὶν ἐν τῇ πίστει καὶ πιστοίνουθετοῦνται δὲ ὑπό τῶν ἀγγέλων εἰς τὸ ἀγαθοποιεῖν, διότι εὑρέθη ἐν αὐτοῖς πονηρίαΟὓς δὲ ἀπέβαλλον καὶ ἐρίπτουν, τίνες εἰσίν; Οὗτοί εἰσιν ἡμαρτηκότες καὶ θέλοντες μετανοῆσαι· διὰ τοῦτο μακρὰν οὐκ ἀπερίφησαν ἔξω τοῦ πύργου, ὅτι εὔχρηστοι ἔσονται εἰς τὴν οἰκοδομήν, ἐὰν μετανοήσωσινοἱ οὖν μέλλοντες μετανοεῖν, ἐὰν μετανοήσωσιν, ἰσχυροὶ ἔσονται ἐν τῇ πίστει, ἐὰν νῦν μετανοήσωσιν, ἐν ᾧ οἰκοδομεῖται ὁ πύργοςἐὰν δὲ τεελσθῇ ἡ οἰκοδομή, οὐκέτι ἔχουσιν τόπον, ἀλλ’ ἔσονται ἔκβολοι· μόνον δὲ τοῦτο ἔχουσιν, παρὰ τῷ πύργῳ κεῖσθαι.
Τοὺς δὲ κατακοπτομένους καὶ μακρὰν ῥιπτομένους ἀπὸ τοῦ πύργου θέλεις γνῶναι; οὗτοι εἰσιν οἱ υἱοὶ τῆς ἀνομίας· ἐπίστευσαν δὲ ἐν ὑποκρίσει, καὶ πᾶσα πονηρία οὐκ ἀπέστη ἀπ’ αὐτῶν· διὰ τοῦτο οὐκ ἔχουσιν σωτηρίαν, ὅτι οὐκ εἰσὶν εὔχρηστοι εἰς οἰκοδομὴν διὰ τὰς πονηρίας αὐτῶνδιὰ τοῦτο συνεκόπησαν καὶ πόρρω ἀπερίφησαν διὰ τὴν ὀργὴν τοῦ κυρίου, ὅτι παρώργισαν αὐτόντοὺς δὲ ἑτέρους, οὓς ἑώρακας πολλοὺς κειμένους, μὴ ὑπάγοντας εἰς τὴν οἰκδομήν, οὗτοι οἱ μὲν ἐψωριακότες εἰσίν, οἱ ἐγνωκότες τὴν ἀλήθειαν, μὴ ἐπιμένοντας δὲ ἐν αὐτῇΟἱ δὲ τὰς σχισμὰς ἔχοντες, τίνες εἰσί; Οὗτοί εἰσιν οἱ κατ’ ἀλλήλων ἐν ἐαυτοῖς, ἀλλήλων ἀποχωρήσωσιν, αἱ πονηρίαι αὐτῶν ἐν ταῖς καρδίαις ἐμμένουσιν, αὖται οὖν αἱ σχισμαί εἰσιν, ἃς ἔχουσιν οἱ λίθοιοἱ δὲ κεκολοβωμένοι, οὗτοί εἰσιν πεπιστευκοτες μὲν καὶ τὸ πλεῖον μέρος ἔχουσιν ἐν τῇ δικαιοσύνῃ, τινὰ δὲ μέρη ἔχουσιν τῆς ἀνομίας· διὰ τοῦτο κολοβοὶ καὶ οὐχ ὁλοτελεῖς εἰσινΟἱ δὲ λευκοὶ καὶ στρογγύλοι καὶ μὴ ἁρμόζοντες εἰς τὴν οἰκοδομήν τίνες εἰσιν, κυρία; ἀποκριθεῖσά μοι λέγει· Ἕως πότε μωρὸς εἶ καὶ ἀσύνετος, καὶ πάντα ἐπερωτᾷς καὶ οὐδὲν νοεῖς; οὗτοί εἰσιν ἔχοντες μὲν πίστιν, ἔχοντες δὲ καὶ πλοῦτον τοῦ αἰῶνος τούτου· ὅταν γένηται θλῖψις, διὰ τὸν πλοῦτον αὐτῶν καὶ διὰ τὰς πραγματείας ἀπαρνοῦνται τὸν κυπριον αὐτῶνκαὶ ἀποκριθεὶς αὐτῇ λέγω· Κυρια, πότε οὖν εὔχρηστοι ἔσονται εἰς τὴν οἰκοδομήν; Ὅταν, φησίν, περικοπῇ αὐτῶν ὁ πλοῦτος ὁ ψυχαγωγῶν αὐτούς, τότε εὔχρηστοι ἔσονται τῷ θεῷὥσπερ γὰρ ὁ λίθος ὁ στρογγύλος, ἐὰν μὴ περικοπῇ καὶ ἀποβάλῃ ἐξ αὐτοῦ τι, οὐ δύναται τετράγωνος γενέσθαι, οὕτω καὶ οἱ πλουτοῦντες ἐν τούτῳ τῷ αἰῶνι, ἐὰν μὴ περικοπῃ αὐτῶν ὁ πλοῦτος, οὐ δύνανται τῷ κυρίῳ εὔχρηστοι γενέσθαιἀπὸ σεαυτοῦ πρῶτον γνῶθι· ὅτε ἐπλούτεις ἄχρηστος ἦς, νῦν δὲ εὔχρηστος εἶ καὶ ὠφέλιμος τῇ ζωῇεὔχρηστοι γίνεσθε τῷ θεῷ· καὶ γὰρ σὺ αὐτὸς χρᾶσαι ἐκ τῶν αὐτῶν λίθων.
Τοὺς δὲ ἑτέρους λίθους, οὓς εἶδες μακρὰν ἀπὸ τοῦ πύργου ῥιπτομένους καὶ πίπτοντας εἰς τὴν ὁδὸν καὶ κυλιομένους ἐκ τῆς ὁδοῦ εἰς τὰς ἀνοδίας· οὗτοί εἰσιν οἱ πεπιστευκότες μέν, ἀπὸ δὲ τῆς διψυχίας αὐτῶν ἀφίουσιν τὴν ὁδὸν αὐτῶν τὴν ἀληθινήν· δοκοῦντες οὖν βελτίονα ὁδὸν δύνασθαι εὑρεῖν, πλανῶνται καὶ ταλαιπωροῦσιν περιπατοῦντες ἐν ταῖς ἀνοδίαιςοἱ δὲ πίπττοντες εἰς τὸ πῦρ καὶ καιόμενοι, οὗτοί εἰσιν οἱ εἰς τέλος ἀποστάντες τοῦ θεοῦ τοῦ ζῶντος, καὶ οὐκέτι αὐτοῖς ἀνέβη ἐπὶ τὴν καρδίαν τοῦ μετανοῆσαι διὰ τὰς ἐπιθυμίας τῆς ἀσελγείας αὐτῶν καὶ τῶν πονηριῶν ὧν εἰργάσαντοτοὺσ δὲ ἑτέρους τοὺς πίπτοντας ἐγγὺς τῶν ὑδάτων καὶ μὴ δυναμένους κυλισθῆναι εἰς τὸ ὕδωρ θέλεις γνῶναι, τίνες εἰσίν; οὗτοί εἰσιν οἱ τὸν λόγον ἀκούσαντες καὶ θέλοντες βαπτισθῆναι εἰς τὸ ὄνομα τοῦ κυρίου· εἶτα ὅταν αὐτοῖς ἔλθῃ εἰς μνείαν ἡ ἁγνότης τῆς ἀηθείας, μετανοοῦσιν καὶ πορνεύονται πάλιν ὀπίσω τῶν ἐπιθυμιῶν αὐτῶν τῶν πονηρῶνἐτέλεσεν οὖν τὴν ἐξήγησιν τοῦ πύργουἀναιδευσάμενος ἔτι αὐτὴν ἐπηρώτησα, εἰ ἄρα πάντες οἱ λίθοι οὗτοι οἱ ἀποβεβλημένοι καὶ μὴ ἁρμόζοντες εἰς τὴν οἰκοδομὴν τοῦ πύργου, εἰ ἔστιν αὐτοῖς μετάνοια καὶ ἔχουσιν τόπον εἰς τὸν πύργον τοῦτονἘχουσιν, φησίν, μετάνοιαν, ἀλλὰ εἰς τοῦτον τὸν πύργον οὐ δύνανται ἁρμόσαι· ἑτέρῳ δὲ τὁπῳ ἁρμόσουσιν πολὺ ἐλάττονι, καὶ τοῦτο ὅταν βασανισθωσιν καὶ ἐκπληρώσωσιν τὰς ἡμέρας τῶν ἁμαρτιῶν αὐτῶνκαὶ διὰ τοῦτο μετατεθήσονται, ὅτι μετέλαβον τοῦ ῥήματος τοῦ δικαίουκαὶ τότε αὐτοῖς συμβήσεται ἔργα ἃ εἰργάσαντο πονηράἐὰν δὲ μὴ ἀναβῇ ἐπὶ τὴν καρδίαν αὐτῶν, οὐ σώζονται διὰ τὴν σκληροκαρδίαν αὐτῶν.
Ὅτε οὖν ἐπαυσάμην ἐρωτῶν αὐτὴν περὶ´πάντων τούτων, λέγει μοι· Θέλεις ἄλλο ἰδεῖν; κατεπίθυμος ὢν τοῦ θεάσασθαι περιχαρὴς ἐγενόμην τοῦ ἰδεῖνἐμβλέψασά μοι ὑπεμειδιασεν καὶ λέγει μοι· Βλέπεις ἑπτὰ γυναῖκας κύκλῳ τοῦ πύργου; Βλέπω, φημί, κυρίαὉ πύργος οὗτος ὑπὸ τούτων βαστάζεται κατ’ ἐπιταγὴν τοῦ κυρίουἄκουε νῦν τὰς ἐνεργείας αὐτῶνἡ μὲν πρώτη αὐτῶν, ἡ κρατοῦσα τὰς χεῖρας, Πίστις καλεῖται· διὰ ταύτης σώζονται οἱ ἐκλεκτοὶ τοῦ θεοῦἡ δὲ ἑτέρα, ἡ περιεζωσμένη καὶ ἀνδριζομένη, Ἐγκράτεια καλεῖται· αὕτη θυγάτηρ ἐστὶν τῆς Πίστεωςὃς ἂν οὖν ἀκολουθήσῃ αὐτῇ, μακάριος γίνεται ἐν τῇ ζωῇ αὐτοῦ, ὅτι, ἐὰν ἀφέξηται πάσης ἐπιθυμίας πονηρᾶς, κληρονομήσει ζωὴν Ἁπλότης, ἡ δὲ Ἐπιστήμη, ἡ δὲ Ἀκακία, ἡ δὲ Σεμνότης, ἡ δὲ Ἀγάπηὅταν οὖν τὰ ἔργα τῆς μητρὸς αὐτῶνἌκουε, φησίν, τὰς δυνάμεις, ἃς ἔχουσινκρατοῦνται δὲ ὑπ’ ἀλλήλαις, καθὼς καὶ γεγεννημέναι εἰσίνἐκ τῆς Πίστεως γεννᾶται Ἐγκράτεια, ἐκ τῆς Ἐγκρατείας Ἁπλότης, ἐκ τῆς Ἁπλότητος Ἀκακία, ἐκ τῆς Ἀκακίας Σεμνότης, ἐκ τῆς Σεμνότητος Ἐπιστήμη, ἐκ τῆς Ἐπιστήμης Ἀγάπητούτων οὖν τὰ ἔργα ἁγνὰ καὶ σεμνὰ καὶ θεῖά ἐστινὃς ἂν οὐν δουλεύσῃ ταύταις καὶ ἰσχύσῃ κρατῆσαι τῶν ἔργων αὐτῶν, ἐν τῷ πύργῳ ἕξει τὴν κατοίκησιν μετὰ τῶν ἁγίων τοῦ θεοῦἐπηρώτων δὲ αὐτὴν περὶ τῶν καιρῶν, εἰ ἤδη συντέλειά ἐστινἡ δὲ ἀνέκραγε φωνῇ μεγάλῃ λέγουσα· Ἀσύνετε ἄνθρωπε, οὐχ ὁρᾷς τὸν πύργον ἔτι οἰκοδομούμενον; ὡς ἐὰν οὖν συντελεσθῇ ὁ πύργος οἰκοδομούμενος, ἔσει τέλοςἀλλὰ ταχὺ ἐποικοδομηθήσεταιμηκέτι με ἐπερώτα μηδέν· ἀρκετή σοι ἡ ὑπόμνησις τῶν πνευμάτων ὑμῶνἀλλ’ οὐ σοὶ μόνῳ ταῦτα ἀπεκαλύφθη, ἀλλ’ ἵνα πᾶσιν δηλώσῃς αὐτά, μετὰ τρεῖς ἡμέρας, νοῆσαί σε γὰρ δεῖ πρῶτονἐντέλλομαι δέ σοι πρῶτον, Ἑρμᾶ, τὰ ῥήματα ταῦτα, ἅ σοι μέλλω λέγειν, λαλῆσαι αὐτὰ πάντα εἰς τὰ ὦτα τῶν ἁγίων, ἵνα ἀκούσαντες αὐτὰ καὶ ποιήσαντες καθαρισθῶσιν ἀπὸ τῶν πονηριῶν αὐτῶν καὶ σὺ δὲ μετ’ αὐτῶν.
Ἀκούσατέ μου, τέκνα· ἐγὼ ὑμᾶς ἐξέθρεψα ἐν πολλῇ ἁπλότητι καὶ ἀκακίᾳ καὶ σεμνότητι διὰ τὸ ἔλεος τοῦ κυρίου τοῦ ἐφ’ ὑμᾶς στάξαντος τὴν δικαιοσύνην, ἵνα δικαιωθῆτε καὶ ἁγιασθῆτε ἀπὸ πάσης πονηρίας καὶ ἀπὸ πάσης σκολιότηος· ὑμεῖς δέ οὐ θέλετε παῆναι ἀπὸ τῆς πονηρίας ὑμῶννῦν οὐν ἀκούσατέ μου καὶ εἰ εἰρηνεύετε ἐν ἑαυτοῖς καὶ ἐισκέπτεσθε ἀλλήλους καί ἀντιλαμβάνεσθε ἀλλήλων, καὶ μὴ μόνοι τὰ κτίσματα τοῦ θεοῦ μεταλαμβάνετε ἐκ καταχύματος, ἀλλὰ μεταδίδοτε καὶ τοῖς ὑστερουμένοις· οἱ μὲν γὰρ ἀπὸ τῶν πολλῶν ἐδεσμάτων ἀσθένειαν τῇ σαρκὶ αὐτῶν ἐπισπῶνται καὶ λυμαίνονται τὴν σάρκα αὐτῶν· τῶν δὲ μὴ ἐχόντων ἐδέσματα λυμαίνεται ἡ σὰρξ αὐτῶν διὰ τὸ μὴ ἔχειν τὸ ἀρκετὸν τῆς τροφῆς, καὶ διαφθείρεται τὸ σῶμα αὐτῶναὕτη οὖν ἡ ἀσυνκρασία βλαβερὰ ὑμῖν τοῖς ἔχουσι καὶ μὴ μεταδιδοῦσιν τοῖς ὑστερουμένοιςβλέπετε τὴν κρίσιν τὴν ἐπερχομένηνοἱ ὑπερέχοντες οὖν ἐκζητεῖτε τοὺς πεινωντας, ἕως οὔπω ὁ πύργος ἐτελέσθη· μετὰ γὰρ τὸ τελεσθῆναι τὸν πύργον θελήσετε ἀγαθοποιεῖν, καὶ οὐχ ἕξετε τόπονβλέπετε οὖν ὑμεῖς οἱ γαυριώμενοι ἐν τῷ πλούτῳ ὑμῶν, μήποτε στενάξουσιν οἱ ὑστερούμενοι καὶ ὁ στεναγμὸς αὐτῶν ἀναβήσεται πρὸς τὸν κύριον καὶ ἐκκλεισθήσεσθε μετὰ τῶν ἀγαθῶν ὑμῶν ἔξω τῆς θύρας τοῦ πύργουνῦν οὖν ὑμῖν λέγω τοῖς προηγουμένοις τῆς ἐκκλησίας καὶ τοῖς πρωτοκαθεδρίταις· μὴ γίνεσθε ὅμοιοι τοῖς φαρμακοῖςοἱ φαρμακοὶ μὲν οὖν τὰ φάρμακα ἑαυτῶν εἰς τὰς πυξίδας βαστάζουσιν, ὑμεῖς δὲ τὸ φάρμακον ὑμῶν καὶ τὸν ἰὸν εἰς τὴν καρδίανἐνεσκιρωμένοι ἐστὲ καὶ οὐ θέλετε καθαρίσαι τὰς καρδίας ὑμῶν καὶ συνκεράσαι ὑμῶν τὴν φρόνησιν ἐπὶ τὸ αὐτὸ ἐν καθαρᾷ καρδίᾳ, ἵνα σχῆτε ἔλεος παρὰ τοῦ βασιλέως τοῦ μεγάλουβλέπετε οὖν, τέκνα, μήποτε αὗται αἱ διχοσασιαι ἀποστερήσουσιν τὴν ζωὴν ὑμῶνπῶς ὑμεῖς παιδεύειν θέλετε τοὺς ἐκλεκτοὺς κυρίου, αὐτοὶ μὴ ἔχοντες παιδείαν; παιδεύετε οὖν ἀλλήλους καὶ εἰρηνεύετε ἐν αὑτοῖς ἵνα κἀγὼ κατέναντι τοῦ πατρὸς ἱλαρὰ σταθεῖσα λόγον ἀποδῶ ὑπὲρ ὑμῶν πάντων τῷ κυρίῳ.
Ὅτε οὖν ἐπαύσατο μετ’ ἐμοῦ λαλοῦσα, ἦλθον οἱ ἓξ νεανίσκοι οἱ οἰκοδομοῦντες καὶ ἀεήνεγκαν αὐτὴν πρὸς τὸν πύργον, καὶ ἄλλοι τέσσαρες ἦραν τὸ συμψέλιον καὶ ἀπήνεγκαν καὶ αὐτὸ πρὸς τὸ πύργοντούτων τὸ πρόσωπον οὐκ εἶδον, ὅτι ἀπεστραμμένοι ἦσανὑπάγουσαν δὲ αὐτὴν ἠρώτων, ἵνα μοι ἀποκαλύψῃ περὶ τῶν τριῶν μορφῶν, ἐν αἷς μοι ἐνεφανίσθηἀποκριθεῖσά μοι λέγει· Περὶ τούτων ἕτερον δεῖ σε ἐπερωτῆσαι, ἵνα σοι ἀποκαλυφθῇὤφθη δέ μοι, ἀδελφοί, τῇ μὲν πρώτῃ ὁράσει τῇ περυσινῇ λίαν πρεσβυτέρα καὶ ἐν καθέδρᾳ καθημένητῇ δὲ ἑτέρα ὁράσει τῆν μὲν ὄψιν νεωτέραν εἶχεν, τὴν δὲ σάρκα καὶ τὰς ἱλαρωτέρα δὲ ἦν ἢ τὸ πρότεροντῇ δὲ τρίτῃ ὁράσει ὅλη νεωτέρα καὶ κάλλει ἐκπρεπεστάτη, μόνας δὲ τὰς τρίχας πρεσβυτέρας εἶχεν· ἱλαρὰ δὲ εἰς τέλος ἦν καὶ ἐπὶ συμψελίου καθημένηπερὶ τούτων περίλυπος ἤμην λίαν τοῦ γνῶναί με τὴν ἀποκάλυψιν ταύτην, καὶ βλέπω τὴν πρεσβυτέραν ἐν ὁράματι τῆς νυκτὸς λέγουσάν μοιΠᾶσα ἐρώτησις ταπεινοφροσύνης χρῄζεινήστευσον οὖν, καὶ λήμψῃ ὃ αἰτεῖς παρὰ τοῦ κυρίουἐνήστευσα οὖν μίαν ἡμέραν, καὶ αὐτῇ τῇ νυκτί μοι ὤφθη νεανίσκος καὶ λέγει μοι· Τί σὺ ὑπὸ χεῖρα αἰτηεῖς ἀποκαλύψεις ἐν δεήσει; βλέπε, μήποτε πολλὰ αἰτούμενος βλάψῃς σου τὴν σάρκαἀρκοῦσίν σοι αἱ ἀποκαλύψεις αὗταιμήτι δύνῃ ἰσχυροτέρας ἀποκαλύψεις ὧν ἑώρακας ἰδεῖν; ἀποκριθείς αὐτῷ λέγω· Κύριε, τοῦτο μόνον αἰτοῦμαι, περὶ τῶν τριῶν μορφῶν τῆς πρεσβυτέρας ἵνα ἀποκάλυψις ὁλοτελὴς γένηταιἀποκριθείς μοι λέγει· Μέχρι τίνος ἀσύνετοί ἐστε; ἀλλ’ αἱ διψυχίαι ὑμῶν ἀσυνέτους ὑμᾶς ποιοῦσιν καὶ τὸ μὴ ἔχειν τὴν καρδίαν ὑμῶν πρὸς τὸν κύριονἀποκριθεὶς αὐτῷ πάλιν εἶπον· Ἀλλ’ ἀπὸ σοῦ, κύριε, ἀκριβέστερον αὐτὰ γνωσόμεθα.
Ἄκουε, φησίν, περὶ τῶν μορφῶν ὧν ἐπιζητεῖςτͅτῇ μὲν πρώτῇ ὁράσει διατί πρεσβυτέρα ὤφθη σοι καὶ ἐπὶ καθέδραν καθημέη; ὅτι τὸ πνεῦμα ὑμῶν πρεσβύτερον καὶ ἤδη μεμαραμμένον καὶ μὴ ἔχον δύναμιν ἀπὸ τῶν μαλακιῶν ὑμῶν καὶ διψυχιῶν· ὥσπερ γὰρ οἱ πρεσβύτεροι, μηκέτι ἔχοντες ἐλπίδα τοῦ ἀνανεῶσαι, οὐδὲν ἄλλο προσδοῶσιν εἰ μὴ τὴν κοίμησιν αὐτῶν, οὕτως καὶ ὑμεῖς μαλακισθέντες ἀπὸ τῶν βιωτικῶν πραγμάτων παρεδώκατε ἑαυτοὺς εἰς τὰς ἀκηδίας καὶ οὐκ ἐπερίψατε ἑαυτῶν τὰς μερίμνας ἐπὶ τὸν κύριον· ἀλλὰ ἐθραύσθη ὑμῶν ἡ διάνοια καὶ ἐπαλαιώθητε ταῖς λύπαις ὑμῶνΔιατί οὖν ἐν καθέδρᾳ ἐκάθητο, ἤθελον γνῶναι, κύριεὍτι πᾶς ἀσθενὴς εἰς καθέδραν καθέζεται διὰ τὴν ἀσθένειαν αὐτοῦ, ἵνα συνκρατηθῇ ἡ ἀσθένεια τοῦ σώματος αὐτοῦἔχεις τὸν τύπον τῆς πρώτης ὁράσεως.
Τῇ δὲ δευτέρᾳ ὁράσει εἶδες αὐτὴν ἑστηκυῖαν καὶ τὴν ὄψιν νεωτέραν ἔχουσαν καὶ ἱλαρωτέραν παρὰ τὸ πρότερον, τὴν δὲ σάρκα καὶ τὰς τρίχας πρεσβυτέραςἄκουε, φησίν, καὶ ταύτην τὴν παραβολήν· ὁταν πρεσβύτερός τις, ἤδη ἀφηλπικὼς ἑαυτὸν διὰ τὴν ἀσθένειαν αὐτοῦ καὶ τὴν πτωχότητα, οὐδὲν ἕτερον προσδέχεται εἰ μὴ τὴν ἐσχάτην ἡμέραν τῆς ζωῆς αὐτοῦ· εἶτα ἐξαίφνης κατελείφθη αὐτῷ κληρονομία, ἀκούσας δὲ ἐξηγέρθη καὶ περιχαρὴς γενόμενος ἐνεδύσατο τὴν ἰσχύν· καὶ οὐκέτι ἀνακεῖται, ἀλλὰ ἕστηκεν, καὶ ἀνανεοῦται αὐτοῦ τὸ πνεῦμα τὸ ἤδη ἐφθαρμένον ἀπὸ τῶν προτέρων αὐτοῦ πράξεων, καὶ οὐκέτι κάθηται, ἀλλὰ ἀνδρίζεται· οὕτως καὶ ὑμεῖς, ἀκούσαντες τὴν ἀποκάλυψιν, ἣν ὑμῖν ὁ κύριος ἀπεκάλυψεν, ὅτι ἐσπλαγχνίσθη ἐφ’ ὑμᾶς, καὶ ἀνενεώσατο τὰ πνεύματα ὑμῶν καὶ ἀπέθεσθε τὰς μαλακίας ὑμῶν, καὶ προσῆλθεν ὑμῖν ἰσχυρότης καὶ ἐνεδυναμώθητε ἐν τῇ πίστει, καὶ ἰδὼν ὁ κύριος τὴν ἰσχυροποίησιν ὑμῶν ἐχάρη· καὶ διὰ τοῦτο ἐδήλωσεν ὑμῖν τὴν οἰκοδομὴν τοῦ πύργου καὶ ἕτερα δηλώσει, ἐὰν ἐξ ὅλης καρδίας εἰρηνεύετε ἐν ἑαυτοῖς.
Τῇ δὲ τρίτῃ ὁράσει εἶδες αὐτὴν νεωτέραν καὶ καλὴν καὶ ἱλαρὰν καὶ καλὴν τὴν μορφὴν αὐτης· ὡς ἐὰν γάρ τινι λυπουμένῳ ἔλθῃ ἀγγελία ἀγαθή τις, εὐθὺς ἐπελάθετο τῶν προτέρων λυπῶν καὶ οὐδὲν ἄλλο προσδέχεται εἰ μὴ τὴν ἀγγελίαν, ἣν ἤκουσεν, καὶ ἰσχυροποιεῖται λοιπὸν εἰς τὸ ἀγαθὸν καὶ ἀνανεοῦται αὐτοῦ τὸ πνεῦμα διὰ τὴν χαράν, ἣν ἔλαβεν· οὕτως καὶ ὑμεῖς ἀνανέωσιν εἰλήφατε τῶν πνευμάτων ὑμῶν ἰδόντες ταῦτα τὰ ἀγαθάκαὶ ὅτι ἐπὶ συμψελίου εἶδες καθημένην, ἰσχυρὰ ἡ θέσις, ὅτι τέσσαρας πόδας ἔχει τὸ συμψέλιον καὶ ἰσχυρῶς ἕστηκεν· καὶ γὰρ ὁ κόσμος διὰ τεσσάρων στοιχείων κρατεῖταιοἱ οὖν μετανοήσαντες ὅλης καρδίας μετανοήσαντεςἀπέχεις ὁλοτελῆ τὴν ἀποκάλυψιν· μηκέτι μηδὲν αἰτήσῃς περὶ ἀποκαλύψεως, ἐάν τι δὲ δέῃ, ἀποκαλυφθήσεταί σοι

Ὅρασις δ’
Ἣν εἶδον, ἀδελφοί, μετὰ ἡμέρας εἴκοσι τῆς προτέρας ὁράσεως τῆς γενομένης, εἰς τύπον τῆς θλίψεως τῆς ἐπερχομένηςὑπῆγον εἰς ἀγρὸν τῂ ὁδῷ τῇ καμπανῇἀπὸ τῆς ὁδοῦ τῆς δημοσίας ἐστὶν ὡσεὶ στάδια δέκα· ῥᾳδίως δὲ ὁδεύεται ὁ τόποςμόνος οὖν περιπατῶν ἀξιῶ τὸν κύριον, ἵνα τὰς ἀποκαλύψεις καὶ τὰ ὁράματα, ἃ μοι ἔδειξεν διὰ τῆς ἁγίας Ἐκκλησίας αὐτοῦ, τελειώσῃ, ἵνα με ἰσχυροποιήσῃ καὶ δῷ τὴν μετάνοιαν τοῖς δούλοις αὐτοῦ τοῖς ἐσκανδαλισμένοις, ἵνα δοξασθῇ τὸ ὄνομα αὐτοῦ τὸ μέγα καὶ ἔνδοξον, ὅτι με ἄξιον ἡγήσατο τοῦ δεῖξαί μοι τὰ θαυμάσια αὐτοῦκαὶ δοξάζοντός μου καὶ εὐχαριστοῦντος αὐτῷ, ὡς ἦχος φωνῆς μοι ἀπεκρίθη· Μὴ διψυχήσεις, Ἑρμᾶεν ἐμαυτῷ ἠρξάμην διαλογιζεσθαι καὶ λέγειν· Ἐγὼ τί ἔχω διψυχῆσαι, οὕτω τεθεμελιωμένος ὑπὸ τοῦ κυρίου καὶ ἰδὼν ἔνδοξα πράγματα; καὶ προσέβην μικρόν, ἀδελφοί, καὶ ἰδού, βλέπω κονιορτὸν ὡς εἰς τὸν οὐρανὸν καὶ ἠρξάμην λέγειν ἐν ἐμαυτῷ· Μήποτε κτήνη ἔρχονται καὶ κονιορτὸν ἐγείρουσιν; οὕτω δὲ ἦν ἀπ’ ἐμοῦ ὡς ἀπὸ σταδίουγινομένου μείζονος καὶ μείζονος κονιορτοῦ ἥλιος καί ἰδού, βλέπω θηρίον μέγιστον ἐξέλαμψεν ὁ ἥλιος καὶ ἰδού, βλέπω θηρίον μέγιστον ὡσεὶ κῆτός τι, και ἐκ τοῦ στόματος αὐτοῦ ἀρίδες πυριναι ἐξεπορεύοντο· ἦν δὲ τὸ θηρίον τῷ μήκει ὡσεὶ ποδῶν ρ’, τὴν δὲ κεφαλὴν εἶχεν ὡσεὶ κεράμουκαὶ ἠρξάμην κλαίειν καὶ ἐρωτᾶν τὸν κύριον, ἵνα με λυτρώσηται ἐξ αὐτοῦ· καὶ ἐπανεμνήσθην τοῦ ῥήματος οὗ ἀκηκόειν· Μὴ διψυχήσεις, Ἑρμᾶἐνδυσάμενος οὖν, ἀδελφοί, τὴν πίστιν τοῦ κυρίου καὶ μνησθεὶς ὧν ἐδίδαξέν με μεγαλείων, θαρσήσας εἰς τὸ θηρίον ἐμαυτὸν ἔδωκαοὕτω δὲ ἤρχετο τὸ θηρίον ῥοίζῳ, ὥστε δύνασθαι αὐτὸ πόλιν λυμᾶναιἔρχομαι ἐγγὺς αὐτοῦ, καὶ τὸ τηλικοῦτο κῆτος ἐκτείνει ἑαυτὸ χαμαὶ καὶ οὐδὲν εἰ μὴ τὴν γλῶσσαν προέβαλλεν καὶ ὅλως οὐκ ἐκινήθη, μέχρις ὅτε παρῆλθον αὐτό· εἶχεν δὲ τὸ θηρίον ἐπὶ τῆς κεφαλῆς χρώματα τέσσαρα· μέλαν, εἶτα πυροειδὲς καὶ αἱματῶδες, εἶτα χρυσοῦν, εἶτα λευκόν.
Μετὰ δὲ τὸ παρελθεῖν με τὸ θηρίον καὶ προελθεῖν ὡσεὶ πόδας λ’, ἰδού, ὑπαντᾷ μοι παρθένος κεκοσμημένη ὡς ἐκ νυμῶνος ἐκπορευμένη, ὅλη ἐν λευκοῖς καὶ ὑποδήμασιν λευκοῖς, κατακεκαλυμμένη ἕως τοῦ μετώπου, ἐν μίτρͅ δὲ ἦν ἡ κατακάλυψις αὐτῆς· εἶχεν δὲ τὰς τρίχας αὐτῆς λευκάςἔγνων ἐγὼ ἐκ τῶν προτέρων ὁραμάτων, ὅτι ἡ Ἐκκλησία ἐστίν, καὶ ἱλαρώτερος ἐγενόμηνἀσπάζεταί με λέγουσα· Χαῖρε σύ, ἄνθρωπεκαὶ ἐγὼ αὐτὴν ἀντησπασάμην· Κυρία, χαῖρεἀποκριθεῖσά μοι λέγει· Οὐδέν σοι ἀπήντησεν; λέγω αὐτῇ· Κυρία, τηλικοῦντο θηρίον, δυνάμενον λαοὺς διαφθεῖραι· ἀλλὰ τͅτῃ δυνάμει τοῦ κυρίου καὶ τῇ πολυσπλαγχνίᾳ αὐτοῦ ἐξέφυγον αὐτόΚαλῶς ἐξέφυγες, φησίν, ὅτι τὴν μέριμνάν σου ἐπὶ τὸν θεὸν ἐπέριψας καὶ τὴν καρδίαν σου ἤνοιξας πρὸς τὸν κύριον, πιστεύσας, ὅτι δι’ οὐδενὸς δύνῃ σωθῆναι εἰ μὴ διὰ τοῦ μεγάλου καὶ ἐνδόξου ὀνόμά ἐστιν Θεγρί, καὶ ἐνέφραξεν τὸ στόμα αὐτοῦ, ἵνα μή σε λυμάνῃμεγάλην θλῖψιν ἐκπέφευγας διὰ τὴν πίστιν σου καὶ ὅτι τηλικοῦντο θηρίον ἰδὼν οὐκ ἐδιψύχησας· ὕπαγε οὖν καὶ ἐξήγησαι τοῖς ἐκλεκτοῖς τοῦ κυρίου τὰ μεγαλεῖα αὐτοῦ καὶ εἰπὲ αὐτοῖς, ὅτι τὸ θηρίον τοῦτο τύπος ἐστὶν θλίψεως τῆς μελλούσης τῆς μεγάλης· ἐὰν οὖν προετοιμάσησθε καὶ μετανοήσητε ἐξ ὅλης καρδίας ὑμῶν πρὸς τὸν κύριον, δυνήσεσθε ἐκφυγεῖν αὐτήν, ἐὰν ἡ καρδία ὑμῶν γένηται καθαρὰ καὶ ἄμωμος καὶ τὰς λοιπὰς τῆς ζωῆς ἡμέρας ὑμῶν δουλεύσητε τῷ κυρίῳ ἀμέμπτωςἐπιρίψατε τὰς μερίμνας ὑμῶν ἐπὶ τὸν κύριον, καὶ αὐτὸς κατορθώσει αὐτάςπιστεύσατε τῷ κυρίῳ, οἱ δίψυχοι, ὅτι δίψυχοι, ὅτι πάντα δύναται καὶ ἀποστρέφει τὴν ὀργὴν αὐτοῦ ἀφ’ ὑμῶν καὶ ἐξαποστέλλει μάστιγας ὑμῖν τοῖς διψύχοιςοὐαὶ τοῖς ἀκούσασιν τὰ ῥήματα ταῦτα καὶ παρακούσασιν· αἱρετώτερον ἦν αὐτοῖς τὸ μὴ γεννηθῆναι.
Ἠρώτησα αὐτὴν περὶ τῶν τεσσάρων χρωμάτων ὧν εἶχεν τὸ θηρίν εἰς τὴν κεφαλήνἡ δὲ ἀποκριθεῖσά μοι λέγει· Πάλιν περίεργος εἶ περὶ τοιούτων πραγμάτωνΝαί, φημί, κυρία· γνώρισόν μοι, τί ἐστιν ταῦταἌκουε, φησίν· τὸ μὲν μέλαν οὗτος ὁ κόσμος ἐστίν, ἐν ᾧ κατοικεῖτε· τὸ δὲ πυροειδες, ὅτι δεῖ τὸν κόσμον τοῦτον δι’ αἵματος καὶ πυρὸς ἀπόλλυσθαι· τὸ δὲ χρυσοῦν μέρος ὑμεῖς ἐστε οἱ ἐκωυγόντες τὸν κόσμον τοῦτονὥσπερ γὰρ τὸ χρυσίον δοκιμάζεται διὰ τοῦ πυρὸς καὶ εὔχρηστον γίνεται, οὕτως καὶ ὑμεῖς δοκιμάζεσθε οἱ κατοικοῦντες ἐν αὐτοῖςοἱ οὖν μείναντες και πυρωθέντες ὑπ’ αὐτῶν καθαρισθήσεσθεὥσπερ τὸ χρυσίον ἀποβάλλει πᾶσαν λύπην καὶ στενοχωρίαν, καὶ καθαρισθήσεσθε καὶ χρήσιμοι ἔσεσθε εἰς τὴν οἰκοδομὴν τοῦ πύργουτὸ δὲ λευκὸν μέρος ὁ αἰὼν ὁ ἐπερχόμεός ἐστιν, ἐν ᾧ κατοικήσουσιν οἱ ἐκλεκτὶ τοῦ θεοῦ· ὅτι ἄσπιλοι καὶ καθαροὶ ἔσονται οἱ ἐκλελεγμένοι ὑπὸ τοῦ θεοῦ εἰς ζωὴν αἰώνιονσὺ οὖν μὴ διαλίῃς λαλῶν εἰς τὰ ὦτα τῶν ἁγίωνἔχετε καὶ τὸν τύπον τῆς θλιπψεως τῆς ἐρχομένης μεγάληςἐὰν δὲ ὑμεῖς θελήσητε, οὐδὲν ἔσταιμνημονεύετε τὰ προγεγραμμέναταῦτα εἴπασα ἀπῆλθεν, καὶ οὐκ εἶδον, ποίῳ τόπον ἀπῆλθεν, καὶ οὐκ εἶδον, ποίῳ τόπῳ ἀπῆλθεν· νέφος γὰρ ἐγένετο· κἀγὼ ἐπεστράφην εἰς τὰ ὀπίσω φοβηθείς, δοκῶν ὅτι τὸ θηρίον ἔρχεται.

Ἀποκάλυψις ε’.
Προσευξαμένου μου ἐν τῷ οἴκῳ καὶ καθίσαντος εἰς τὴν κλίνην εἰςῆλθεν ἀνήρ ἔδοξος τῇ ὄψει, σχήματι ποιμενικῷ, περικείμενος δέρμα ἄγειον λευκὸν καὶ πήραν ἔχων ἐπὶ τῶν ὤμων καὶ ῥάβδον εἰς τὴν χεῖρακαὶ ἠσπάσατό με, κἀγὼ ἀντηπασάμην αὐτόνκαὶ εὐθὺς παρεκάθισέν μοι καὶ λέγει μοι· Ἀπεστάλην ὑπὸ τοῦ σεμνοτάτου ἀγγέλου, ἵνα μετὰ σοῦ οἰκήσω τὰς λοιπὰς ἡμέρας τῆς ζωῆς σουἔδοξα ἐγώ, ὅτι πάρεστιν ἐκπειράζων με, καὶ λέγω αὐτῷ· Σὺ γὰρ τίς εἶ; ἐγὼ γάρ, φημί, γινώσκω ᾧ παρεδόθηνλέγει μοι· Οὐκ ἐπιγινώσκω, ᾧ παρεδόθηνλέγει μοι· Οὐκ ἐπιγινώσκεις με; Οὔ, φημίἘγώ, φησίν, εἰμὶ ὁ ποιμήν, ᾧ παρεδόθηςἔτι λαλοῦντος αὐτοῦ ἠλλοιώθη ἡ ἰδέα αὐτοῦ, καὶ ἐπέγνων αὐτόν, ὅτι ἐκεῖνος ἦν, ᾧ παρεδόθην, καὶ εὐθὺς συνεχύθην καὶ φόβος με ἔλαβεν καὶ ὅλος συνεκόπην ἀπὸ τῆς λύπης, ὅτι οὕτως αὐτῷ ἀπεκρίθην πονηρῶς καὶ ἀφρόνωςὁ δὲ ἀποκριθείς μοι λέγει· Μὴ συγχύννου, ἀλλὰ ἰσχυροποιοῦ ἐν ταῖς ἐντολαῖς μου αἷς σοι μέλλω ἐντέλλεσθαιἀπεστάλην γάρ, φησίν, ἵνα ἃ εἶδες πρότερον πάντα σοι πάλιν δείξω, αὐτὰ τὰ κεφάλαια τὰ ὄντα ὑμῖν σύμφοραπρῶτον πάντων τὰς ἐντολάς μου γράψον καὶ τὰς παραβολάς· τὰ δὲ ἕτερα, καθώς σοι δείξω, οὕτως γράψεις· διὰ τοῦτο, φησίν, ἐντέλλομαί σοι πρῶτον γράψαι τὰς ἐντολὰς καὶ παραβολάς, ἵνα ὑπὸ χεῖρα ἀναγινώσκης αὐτὰς καὶ δυνηθῇς φυλάξαι αὐτάςἔγραψα οὖν τὰς ἐντολὰς καὶ παραβολάς, καθὼς ἐνετείλατό μοιἐὰν οὖν ἀκούσαντες αὐτὰς φυλάξητε καὶ ἐν αὐταῖς πορευθῆτε καὶ ἐργάσησθε αὐτὰς ἐν καθαρᾷ καρδίᾳ, ἀπολήμψεσθε ἀπὸ τοῦ κυρίου, ὅσα ἐπηγγείλατο ὑμῖν· ἐὰν δὲ ἀκούσαντες μὴ μετανοήτε, ἀλλ’ ἔτι προσθῆτε ταῖς ἁμαρτίαις ὑμῶν, ἀπολήμψεσθε παρὰ τοῦ κυρίου τὰ ἐναντίαταῦτά μοι πάντα οὕτως γράψαι ὁ ποιμὴν ἐνετείλατο, ὁ ἄγγελος τῆς μετανοίας.

Ἐντολὴ α’.
Πρῶτον πάντων πίστευσον, ὅτι εἷς ἐστὶν ὁ θεός, ὁ τὰ πάντα κτίσας καὶ καταρτίσας καὶ ποιήσας ἐκ τοῦ μὴ ὄντος εἰς τὸ εἶναι τὰ πάντα καὶ´πάντα χωρῶν, μόνος δὲ ἀχώρητος ὤνπίστευσον οὖν αὐτῷ καὶ φοβήθητι αὐτόν, φοβηθεὶς δὲ ἐγκράτευσαιταῦτα φύλασσε, καὶ ἀποβαλεῖς πᾶσαν πονηρίαν ἀπὸ σεαυτοῦ καὶ ἐνδύσῃ´πᾶσαν ἀρετὴν δικαιοσύνης καὶ ζήσῃ τῷ θεῷ, ἐὰν φυλάξης τὴν ἐντολὴν ταύτην.

Ἐντολὴ β’.
Λέγει μοι· Ἁπλότητα ἔχε καὶ ἄκακος γίνου, καὶ ἔσῃ ὡς τὰ νήπια τὰ μὴ γινώσκοντα τὴν πονηρίαν τὴν ἀπολλύουσαν τὴν ζωὴν τῶν ἀνθρώπωνπρῶτον μὲν μηδενὸς καταλάλει μηδὲ ἡδέως ἄκουε καταλαλοῦντος· εἰ δὲ μή, καὶ σὺ ὁ ἀκούων ἔνοχος ἔσῃ τῆς ἁμαρτίας τοῦ καταλαλοῦντος, ἐὰν πιστεύσῃς τῇ καταλαλιᾷ ᾗ ἂν ἀκούσῃς· πιστεύσας γὰρ καὶ σὺ αὐτὸς ἕξεις κατὰ τοῦ ἀδελφοῦ σου· οὕτως οὖν ἔνοχος ἔσῃ τῆς ἁμαρτίας τοῦ καταλαλοῦντοςπονηρὰ ἡ καταλαλιά· ἀκατάστατον δαιμόνιόν ἐστιν, μηδέποτε εἰρηνεῦον, ἀλλὰ πάντοτε ἐν διχοστασίαις κατοικοῦνἀπέχου οὖν ἀπ’ αὐτοῦ, καὶ εὐθηνίαν πάντοτε ἕξεις μετὰ πάντωνἔνδυσαι δὲ τὴν σεμνότητα, ἐν ᾗ οὐδὲν πρόσκομμά ἐστιν πονηρόν, ἀλλὰ πάντα ὁμαλὰ καὶ ἱλαράἐργάζου τὸ ἀγαθὸν καὶ ἐκ τῶν κόπων σου ὧν ὁ θεὸς δίδωσίν σοι πᾶσιν ὑστερουμένοις δίδου ἁπλῶς, μὴ διστάζων, τίνι δῷς ἢ τίνι μὴ δῷςπᾶσιν δίδου· πᾶσιν γάρ ὁ θεὸς δίδοσθαι θέλει ἐκ τῶν ἰδίων δωρημάτωνοἱ οὖν λαμβάνοντες ἀποδώσουσιν λόγον τῷ θεῷ, διατί ἔλαβον καὶ εἰς τί· οἱ μὲν γὰρ λαμβάνοντες θλιβόμενοι οὐ δικασθήσονται, οἱ δὲ ἐν ὑποκρίσει λαμβάνοντες τίσουσιν δίκηνὁ οὖν διδοὺς ἀθῷός ἐστιν· ὡς γὰρ ἔλαβεν παρὰ τοῦ κυρίου τὴν διακονίαν τελέσαι, ἁπλῶς αὐτὴν ἐτέλεσεν, μηθὲν διακρίνων, τίνι δῷ ἢ μὴ δῷἐγένετο οὖν ἡ διακονία αὕτη ἁπλῶς τελεσθεῖσα ἔνδοξος παρά τῷ θεῷὁ οὖν οὕτως ἁπλῶς διακονῶν τῷ θεῷ ζήσεταιφύλασσε οὖν τὴν ἐντολὴν ταύτην, ὥς σοι λελάληκα, ἵνα ἡ μετάνοιά σου καὶ τοῦ οἴκου σου ἐν ἁπλότητι εὑρεθῇ, καὶ ἀκακία καθαρὰ καὶ ἀμίαντος.

Ἐντολὴ γ’.
Πάλιν μοι λέγει· Ἀλήθειαν ἀγάπα καὶ πᾶσα ἀλήθεια ἐκ τοῦ στόματός σου ἐκορευέσθω, ἵνα τὸ πνεῦμα ὃ ὁ θεὸς κατῴκισεν ἐν τῇ σαρκὶ ταύτῃ, ἀληθὲς εὑρεθῇ παρὰ πᾶσιν ἀνθρώποις, καὶ οὕτως δοξασθήσεται ὁ κύριος ὁ ἐν σοὶ κατοκῶν, ὅτι ὁ κύριος ἀληθινὸς ἐν παντὶ ῥήματι καὶ οὐδὲν παρ’ αὐτῷ ψεῦδοςοἱ οὖν ψευδόμενοι ἀθετοῦσι τὸν κύριον καὶ γίνονται ἀποστερηταὶ τοῦ κυρίου, μὴ παραδιδόντες αὐτῷ τὴν παρακαταθήκην, ἣν ἔλαβονἔλαβον γὰ παρ’ αὐτοῦ πνεῦμα ἄψευστοντοῦτο ἐὰν ψευδὲς ἀποδώσωσιν, ἐμίαναν τὴν ἐντολὴν τοῦ κυρίου καὶ ἐγένοντο α0ποστερηταίταῦτα οὖν ἀκούσας ἐγὼ ἔλαυσα λίανἐδὼν δέ με κλαίοντα λέγει· Τί κλαίεις; Ὅτι, φημί, κύριε, οὐκ οἶδα, εἰ δύναμαι σωθῆναιΔιατί; φησίνΟὐδέπω γάρ, φημί, κύριε, ἐν τῇ ἐμῇ ζωῇ ἀληθὲς ἐλάλησα ῥήμα, ἀλλὰ πάντοτε πανούρως ἐλάλησα μετὰ πάντων καὶ τὸ ψεῦδός μου ἀληθὲς ἐπέδειξα παρὰ πᾶσιν ἀνθρώποις· καὶ οὐδέποτέ μοι οὐδεὶς ἀντεῖπεν, ἀλλ’ ἐπιστεύθη τῷ λόγῳ μουπῶς οὖν, φημί, κύριε, δύναμαι ζῆσαι ταῦτα πράξας; Σὺ μέν, φησί, καλῶς καὶ ἀληθῶς φρονεῖς· ἔδει γάρ σε ὡς θεοῦ δουλον ἐν ἀληθειπᾳ πορεύεσθαι, καὶ πονηρὰν συνείδησιν μετὰ τοῦ πνεύματος τῆς ἀληθείας μὴ κατοικεῖν μηδὲ λύπην ἐπάγειν τῷ πνεύματι τῷ σεμνῷ καὶ ἀληθεῖΟὐδέποτε, φημί, κύριε, τοιαῦτα ῥήματα ἀκριβῶς ἤκουσαΝῦν οὖν, φησίν, ἀκούεις· φύλασσε αὐτά, ἵνα καὶ τὰ πρότερον ἃ ἐλάλησας ψευδὴ ἐν ταῖς πραγματείας σου, τούτων εὑρεθέντων ἀληθινῶν, κἀκεῖνα πιστὰ γένηται· δύναται γὰρ κἀκεῖνα πιστὰ γενέσθαιἐὰν ταῦτα φυλάξῃς καὶ ἀπὸ τοῦ νῦν πᾶσαν ἀλήθειαν λαλήσῃς, δυνήσῃ σεαυτῷ ζωὴν περιποιήσασθαι· και ὃς ἂν ἀκούσῃ τὴν ἐντολὴν ταύτην καὶ ἀπέξεται τοῦ πονηροτάτου ψεύσματος ζήσεται τῷ θεῷ.

Ἐντολὴ δ’.
Ἐντέλλομαί σοι, φησίν, φυλάσσειν τὴ ἁγνείαν, καὶ μὴ ἀναβαινέτω σου ἐπὶ τὴν καρδίαν περὶ γυναικὸς ἀλλοτρίας ἡ περὶ πορνείας τινὸς ἢ περὶ τοιούτων τινῶν ὁμοιωμάτων πονηρῶν, τοῦτο γὰρ ποιῶν μεγάλην ἁμαρτίαν ἐργάζῃτῆς δὲ σῆς μνημονεύων πάντοτε γυναικὸς οὐδέποτε διαμαρτυρήσειςἐὰν γὰρ αὕτη ἡ ἐθύμησις ἐπὶ τὴν καρδίαν σου ἀναβῇ, διαμαρτήσεις, καὶ ἐὰν ἕτερα οὕτως πονηρά, ἁμαρτίαν ἐργάζῃ· ἡ γὰρ ἐνθύμησις αὕτη θεοῦ δούλῳ ἁμαρτία μεγάλη ἐστίν· ἐὰν δὲ τις ἐργάσηται τὸ ἔργον τὸ πονηρὸν τοῦτο, θάνατον ἑαυτῷ κατεργάζεταιβλέπε οὖν σύ· ἀπέχου ἀπὸ τῆς ἐνθυμήσεως ταύτης· ὅπου γὰρ σεμνότης κατοικεῖ, ἐκεῖ ἀνομία οὐκ ὀφείλει ἀναβαίνειν ἐπὶ καρδίαν ἀνδρὸς δικαίουλέγω αὐτῷ· Κύριε, ἐπίτρεψόν μοι ὀλίγα ἐρωτῆσαί σεΛέγε, φησίνΚύριε, φημί, εἰ γυναῖκα ἔχῃ τις πιστὴν ἐν κυρίῳ καὶ ταύτην εὕρῃ ἐν μοιχείᾳ τινί, ἆρα ἁματάνει ὁ ἀνὴρ συνζῶν μετ’ αὐτῆς; Ἄχρι τῆς ἀγονοίς, φησίν, οὐχ ἁμαρτάνει· ἐὰν δὲ γνῷ ὁ ἀνὴρ τὴν ἁμαρτίαν αὐτῆς καὶ μὴ μετανοήσῃ ἡ γυνή, ἀλλ’ ἐπινένῃ τῇ πορνείᾳ αὐτῆς καὶ συνζῇ ὁ ἀνὴρ μετ’ αὐτῆς, ἔνοχος γίνεται τῆς ἁμαρτίας αὐτῆς καὶ κοινωνὸς τῆς μοιχείας αὐτῆςΤί οὖν, φημί, κύριε, ποιήσῃ ὁ ἀνήρ, ἐὰν ἐπιμείνῃ τῷ πάθει τούτῳ ἡ γυνή; Ἀπολυσάτω, φησίν, αὐτην καὶ ὁ ἀνὴρ ἐφ’ ἑαυτῷ μενέτω· ἐὰν δὲ ἀπολύσας τὴν γυναῖκα μετανοήσῃ ἡ γυνὴ καὶ θελήσῃ ἐπὶ τὸν ἑαυτῆς ἄνδρα ὑποστρέψαι, οὐ παραδεχθήσεται; Καὶ μήν, φησίν, ἐὰν μὴ παραδέξηται αὐτὴν ὁ ἀνή, ἁμαρτάνει καὶ μεγάλην ἁμαρτίαν ἑαυτῷ ἐπιπᾶται, ἀλλὰ δεῖ παραδεχθῆναι τὸν ἡμαρτηκότα καὶ μετανοοῦντα, μὴ ἐπὶ πολὺ δέ· τοῖς γὰρ δούλοις τοῦ θεοῦ μετάνοιά ἐστιν μίαδιὰ τὴν μετάνοιαν οὖν ὀφείλει γαμεῖν ὁ ἀνήραὕτη ἡ πρᾶξις ἐπὶ γυναικὶ καὶ ἀνδρὶ κεῖταιοὐ μόνον, φησίν, μοιχεία ἐστίν, ἐὰν τις τὴν σάρκα αὐτοῦ μιάνῃ, ἀλλὰ καὶ ὃς ἂν τὰ ὁμοιώματα ποιῇ τοῖς ἔθνεσιν, μοιχᾶταιὥστε καὶ ἐν τοῖς τοιοὐτοις ἔργοις ἐὰν ἐμμένῃ τις καὶ ἐν τοῖς τοιοὐτοις ἔργοις ἐὰν ἐμμένῃ τις καὶ μὴ μετανοῇ, ἀπέχου ἀπ’ αὐτοῦ καὶ μὴ συνζῆθι αὐτῷ· εἰ δὲ μή, καὶ σὺ μέτοχος εἶ τῆς ἁμαρτίας αὐτοῦδιὰ τοῦτο προσετάγη ὑμῖν ἐφ’ ἐαυτοῖς μένειν, εἴτε ἀνὴρ εἴτε γυνή· δύνατια γὰρ ἐ τοῖς τοιούτοις μετάνοια εἶναιἐγὼ οὖν, φησίν, οὐ δίδωμι ἀφορμήν, ἵνα αὕτη ἡ πρᾶξις οὕτως συντελῆται, ἀλλὰ εἰς τὸ μηκέτι ἁμαρτάνειν τὸν ἡμαρτηκόταπερὶ δὲ τῆς προτέρας ἁμαρτίας αὐτοῦ ἔστιν ὁ δυνάμενος ἴασιν δοῦναι· αὐτὸς γάρ ἐστιν ὁ ἔχων πάντων τὴν ἐξουσίαν.
Ἠρώτησα δὲ αὐτὸν πάλιν λέγων· Ἐπεὶ ὁ κύριος ἄξίον με ἡγήσατο, ἵνα μετ’ ἐμοῦ πάντοτε κατοικῇς ὀλίγα μου ῥήματα ἔτι ἀνάσχου, ἐπεὶ οὐ συνίω οὐδὲν καὶ ἡ καρδία μου πεπώρωται ἀπὸ τῶν προτέρων μου πράξεων· συνέτισόν με, ὅτι λίαν ἄφρων εἰμὶ καὶ ὅλως οὐθὲν νοῶἀποκριθείς μοι λέγει· Ἐγώ, φησίν, ἐπὶ τῆς μετανοίας εἰμὶ καὶ πᾶσιν τοῖς μετανοοῦσιν σύνεσιν δίδωμιἡ οὐ δοκεῖ σοι, φησίν, αὐτὸ τοῦτο τὸ μετανοῆσαι σύνεσιν εἶναι; τὸ μετανοῆσαι, φησίν, σύνεσίς ἐστιν μεγάλη· συνίει γὰρ ὁ ἁμαρτήσας, ὅτι πεποίκεν τὸ πονηρὸν ἔμπροσθεν τοῦ κυρίου, καὶ ἀναβαίνει ἐπὶ τὴν καρδίαν αὐτοῦ ἡ πρᾶξις, ἣν ἔπραξεν, καὶ μετανοεῖ καὶ οὐκέτι ἐργάζεται τὸ πονηρόν, ἀλλὰ τὸ ἀγαθὸν πολυτελῶς ἐργάζεται καὶ ταπεινοῖ τὴν ἑαυτοῦψυχὴν καὶ βασανίζει, ὅτι ἥμαρτενβλέπεις οὖν, ὅτι ἡ μετάνοια σύνεσίς ἐστιν μεγάληΔιὰ τοῦτο οὖν, φημί, κύριε, ἐξακριβάζομαι παρὰ σοῦ πάντα· πρῶτον μέν, ὅτι ἁμαρτωλός εἰμι, ἵνα γνῶ, ποῖα ἔργα ἐργαζόμενος ζήσομαι, ὅτι πολλαί μου εἰσὶν αἱ ἁμαρτίαι καὶ ποικίλαιΖήσῃ, φησίν, ἐὰν τὰς ἐντολάς μου φυλάξῃς καὶ πορευθῇς ἐν αὐταῖς· καὶ ὃς ἂν ἀκούσας τὰς ἐντολὰς ταύτας φυλάξῃ, ζήσεται τῷ θεῷ.
Ἔτι φημί, κύριε, προσθήσω τοῦ ἐρρωτῆσαιΛέγε, φησίνἬκουσα, φημί, κύριε παρά τινων διδασκάλων, ὅτι εἰς ὕδωρ κατέβημεν και ἐλάβομεν ἄφεσιν ἁμαρτιῶν ἡμῶν τῶν προτέρωνλέγει μοι· Καλῶς ἤκουσας· οὕτω γὰρ ἔχειἔδει γὰρ τὸν εἰληφότα ἄφεσιν ἁμαρτιῶν μηκέτι ἁμαρτάνειν, ἀλλ’ ἐν ἁγνείᾳ κατοικεῖνἐπεὶ δὲ πάντα ἐξακριβάζῃ, καὶ τοῦτό σοι δηλώσω, μὴ διδοὺ ἀφορμὴν τοῖς μέλλουσι πιστεύειν ἢ τοῖς νῦν πιστεύσασιν εἰς τὸν κύριονοἱ γὰρ νῦν πιστεύσαντες ἡ μέλλοντες πιστεύειν μετάνοιαν ἁμαρτιῶν οὐκ ἔχουσιν, ἄφεσιν δὲ ἔχουσι τῶν προτέρων ἁμαρτιῶν αὐτῶντοῖς οὖν κληθεῖσι πρὸ τούτων τῶν ἡμερῶν ἔθηκεν ὁ κύριος μετάμοιαν· καρδιογνώστης γὰρ ὢν ὁ κύριος καὶ πάντα προγιώσων ἐγνω τὴν ἀσθένειαν τῶν ἀνθρώπων καὶ τὴν πολυπλκίαν τοῦ διαωβόλου, ὅτι ποιήσει τι κακὸν τοῖ δούλοις τοῦ θεοῦ καὶ πονηρευσεται εἰς αὐτούςπολύσπλγχνος οὖν ὢν ὁ κύριος ἐσπλαγχνίσθη ἐπὶ τὴν ποίησιν αὐτοῦ καὶ ἔθηκεν τὴν μετάνοιαν ταύτην, καὶ ἐμοὶ ἡ ἐξουσία τῆς μετανοίας ταύτης ἐδόθηἀλλὰ ἐγὼ σοι λέγω, φησί· μετὰ τὴ κλῆσιν ἐκείνην τὴν μεγάλην καὶ σεμνὴν ἐὰν τις ἐκπειρασθεὶς ὑπὸ τοῦ διαβόλου ἁμαρτήσῃ, μίαν μετάνοιαν ἔχει· ἐὰν δὲ ὑπὸ χεῖρα ἁμαρτάνῃ καὶ μετανοήσῃ, ἀσύμφορόν ἐστι τῷ ἀνθρώπῳ τῷ τοιούτῳ· δυσκόλως γὰρ ζήσεταιλέγω αὐτῷ· Ἐζωοποιήθην ταῦτα παρὰ σοῦ ἀκούσας οὕτως ἀκριβῶς· οἶδα γὰρ ὅτι, ἐὰν μηκέτι προσθήσω ταῖς ἁμαρτίαις μου, σωθήσομαιΣωθήσῃ, φησίν, καὶ πάντες, ὅσοι ἐὰν ταῦτα ποιήσωσιν.
Ἠρώτησα αὐτὸν πάλιν λέγων· Κύριε, ἐπεὶ ἅπαξ ἀνέχῃ μου, ἔτι μοι καὶ τοῦτο δήλωσονΛέγε, φησίνἘὰν γυνή, φημί, κύριε, ἢ πάλιν ἀνήρ τις κοιμηθῇ καὶ γαμήσῃ τις ἐξ αὐτῶν μήτι ἁμαρτάνει ὁ γαμῶν; Οὐχ ἁμαρτάνε φησίν· ἐὰν δὲ ἐφ’ ἑαυτῷ μείνῃ τις, περισσοτέραν ἑαυτῷ τιμὴν καὶ μεγάλην δόξαν περιποιεῖται πρὸς τὸν κύριον· ἐὰν δὲ καὶ γαμήσῃ, οὐχ ἁμαρτανειτήρει οὖν ἁγνείαν καὶ τὴν σεμνότητα, καὶ ζήσῃ τῷ θεῷταῦτά σοι ὅσα λαλῶ καὶ μέλλω λαλεῖν, φύλασσε ἀπὸ τοῦ νῦν, ἀφ’ ἧς μοι παρεδόθης ἡμέρας, καὶ εἰς τὸν οἶκόν σου κατοικήσωτοῖς δὲ προτέροις σου παραπτώμασιν ἄφεσις ἔσται, ἐὰν τὰς ἐντολάς μου φυλάξῃς· καὶ´πᾶσι δὲ ἄφεσις ἔσται ἐὰν τὰς ἐντολάς μου ταύτας φυλάξωσι καὶ πορευθῶσιν ἐν τῇ ἁγνότητι ταύτῃ.

Ἐντολὴ ε’.
Μακρόθυμος, φησί, γίνου καὶ συνετός, καὶ πάντων τῶν πονηρῶν ἐργων κατακυριεύσεις καὶ ἐργάσῃ πᾶσαν δικαιοσύνηνἐὰν γὰρ μακρόθυμος ἔσῃ, τὸ πνεῦμα τὸ ἅγιον τὸ κατιοικοῦν ἐν σοὶ καθαρὸν ἔσται, μὴ ἐπισκοτούμενον ὑπὸ ἑτέρου πονηροῦ πνεύματος, ἀλλ’ ἐν εὐρυχώρῳ κατοικοῦν ἀγαλλιάσσεται καὶ εὐφρανθήσεται μετὰ τοῦ σκεύους, ἐν ᾧ κατοικεῖ, καὶ λειτουργήσεται μετὰ τῷ θεῷ ἐν ἱλαρότητι πολλῇ, ἔχον τὴν ευθηνίαν ἐν ἑαυτῷἐὰν δὲ ὀξυχολία τις προσέλθῃ, εὐθὺς τὸ πνεῦμα τὸ ἅγιον, τρυφερὸν ὄν, στενοχωρεῖται, μὴ ἔχον τὸν τόπον καθαρόν, καὶ ζητεῖ ἀποστῆναι ἐκ τοῦ τόπου· πνίγεται γὰρ ὑπὸ τοῦ πονηροῦ πνεύματος, μὴ ἔχον τόπον λειτουργῆσαι τῷ κυρίῳ, καθὼς βούλεται, μιαινόμενον ὑπὸ τῆς ὀξυχολίαςἐν γὰρ τῇ μακροθυμίᾳ ὁ κύριος κατοικεῖ, ἐν δὲ τῇ οξυχολίᾳ ὁ διάβολοςἀμφότερα οὖν τὰ πνεύματα ἐπὶ τὸ αὐτὸ κατοικοῦντα, ἀσύμφορόν ἐστιν καὶ πονηρὸν τῷ ἀνθρώπῳ ἐκείνῳ, ἐν ᾧ κατοικοῦσινἐὰν γὰρ λάβῃς ἀψινθίου μικρὸν λίαν καὶ εἰς κεράμιον μέλιτος ἐιχεῃς, οὐχὶ ὅλον τὸ μέλι ἀφανίζεται, καὶ τοσοῦτον μέλι ὑπὸ τοῦ ἐλαχίστου ἀψινθίου ἀπόλλυται καὶ ἀπόλλυσι τὴν γλυκύτητα τοῦ μέλιτος, καὶ οὐκέτι τὴν αὐτὴν χάριν ἔχει παρὰ τῷ δεσπότῃ, ὅτι ἐπικράθη καὶ τῆν χρῆσιν αὐτοῦ ἀπώλεσεν; ἐὰν δὲ εἰς τὸ μέλι μὴ βληθῇ τὸ ἀψίνθιον, γλυκὺ εὑρίσκεται τὸ μέλι καὶ εὔχρηστον γίνεται τῷ δεσπότῃ αὐτοῦ.́1 βλέπεις ὅτι ἡ μακροθυμία γλυκυτάτη ἐστὶν ὑπὲρ τὸ μέλι καὶ εὔχρηστός ἐστι τῷ κυρίῳ, καὶ ἐν αὐτῇ κατοικεῖὑ δὲ ὀξυχολία πικρὰ καὶ ἄχρηστός ἐστινἐὰν οὖν μιγῇ ἡ ὀξυχολία τῇ μακροθυμίᾳ, μιαίνεται ἡ μακροθυμία καὶ οὐκέτι εὔχρηστός ἐστι τῷ θεῷ ἡ ἔντευξις αὐτῆςἬθελον, φημί, κύριε, γνῶναι τὴν ἐνέργειαν τῆς ὀξυχολίας, ἵνα φυλάξωμαι ἀπ’ αὐτῆςΚαὶ μήν, φησίν, ἐὰν μὴ φυλάξῃ ἀπ’ αὐτῆς σὺ καὶ ὁ οἶκός σου, ἀπώλεσάς σου τὴν πᾶσαν ἐλπίδαἀλλὰ φύλαξαι ἀπ’ αὐτῆς· ἐγὼ γὰρ μετὰ σοῦ εἰμίκαὶ πάντες δὲ ἀφέξονται ἀπ’ αὐτῆς, ὅσοι ἂν μετανοήσωσιν ἐξ ὅλης τῆς καρδίας αὐτῶν· μετ’ αὐτῶν γὰρ ἔσομαι καὶ συντηρήσω αὐτούς· ἐδικαιώθησαν γὰρ πὰντες ὑπὸ τοῦ σεμνοτάτου ἀγγέλου.
Ἄκουε νῦν φησί, τὴν ἐνέρειαν τῆς ὀξυχολίας πῶς πονηρά ἐστι, καὶ πῶς τοὺς δούλους μοῦ καταστρέφει τῇ ἑαυτῆς ἐνεργείᾳ καὶ πῶς ἀποπλανᾷ δὲ τοὺς πλήρεις ὄντας ἐν τῇ πίστει οὐδὲ ἐνεργῆσαι δύναται εἰς αὐτούς, ὅτι ἡ δύναμις μου μετ’ αὐτῶν ἐστιν· ἀποπλανᾷ δὲ τοὺς ἀποκένους καί διψύχους ὄνταςὅταν δὲ ἴδῃ τοὺς τοιούτους ἀνθρώπους εὐσταθοῦντας, παρεμβάλλει ἑαυτὴν εἰς τὴν καρδίαν τοῦ ἀνθρώπου ἐκείνου, καὶ ἐκ τοῦ μηδενὸς ὁ ἀνὴρ ἢ γυνὴ ἐν πικρίᾳ γινετα ἕνεκεν βιωτικῶν πραγμάτων ἢ περὶ ἐδεσμάτων ἢ μικρολογίας τινὸς ἢ περὶ φίλου τινὸς ἢ περὶ δόσεως ἢ λήψεως ἢ περὶ τοιούτων μωρῶν πραγμάτων· ταῦτα γὰρ πάντα μωρὰ ἐστι καὶ κενὰ καὶ ἄφρονα καὶ ἀσύμφορα τοῖς δούλοις τοῦ θεοῦἡ δὲ μακροθυμία μεγάλη ἐστὶ καὶ ἰσχυρὰ καὶ δύναμιν ἔχουσα καὶ στιβαρὰν καὶ εὐθηνουμένην ἐν πλατυσμῷ μεγάλῳ, ἱλαρά, ἀγαλλιωμένη, ἀμέριμνος οὖσα, δοξάζουσα τὸν κύριον ἐν παντὶ καιρῷ, μηδὲν ἐν ἑαυτῇ ἔχουσα πικρόν, παραμένουσα διὰ παντὸς πραεῖα καὶ ἡσύχιος· αὕτη οὖν ἡ μακροθυμία κατοικεῖ μετὰ τῶν τὴν πίστιν ἐχόντων ὁλόκληρονἡ δὲ ὀξυχολία πρῶτον μὲν μωρά ἐστιν, ἐλαφρά τε καὶ ἄφρωνεἶτα ἐκ τῆς ἀφροσύνης γίνεται πικρία, ἐκ δὲ τῆς πικρίας θυμός, ἐκ δὲ τοῦ θυμοῦ ὀργή, ἐκ δὲ τῆς ὀργῆς μῆνις· εἶταἡ̓ἡ μῆνις αὕτη ἐκ τοσούτων κακῶν συνισταμένη γίνεται ἁμαρτία μεγάλη καὶ ἀνίατοςὅταν γὰ ταῦτα τὰ πνεύματα ἐν ἑνὶ ἀγγείῳ κατοικῇ, οὗ καὶ τὸ πνεῦμα τὸ ἅγιον κατοικεῖ, οὐ χωριε͂ τὸ ἄγγος ἐκεῖνο, ἀλλ’ ὑπερπλεονάζειτὸ τρυφερὸν οὖν πνεῦμα, μἡ ἔχον συνήθειαν μετὰ πονηροῦ πνεύματος κατοικεῖν μηδὲ μετὰ σκληροτητος, αποχωρει ἀπὸ τοῦ ἀνθρώπου τοῦ τοιούτου καὶ ζητεῖ κατοικεῖν μετὰ πραότητος καὶ ἡσυχίαςεἶτα ὅταν ἀποστῇ ἀπὸ τοῦ ἀνθρώπου ἐκεῖνου, οὗ κατοικεῖ, γίνεται ὁ ἄθρωπος ἐκεῖνος κενὸς ἀπὸ τοῦ πνεύματος τοῦ δικαίου, καὶ τὸ λοιπὸν πεπληρωμένος τοῖς πνεύμασι τοῖς πονηροῖς ἀκαταστατεῖ ἐν πάσῃ πράξει αὐτοῦ, περισπώμενος ὧδε κἀκεῖσε ἀπὸ τῶν πνευμάτων τῶν πονηρῶν καὶ ὅλως ἀποτυφλοῦται ἀπὸ τῶν πνευμάτων τῶν πονηρῶν, καὶ ὅλως ἀποτυφλοῦται ἀπὸ τῆς διανοίας τῆς ἀγαθῆςοὕτως οὖν συμβαίνει πᾶσι τοῖς ὀξυχόλοιςἀπέχου οὖν ἀπὸ τῆς ὀξυχολίᾳ καὶ τῇ πικρίᾳ, καὶ ἔσῃ εὑρισκόμενος μετὰ τῆς σεμνότητος τῆς ἠγαπημένης ὑπὸ τοῦ κυρίουβλέπε οὖν μήποτε παρενθυμηθῇς τὴν ἐντολὴν ταύτην· ἐὰν γὰρ ταύτης τῆς ἐντολῆς κυριεύσῃς, καὶ τὰς λοιπὰς ἐντολὰς δυνήσῃ φυλάξαι, ἅς σοι μέλλω ἐντέλλεσθαιἰσχυροῦ ἐν αὐταῖς καὶ ἐνδυναμοῦ, καὶ πάντες ἐνδυναμούσθωσαν, ὅσοι ἐὰν θέλωσιν ἐν αὐταῖς πορεύεσθαι.

Ἐντολὴ ς’.
Ἐντειλάμην σοι, φησίν, ἐν τῇ πρώτῃ ἐντολῇ, ἵνα φυλάξῃς τὴν πίστιν καὶ τὸν φόβον καὶ τὴν ἐγκράτειανΝαί, φημί, κύριεἈλλὰ νῦν θέλω σοι, φησίν, δηλῶσαι καὶ τὰς δυνάμεις αὐτῶν, ἵνα νοήσῃς τίς αὐτῶν τίνα δύναμιν ἔχει καὶ ἐνέργειαν· διπλαῖ γάρ εἰσιν αἱ ἐνέργειαι αὐτῶνκεῖνται οὖν ἐπὶ δικαίῳ καὶ ἀδίκῳ· σὺ οὖν πίστευε τῷ δικαίῳ, τῷ δὲ ἀδίκῳ μὴ πιστεύσῃς· τὸ γὰρ δίκαιον ὀρθὴν ὁδὸν ἔχει, τὸ δὲ ἄδικον στρεβλήνἀλλὰ σὺ τῇ ὀρθῇ ὁδῷ πορεύου καὶ ὁμαλῇ, τὴν δὲ στρεβλὴν ἔασονἡ γὰρ στρεβλὴ ὁδὸς τρίβους οὐκ ἔχει, ἀλλ’ ἀνοδίας καὶ προσκόμματα πολλὰ καὶ τραχεῖά ἐστι καὶ ἀκανθώδηςβλαβερὰ οὖν ἐστι τοῖς ἐν αὐτῇ ορευομένοιςοἱ δὲ τῇ ὀρθῇ ὁδῷ πορευόμενοι ὁμαλῶς περιπατοῦσι καὶ ἀπροσκόπως· οὔτε γὰρ τραχεῖά ἐστιν οὔτε ἀκανθώδηςβλέπεις οὖν, ὅτι συμφορώτερόν ἐστι ταύτῃ τῇ ὁδῷ πορεύεσθαιΠορεύσῃ, φησί, καὶ ὃς ἂν ἐξ ὅλης καρδίας ἐπιστρέψῃ πρὸς κύριον, πορεύσεται ἐν αὐτῇ.
Ἄκουε νῦν, φησί, περὶ τῆς πίστεως, δύο εἰσὶν ἄγγελοι μετὰ τοῦ ἀνθρώπου, εἷς τῆς δικαιοσύνης καὶ εἷς τῆς πονηρίαςΠῶς οὖν, φημί, κύριε, γνώσομαι τὰς αὐτῶν ἐνεργείας, ὅτι ἀμφότεροι ἄγγελοι μετ’ ἐμοῦ κατοικοῦσιν; Ἄκουε, φησί, καὶ συνιεῖς αὐτάςὁ μὲν τῆς δικαιοσύνης ἄγγελος τρυφερόσ ἐστι καὶ αἰσχυντηρὸς καὶ πραῢς καὶ ἡσύχιος· ὅταν οὖν οὗτος ἐπὶ τὴν καρδίαν σον ἀναβῇ, εὐθέως λαλεῖ μετὰ σοῦ περὶ δικαιοσύνης, περὶ ἁγνείας, περὶ σεμνότητος καὶ περὶ αὐταρκείας καὶ περὶ παντός ἔργου δικαίου καὶ περὶ πάσης ἀρετῆς ἐνδόξυταῦτα πάντα ὅταν εἰς τὴν καρδίαν σου ἀναβῇ, γίνωσκε, ὅτι ὁ ἄγγελος τῆς δικαιοσύνης μετὰ ςοῦ ἐστίταῦτα οὖν ἐστι τὰ ἔργα τοῦ ἀγγέλου τῆς δικαιοσύνηςτούτῳ οὖν πίστευε καὶ τοῖς ἔργοις αὐτοῦὅρα οὖν καὶ τοῦ ἀγγέλου τὰ ἔργαπρῶτον πάντων ὀξύχολός ἐστι καὶ πικρὸς καὶ ἄφρων, καὶ τὰ ἔργα αὐτοῦ πονηρά, καταστρεφοντα τοὺς δούλους τοῦ θεοῦ· ὅταν οὖν οὗτος ἐπὶ τὴν καρδίαν σου ἀναβῇ, γνῶθι αὐτὸν ἀπὸ τῶν ἔργων αὐτοῦΠῶς, φημί, κύριε νοήσω αὐτόν, οὐκ ἐπίσταμαιἌκουε, φησίνὅταν ὀξυχολία σοί τις προσπέσῃ ἢ πικρία, γίνωσκε, ὅτι αὐτός ἐστιν ἐν σοί· εἶτα ἐπιθυμία πράξεων πολλῶν καὶ πολυτέλειαι ἐδεσμάτων πολλῶν καὶ μεθυσμάτων καὶ κραιπαλῶν πολλῶν καὶ ποικίων τροφῶν καὶ οὐ δεόντων καὶ ἐπιθυμίαι γυναικῶν καὶ πλεονεξιῶν καὶ ὑπερηφανία πολλή τις καὶ ἀλαζονεία καὶ ὅσα τούτοις παραπλήσιά ἐστι καὶ ὅμοια· ταῦτα οὖν ὅταν ἐπὶ τὴν καρδίαν σου ἀναβῇ, γίνωσκε, ὅτι ὁ ἄγγελος τῆς πονηρίας ἐστὶν ἐν σοίσὺ οὖν ἐπιγνοὺς τὰ ἔργα αὐτοῦ ἀπόστα ἀπ’ αὐτοῦ πονηρά εἰσι καὶ ἀσύμποφα τοῖς δοῦλοις τοῦ θεοῦ, ἔχεις οὖν ἀμφοτέρων τῶν ἀγγέλων τὰς ἐνεργείας· αυνιε αὐτὰς καὶ πίστευε τῷ ἀγγέλῳ τῆς δικαιοσύνης· ἀπὸ δὲ τοῦ ἀγγέλου τῆς πονηρίας ἀπόστηθι, ὅτι ἡ διδαχὴ αὐτου τούτου ἀναβῇ ἐπὶ τὴν καρδίαν αὐτοῦ τὰ ἔργα τοῦ ἀγγέου τῆς δικαιοσύνης, ἐξ ἀνάγκης δεῖ αὐτὸν ἀγαθόν τι ποιῆσαιβλέπεις οὖν, φησίν, ὅτι καλόν ἐστι τῷ ἀγγελῳ τῆς δικαιοσύνης ἀκολουθεῖν, τῷ δὲ ἀγγέλῳ τῆς πονηρίας ἀποτάξασθαιτὰ μὲν περὶ τῆς πιπστεως αὕτη ἡ ἐντολὴ δηλοῖ, ἵνα τοῖς ἔργοις τοῦ ἀγγέλου τῆς δικαιοσύνης πιστεύσῃς, καὶ ἐργασάμενος αὐτὰ τοῦ ἀγγέλλου τῆς πονηρίας χαλεπά ἐστι· μὴ ἐργαζόμενος οὖν αὐτὰ ζήσῃ τῷ θεῷ.

Ἐντολὴ ζ’.
Φοβήθητι, φησί, τὸν κύριον καὶ φύλασσε τὰς ἐντολὰς αὐτοῦφυλάσσων οὖν τὰς ἐντολὰς τοῦ θεοῦ ἔσῃ δυνατὸς ἐν πάσῃ πράξει, καὶ ἡ πρᾶξίς σου ἀσύγκριτος ἔσταιφοβούμενος γὰρ τὸν κύριον πάντα καλῶς ἐργάσῃ· οὗτος δέ ἐστιν ὁ φόβοςὃν δεῖ σε φοβηθῆναι, καὶ σωθῆναιτὸν δὲ διάβολον μὴ φοβηθῇς· φοβούμενος γὰρ τὸν κύριον κατακυριεύσεις τοῦ διαβόλου, ὅτι δύναμις ἐν αὐτῷ οὐκ ἔστινἐν ᾧ δὲ δύναμις ἡ ἔνδοξος, καὶ φόβος ἐν αὐτῷπᾶς γὰρ ὁ δύναμιν ἔχων φόβον ἔχει· ὁ δὲ μὴ ἔχων δύναμιν ὑπὸ πάντων καταφρονεῖταιφοβήθητι δὲ τὰ ἔργα τοῦ διαβόλου, ὅτι πονηρά ἐστιφοβούμενος οὖν τὸν κύριον οὐκ ἐργάσῃ αὐτά, ἀλλ’ ἐφέξῃ ἀπ’ αὐτῶνδισσοὶ οὖν εἰσιν οἱ φόβοι· ἐὰν γὰρ θέλῃς τὸ πονηρὸν ἐργάσῃ αὐτό· ἐὰν δὲ θέλῃς πάλιν τὸ ἀγαθὸν ἐργάσασθαι, φοβοῦ τὸν κύριον, καὶ ἐργάσῃ αὐτόὥστε ὁ φόβος τοῦ κυρίου ἰσχυρός ἐστι καὶ μέγας καὶ ἔνδόξοςφοβήθητι οὖν τὸν κύριον, καὶ ζήσῃ αὐτῷ· καὶ ὅσοι ἂν φοβηθῶσιν αὐτὸν καὶ τηρήσωσι τὰς ἐντολὰς αὐτοῦ, ζήσονται τῷ θεῷΔιατί, φημί, κύριε, εἶπας περὶ τῶν τηρούντων τὰς ἐντολὰς αὐτοῦ· Ζήσονται τῷ θεῷ; Ὅτι, φησίν, πᾶσα ἡ κτίσις φοβεῖται τὸν κύριον τὰς δὲ ἐντολὰς αὐτοῦ οὐ φυλάσσειτῶν οὖν φοβουμένων αὐτὸν καὶ φυλασσόντων τὰς ἐντολὰς αὐτοῦ, ἐκείνων ἡ ζωή ἐστι παρὰ τῷ θεῷ· τῶν δὲ μὴ φυλασσόντων τὰς ἐντολὰς αὐτοῦ, οὐδὲ ζωὴ ἐν αὐτῷ.

Ἐντολὴ η’.
Εἶπόν σοι, φησίν, ὅτι κτίσματα τοῦ θεού διπλᾶ ἐστι· καὶ γὰρ ἡ ἐγκράτεια διπλῆ ἐστινἐπί τινων γὰρ δεῖ ἐγκρατεύσεύσθαι, ἐπίτινων δὲ οὐ δεῖ· Γνώρισόν μοι, φημί, κυριε, ἐπὶ τίνων δεῖ ἐγκρατεύεσθαι, ἐπὶ τίνον δὲ οὐ δεῖἌκουε, φησί, τὸ πονηρὸν ἐγκρατεύου καὶ μὴ ποίει αὐτό· τὸ δὲ ἀγαθὸν μὴ ἐγκρατεύου, ἀλλὰ ποιει αὐτόἐὰν γὰρ ἐγκρατεύσῃ τὸ ἀγαθὸν μὴ ποιεῖν, ἁμαρτίαν μεγάλην ἐργάζῃ· ἐὰν δὲ ἐγκρατεύσῃ τὸ πονηρὸν μὴ ποιεῖν, δικαιοσύνην μεγάλην ἐργάζῃἐγκράτευσαι οὖν ἀπὸ πονηρίας πάσης ἐργαζόμενος τὸ ἀγαθόνΠοταπαί, φημί, κύριε, εἰσὶν αἱ πονηρίαι, ἀφ’ ὧν ἡμᾶς δεῖ ἐγκρατεύεσθαι; Ἄκουε, φησίν· ἀπὸ μοιχείας καὶ πορνείας ἀπὸ μεθύσματος ἀνομίας, ἀπὸ μοιχείας καὶ πορνείας, ἀπό ἐδεσμάτων πολλῶν καὶ ὑψηλοφροσύνης καὶ ὑπερηφανίας καὶ απὸ ψεύματος και καταλαλιᾶς καὶ ὑποκρίσεως, μνησικακίας καὶ πάσης βλασφημίαςταῦτα τὰ ἔργα πάντων πονηρότατά εἰσιν ἐν τῇ ζωῃ τῶν ἀνθρώπωνἀπὸ τούτων οὖν τῶν ἔργων δεῖ ἐγκρατεύεσθαι τὸν δοῦλον τοῦ θεοῦ· ὁ γὰρ μὴ ἐγκρατευόμενος ἀπὸ τούτων οὐ δύναται ζῆσαι τῷ θεῷἄκουε οὖν καὶ τὰ ἀκολουθα τούτων, Ἔτι γάρ, φημί, κύριε, πονηρὰ ἔργα ἐστί; Καί γε πολλά, φησίν, ἔστιν, ἀφ’ ὧν δεῖ τὸν δοῦλον τοῦ θεοῦ ἐγκρατεύεσθαι· κλέμμα, ψεῦδος, ἀποστρέρησις, ψευδομαρτυρία, πλεονεξία, ἐπιθυμία πονηρά, ἀπάτη, κενοδοξία, ἀλαζονεία καὶ ὅσα τούτοις ὅμοιά εἰσινοὐ δοκεῖ σοι ταῦτα πονηρὰ εἶναι; καὶ λίαν πονηρά, φημί, τοῖς δούλοις τοῦ θεοῦτούτων πάντων πάντων δεῖ ἐγκρατεύεσθαι τὸ δουλεύοντα τῷ θεῳἐγκράτευσαι οὖν δεῖ σε ἐγκρατευεσθαι, ταῦτά ἐστινἃ δὲ δεῖ σε μὴ ἐγκρατεύεσθαι, φησίν, ἀλλὰ ποίει αὐτόΚαὶ τῶν ἀγαθῶν μοι, φημί, κύριε, δήλωσον τὴν δύναμιν, ἵινα πορευθῶ ἐν αὐτοῖς καὶ δουλεύσω αὐτοῖς, ἵνα πορευθῶ ἐν αὐτοῖς καὶ δουλεύσω αὐτοῖς, ἵνα ἐργασάμενος αὐτὰ δυνηθῶ σωθῆναιἌκουε, φησί, καὶ τῶν ἀγαθῶν τὰ ἔργα, ἃ σε δεῖ ἐργάζεσθαι καὶ μὴ ἐγκρατεύεσθαιπρῶτον πάντων πίστις, φόβος κυρίου, ἀγάπη, ὁμόνοια, ῥήματα δικαιοσύνης, ἀλήθεια, ὑπομονή· τούτων ἀγαθώτερον οὐδέν ἐστιν ἐν τῇ ζωῇ τῶν ἀνθρώπωνταῦτα ἐὰν τις φυλάσσῃ καὶ μὴ ἐγκρατεύηται ἀπ’ αὐτῶν, μακάριος γίνεται ἐν τῇ ζωῇ αὐτοῦεἶτα τοῦτων τὰ ἀκόλουθα ἄκουσον· χήραις ὑπηρετεῖν, ὀρφανοὺς καὶ ὑστερουμένους ἐπισκέτεσθαι, ἐξ ἀναγκῶν λυτροῦσθαι τοὺς δούλους τοῦ θεοῦ, φιλόξενον εἶναι (ἐν γὰρ τῇ φιλοξενίᾳ εὑρίσκεται ἀγαθοποίησίς ποτε), μηδενὶ ἀντιτάσσεσθαι, ἡσύχιον εἶναι, ἐνδεέστερον γίνεσθαι πάντων ἀνθρωπων, πρεσβύτας σέβεσθαιδικαιοσύνην ἀσκεῖν, ἀδελφότητα συντηρεῖν, ὕβριν ὑποφέειν, μακρόθυμον εἶναι, μνησικακιαν μὴ ἔχειν, κάμνοντας τῇ ψυχῇ παρακαλεῖν, ἐσκανδαλισμένους ἀπὸ τῆς πίστεως μὴ ἀποβάλλεσθαι, ἀλλ’ ἐπιστρέφειν καὶ εὐθύμους ποιεῖν, ἁμαρτάνοντας νουθετεῖν, χρεώστας μὴ θλίβειν καὶ ἐνδεεῖς, καὶ εἴ τινα τούτοις ὅμοιά ἐστιδοκεῖ σοι, φησί, ταῦτα ἀγαθὰ εἶναι; Τί γάρ, φημί, κύριε, τούτων ἀγαθώτερον; Πορεύου οὖν, φησίν, ἐν αὐτοῖς καὶ μὴ ἐγκρατεύου ἀπ’ αὐτῶν, καὶ ζήσῃ τῷ θεῷ· φύλασσε οὖν τὴν ἐντολὴν ταύτην· ἐὰν τὸ ἀγαθὸν ποιῇς καὶ μὴ ἐγκρατεύσῃ ἀπ’ αὐτοῦ, ζήσῃ τῷ θεῷ, καὶ´πάντες ζήσονται τῷ θεῷ οἱ οὕτω ποιοῦντεςκαὶ πάλιν ἐὰν τὸ πονηρὸν μὴ ποιῇς καὶ ἐγκρατεύσῃ ἀπ’ αὐτοῦ, ζήσῃ τῷ θεῷ, καὶ πάντες ζήσονται τῷ θεῷ, ὅσοι ἐὰν ταύτας τὰς ἐντολὰς φυλάξωσι καὶ πορευθῶσιν ἐν αὐταῖς.

Ἐντολὴ θ’.
Λέγει μοι· Ἆρον ἀπὸ σεαυτοῦ τὴν διψυχίαν καὶ μὲν ὅλως διψυχήσῃς αἰτήσασθαί τι παρὰ τοῦ θεου, λέγων ἐν σεαυτῷ ὅτι πῶς δύναμαι αἰτήσασθαι παρὰ τοῦ κυρίου καὶ λαβεῖν, ἡμαρτηκὼς τοσαῦτα εἰς αὐτόν; μὴ διαλογίζου ταῦτα, ἀλλ’ ἐξ ὅλης τῆς καρδίας σου ἐπίστρεψον ἐπὶ τὸν κύριον καὶ αἰτοῦ παρ’ αὐτοῦ ἀδιστάκτως, καὶ γνώσῃ τὴν πολλὴν εὐσπλαγχνίαν αὐτοῦ, ὅτι οὐ μή σε ἐγκαταλίπῃ, ἀλλὰ τὸ αἴτημα τῆς ψυχῆς σου πληροφορησειοὐκ ἔστι γὰρ ὁ θεὸς ὡς οἱ ἄνθρωποι μνησικακοῦντες, ἀλλ’ ἀυτὸς ἀμνησίκακός ἐστι καὶ σπλαγχνίζεται ἐπὶ τὴν ποίησιν αὐτουσὺ οὖν καθάρισόν σου τὴν καρδίαν ἀπὸ πάντων τῶν ματαωμάτων τοῦ αἰῶνος τούτου καὶ τῶν προειρημένων σοι ῥημάτων καὶ αἰτοῦ παρὰ τοῦ κυρίου, καὶ ἀπολήψῃ πάντα καὶ ἀπὸ πάντων τῶν αἰτημάτων σου ἀνυστέρητος ἔσῃ, ἐὰν ἀδιστάκτως αἰτήσῃς παρὰ τοῦ κυρίουἐὰν δὲ διστάσῃς ἐν τῇ καρδίᾳ σου, οὐδὲν οὐ μὴ λήψῃ τῶν αἰτημάτων σουοἱ γὰρ διστάζοντες εἰς τὸν θεόν, οὗτοί εἰσιν οἱ δίψυχοι καὶ οὐδὲν ὅλως ἐπιτυγχάνουσι τῶν αἰτημάτων αὐτῶνοἱ δὲ ὁποτελεῖς ὄντες ἐν τῇ πίστει πάντα αἰτοῦνται πεποιθότες ἐπὶ τὸν κύριον καὶ λαμβάνουσιν, ὅτι ἀδιστάκτως αἰτοῦνται, μηδὲν διψυχοῦντεςπᾶς γὰρ δίψυχος ἀνηρ, ἐὰν μὴ μετανοήσῃ, δυσκόλως σωθήσεταικαθάρισον οὖν τὴν καρδίαν σου ἀπὸ τῆς διψυχίας, ἔνδυσαι δὲ τὴν πίστιν, ὅτι ισχυρά ἐστι, καὶ πίστευε τῷ θεῷ, ὅτι πάντα τὰ αἰτήματά σου ἃ αἰτεῖς λήψῃ, καὶ ἐὰν αἰτησάμενός ποτε παρὰ τοῦ κυρίου αἴτημά τι βραδύτερον λαμβάνῃς, μὴ διψυχήσῃς, ὅτι ταχὺ οὐκ ἔλαβες τὸ αἰτημα τῆς ψυχῆς σου· πάντως γὰρ διὰ πειρασμόν τινα ἢ παράπτωμά τι, ὃ σὺ ἀγνοεῖς, βραδύτερον λαμβάνεις τὸ αἴτημά σουσὺ οὖν μὴ διαλίπῃς αὐτό· ἐὰν δὲ ἐκκακήσῆς καὶ διψυχήσῃς αἰτούμενος, σεαυτὸν αἰτιῶ καὶ μὴ τὸν διδόντα σοιβλέε τὴν διψυσίαν ταύτην· πονηρὰ γάρ ἐστι καὶ ἀσύνετος καὶ πολλοὺς ἐκριζοῖ ἀπὸ τῆς πίστεως καί γε λίαν πιστοὺς καὶ ἰχυρούςκαὶ γὰρ αὕτη ἡ διψυχία θυγάτηρ ἐστὶ τοῦ διαβόλου καὶ λίαν πονηρεύεται εἰς τοὺς δούλους τοῦ θεοῦκαταφρόνησον οὖν τῆς διψυχίας καὶ κατακυρίευσον αὐτῆς ἐν παντὶ πράγματι, ἐνδυσάμενος τὴν πίστιν τὴν ἰσχυρὰν καὶ δυνατήν· ἡ γὰρ πίστις πάντα ἐπαγγέλλεται, πάντα τελειοῖ, ἡ δὲ διψυχία μὴ καταπιστεύουσα ἑαυτῇ πάντων ἀποτυγχάνει τῶν ἔργων αὐτῇς ὧν πράσσειβλέπεις οὖν, φησίν, ὅτι ἡ πίστις ἄνωθέν ἐστι παρὰ τοῦ κυρίου καὶ ἔχει δύναμιν μεγάλην· ἡ δὲ διψυσία ἐπίγειον πνεῦμά ἐστι παρὰ τοῦ διαβόλου, δύναμιν μὴ ἔχουσασὺ οὖν δούλευε τῇ ἐχούσῃ δύναμιν τῇ πίστει καὶ ἀπὸ τῆς διψυχίας ἀπόσχου τῆς μὴ ἐχούσης δύναμιν, καὶ ζήσῃ τῷ θεῷ, καὶ πάντες ζήσονται τῷ θεῷ οἱ ταῦτα φρονοῦντες

Ἐντολὴ ι’
Ἄρον ἀπὸ σεαθτοῦ, φησί, τὴν λύπην· καὶ γὰρ αὕτη ἀδελφή ἐστι τῆς δίψυχίας καὶ τῆς ὀξυχολίαςΠῶς, φημί, κύριε, ἀδελφή ἐστι τούτων; ἄλλο γάρ μοι δοκεῖ εἶναι ὀξυχολία καὶ ἄλλο διψυχία καὶ ἄλλο λύπηἈσύνετος εἶ ἄνθρωπε, φησί, καὶ οὐ νοεῖς, ὅτι ἡ λύπη πάντων τῶν πνευματων πονηροτέρα ἐστὶ καὶ δεινοτάτη τοῖς δούλοις τοῦ θεοῦ καὶ παρὰ πάντα τὰ πνεύματα καταφθείρει τὸν ἄνθρωπον καὶ ἐκτρίβει τὸ πνεῦμα τὸ ἅγιον καὶ πάλιν σώζει; Ἐγώ, φημί, κύριε, ἀσύνετός εἰμι καὶ οὐ συνίω τὰς παραβολὰς ταύταςπῶς γὰρ δύναται ἐκτρίβειν και πάλιν σώζειν, οὐ νοῶἌκουε, φησίν· οἱ μηδέποτε ἐρευνήσαντες περὶ τῆς ἀηθείας μηδὲ ἐπιζητήσαντες περὶ τῆς θεότητος, πιστεύσαντες δὲ μόνον, ἐμπεφυρμένοι δὲ πραγματείαις καὶ πλούτῳ καὶ φιλίαις ἐθνικαῖς καὶ ἄλλαις πολλαῖς πραγματείαις τοῦ αἰῶνος τούτου· ὅσοι οὖν τούτοις προσκεινται, οὐ νοοῦσι τὰς παραβολὰς τῆς θεότητος· ἐπισκοτοῦνται γὰρ ὑπό τούτων τῶν πράξεων καὶ καταφθείρονται καὶ γίνονται κεχερσωμένοικαθὼς οἱ ἀμπελῶνες οἱ καλοί, ὅταν ἀμελείας τύχωσι, χερσοῦνται ἀπὸ τῶν ἀκανθῶν καὶ βοτανῶν ποικίλων, οὕτως οἱ ἀνθρωποι οἱ πιστεύσαντες καὶ εἰς ταύτας τὰς πράξεις τὰς πολλὰς ἐμπίπτοντες τὰς προειρημένας, ἀποπλανῶνται ἀπὸ τῆς διανοίας αὐτῶν, καὶ οὐδὲν ὅλως νοοῦσι περὶ δικαιοσύνης, ἀλλὰ καὶ ὅταν ἀκούσωσι περὶ θεότητος καὶ ἀληθείας, καὶ ἀληθείας καὶ οὐδὲν ὅλως νοοῦσινοἱ δὲ φόβον ἔχοντες θεοῦ καὶ ἐρευνῶντες περὶ θεότητος καὶ ἀληθείας καὶ τὴν καρδίαν ἔχοντες πρὸς τὸν κύριον, πάντα τὰ λεγόμενα αὐτοῖς τάχιον νοοῦσι καὶ συνίουσιν, ὅτι ἔχουσι τὸν φόβον τοῦ κυρίου ἐν ἑαυτοῖς· ὅπου γὰρ ὁ κύριος κατοικεῖ, ἐκεῖ καὶ σύνεσις πολλήκοολήθητι οὖν τῷ κυρίῳ, καὶ πάντα συνήσεις καὶ νοήσεις.
Ἄκουε οὖν, φησίν, ἀνόητε, πῶς ἡ λύπη ἐκτρίβει τὸ πνεῦμα τὸ ἅγιον καὶ πάλιν σώζει· ὅταν ὁ δίψυχος ἐπιβάληται πρᾶξίν τινα καὶ ταύτης ἀποτύχῃ διὰ τὴν διωυσίαν αὐτοῦ, ἡ λύπη αὕτη εἰσπορεύεται εἰς τὸν ἄνθρώπον καὶ λυπεῖ τὸ πνεῦμα τὸ ἅγιον καὶ ἐκτρίβει αὐτόεἶτα πάλιν ἡ ὀξυχολία ὅταν κολληθῇ τῷ ἀνθρώπῳ περὶ πράγματός τινος, καὶ λίαν πικρανθῇ, πάλιν ἡ λύπη εἰσπορεύεται εἰς τὴν καρδίαν τοῦ ἀνθρώπου τοῦ ὀξυχολήσαντος, καὶ λυπεῖται ἐπὶ τῇ πράξει αὐτοῦ ᾗ ἔπραξε καὶ μετανοεῖ, ὅτι πονηρὸν εἰργάσατοαὕτη οὖν ἡ λύπη δοκεῖ σωτηρίαν ἔχειν, ὅτι τὸ πονηρὸν πράξας μετενόησενἀμφότεραι οὖν αἱ πράξεις λυποῦσι τὸ πνεῦμα· ἡ μὲν διψυχία, ὅτι οὐκ ἐπέτυχε τῆς πράξεως αὐτῆς, ἡ δὲ ὀξυχολία λυπεῖ τὸ πνεῦμα, ὅτι ἔπραξε τὸ πονηρόνἀμφότερα οὖν λυπηρά ἐστι τῷ πνεύματι τῷ ἁγίῳ, ἡ διψυχία καὶ ἡ ὀξυχολίαἆρον οὖν ἀπὸ σεαυτοῦ τὴν λύπην καὶ μ̀μὴ θλῖβε τὸ πνεῦμα τὸ ἅγιον τὸ ἐν σοὶ κατοικοῦν, μήποτε ἐντεύξηται τῷ θεῷ καὶ ἀποστῇ ἀπὸ σοῦτὸ γὰρ πνεῦμα τοῦ θεοῦ τὸ δοθὲν εἰς τὴν σάρκα ταύτην λύπην οὐχ ὑποφέρει οὐδὲ στενοχωρίαν.
Ἔνδυσαι οὖν τὴν ἱλαρότητα, τὴν πάντοτε ἔχουσαν χάριν παρὰ τῷ θεῷ καὶ εὐπρόσδεκτον οὖσαν αὐτῷ, καὶ ἐντρύφα ἐν αὐτῇπᾶς γὰρ ἱλαρὸς ἀνὴρ ἀγαθὰ ἐργάζεται καὶ ἀγαθὰ φρονεῖ καὶ καταφρονεῖ τῆς λύπηςὁ δὲ λυπηρὸς ἀνὴρ πάντοτε πονηρεύεται· πρῶτον μὲν πονηρεύεται, ὅτι λυπεῖ τὸ πνεῦμα τὸ ἅγιον τὸ δοθὲν τῷ ἀνθρώπῳ ἱλαρὸν· δεύτερον δὲ λυπῶν τὸ πνεῦμα τὸ ἅγιον ἀνομίαν ἐργάζεται, μὴ ἐντυγχάνων μηδὲ ἐξομολογούμενος τῷ κυρίῳΠάντοτε γὰρ λυπηροῦ ἀνδρὸς ἡ ἔντευξις οὐκ ἔχει δύναμιν τοῦ ἀναβῆναι ἐπὶ τὸ ἔντευξις οὐκ ἔχει δύναμιν τοῦ ἀναβῆναι ἐπὶ τὸ θυσιαστήριον τοῦ θεοῦΔιατί, φημί, οὐκ ἀναβαίνει ἐπὶ τὸ θυσιαστήριον ἡ ἔντευξις τοῦ λυπουμένου; Ὅτι, φησίν, ἡ λύπη ἐγκάθηται εἰς τὴν καρδίαν αὐτουμεμιγμένη οὖν ἡ λύπη μετὰ τῆς ἐντεύξεως οὐκ ἀφίησι τὴν ἔντευξιν ἀναβῆναι καθαρὰν ἐπὶ τὸ θυσιαστήριονὥσπερ γὰρ ὄξος καὶ οἶνος μεμιγμένα ἐπὶ τὸ αὐτὸ τὴν αὐτὴν ἡδονὴν οὐκ ἔχουσιν, οὕτω καὶ ἡ λύπη μεμιγμένη μετὰ τοῦ ἁγίου πνεύματος τὴν αὐτὴν ἔντευξιν οὐκ ἔχεικαθάρισον οὖν σεαυτὸν ἀπὸ τῆς λύπης τῆς πονηρᾶς ταύτης, καὶ ζήσῃ τῷ θεῷ· καὶ πάντες ζήσονται τῷ θεῷ, ὅσοι ἂν ἀποβάλωσιν ἀφ’ ἑαυτῶν τὴν λύπην καὶ ἐνδύσωνται πᾶσαν ἱλαρότητα.

Ἐντολὴ ια’.
Ἔδειξέ μοι ἐπὶ συμψελλίου καθημένους ἀνθρώπους καὶ ἕτερον ἄνθρωπον καθήμενον ἐπὶ καθέδραν, καὶ λέγει μοι· Βλέπεις τοὺς ἐπὶ τοῦ συμψελλίου καθημένους; Βλέπω, φημί, κύριεΟὗτοι, φησί, πιστοί εἰσι, καὶ ὁ καθήμενος ἐπὶ τὴν καθέδραν ψευδοπροφήτης ἐστίν, ὃς ἀπόλλυσι τὴν διάνοιαν τῶν δούλων τοῦ θεοῦ· τῶν διψύχων δὲ ἀπόλλυσιν, οὐ τῶν πιστῶνοὗτοι οὖν οἱ δίψυχοι ὡς ἐπὶ μάντιν ἔρχονται καὶ ἐπερωτῶσιν αὐτόν, τί ἄρα ἔσται αὐτοῖς· κἀκεῖνος ὁ ψευδοπροφήτηςμηδεμίαν ἔχων ἐν ἑαυτῷ δύναμιν πνεύματος θείου, λαλεῖ μετ’ αὐτῶν κατὰ τὰ ἐπερωτήματα αὐτῶν και κατὰ τὰς ἐπιθυμίας τῆς πονηρίας αὐτῶν καί πληροῖ τὰς ψυχὰς αὐτῶν, καθὼς αὐτοὶ βούλανταιαὐτὸς γὰρ κενὸς ὢν κενὰ καὶ ἀποκρίνεται κενοῖς· ὃ γὰρ ἐὰν ἐπερωτηθῇ, πρὸς τὸ κένωμα τοῦ ἀνθρώπου ἀπροκρίνεταιτινὰ δὲ καὶ ῥήματα ἀληθῆ λαλεῖ· ὁ γὰρ διάβολος πληροῖ αὐτὸν τῷ αὐτοῦ πνεύματι, εἴ τινα δυνήσεται ῥῆξαι τῶν δικάωνὅσοι οὖν ἰσχυροί εἰσιν ἐν τῇ πίστει τοῦ κυρίου, ἐνδεδυμένοι τὴν ἀλήθειαν, τοῖς τοιούτοις πνεύμασιν οὐ κολλῶνται, ἀλλ’ ἀπέχονται ἀπ’ αὐτῶν· ὅσοι δὲ δίψυχοί εἰσι καὶ πυκνῶς μετανοοῦσι, μαντεύονται ὡς καὶ τὰ ἔθνη καὶ ἑαυτοῖς μείζονα ἁμαρτίαν ἐπιφέρουσιν εἰδωλολατροῦντες· ὁ γὰρ ἐπερωτῶν ψευδοπροφήτην περὶ πράξεώς τινος εἰδωλολοάτρης ἐστὶ καὶ κενὸς ἀπὸ τῆς ἀληθείας καὶ ἄφρων´πᾶν γὰρ πνεῦμα ἀπὸ θεοῦ δοθὲν οὐκ ἐπερωτᾶται, ἀλλὰ ἔχον τὴν δύναμιν τῆς θεότητος ἀφ’ ἑαυτοῦ λαλεῖ π́πάντα, ὅτι ἄνωθέν ἐστιν ἀπὸ τῆς δυνάμεως τοῦ θείου πνεύματοςτὸ δὲ πνεῦματ τὸ ἐπερωτώμενον καὶ λαλοῦν κατὰ τὰς ἐπιθυμίας τῶν ἀνθρώπων ἐπίγειόν ἐστι καὶ ἐλαφρόν, δύναμιν μὴ ἔχον· καὶ ὅλως οὐ λαλεῖ, ἐὰν μὴ ἐπερωτηθῇΠῶς οὖν, φημί, κύριε, ἄνθρωπος γνώσεται, τίς αὐτῶν προφήτης καὶ τίς ψευδοπροφήτης ἐστίν; Ἄκουε, φησί, περὶ ἀμφοτέρων τῶν προφητῶν· καὶ ὥς σοι μέλλω λέγειν, οὕτω δοκιμάσεις τὸν προφήτην καὶ τὸν ψευδοπροφήτηνἀπὸ τῆς ζωῆς δοκίμαζε τὸν ἄνθρωπον τὸν ἔχοντα τὸ πνεῦμα τὸ θεῖονπρῶτον μὲν ὁ ἔχων τὸ πνεῦμα τὸ ἄνωθεν πραΰς ἐστι καὶ ἡσύχιος καὶ ταπεινόφρων καὶ ἀπεχόμενος ἀπὸ πάσης πονηρίας καὶ ἐπιθυμίας ματαίας τοῦ αἰῶνος τούτου καὶ ἑαυτὸν ἐνδεέστερον ποιεῖ πάντων τῶν ἀνθρώπων καὶ οὐδενὶ οὐὲν ἀποκρίνεται ἐπερωτώμενος, οὐδὲ καταμόνας λαλεῖ, οὐδὲ ὅταν θέλῃ ἄνθρωπος λαλεῖν, λαλεῖ τὸ πνεῦμα τὸ ἅγιον, ἀλλὰ τότε λαλεῖ, ὅταν θελήσῃ αὐτὸν ὁ θεὸς λαλῆσαιὅταν οὖν ἔλθῃ ὁ ἄνθρωπος ὁ ἔχων τὸ πνεῦμα τὸ θεῖον εἰς συναγωγὴν ἀνδρῶν δικαίων τῶν ἐχόντων πίστιν θείον πνεύματος καὶ ἔντευξις γένηται πρὸς τὸ θεὸν τῆς συναγωγῆς τῶν ἀνδρῶν ἐκείνων, τότε ὁ ἄγγελος τοῦ προφητικοῦ πνεύματος ὁ κείμενος πρὸς αὐτὸν πληροῖ τὸν ἄνθρωπον, καὶ πληρωθεὶς ὁ ἄνθρωπος τῷ πνεύματι τῷ πνεύματι τῷ ἁγίῳ λαλεῖ εἰς τὸ πλῆθος, καθὼς ὁ κύριος βούλεταιοὕτως οὖν φανερὸν ἔσται τὸ πνεῦμα τῆς θεότητοςὅση οὖν περὶ τοῦ πνεύματος τῆς θεότητος τοῦ κυρίου ἡ δύναμις αὕτηἄκουε νῦν, φησί, περὶ τοῦ πνεῦματος τοῦ ἐπιγείου καὶ κενοῦ καὶ δύναμιν μὴ ἔχοντος, ἀλλὰ ὄντος μωροῦπρῶτον μὲν ὁ ἄνθρωπος ἐκεῖνος ὁ δοκῶν πνεῦμα ἔχειν ὑψοῖ ἑαυτὸν καὶ θέλει πρωτοκαθεδρίαν ἔχειν, καὶ εὐθὺς ἰταμός ἐστι καὶ ἀναιδὴς καὶ πολύλαλος καὶ ἐν τρυφαῖς πολλαῖς ἀναστρεφόμενος καὶ ἐν ἑτέραις πολλαῖς ἀπάταις καὶ μισθοὺς λαμβάνων τῆς προφητείας αὐτοῦ· ἐὰν δὲ μὴ λάβῃ, οὐ προφητεύειδύναται οὖν πνεῦμα θεῖον μισθοὺς λαμβάνειν και προφητεύειν; οὐκ ἐνδεχεται τοῦτο ποιεῖν θεοῦ προφήτην, ἀλλὰ τῶν τοιούτων προφητῶν ἐίγειόν ἐστι τὸ πνεῦμαεἶτα ὅλως εἰς συαγωγὴν ἀνδρῶν δικαίων οὐκ ἐγγίζει, ἀλλ’ ἀποφεύγει αὐτούς· κολᾶται δὲ τοῖς διψύχοις καὶ κενοῖς καὶ κατὰ γωνίαν αὐτοῖς προφητεύει καὶ ἀπατᾷ αὐτοὺς λαλῶν κατὰ τὰς ἐπιθυμίας αὐτῶν πάντα κενῶς· κενοῖς γὰρ καὶ ἀποκρίνεται· τὸ γὰρ κενὸν σκεῦος μετὰ τῶν κενῶν συντιθέμενον οὐ θραύεται, ἀλλὰ συμφωνοῦσιν ἀλλήλοιςὅταν δὲ ἔλθῃ εἰς συναγωγὴν πλήρη ἀνδρῶν δικαίων ἐχόντων πνεῦμα τὸ ἐπιγειον ἀπὸ τοῦ φόβου φεύγει ἀπ’ αὐτοῦ, καὶ κωφοῦται ὁ ἄνθρωπος ἐκεῖνος καὶ ὅλως συνθραύεται, μηδὲν δυνάμενος λαλῆσαιἐὰν γὰρ εἰς αὐτοῖς θῇς κεράμιον κενόν, καὶ πάλιν ἀποτιβάσαι θελήσῃς τὴν ἀποθήκην, τὸ κεράμιον ἐκεῖνο, ὃ ἔθηκας κενον, κενὸν καὶ εὑρήσεις· οὕτω καὶ οἱ προφῆται οἱ κενοὶ ὅταν ἔλθωσιν εἰς πνεύματα δικαίων, ὁποῖοι ἦλθον, τοιοῦτοι καὶ εὑρίκονταιἔχεις ἀμφοτέρων τῶν προφητῶν τὴν ζωήνδοκίμαζε οὖν ἀπὸ τῶν ἔργων καὶ τῆς ζωῆς τὸν ἄνθρωπον τὸν λέγοντα τῷ πνεύματι τῷ ἐρχομένῳ ἀπὸ τοῦ θεοῦ καὶ ἔχοντι δύναμιν· τῷ ἐρχομένῳ ἀπὸ τοῦ θεοῦ καὶ κενῷ μηδὲν πίστευε, ὅτι ἐν αὐτῷ δύναμις οὐκ ἔστιν· ἀπὸ τοῦ διαβόλου γὰρ ἔρχεται ἄκουσον οὖν τὴν παραβολήν, ἣν μέλλω σοι λέγειν· λάβε λίθον καὶ βάλε εἰς τὸν οὐρανόν, ἴδε, εἰ δύνασαι ἅψασθαι αὐτοῦ· ἢ πάλιν λάβε σίφωνα ὕδατος καὶ σιφώνισον εἰς τὸν οὐρανόν, ἴδε, εἰ δύνασαι τρυπῆσαι τὸν οὐρανόνΠῶς, φημί κύριε, δύναται ταῦτα εἴρηκαςὩς ταῦτα οὖν, φησίν, ἀδύνατά ἐστιν, οὕτω καὶ τὰ πνεύματα τὰ ἐπίγεια ἀδύνατά ἐστι καὶ ἀδρανῆλάβε οὖν τὴν δύναμιν τὴν ἄνωθεν ἐρχομένην· ἡ χάλαζα ἐλάχιστόν ἐστι κοκκάριον, καὶ ὅταν ἐπιπέσῃ ἐπὶ κεφαλὴν ἀνθρώπου, πῶς πόνον παρέχει; ἢ πάλιν λὰβε σταγόνα, ἢ ἀπὸ τοῦ κεράμου πίπτει χαμαὶ ἄνωθεν ἐλάχιστα πίπτοντα ἐπὶ τὴν γῆν μεγάλην δύναμιν ἔχει· οὕτω καὶ τὸ πνεῦμα τὸ θεῖον ἄνωθεν ἐρχόμενον δυνατόν ἐστι· τούτῳ οὖν τῷ πνεύματι πίστευε, ἀπὸ δὲ τοῦ ἑτέρου ἀπέχου

Ἐντολὴ ιβ’.
Λέγει μοι· Ἄρον ἀπὸ σεαυτοῦ πᾶσαν ἐπιθυμίαν πονηράν, ἔνδυσαι δὲ τὴν ἐπιθυμίαν τὴν ἀγαθὴν καὶ σεμνήν· ἐνδεδυμένος γὰρ τὴν ἐπὶθυμίαν ταύτην μισήσεις τὴν πονηρὰν ἐπιθυμίαν καὶ χαλιναγωγήσεις αὐτήν, καθὼς βούλειἀγρία γάρ ἐστιν ἡ ἐπιθυμία ἡ πονηρὰ καὶ δυσκόλως ἡμεροῦνταιφοβερὰ γάρ ἐστι καὶ λίαν τῇ ἀγριότητι αὐτῆς δαπανᾷ τοῖς ἀνθρώπους· μάλιστα δὲ ἐὰν ἐμπέσῃ εἰς αὐτὴν δοῦλος θεοῦ καὶ μὴ ᾖ συνετός, δαπανᾶται ὑπ’ αὐτῆς δεινῶς· δαπανᾷ δὲ τοὺς τοιούτους τοὺς μὴ ἔχοντας ἔνδυμα τῆς ἐπιθυμίας τῆς ἀγαθῆς, ἀλλὰ ἐμπεφυρμένους τῷ αἰῶνι τούτῷ· τούτους οὖν παραδίδωσιν εἰς θάνατονΠοῖα, φημί, κύριε, ἔργα ἐστὶν τῆς ἐπιθυμίας τῆς πονηρᾶς τὰ παραδιδόντα τοὺς ἀντρώπους εἰς θάνατον; γνώρισόν μοι, ἵνα ἀφέξωμαι ἀπ’ αὐτῶνἌκουσον, φησίν, ἐν ποίοις ἔργοις θανατοῖ ἡ ἐπιθυμία ἡ πονηρὰ τοὺς δούλους τοῦ θεοῦ.
Πάτων προέχουσα ἐπιθυμία γυναικὸς ἀλλοτρίας ἡ ἀνδρὸς καὶ πολυτελείας πλούτου καὶ ἐδεσμάτων πολλῶν παταίων καὶ μεθυσμάτων καὶ ἑτέρων τρυφῶν πολλῶν καὶ μωρῶν· πᾶσα γὰρ τρυρὴ μωρά ἐστι καὶ κενὴ τοῖς δούλοις τοῦ θεοῦαὗται οὖν αἱ ἐπιθυμίαι πονηραί εἰσι, θανατοῦσαι τοὺς δούλους τοῦ θεοῦ· αὑτη γὰρ ἡ ἐπιθυμία ἡ πονηρὰ τοῦ διαβόλου θυγάτηρ ἐστίνἀπέχεσθαι οὖν δεῖ ἀπὸ τῶν ἐπιθυμιῶν τῶν πονηρῶν, ἵνα ἀποσχόμενοι ζήσητε τῷ θεῷὅσοι δὲ ἂν κατακυριευθῶσιν ὑπ’ αὐτῶν καὶ μὴ ἀντίσταθῶσιν αὐταῖς, ἀποθανοῦνται εἰς τέλος· θανατώδεις γάρ εἰσιν αἱ ἐπιθυμίαι αὗταισὺ δὲ ἔδυσαι τὴν ἐπιθυμίαν τῆς δικαιοσύνης, καὶ καθοπλισάμενος τὸν φόβον τοῦ κυρίου ἀντίστηθι αὐταῖς· ὁ γὰρ φόβος τοῦ θεοῦ κατοικεῖ ἐν τῇ ἐπιθυμίᾳ τῇ ἀγαθῇἡ ἐπιθυμία ἡ πονηρὰ ἐὰν ἴδῃ σε καθωπλισμένον τῷ φόβῳ τοῦ θεοῦ καὶ ἀνθεστηκότα αὐτῇ, φεύξεται ἀπὸ σου μακρὰν καὶ οὐκέτι σοι ὀφθησετα φοβουμένη τὰ ὅπλα σουσὺ οὖν νικήσας καὶ στεφανωθεὶς κατ’ αὐτῆς ἐλθὲ πρὸς τὴν ἐπιθυμίαν τῆς δικαιοσύνης, καί παραδοὺς αὐτὴ βούλεταιἐὰν δουλεύσῃς τῇ ἐπιθυμίᾳ τῇ ἀγαθῇ καὶ ὑποταγῇς αὐτῃ, δυνήσῃ τῆς ἐπιθυμίας τῆς πονηρᾶς κατακυριεῦσαι καὶ ὑποτάξαι αὐτήν, καθὼς βούλει.
Ἤθελον, φημί, κύριε, γνῶναι, ποίοις τρόποις με δεῖ δουλεῦσαι τῇ ἐπιθυμίᾳ τῇ ἀρετήν, ἀλήθειαν καὶ φόβον κυρίου, πίστιν καὶ πραόρητα καὶ ὅσα τούτοις ὅμοιά ἐστιν ἀγαθάταῦτα ἐργαζομενος εὐάρεστος ἔσῃ δοῦλος τοῦ θεοῦ καὶ ζήσῃ αὐτῷ· καὶ πᾶς, ὃς ἂν δουλεύσῃ τῇ ἐπιθυμίᾳ τῇ ἀγαθῇ, ζήσεται τῷ θεῷσυντελέσεν οὖν τὰς ἐντολὰς ταύτας· πορεύου ἐν αὐταῖς καὶ τοὺς ἀκούοντας παρακάλει, ἵνα ἡ μετάνοια αὐτῶν καθαρὰ γένηται τὰς λοιπὰς ἡμέρας τῆς ζωῆς αὐτῶντὴν διακονίαν ταύτην, ἥν σοι δίδωμι, ἐκτέλει ἐπιμελῶσ, καὶ πολὺ ἐράσῃ· εὑήσεις γὰρ χάριν ἐν τοῖς μέλλουσι μετανοεῖν, και πεισθήσονταί σου τοῖς ῥήμασιν· ἐγὼ γὰρ μετὰ σοῦ ἔσομαι καὶ ἀανγκάσω αὐτοὺς πεισθῆναί σοιΛέγω αὐτῷ· Κύριε, αἱ ἐντολαὶ αὕται μεγάλαι καὶ καλαὶ καὶ ἔνδοξοί εἰσι καὶ δυνάμεναι εὐφρᾶναι καρδίαν ἀνθρώπου του δυναμένου τηρ͂σαι αὐτάςοὐκ οἶδα δέ, εἰ δύνανται αἱ ἐντολαὶ αὗται ὑπὸ ἀνθρώπου φυλαχθῆναι, διότι σκληραί εἰσιν λίανἀποκριθεὶς λέγει μοι· Ἐὰν σὺ σεαυτῷ προθῃς, ὅτι δύνανται φυλαχθῆναι, εὐκολως αὐτὰς φυλάξεις καὶ οὐκ ἔσονται σκληραί· ἐὰν δὲ ἐπὶ τὴν καρδίαν σου ἤδη ἀναβῇ μὴ δύνασθαι, αὐτὰς ὑπὸ ἀνθρώπου φυλαχθῆναι, οὐ φυλάξεις αὐτάςνῦν δέ σοι λέγω· ἐὰν ταύτας μὴ φυλάξῃς, ἀλλὰ παρενθυμηθῇς, οὐχ ἔξεις σωτηρίαν οὔτε τὰ τέκνα σου οὔτε ὁ οἶκός σουἐπεὶ ἤδη σεαυτῷ κέκρικας τοῦ μὴ δύνασθαι τὰς ἐντολὰς ταύτας ὑπὸ ἀνθρώπου φυλαχθῆναι.
Καὶ ταῦτά μοι λίαν ὀργίλως ἐλάλησεν, ὥστε με συγχυθῆναι καὶ λίαν αὐτὸν φοβηθῆναι· ἡ μορφὴ γὰρ αὐτοῦ ἠλλοιώθη, ὥστε μὴ δύνασθαι ἄνθρωπον ὑπενεγκεῖν τὴν ὀργὴν αὐτοῦἰδὼν δέ με τεταραγμένον ὅλον καὶ συγκεχυμένον ἤρξατό μοι ἐπιεικέστερον καὶ ἱλαρώτερον λαλεῖν καὶ λέγει· Ἄφρον, ἀσύετε καὶ δίψυχε, οὐ νοεῖς τὴν δόξαν τοῦ θεοῦ, πῶς μεγάλη ἐστὶ καὶ ἰσχυρὰ καὶ θαυμαστή, ὅτι ἔκτισε τὸν κόσμον ἕνεκα τού ἀνθρώπου καὶ πᾶσαν τὴν ἐξουσίαν πᾶσαν ἔδωκεν αὐτῷ τοῦ κατακυριεύειν τῶν ὑπὸ τὸν οὐραντὸν πάντων; εἰ οὖν, φησίν, πάντων ὁ ἄνθρωος κύριός ἐστι τῶν κτισμάτων τοῦ θεοῦ καὶ πάντων κατακυριεύει, οὐ δύναται καὶ τούτων τῶν ἐντολῶν κατακυριεῦσαι; κύναται, φησί, πάντων καὶ πασῶν τῶν ἐτολῶν τούτων κατακυριεῦσαι ὁ ἄνθρωπος ὁ ἔχων τὸν κύριον ἐν τῇ καρδίᾳ αὐτοῦοἱ δὲ ἐπὶ τοῖς χείλεσιν ἔχοντες τὸν κύριον, τὴν δὲ καρδίαν αὐτῶν πεπωρωμένην καὶ μακρᾶν ὄντες ἀπὸ τοῦ κυρίου, ἐκείνοις αἱ ἐντολαὶ αὗται σκληραί εἰσι καὶ κύσβατοιθέσθε οὖν ὑμεῖς, οἱ κενοὶ καὶ ἐλαφροὶ ὄντες ἐν τῇ πίστει, τὸν κύριον ὑμῶν εἰς τὴν καρδίαν, καὶ γνώσεσθε, ὅτι οὐδέν ἐστιν εὐκοπώτερονἐπιστράφητε ὑμεῖς οἱ ταῖς ἐντολαῖς πορευόμενοι τοῦ διαβόλου, ταῖς δυσκόλοις καὶ πικραῖς καὶ ἀγρίαις καὶ ἀσελγέσι, καὶ μὴ φοβήθητε τὸν διάβολον, ὅτι ἐν αὐτῷ δύναμις οὐκ ἔστιν καθ’ ὑμῶν· ἐγὼ γὰρ ἔσομαι μεθ’ ὑμῶν, ὁ ἄγγελος τῆς μετανοίας ὁ κατακυριεύων αὐτοῦὁ διάβολος μόνον φόβον ἔχει, ὁ δὲ φόβος αὐτοῦ τόνον οὐκ ἔκει· μὴ φοβήθητε οὖν αὐτόν, καὶ φεύξεται ἀφ’ ὑμῶν.
Λέγω αὐτῷ· Κύριε, ἄκουσόν μου ὀλίγων ῥημάτωνΛέγε, φησίν, ὃ βούλειὈ̔Ὁ μὲν ἄνθρωπος, φημί, κύριε, πρόθυμός ἐστι τὰς ἐντολὰς τοῦ θεοῦ φυλάσσειν, καὶ οὐδείς ἐστιν ὁ μὴ αἰτούμενος παρὰ τοῦ κυρίου, ἵνα ἐνδυναμωθῇ ἐν ταῖς ἐντολαῖς αὐτοῦ καὶ ὑποταγῇ αὐταῖς· ἀλλ’ ὁ διάβολος σκληρός ἐστι καὶ κατακυναστεύει αὐτῶνΟὐ δύναται, φησί, καταδυναστεύειν τῶν δούλων τοῦ θεοῦ τῶν ἐξ ὅλης καρδίας ἀντιπαλαῖσαι, καταπαλαῖσαι δὲ οὐ δύναταιἐὰν οὖν ἀντισταθῆτε αὐτῷ, νικηθεὶς φεύξεται ἀφ’ ὑμῶνκατῃσχυμμένοςὅσοι δέ, φησίν, ἀπόκενοί εἰσι, φοβοῦνται τὸν διάβολον ὡς δύναμιν ἔχονταὅταν ὁ ἄνθρωπος κεράμια ἱκανώτατ γεμίσῃ οἴνου καλοῦ καὶ ἐν τοῖς κεραμίοις ἐκείνοις ὀλίγα ἀπόκενα ᾖ, ἔρχεται ἐπὶ τὰ κεράμια και οὐ κατανοεῖ τὰ πλήρη· οἶδε γάρ, ὅτι πλήρη εἰσί· κατανοεῖ δὲ τὰ ἀπόκενα, φοβούμενος, μήποτε ὤξισαν· ταχὺ γὰρ τὰ ἀπόκενα κεράμια ὀξίζουσι, καὶ ἀπόλλυται ἡ ἡδονὴ τοῦ οἴνουοὕτω καὶ ὁ διάβολος ἔρχεται ἐπὶ πάντας τοὺς δούλους τοῦ θεοῦ ἐκπειράζων αὐτούςὅσοι οὖν πλήρεις εἰσὶν ἐν τῇ πίστει, ἀνθεστήκασιν αὐτῷ ἰσχυρῶς, κἀκεῖνος ἀποχωρεῖ ἀπ’ αὐτῶν μὴ ἔχων τόπον, ποῦ εἰσέλθῃἔρχεται οὖν τότε πρὸς τοὺς ἀποκένους καὶ ἔχων τόπον εἰσπορεύεται εἰς αὐτούς, καὶ ὃ δὲ βούλεται ἐν αὐτοῖς ἐργάζεται, καὶ γίνονται αὐτῷ ὑπόδουλοι.
Ἐγὼ δὲ ὑμῖν λέγω, ὁ ἄγγελος τῆς μετανοίας· μὴ φοβήθητε τὸν διάβολονἀπεστάλην γάρ, φησί, μεθ’ ὑμῶν εἶναι τῶν μετανοούντων ἐξ ὅλης καρδίας αὐτῶν καὶ ἰσχυροποιῆσαι αὐτοὺς ἐν τῇ πίστειπιστεύσατε οὖν τῷ θεῷ ὑμεῖς οἱ διὰ τὰς ἁμαρτίας ὑμῶν ἀπεγνωκότες τὴν ζωὴν ὑμῶν καὶ προστιθέντες ἁμαρτίαις καὶ καταβαρύνοντες τὴν ζωὴν ὑμῶν, ὅτι, ἐὰν ἐπιστραφῆτε πρὸς τὸν κύριον ἐξ ὅλης τῆς καρδίας ὑμῶν καὶ ἐργάσσησθε τὴν δικαιοσύνην, τὰς λοιπὰς ἡμέρας τῆς ζωῆς ὑμῶν καὶ δουλεύσητε αὐτῷ ὀρθῶς κατὰ τὸ θέλημα αὐτοῦ, ποιήσει ἴασιν τοῖς προτέροις ὑμῶν ἁμαρτήμασι καὶ ἕξετε δύναμιν τοῦ κατακυριεῦσαι τῶν ἔργων τοῦ διαβόλουτὴν δὲ ἀπειλὴν τοῦ διαβόλου ὅλως μὴ φοβήθητε· ἄτονος γάρ ἐστιν ὥσπερ νεκροῦ νεῦραἀκούσατε οὖν μου καὶ φοβήθητε τὸν πάντα δυνάμενον, σῶσαι καὶ ἀπολέσαι, καὶ τηρεῖτε τὰς ἐντολὰς ταύτας, καὶ ζήσεσθε τῷ θεῷλέγω αὐτῷ· Κύριε, νῦν ἐνεδυναμώθην ἐν πᾶσι τοῖς δικαιώμασι τοῦ κυρίου, ὅτι σὺ μετ’ ἐμοῦ εἶ· καὶ οἶδα, ὅτι συγκόψεις τὴν δύναμιν τοῦ διαβόλου πᾶσαν καὶ ἡμεῖς αὐτοῦ κατακυριεύσομεν καὶ κατισχύσομεν πάντων τῶν ἔργων αὐτοῦκαὶ ἐλπίζω, κύριε, δυνασθαί με τὰς ἐντολὰς ταύτας, ἃς ἐντεταλσαι, τοῦ κυρίου ἐνδυναμοῦντος φυλάξαιΦυλάξεις, φησίν, ἐὰν ἡ καρδία σου καθαρὰ γένηται πρὸς κύριον· καὶ´πάντες δὲ φυλάξοθσιν, ὅσοι ἂν καθαρίσωσιν ἑαυτῶν τὰς καρδίας ἀπὸ τῶν ματαίων ἐπιθυμιῶν τοῦ αἰῶνος τούτου, καὶ ζήσονται τῷ θεῷ.

ΠΑΡΑΒΟΛΑΙ ΑΣ ΕΛΑΛΗΣΕ ΜΕΤ’ ΕΜΟΥ
Λέγει μοι· Οἴδατε, φησίν, ὅτι ἐπὶ ξένης κατοικεῖτε ὑμεῖς οἱ δοῦλοι τοῦ θεοῦ· ἡ γὰρ πόλις οὖν οἴδατε, φησί, τὴν πόλιν ὑμῶν, ἐν ᾗ μέλλετε κατοικεῖν, τί ὧδε ὑμεῖς ἑτοιμάζετε ἀγροὺς καὶ παρατάξεις πολυτελεῖς καὶ οἰκοδομὰς καὶ οἰκήματα μάταια; ταῦτα οὖν ὁ ἑτοιμάζων εἰς ταύτην τὴν πόλιν οὐ δύναται ἐπανακάμψυχε εἰς τὴν ἰδίαν πόλινἄφρον καὶ δίψυχε καὶ ταλαιπωρε ἄθρωπε, οὐ νοεῖς, ὅτι ταῦτα πάντα ἀλλότριά εἰσι καὶ ὑπ’ ἐξουσίαν ἑτέρου εἰσίν; ἐρεῖ γὰρ ὁ κύριος τῆς πόλεως ταύτης· Οὐ θέλω σε κατοικεῖν εἰς τὴν πόλιν μου, ἀλλ’ ἔξελθε ἐκ τῆς πόλεως ταύτης, ὅτι τοῖς νόμοις μου οὐ χρᾶσαισὺ οὖν ἔχων ἀγροὺς καὶ οἰκήσεις καὶ ἑτέρας ὑπάρξεις πολλάς, ἐκβαλλόμενος ὑπ’ αὐτοῦ τί ποιήσεις σου τὸν ἀγρὸν καὶ τὴν οἰκίαν καὶ τὰ λοιπά, ὅσα ἡτοίμασας σεαυτῷ; λέγει γάρ σοι δικαίως ὁ κύριος τῆς χώρας ταύτης· Ἡ τοῖς νόνοις μου τί μέλλεις ποιεῖν, ἔχων νόμον ἐν τῇ σῇ πόλει; ἕνεκεν τῶν ἀγρῶν σου καὶ τῆς λοιπῆς ὑπάρξεως τὸν νόμον σου πάντως ἀπαρνήσῃ καὶ πορεύσῃ τῷ νόμῳ τῆς πόλεως ταύτης; βλέπε, μὴ ἀσύμφορόν ἐστιν ἀπαρνῆσαι τὸν νόμον σου· ἐὰν γὰρ ἐπανακάμψαι θελήσῃς εἰς τὴν πόλιν σου, οὐ μὴ παραδεχθήσῃ, ὅτι ἀπηρνήσω τὸν νόμον τῆς πόλεως σου, καὶ ἐκκλεισθήσῃ ἀπ’ αὐτῆςβλέπε οὐν σύ· ὡς ἐπὶ ξένης κατοικῶν μηδὲν πλέον ἑτοιμαζε σεαθτῷ εἰ μὴ τὴν αὐτάρκειαν τὴν ἀρκετήν σοι, καὶ ἔτοιμος γίνου, ἵνα, ὅταν θέλῃ ὁ δεσπότης τῆς πόλεως ταύτης ἐκβαλεῖν σε ἀντιταξάμενον τῷ νόμῳ αὐτοῦ, ἐξέλθῃς ἐν τῆς πόλεως αὐτοῦ καὶ ἀπέλθῃς ἐν τῇ πόλει σου καὶ τῷ σῷ νόμῳ χρήσῃ ἀνυβρίστως ἀγαλλιώμενοςβλέπετε οὖν ὑμεῖς οἱ δουλεύοντες τῷ κυρίῳ καὶ ἔχοντες αὐτὸν εἰς τὴν καρδίαν· ἐργάζεσθε τὰ ἔργα τοῦ θεοῦ μνημονεύοντες τῶν ἐντολῶν αὐτοῦ καὶ τῶν ἐπαγγελιῶν ὧν ἐπηγγείλατο, καὶ πιστεύσατε αὐτῷ, ὅτι ποιήσει αὐτάς, ἐὰν αἱ ἐντολαὶ αὐτοῦ φυλαχθῶσινἀντὶ ἀγρῶν οὖν ἀγοράζετε ψυχᾶς καὶ ὀρφανοὺς ἐπισκέπτεσθε καὶ μὴ παραβλέπετε αὐτούς, καὶ τὸν πλοῦτον ὑμῶν καὶ τὰς παρατάξεις πάσας εἰς τοιούτους ἀγροὺς καὶ οἰκίας δαπανᾶτε, ἃς ἐλάβετε παρὰ τοῦ θεοῦεἰς τοῦτο γὰρ ἐπλούτισεν ὑμᾶς ὁ δεσπότης, ἵνα τούτας τὰς διακονίας τελέσητε αὐτῷ· πολὺ βέλτιόν ἐστι τοιούτους ἀγροὺς ἀγοριζειν καὶ κτήματα καὶ οἴκους, οὓς εὑρήσεις ἐν τῇ πόλει σου, ὅταν ἐπιδημήσῃς εἰς αὐτήναὕτη ἡ πολυτέλεια καλὴ καὶ ἱερά, λύπην μὴ ἔχουσα μηδὲ φόβον, ἔχουσα δὲ χαράντὴν οὖν πολυτέλειαν τῶν ἐθνῶν μὴ πράσσετε· ἀσύμφορον γάρ ἐστιν ὐμῖν τοῖς δούλοις τοῦ θεοῦτὴν δὲ ἰδίαν πολυτέλειαν πράσσετε μηδὲ τοῦ ἀλλοτρίου ἅψησθε μηδὲ ἐπιθυμεῖτε αὐτοῦ· πονηρὸν γάρ ἐστιν ἀλλοτρίων ἐπιθυμεῖντὸ δὲ σὸν ἔργον ἐργάζου, καὶ σωθήσῃ.

Ἄλλη παραβολή
Περιπατοῦντός μου εἰς τὸν ἀγρὸν καὶ κατανοοῦντος πτελέαν καὶ ἄμπελον καὶ διακρίνοντος περὶ τῆς πτελέας καὶ τῆς ἀμπέλου; Συζητῶ, φημί, κύριε, ὅτι εὐπρεπέταταί εἰσιν ἀλλήλαις, Ταῦτα τὰ δύο δένδρα, φησίν, εἰς τύπον κεῖνται τοῖς δούλοις τοῦ θεουἬθελον, φημί, γνῶναι τὸν τύπον τῶν δένδρων τούτοων ὧν λέγειςΒλέπεις, φησί, τὴν πτελέαν καὶ τὴν ἄμπελον; Βλέπω, φημί, κύριεἩ ἄμπελος, φησίν, αὕτη καρπὸν φέρει, ἡ δὲ πτελέα ξύλον ἄκαρπόν ἐστιν· ἀλλ’ ἡ ἄμπελος αὕτη ἐὰν μὴ ἀναβῇ ἐπὶ τὴν πτελέαν, οὐ δύναται καρποφοῆσαι πολὺ ἐρριμμένη χαμαί, καὶ ὃν φέρει καρπόν, σεσηκότα φέρει μὴ κρεμαμένη ἐπὶ τῆς πτελέας, ὅταν οὖν ἐπιρριφῇ ἡ ἄμπελος ἐπὶ τὴν πτελέαν, καὶ παρ’ ἑαυτῆς φέρει καρπὸν καὶ παρὰ τῆς πτελέαςβλέπεις οὖν, ὅτι καὶ ἡ πτελέα πολὺν καρπὸν δίδωσιν, οὐκ ἐλάσσονα τῆς ἀμπέλου, μᾶλλον δὲ καὶ πλείοναΠῶς φημί, κύριε, πλείονα; Ὅτι φησίν, ἡ ἄμπελος κρεμαμένη ἐπὶ τὴν πτελέαν τὸν καρπὸν πολὺν καὶ καλὸν δίδωσιν, ἐρριμμένη δὲ χαμαὶ ὀλίγον καί σαπρὸν φέρειαὕτη οὖν ἡ παραβολὴ εἰς τοὺς δούλους τοῦ θεοῦ κεῖται, εἰς πτωχὸν καὶ πλούσιονΠῶς, φημί, κύριε, γνώρισον μοιἌκουε, φησίν· ὁ μὲν πλούσιος ἔχει χρήματα, τὰ δὲ πρὸς τὸν κύριον πτωσεύει, περισπώμενος περὶ τὸν πλοῦτον ἑαυτοῦ, καὶ λίαν μικρὰν ἔχει τὴν ἔντευξιν καὶ τὴν ἐξομολόγησιν πρὸς τὸν κύριον, καὶ ἣν ἔχει, βληχρὰν καὶ μικρὰν καὶ ἄλλην μὴ ἔχουσαν δύναμινὅταν οὖν ἐπαναπάῃ ἐπὶ τὸν πένητα ὁ πλούσιος καὶ χορηγήσῃ αὐτῷ τὰ δέοντα πιστεύει, ὅτι ἐὰν ἐργάσηται εἰς τὸν πένητα δυνηθήσεται τὸν μισθὸν εὑρεῖν παρὰ τῷ θεῷ· ὅτι ὁ πένης πλούσιός ἐστιν ἐν τῇ ἐντεύξει καὶ ἐν τῇ ἐξομολογήσει καὶ δύναμιν μεγάλην ἔχει παρὰ τῷ θεῷ ἡ ἔντευξις αὐτοῦἐπιχορηγεῖ οὖν ὁ πλούσιος τῷ πένητι πάντα ἀδιστάκτωςὁ πένης δὲ ἐπιχορηγούμενος ὑπὸ τοῦ πλουσίου ἐντυγχάνει τῷ θεῷ εὐχαριστῶν αὐτῷ, ὑπὲρ τοῦ διδόντος αὐτῷ· κἀκεῖνος ἔτι ἐπισπουδάζει περὶ τοῦ πένητος, ἵνα ἀδιάλειπτος γένηται ἐν τῇ ζωῇ αὐτοῦ· οἶδε γάρ, ὅτι ἡ τοῦ πένητος ἔντευξις προσδεκτή ἐστι καὶ πλουσία πρὸς κύριονἀμφότεροι οὖν τὸ ἔργον τελοῦσιν· ὁ μὲν πένης ἐργάζεται τῇ ἐντεύξει, ἐν ᾗ πλουτεῖ, ἣν ἔλαβεν παρὰ τοῦ κυρίου· ταύτην ἀποδίδωσι τῷ κυρίῳ τῷ ἐπιχορηγοῦντι αὐτῷ· καὶ ὁ πλούσιος ὡσαύτως το πλοῦτος, ὃ́ͅὃ ἔλαβεν παρὰ τοῦ κυρίου, ἀδιστάκτως παρέχεται τῷ πένητικαὶ τοῦτο ἔργον μέγα ἐστὶ καὶ δεκτὸν παρὰ τῷ θεῷ, ὅτι συνῆκεν ἐπὶ πλούτῳ αὐτοῦ καὶ εἰργάσατο εἰς τὸν´πένητα ἐκ τῶν δωρημάτων τοῦ κυρίου καὶ ἐτέλεσε τὴν διακονίαν ὀρθῶςπαρὰ τοῖς οὖν ἀνθρώποις ἡ πτελέα δοκεῖ καρπὸν μὴ φέρειν, καὶ οὐκ οἴδασιν οὐδὲ νοοῦσιν, ὅτι, ὅταν ἀβροχία γένηται, ἡ πτελέα ἔχουσα ὕδωρ τρέφει τὴν ἄμπελον καὶ ἡ ἄμπελος ἀδιάλειπτον ἔχουσα τὸ ὕδωρ διπλοῦν τὸν καρπὸν ἀποδίδωσι, καὶ ὑπὲρ τῆς πτελέαςοὕτως καὶ οἱ πένητες ὑπὲρ τῶν πλουσίων ἐντυγχάνοντες πρὸς τὸν κύριον πληροφοροῦσι τὸ πλοῦτος αὐτῶν, καὶ πάλιν οἱ πλούσιοι χορηγοῦντες τοῖς πένησι τὰ δέοντα πληροφοροῦσι τὰς εὐχὰς αὐτῶν γίνονται οὖν ἀμφότεροι κοινωνοὶ τοῦ ἔργου τοῦ δικαίουταῦτα οὖν ὁ ποιῶν οὐκ ἐγκαταλειφθήσεται ὑπὸ τοῦ θεοῦ, ἀλλ’ ἔσται γεγραμμένος εἰς τὰς βίβλους τῶν ζώντωνμακάριοι οἱ ἔχοντες καὶ συνιέντες, ὅτι παρὰ τοῦ κυρίου πλουτίζονται, ὁ γὰρ συνίων τοῦτο δυνήσεται καὶ τι ἀγαθόν.
Ἄλλη παραβολή 
Ἔδειξέ μοι δένδρα πολλὰ μη ἔχοντα φύλλαἀλλ’ ὡσεὶ ξηρὰ ἐδόκε μοι εἶναι· ὅμοια γὰρ ἦν πάντακαὶ λέγει μοι· Βλέπεις τὰ δένδρα ταῦτα; Βλέπω, φημί, κύριε, ὅμοια ὄντα καὶ ξηράἀποκριθείς μοι λέγει· Ταῦτα τὰ δένδρα, ἃ βλέπεις, οἱ κατοικοῦντές εἰσιν ἐν τῷ αἰῶνι τούτῳΔιατί οὖν; φημί, κύριε, ὡσεὶ ξηρά εἰσι καὶ ὅμοια; Ὅτι, φησίν, οὔτε οἱ δικαίοις φαίνονται οὔτε οἱ ἁμαρτωλοὶ ἐν τῷ αἰῶνι τούτῳ, ἀλλ’ ὅμοιοί εἰσιν· ὁ γὰρ αἰὼν οὗτος τοῖς δικαίοις χειμών ἐστι, καὶ οὐ φαίνονται μετὰ τῶν ἁμαρτωλῶν κατοικοῦντεςὥσπερ γὰρ ἐν τῷ χειμῶνι τὰ δένδρα ἀποβεβληκότα τὰ φύλλα ὅμοιά εἰσι καὶ οὐ φαίνονται τὰ ξηρὰ ποῖά εἰσιν ἢ τὰ ζῶντα, οὕτως ἐν τῷ αἰῶνι τούτῳ οὐ φαίνονται οὔτε οἱ δίκαιοι οὔτε οἱ ἁμαρτωλοί, ἀλλὰ πάντες ὅμοιοί εἰσιν.

Ἄλλη παραβολή 
Ἔδειξέ μοι πάλιν δέδρα πολλά, ἃ μὲν βλαστῶντα, ἃ δὲ ξηρά, καὶ λέγει μοι· Βλεπεις, φησί, τὰ δένδρα ταῦτα; Βλέπω, φημί, κύριε, τὰ μὲν βλαστῶντα τὰ δὲ ξηράΤαῦτα, φησί, τὰ δένδρα τὰ βλαστῶντα οἱ δίκαιοί εἰσιν οἱ μέλλοντες κατοικεῖν εἰς τὸν αἰῶνα τὸν ἐρχόμενον· ὁ γὰρ αἰὼν ὁ ἐρχόμενος θερία ἐστὶ τοῖς δικαίοις, τοῖς δὲ ἁμαρτωλοῖς χειμώνὅταν οὖν ἐπιλάμψῃ τὸ ἔλεος τοῦ κυρίου, τότε φανερωθήσονταιὥσπερ γὰρ τῷ θέρει ἑνὸς ἑκάστου δένδρου οἱ καρποὶ φανεροῦνται καὶ ἐπιγινώσκονται ποταποί εἰσιν, οὕτω καὶ γνωσθήσονται πάντες εὐθαλεῖς ὄντες ἐν τῷ αἰῶνι ἐκείνῳτὰ δὲ ἔθνη καὶ οἱ ἁμαρτωλοί, ἃ εἶδες τὰ δένδρα τὰ ξηρά, τοιοῦτοι εὑρεθήσονται ξηροὶ καὶ ἄκαρποι ἐν ἐκείνῳ τῷ αἰῶνι καὶ ὡς ξύλα κατακαυθήσονται καὶ φανεροὶ ἔσονται, ὅτι ἡ πρᾶξις αὐτῶν πονηρὰ γέγονεν ἐν τῇ ζωῇ αὐτῶνοἱ μὲν γὰρ ἁμαρτωλοὶ καυθήσονται, ὅτι ἥματον καὶ οὐ μετενόησαν· τὰ δὲ ἔθνη καυθήσονται, ὅτι οὐκ ἔγνωσαν τὸν κτίσαντα αὐτούςσύ οὖν καρποφόρησον, ἵνα ἐν τῷ θέρει ἐκείνῳ γνωσθῇ σου ὁ καρπός· ἀπέχου δὲ ἀπὸ πολλῶν πράξεων καὶ οὐδὲν καμαρτήσειςοἱ γὰρ τὰ πολλὰ πράσσοντες πολλὰ καὶ ἁμαρτάνουσι, περισπώμενοι περὶ τὰς πράξεις αὐτῶν καὶ μηδὲν δουλεύοντες τῷ κυρίῳ ἑαυτῶνπῶς οὖν, φησίν, ὁ τοιοῦτος δύνατιά τι αἰτήσασθαι παρὰ τοῦ κυρίου καὶ λαβεῖν, μ̀μὴ δουλεύων τῷ δυρίῳ; ἐκεῖνοι οὐδὲν λήψονταιἐὰν δὲ μίαν τις πρᾶξιν ἐργάσηται, δυναται καὶ τῷ κυρίῳ δουλεῦσαι· οὐ γὰρ διαφθαρήσεται ἡ διάνοια αὐτοῦ ἀπὸ τοῦ κυρίου, ἀλλὰ δουλεύσει αὐτῷ ἔχων τὴν διάνοιαν αὐτοῦ καθαράνταῦτα οὖν ἐὰν ποιήσῃς, δύνασαι καρποφορῆσαι εἰς τὸν αἰῶνα τὸν ἐρχόμενον· καὶ ὃ ἂν ταῦτα ποιήσῃ, καρπορορήσει.

Ἄλλη παραβολή 
Νηστεύων καὶ καθήμενος εἰς τι κα εὐχαριστῶν τῷ κυρίῳ περὶ πάντων ὧν ἐποίησε μετ’ ἐμου, βλέπω τὸν ποιμένα παρακαθήμενόν μοι καὶ λέγοντα· Τί ὀρθρινὸς ὧδε ἐλήλυθας; Ὅτι, φημί, κύριε, στατίωνα ἔχωΤί, φησίν, ἐστὶ στατίων; Νηστεία δὲ, φησί, τί ἐστιν αὕτη, ἣν νηστεύετε; Ὡς εἰώθειν, φημί κύριε, οὕτω νηστεύωΟὐκ οἴδατε, φησί, νηστεύειν τῷ κυρίῳ, οὐδέ ἐστιν νηστεία αὕτη ἡ ἀνωφελής, ἣν νηστεύετε αὐτῷΔιάτι, φημί, κύριε, τοῦτο λέγεις; Λέγω σοι, φησίν, ὅτι οὐκ ἔστιν αὕτη νηστεία δεκτὴ καὶ πλήρης τῷ κυρίῳἌκουε, φησίνὁ θεὸς οὐ βούλεται τοιαύτην νηστείαν ματαίαν· οὕτω γὰρ νηστεύων τῷ θεῷ οὐδὲν ἐργάσῃ τῇ ζωῇ ζωῇ σου, ἀλλὰ δούλευσον τῷ κυρίῳ ἐν καθαρᾷ καρδίᾳ· τήρησον τὰς ἐντολὰς αὐτοῦ πορευόμενος ἐν τοὶς προστάγμασιν αὐτοῦ καὶ μηδεμία ἐπιθυμία πονηρὰ ἀναβήτω ἐν τῇ καρδίᾳ σου· πίτευσον δὲ τῷ θεῷ, ὅτι, ἐὰν ταῦτα ἐργάσῃ καὶ φοβηθῇς αὐτὸν καὶ ἐγκρατεύσῃ ἀπὸ παντὸς πονηροῦ πράγματος, ζήσῃ τῷ θεῷ· καὶ ταῦτα ἐὰν ἐργάσῃ, μεγάλην νηστείαν ποιήσεις καὶ δεκτὴν τῷ θεῷ.
Ἄκουε τὴν παραβολήν, ἣν μέλλω σοι λέγειν, ἀνήκουσαν τῇ νηστηείᾳεἶχέ τις ἀγραὸν καὶ δούλους πολλοὺς καὶ μέρος τι τοῦ ἀγροῦ ἐφύτευσεν ἀμπελῶν· καὶ ἐκλεξάμενος δοῦλόν τινα πιστὸν καὶ εὐάρεστον ἔντιμον, προσεκαλέσατο αὐτὸν καὶ λέγει αὐτῷ· Λάβε τὸν ἀμπελῶνα τοῦτον, ὃν εφύτεσα, καὶ χαράκωσον αὐτόν, ἕως ἔρχομαι, ταύτην μου τὴν ἐντολὴν φύλαξον, καὶ ἐλεύθερος ἔσῃ παρ’ ἐμοίἐξῆλθε δε ὁ δεσπότης τοῦ δούλου εἰς τὴν ἀποδημίανἐξελθόντος δὲ αὐτοῦ ἔλαβεν ὁ δοῦλος καὶ ἐχαράκωσε τὸν ἀμπελῶνακαὶ τελέσας τὴν χαράκωσιν τοῦ ἀμπελῶνος εἶδε τὸν ἀμπλῶνα βοτανῶν πλήρη ὄνταἐν ἑαυτῷ οὖν ἐλογίσατο λέγων· Ταύτην τὴν ἐντολὴν τοῦ κυρίου τετέλεκα· σκάψω λοιπὸν τὸν ἀμπελῶνα τοῦτον, καὶ ἔσται εὐπρεπέστερος ἐσκαμμένος, καὶ βοτάνας μὴ ἔχων δώσει καρπὸν πλείονα, μὴ πνιγόμενος ὑπὸ τῶν βοτανῶνλαβὼν ἔσκαψε ἐν τῷ ἀμπελῶνι ἐξέτιλλεκαὶ ἐγένετο ὁ ἀμπελὼν ἐκεῖνος εὐπρεπέστατος καὶ εὐθαλής, μὴ ἔχων βοτάνας πνιγούσας αὐτόνμετὰ χρόνον ἦλθεν ὁ δεσπότης τοῦ δούλου καὶ τοῦ ἀγροῦ καὶ εἰσῆλθεν εἰς τὸν ἀμπελῶνακαὶ ἰδὼν τὸν ἀμπελῶνα κεχαρακωμένον εὐπρεπῶς, ἔτι δὲ καὶ ἐσκαμμένον καὶ πάσας τὰς βοτάνας ἐκτετιλμένας καὶ εὐθαλεῖς οὔσας τὰς ἀμπέλους, ἐχάρη λίαν ἐπὶ τοῖς ἔργοις τοῦ δούλουπροσκαλεσάμενος οὖν τὸν υἱὸν αὐτοῦ τὸν ἀγαπητόν, ὃν εἶχε κληρονόμον, καὶ τοὺς φίλους, οὓς εἶχε συμβούλους, λέγει αὐτοῖς, ὅσα εινετείλατο τῷ δούλῳ αὐτοῦ καὶ ὅσα εὗρε γεγονότακἀκεῖνοι συνεχάρησαν τῷ δούλῳ ἐπὶ τῇ μαρτυρίᾳ ᾗ ἐμαρτύρησεν αὐτῷ ὁ δεσπότηςκαὶ λέγει αὐτοῖς· Ἐγὼ τῷ δούλῳ τούτῳ ἐλευθερίαν ἐπηγγειλάμην, ἐάν μου τὴν ἐντολὴν φυλάξῃ, ἣνἐνετειλάμην αὐτῷ· ἐφύλαξε δέ μου τὴν ἐντολὴν καὶ προσέθηκε τῷ ἀμπελῶνι ἔργον καλόν, καὶ ἐμοὶ λίαν ἤρεσενἀντὶ τούτου οὖν τοῦ ἔργου οὗ εἰργάσατο θέλω αὐτὸν συγκληρονόμον τῷ υἱῷ μου ποιῆσαι, ὅτι τὸ καλὸν φρονήσας οὐ παρενεθυμήθη, ἀλλ’ ἐτέλεσεν αὐτόταύτῃ τῇ γνώμῃ ὁ υἱὸς τοῦ δεσπότου συνηυδόκησεν αὐτῷ, ἵνα συγκληρονόμος γένηται ὁ δοῦλος τῷ υἱῷμετὰ ἡμέρας ὀλίγας δεῖπνον ἐποίησεν καὶ ἔπεμψεν αὐτῷ παρὰ τοῦ δεσπότου τὰ τὰ ἀρκοῦντα αὐτῷ ἦρε, τὰ λοιπὰ δὲ τοῖς συνδούλοις αὐτοῦ διέδωκενοἱ δὲ συνδουλοι αὐτοῦ λαβόντες τὰ ἐδέσματα ἐχάρησαν καὶ ἤρξαντο εὔχεσθαι ὑπὲρ αὐτοῦ, ἵνα χάριν μείζονα εὕρῃ παρὰ τῷ δεσπότῃ, ὅτι οὕτως ἐχρήσατο αὐτοῖςταῦτα πάτα τὰ γεγονότα ὁ δεσπότης αὐτοῦ ἤκουσε καὶ πάλιν λίαν ἐχάρη ἐπὶ τῇ πράξει αὐτοῦσυγκαλεσάμενος πάλιν τοὺς φίλους ὁ δεσπότης καὶ τὸν υἱὸν αὐτοῦ ἀπήγγειλεν αὐτοῖς τὴν πρᾶξιν αὐτοῦ, ἣν ἔπραξεν ἐπὶ τοῖς δ̓ἐδέσμασιν αὐτοῦ οἷς ἔλαβεν· οἱ δὲ ἔτι μᾶλλον συνευδόκησαν γενέσθαι τὸν δοῦλον συγκληρονόμον τῷ υἱῷ αὐτοῦ.
Λέγω· Κύριε, ἐγὼ ταύτας τὰς παραβολὰς οὐ γινώσκω οὐδὲ δύναμαι νοῆσαι, ἐὰν μή μοι ἐπιλύσῃς αὐτὰςΠάντα σοι ἐπιλύσω, φησί, καὶ ὅσα ἂν λαλήσω μετὰ σοῦδείξω σοι τὰς ἐντολὰς αὐτοῦ ἐὰν δέ τι ἀγαθὸν ποιήσῃς ἐκτὸς τῆς ἐντολῆς τοῦ θεοῦ, σεαυτῷ περιποιήσῃ δόξαν περισσοτέραν καὶ ἔσῃ ἐνδοξότερος παρὰ τῷ θεῷ οὗ ἔμελλες εἶναιἐὰν οὖν φυλάσσων τὰς ἐντολὰς τοῦ θεοῦ προσθῇς καὶ τὰς λειτουγίας ταύτας, χαρήσῃ, ἐὰν τηρήσῃς αὐτὰς κατὰ τὴν ἐμὴν ἐντολήνλέγω αὐτῷ· Κύριε,`ὃ ἐὰν μοι ἐντείλῂ, φυλάξω αὐτό· οἶδα γάρ, ὅτι σὺ μετ’ ἐμοῦ εἶἜσομαι, φησί, μετὰ σοῦ, ὅτι τοιαύτην προθυμίαν ἔχουσινἡ νηστεία αὕτη, φησί, τηρουμένων τῶν ἐντολῶν τοῦ κυρίου, λίαν καλή ἐστινοὕτως οὖν φυλάξεις τὴν νηστείαν ταύτην, ἣν μέλλεις τηρεῖν· πρῶτον πάντων φύλαξαι ἀπὸ παντὸς ῥήματος πονηροῦ καὶ πάσης ἐπιθυμίας πονηρᾶς καὶ καθάρισόν σου τὴν καρδίαν ἀπὸ πάντων τῶν ματαωμάτων τοῦ αἰῶνος τούτουἐὰν ταῦτα φυλάξῃς, ἔσται σοι αὕτη ἡ νηστεία τελείαοὕτω δὲ ποιήσεις· συντελέσας τὰ γεγραμμένα ἐν ἐκείνῃ τῇ ἡμέρᾳ ᾗ νηστεύεις μηδὲν γεύσῃ εἰ μὴ ἄρτον καὶ ὕδωρ, καὶ ἐκ τῶν ἐδεσμάτων σου ὧν ἔμελλες τρώγειν συμψηφίσας τὴν ποσότητα τῆς δαπάνης ἐκείνης τῆς ἡμέρας ἧς ἔμελλες ποιεῖν, δώσεις αὐτὸ χήρᾳ ἢ ὀρφανῷ ἢ ὑστερουμένῳ, καὶ οὕτω ταπεινοφρονήσεις, ἵν’ ἐκ τῆς ταπεινοφροσύνης σου ὁ εἰληφὼς ἐμπλήσῃ τὴν ἑαυτοῦ ψυχὴν καὶ εὔξηται ὑπὲρ σοῦ πρὸς τὸν κύριονἐὰν οὖν οὕτω τελέσῃς τὴν νηστείαν, ὥς σοι ἐνετειλάμην, ἔσται ἡ θυσία σου δεκτὴ παρὰ τῷ θεῷ, καὶ ἔγγραφος ἔσται ἡ νηστεία αὑτη, καὶ ἡ λειτουργία οὕτως ἐργαζομένη καλὴ καὶ ἱλαρά ἐστι καὶ εὐπρόσδεκτος τῷ κυρίῳταῦτα οὕτω τηρήσεις σὺ μετὰ τῶν τένων σου καὶ ὅλου τοῦ οἴκου σου· τηρήσας δὲ αὐτὰ μακάριος ἔσῃ· καὶ ὅσοι ἂν ἀκούσαντες αὐτὰ τηρήσωσι, μακάριοι ἔσονται, καὶ ὅσα ἂν αἰτήσωνται παρὰ τοῦ κυρίου λήψονται.
Ἐδεήθην αὐτοῦ πολλά, ἵνα μοι δηλώσῃ τὴν παραβολὴν τοῦ ἀγροῦ καὶ τοῦ δεσπότου καὶ τοῦ ἀμπελῶνος καὶ τοῦ δούλου τοῦ χαρακώσαντος τὸν ἀμπελῶνα καὶ τῶν χαράκων καὶ τῶν βοτανῶν τῶν ἐκτετειλμένων ἐκ τοῦ ἀμπελῶνος καὶ τοῦ υἱοῦ καὶ τῶν φίλων τῶν συμβούλων· συνῆκα γάρ, ὅτι παραβολή τίς ἐστι ταῦτα πάνταὁ δὲ ἀποκριθείς μοι εἶπεν· Αὐθάδησ εἶ λίαν εἰς το ἐπερωτᾶνοὐκ ὀφείλεις, φησίν, ἐπερωτᾶν οὐδὲν ὅλως· ἐὰν γάρ σοι δέῃ δηλωθῆσεταιλέγω αὐτῷ· Κύριε, ὅσα ἄν μοι δείξῃς καὶ μὴ δηλώσῃς, μάτην ἔσομαι ἑωρακὼς αὐτὰ καὶ μὴ νοῶν, τί ἐστιν· ὡσαύτως καὶ ἐὰν μοι παραβολὰς λαλήσῃς καὶ μὴ ἐπιλύσῃς μοι αὐτάς, εἰς μάτην ἔσομαι ἀκηκοώς τι παρὰ σοῦὁ δὲ πάλιν ἀπεκρίθη μοι λέγων· Ὃς ἂν, φησί, δοῦλος ᾖ τοῦ θεοῦ καὶ ἔχῃ τὸν κύριον ἑαυτοῦ ἐν τῇ καρδίᾳ, αἰτεῖαι παρ’ αὐτοῦ σύνεσιν καὶ λαμβάνει καὶ πᾶσαν παραβολὴν ἐπιλύει, καὶ γνωστὰ αὐτῷ γίνονται τὰ ῥηματα τοῦ κυρίου τὰ λεγόμενα διά παραβολῶν· ὅσοι δὲ βληχροί εἰσι καὶ ἀργοὶ πρὸς τὴν ἔντευξιν, ἐκεῖνοι διστάζουσιν αἰτεῖσθαι παρὰ τοῦ κυρίου· ὁ δὲ κύριος πολυεύπλαγχνός ἐστι καὶ πᾶσι τοῖς αιτουμένοις παρ’ αὐτοῦ αδιαλείπτως δίδωσισὺ δὲ ἐνδεδυναμωμένος ὑπὸ τοῦ ἁγίου ἀγγέλου καὶ εἰληφὼς παρ’ αὐτοῦ τοιαύτην ἔντευξιν καὶ μὴ ὢν ἀργός, διατί οὐκ αἰτῇ παρὰ τοῦ κυρίου σύνεσιν καὶ λαμβάνεις παρ’ αὐτοῦ; λέγω αὐτῷ· Κυριε, ἐγὼ ἔχων σὲ μεθ’ ἑαυτοῦ ἀαν́γκην ἔχω σὲ αἰτεῖσθαι καὶ σὲ ἐπερωτᾶν· σὺ γάρ μοι δεικνύεις πάνατα καὶ λαλεῖς μετ’ ἐμοῦ· εἰ δὲ ἄτερ σου ἔβλεπον ἢ ἤκουον αὐτά, ἠρώτων ἂν τὸν κύριον, ἵνα μοι δηλωθῇ.
Εἶπον σοι, φησί, καὶ ἄρτι, ὅτι πανοῦργος εἶ καὶ αὐθάδης, ἐπερωτῶν τὰς ἐπιλύσεις τῶν παραβολῶνἐπειδὴ δὲ οὕτω παράμονος εἶ, ἐπιλύσω σοι τὴν παραβολὴν τοῦ ἀγροῦ καὶ τῶν λοιπῶν τῶν ἀκολούθων πάντων, ἵνα γνωστὰ πᾶσι ποιήσῃς αὐτάἄκουε νῦν, φησί, καὶ σύνιε αὐτάὁ ἀγρὸς ὁ κόσμος οὗτός ἐστιν· ὁ δὲ κύριος τοῦ ἀγροῦ ὁ κτίσας τὰ πάντα καὶ ἀπαρτίσας αὐτὰ καὶ δυναμώσας· ὁ δὲ δοῦλος ὁ υἱὸς τοῦ θεοῦ ἐστιν· αἱ δὲ ἄμελοι ὁ λαὸς οὗτός ἐστιν, ὃν αὐτὸς ἐφύτευσεν· οἱ δὲ χάρακες οἱ ἅγιοι ἄγγελοί εἰσι τοῦ κυρίου οἱ συγκρατοῦντες τὸν λαὸν αὐτοῦ· αἱ δὲ βοτάναι αἱ ἐκτετιλμέναι ἐκ τοῦ ἀμπελῶνος ἀνομίαι εἰσὶ τῶν δούλων τοῦ θεοῦ· τὰ δὲ ἐδέσματα, ἃ ἔπεμψεν αὐτῷ ἐκ τοῦ δείπνου, αἱ ἐντολαί εἰσιν, ἃ ς ἔδωκε τῷ λαῷ αὐτοῦ διὰ τοῦ υἱοῦ αὐτοῦ· οἱ δὲ φίλοι καὶ συμβουλοι οἱ ἅγιοι ἄγγελοι οἱ πρῶτοι κτισθέντες· ἡ δὲ ἀποδημία τοῦ δεσπότου ὁ χρόνος ὁ περισσεύων εἰς τὴν παρουσίαν αὐτοῦλέγω αὐτῷ· Κύριε, μεγάλως καὶ θαυμαστῶς πάντα ἐστὶ καὶ ἐνδόξως πάντα ἔχειμὴ οὖν, φημί, ἐγὼ ἠδυνάμην ταῦτα νοῆσαι; οὐδὲ ἕτερος τῶν ἀνθρώπων, κἂν λίαν συνετὸς ᾖ τις, οὐ δύναται νοῆσαι αὐτάἔτι, φημί, κύριε, δήλωσόν μοι, ὃ μέλλω σε ἐπερωτᾶνΛέγε, φησίν, εἴ τι βούλειΔιατί, φημί, κύριε, ὁ υἱὸς τοῦ θεοῦ εἰς δούλου τρόπον κεῖται ἐν τῇ παραβολῇ;
Ἄκουεφησίν· εἰς δούλου τρόπον οὐ κεῖται ὁ υἱὸς τοῦ θεοῦ, ἀλλ’ εἰς ἐξουσίαν μεγάλην κεῖται καὶ κυριότηταΠῶς, φημί, κύριε, οὐ νοῶὍτι, φησίν, ὁ θεὸς τὸν ἀμπελῶνα ἐφύτευσε, τοῦτ’ ἔστι τὸν λαὸν ἔκτισε καὶ παρέδωκε τῷ υἱῷ αὐτοῦ· καὶ ὁ υἱὸς κατέστησε τοὺς ἀγγέλους ἐπ’ αὐτοὺς τοῦ συντητεῖν αὐτούς· καὶ αὐτὸς τὰς ἁμαρτίας αὐτῶν ἐκαθάρισε πολλὰ κοπιάσας καὶ πολλοὺς κόπους ἠντληκώς· οὐδεὶς γὰρ ἀμπελὼν δύναται σκαφῆναι ἄτερ κόπου ἢ μόχθουαὐτὸς οὖν καθαρίσας τὰς ἁμαρτίας τοῦ λαοῦ ἔδειξεν αὐτοῖς τὰς τρίβους τῆς ζωῆς, δοὺς αὐτοῖς τὸν νόμον, ὃν ἔλαβε παρὰ τοῦ πατρὸς αὐτοῦὅτι δὲ ὁ κύριος σύμβουλον ἔλαβε τὸν νόμον, ὃν ἔλαβε τὸν υἱὸν αὐτοῦ καὶ τοὺς ἐνδόξους ἀγγέλους περὶ τῆς κληρονομίας τοῦ δούλου, ἄκουε· τὸ πνεῦμα τὸ ἅγιον τὸ προόν, τὸ κτίσαν πᾶσαν τὴν κτίσιν, κατῴκησε τὸ πνεῦμα τὸ ἅτιον, ἐδούλευσε τῷ πνεύματι καλῶς ἐν σεμνότητι καὶ ἁγνείᾳ πορευθεῖσα, μηδὲν ὅλως μιάνασα τὸ πνεῦμαπολιτευσαμένην οὖν αὐτὴν καλῶς καὶ ἁγνῶς καὶ συγκοπιάσασαν τῷ πνεύματι καὶ ἁγνῶς καὶ συγκοπιάσασαν τῷ πνεύματι καὶ συνεργήσασαν ἐν παντὶ πράγματι, ἰσχυρῶς καὶ ἀνδρείως ἀναστραφεῖσαν, μετὰ τοῦ πνεύματος τοῦ ἁγίου εἵλατο κοινωνόν· ἤρεσε γὰρ ἡ πορεία τῆς σαρκὸς ταύτης, ὅτι οὐκ ἐμιάνθη ἐπὶ τῆς γῆς ἔχουσα τὸ ἅγιονσύμβουλον οὖν ἔλαβε τὸν υἱὸν καὶ τοὺς ἀγγέλους τοὺς ἐνδόξους, ἵνα καὶ ἡ σὰρξ αὕτη, δουλεύσασα τῷ πνεύματι ἀμέμπτως, σχῇ τόπον τινὰ κατασκηνώσεως καὶ μὴ δόξῃ τὸν μισθὸν τῆς δουλείας αὐτῆς ἀπολωλεκέναι· πᾶσα γὰρ σὰρξ ἀπολήψεται μισθὸν ἡ εὑρεθεῖςα ἀμίαντος καὶ ἄσπιλος ἐν ᾗ τὸ πνεῦμα τὸ ἅγιον κατῴκησενἔχεις καὶ ταύτης τῆς παραβολῆς τὴν ἐπίλυσιν.
Ηὐφράνθην, φημί, κύριε, ταύτην τὴν ἐπίλυσιν ἀκούσαςἌκουε νῦν, φησί· τὴν σάρκα σου ταύτην φύλασσε καθαρὰν καὶ ἀμίαντον, ἵνα τὸ πνεῦμα τὸ κατοικοῦν ἐν αὐτῇ ματυρήσῃ αὐτῃ καὶ δικαιωθῇ σου ἡ σάρξβλέπε, μήποτε ἀναβῇ ἐπὶ τὴν καρδίαν σου τὴν σάρκα σου ταύτην φθαρτὴν εἶναι καὶ παραχρήσῃ αὐτῇ ἐν μιασμῷ τινίἐὰν μιάνῃς τὴν σάρκα, οὐ ζήσῃΕἰ δέ τις, φημί, κύριε, γέγονεν ἄγνοια προτέρα, πρὶν ἀκουσθῶσι τὰ ῥήματα ταῦτα, πῶς σωθῇ ὁ ἄνθρωπος ὁ μιάνας τὴν σάρκα αὐτοῦ; Περὶ τῶν προτέρων, φησίν, ἀγνοημάτων τῷ θεῷ μόνῳ δυνατὸν ἴασιν δοῦναι, αὐτοῦ γάρ ἐστι πᾶσα ἐξουσία, ἐὰν τὸ λοιπὸν μὴ μιάνῃς σου τὴν σάρκα μηδὲ τὸ πνεῦμα· ἀμφότερα γὰρ κοινά ἐστι καὶ ἄτερ ἀλλήλων μιανθῆναι οὐ δύναταιἀμφότερα οὖν καθαρὰ φύλασσε, καὶ ζήσῃ τῷ θεῷ.

Παραβολὴ ς’ 
Καθήμενος ἐν τῷ ὄκῳ μου καί δοξάζων τὸν κύριον περὶ πάντων ὧν ἑ̓ἑωράκειν καὶ συζητῶν περὶ τῶν ἐντολῶν, ὅτι καλαὶ καὶ δυναταὶ καὶ ἱλαραὶ καὶ ἔνδοξοι καὶ κυνάμεναι σῶσαι ψυχὴν ἀνθρώπου, ἔλεγον ἐν ἐμαυτῷ· Μακάριος δ̓́ἔσομαι, ἐὰν ταῖς ἐντολαῖς ταύταις πορευθῶ, καὶ ὃς ἂν ταύταις πορευθῇ, μακάριος ἔσταιὡς ταῦτα ἐν ἐμαυτῷ ἐλάλουν, βλέπω αὐτὸν ἐξαίφνης παρακαθήμενόν μοι καὶ λέγοντα ταῦτα· Τί διψυχεῖς περὶ τῶν ἐντολῶν ὧν σοι ἐνετειλάμην; καλαί εἰσιν· ὅλως μὴ διψυχήσῃς, ἀλλ’ ἔνυσαι τὴν πίστιν τοῦ κυρίου, καὶ ἐν αὐταῖς πορεύσῃ· ἐγὼ γάρ σε ἐνδυναμώσω ἐν αὐταῖςαὗται αἱ ἐντολαὶ σύμφοροί εἰσι τοῖς μέλλουσι μεταμοεῖν· ἐὰν γὰρ μὴ πορευθῶσιν ἐν αὐταῖς, εἰς μάτην ἐστὶν ἡ μετάνοια αὐτῶνοἱ οὖν μετανοοῦντες ἀποβάλλετε τὰς πονηρίας τοῦ αἰῶνος τούτου τὰς ἐκτριβούσας ὑμᾶς· ἐνδυσάμενοι δὲ πᾶσιν ἀρετὴν δικαιοσύνης δυνήσεσθε τηρῆσαι τὰς ἐντολὰς ταύτας καὶ μηκέτι προστιθέναι ταῖς ἁμαρτίαις ὑμῶνπορεύεσθε οὖν ταῖς ἐντολαῖς μου ταύταις, καὶ ζήσεσθε τῷ θεῷταῦτα πάντα παρ’ ἐμοῦ λελάλητ αι ὑμῖνκαὶ μετὰ τὸ ταῦτα λαλῆσαι αὐτὸν μετ’ ἐμοῦ, λέγει μοι· Ἄγωμεν εἰς ἀγρόν καὶ δείξω σοι τοὺς ποιμένας τῶν προβάτωνἌγωμεν, φημί, κύριεκαὶ ἤλθομεν εἴς τι πεδίον, καὶ δεικνυμένον σύνθεσιν ἱματίων τῷ χρώματι κροκώδηἔβοσκε δὲ πρόβατα πολλὰ λίαν, καὶ τὰ πρόβατα ταῦτα ὡσεὶ τρυφῶντα ἧν καὶ λίαν σπαταλῶντα καὶ αὐτὸς ὁ ποιμὴν πάνυ ἱλαρὸς ἦν ἐπὶ τῷ ποιμνίῳ αὐτοῦ· καὶ αὐτὴ ἡ ἰδέα τοῦ ποιμένος ἱλαρὰ ἧν λίαν, καὶ ἐν τοῖς προβάτοις περιέτρεχε.
Καὶ λέγει μοι· Βλέπεις τὸν ποιμένα τοῦτον; Βλεπω φημί, κύριεΟὗτος, φησίν, ἄγγελος τρυφῆς καὶ ἀπάτης ἐστίνοὗτος ἐκτρίβει τὰς ψυχὰς τῶν δούλων τοῦ θεοῦ καὶ καταστρέφει αὐτοὺς ἀπὸ τῆς ἀληθείας, ἀπατῶν αὐτοὺς ταῖς ἐπιθομίαις ταῖς πονηραῖς, ἐν αἷς ἀπόλλυνταιἐπιλανθάνονται γὰρ τῶν ἐντολῶν τοῦ θεοῦ τοῦ ζῶντος καὶ πορεύονται ἀπάταις καὶ τρυφαῖς ματαίαις καὶ ἀπόλλυνται ὑπὸ τοῦ ἀγγέλου τούτου, τινὰ μὲν εἰς θάνατον, τινὰ δὲ εἰς καταφθοράνλέγε αὐτῷ· Κύριε, οὐ γινώσκω ἐγώ, τί ἐστιν εἰς θάνατον καὶ τί εἰς καταφθοράνἌκουε, φησίν· ἃ εἶδες πρόβατα ἱλαρὰ καὶ σκιρτῶνα, οὗτοί εἰσιν οἱ ἀπεσπασμένοι ἀπὸ τοῦ θεοῦ εἰς τέλος καὶ παραδεδωκότες ἑαυτοὺς ταῖς ἐπιθυμίαις τοῦ αἰῶνος τούτουἐν τούτοις οὖν μετάνοια ζωῆς οὐκ ἔστιν, ὅτι προσέθηκαν ταῖς ἁματίαις αὐτῶν καὶ εἰς τὸ ὄνομα τοῦ θεοῦ ἐβλασφημησαντῶν τοιούτων οὖν ὁ θάνατός ἐστινἃ δὲ εἶδες πρόβατα μὴ σκιρτῶντα, ἀλλ’ ἐν τόπῳ ἑνὶ βοσκόμενα, οὗτοί εἰσιν οἱ παραδεδωκότες μὲν ἑαυτοὺς ταῖς τρυφαῖς καὶ ἀπάταις, εἰς δὲ τὸν κύριον οὐδὲν ἐβλασφήμησαν· οὗτοι οὖν κατεφθαρμένοι εἰσὶν ἀπὸ τῆς ἀληθείαςἐν τούτοις ἐλπίς ἐστι μετανοίας, ἐν ᾗ δύνανται ζῆσαιἡ καταφθορὰ οὖν ἐλπίδα ἔχει ἀνανεώσεώς τινος, ὁ δὲ θάνατος ἀπώλειαν ἔχει αἰώνιονπάλιν προέβην μικρόν, καὶ δεικνύει μοι ποιμένα μέγαν ὡσεὶ ἄγριον τῇ ἰδέᾳ, περικείμενον δέρμα αἴγειον λευκόν, καὶ πήραν τινὰ εἶχεν ἐπὶ τῶ ὤμων καὶ ῥάβδον σκληρὰν λίαν καὶ ὄζους ἔχουσαν καὶ μάστιγα μεγάλην· καὶ τὸ βλέμμα εἶχε περίπικρον, ὥστε φοβηθῆναί με αὐτόν· τοιοῦτον εἶχε τὸ βλέμμαοὗτος οὖν ὁ ποιμὴν παρελάμβανε τὰ πρόβατα ἀπὸ τοῦ ποιμένος τοῦ νεανίσκου, ἐκεῖνα τὰ σπαταλῶντα καὶ τρυφῶντα, μὴ σκιρτῶντα δέ, καὶ ἔβαλεν αὐτὰ εἴς τινα τόπον κρημνώδη καὶ ἀκανθώδη καὶ τριβολώδη, ὥστε ἀπὸ τῶν ἀκανθῶν καὶ τριβόλων μὴ δύνασθαι ἐκπλέξαι τὰ προβατα, ἀλλ’ ἐμπλέκεσθαι εἰς τὰς ἀκάνθας καὶ τριβόλουςταῦτα οὖν ἐμπεπλεγμένα ἐβόσκοντο ἐν ταῖς ἀκάνθαις καὶ τριβόλοις καὶ λίαν ἐταλαιπώρουν δαιρόμενα ὑπ’ αὐτοῦ· καὶ ὧδε κἀκεῖσε περιήλαυνεν αὐτὰ καὶ ἀνάπαυσιν αὐτοῖς οὐκ ἐδίδου, καὶ ὅλως οὐκ εὐσταθοῦσαν τὰ πρόβατα ἐκεῖνα.
Βλέων οὖν αὐτὰ οὕτω μαστιγούμενα καὶ ταλαιπωρούμενα ἐλυπούμην ἐπ’ αὐτοῖς, ὅτι οὕτως ἐβασανίζοντο καὶ ἀνοχὴν ὅλως οὐκ εἶχονλέγω τῷ ποιμένι τῷ μετ’ ἐμοῦ λαλοῦντι· Κύριε, τίς ἐστιν οὗτος ὁ ποιμὴν ὁ οὕτως ἄσπλαγχνος καὶ πικρὸς καὶ ὅλως μή σπλαγχνιζόμενος ἐπὶ τὰ πρόβατα ταῦτα; Οὗτος, φησίν, ἐστὶν ὁ ἄγγελος τῆς τιμωρίας· ἐκ δὲ τῶν ἀγγέλων τῶν δικααίων ἐστί, κείμενος δὲ ἐπὶ τῆς τιμωρίαςπαραλαμβάνει οὖν τοὺς ἀποπλανωμένους ἀπὸ τοῦ θεοῦ καὶ πορευθέντας ταῖς ἐπιθυμίαις τιμωρίαιςἬθελον, φημί, κύριε, γνῶναι τὰς ποικίλας ταύτας τιμωρίας, ποταπαί εἰσινἌκουε, φησί, τὰς ποικίλας βασάνους καὶ τιμωρίαςβιωτικαί εἰσιν αἱ βάσανοι· τιμωροῦνται γὰρ οἱ μὲν ζημίαις, οἱ δὲ πάσῃ ἀκαταστασίᾳ, οἱ δὲ ὑβριζόμενοι ὑπὸ ἀναξίων καὶ ἑτέραις πολλαῖς πράξεσι πάσχοντεςπολλοὶ γὰρ ἀκαταστατοῦντες ταῖς βουλαῖς αὐτῶν ἐπιβάλλονται πολλά, καὶ οὐδὲν αὐτοῖς ὅλως προχωρεῖκαὶ λέγουσιν ἑαυτοὺς μὴ εὐοδοῦσθαι ἐν ταῖς πράξεσιν αὐτῶν ἐπὶ τὴν καρδίαν, ὅτι ἔπραξαν πονηρὰ ἔργα, ἀλλ’ αἰτῶνται τὸν κυπριονὅταν οὖν θλιβῶσι πάσῃ θλίψει, τότε ἐμοὶ παραδίδονται εἰς ἀγαθὴν παιδείαν καὶ ἰσχυροποιοῦνται ἐν τῇ πίστει τοῦ κυρίου καὶ τὰς λοιπὰς ἡμέρας τῆς ζωῆς αὐτῶν δουλεύουσι τῷ κυρίῳ ἐν καθαρᾷ καρδίᾳ· ἐὰν δὲ μετανοήσωσι, τότε ἀναβαίνει ἐπὶ τὴν καρδίαν αὐτῶν τὰ ἔργα ἃ ἔπραξαν πονηρά, καὶ τότε δοξάζουσι τὸν θεόν, λέγοντες, ὅτι δίκαιος κριτής ἐστι καὶ δικαίως ἔπαθον ἕκαστος κατὰ τὰς πράξεις αὐτοῦ· δουλεύουσι δὲ λοιπὸν τῷ κυρίῳ ἐν καθαρᾷ καρδίᾳ αὐτῶν καὶ εὐοδοῦνται ἐν πάσῃ πράξει αὐτῶν, λαμβάνοντες παρὰ τοῦ κυρίου πάντα, ὅσα ἂν αἰτῶνται· καὶ τότε δοξάζουσι τὸν κύριον, ὅτι ἐμοὶ παρεδόθησαν, καὶ οὐκέτι οὐδὲν πάσχουσι τῶν πονηρῶν.
Λέγω αὐτῷ· Κύριε, ἔτι μοι τοῦτο δήλωσονΤί, φησίν, ἐπιζητεῖς; Εἰ ἄρα φημί, κύριε, τὸν αὐτὸν χρόνον βασανίζονται οἱ τρυφῶντες καὶ ἀπατώμενοι, ὅσον τρυφῶσι καὶ ἀπατῶνται; λέγει μοι· Τὸν αὐτὸν χρόνον βασανίζονταιἘλάχιστον, φημί, κύριε, βασανίζονται· ἔδει γὰρ τοὺς οὕτω τρυφῶντας καὶ ἐπιλανθανομένους τοῦ θεοῦ ἑπταπλασίως βασανίζεσθαιλέγει μοι· Ἄφρων εἶ καὶ οὐ νοεῖς τῆς βασάνου τὴν δύναμινΕἰ γὰρ ἐνόουν φημί, κύριε, οὐκ ἂν ἐπηρώτῳν ἵνα μοι δηλώσῃςἌκουε, φησίν, ἀμφοτέρων τὴν δύναμιντῆς τρυφῆς καὶ ἀπάτης ὁ χρόνος ὥρα ἐστὶ μία· τῆς δὲ βασάνου ἡ ὥρα τριάκοντα ἡμερῶν δύναμιν ἔχειἐὰν οὖν μίαν ἡμέραν τρυφήσῃ τις καὶ ἀπατηθῇ, μίαν δὲ ἡμέραν βασανισθῇ, ὅλον ἐνιαυτὸν ἰσχύει ἡ ἡμερα τῆς βασάνουὅσας οὖν ἡμέρας τρυφήσῃ τις, τοσούτους ἐνιαυτοὺς βασανίζεταιβλέπεις οὖν, φησίν, ὅτι τῆς τρυφῆς καὶ ἀπάτης ὁ χρόνος ἐλάχιστός ἐστι, τῆς τιμωρίας καὶ βασάνου πολύς.
Ἔτι, φημί, κύριε, οὐ νενόηκα ὅλως περὶ τοῦ χρόνου τῆς ἀπάτης καὶ τρυφῆς καὶ βασάνου· τηλαυγέστερόν μοι δήλωσονἀποκριθείς μοι λέγει· Ἡ ἀφρονσύνη σου παράμονός ἐστι, καὶ οὐ θέλεις σου τὴν καρδίαν καθαρίσαι καὶ δουλεύειν τῷ θεῷβλέπε, φησί, μήποτε ὁ χρόνος πληρωθῇ καὶ σὺ ἄφρων εὑρεθῇςἄκουε οὖν, φησί, καθὼς βούλει, ἵινα νοήσῃς αὐτάὁ τρυφῶν καὶ ἀπατώμενος μίαν ἡμέραν καὶ πράσσων, ἃ βούλεται, πολλὴν ἀφρονσύνην ἐνδέδυται καὶ οὐ νοεῖ τὴν πρᾶξιν, ἣν ποιεῖ· εἰς τὴν αὔριον ἐπιλανθάνεται γάρ, τί πρὸ μιᾶς ἔπραξεν· ἡ γὰρ τρυφὴ καὶ ἀπάτη μνήμας οὐκ ἔχει διὰ τὴν ἀφροσύνην, ἣν ἐνδέδυται, ἡ δὲ τιμωρία καὶ ἡ βάσανος ὅταν κολληθῇ τῷ ἀνθρώπῳ μίαν ἡμέραν μέχρις ἐνιαυτοῦ τιμωρεῖται καὶ βασανίζεται· μνήμας γὰρ μεγάλας ἔχει ἡ τιμωρία καὶ ἡ βάσανοςβασανιζόμενος οὖν καὶ τιμωρούμενος ὅλον τὸν ἐνιαυτόν, μνημονεύει τότε τῆς τρυφῆς καὶ ἀπάτης καὶ γινώσκει, ὅτι δι’ αὐτὰ πάσχει τὰ πονηρά´πᾶς οὖν ἄνθρωπος ὁ τρυφῶν καὶ ἀπατώμενος οὕτω βασανίζεται, ὅτι ἔχοντες ζωὴν εἰς θάνατον ἑαυτοὺς παραδεδώκασιΠοῖαι, φημί, κύριε, τρυφαί εἰσι βλαβεραί; Πᾶσα, φησί, πρᾶξις τρυφή ἐστι τῷ ἀνθρώπῳ, ὃ ἐὰν ἡδέως ποιῇ· καὶ γὰρ ὁ ὀξύχολος τῷ ἑαυτοῦ πάθει τὸ ἱκανὸν ποιῶν τρυφᾷ· καὶ ὁ μοιχὸς καὶ ὁ μέθυσος καὶ ὁ κατάλαλος καὶ ὁ ψεύστης καὶ ὁ πλεονέκτης καὶ ὁ ἀποστερητὴς και ὁ τούτοις τὰ ὅμοια ποιῶν τῇ ἰδίᾳ νόσῳ τὸ ἱκανὸν ποιεῖ· τρυφᾷ οὖν ἐπὶ τῇ πράξει αὐτοῦαὗται πᾶσαι αἱ τρυφαὶ βλαβεραί εἰσι τοῖς δούλοις τοῦ θεοῦδιὰ ταύτας οὖν τὰς ἀπάτας πάσχουσιν οἱ τιμωρούμενοι καὶ βασανιζόμενοιεἰσὶν δὲ καὶ τρυφαὶ σώζουσαι τοὺς ἀνθρώπους· πολλοὶ γὰρ ἀγαθὸν ἐργαζόμενοι τρυφῶσι τῇ ἑαυτῶν ἡδονῇ φερόμενοιαὕτη οὖν ἡ τρυφὴ σύμφορός ἐστι τοῖς δούλοις τοῦ θεοῦ καὶ ζωὴν περιποιεῖται τῷ ἀνθρώπῳ τῷ τοιούτῳ· αἱ δὲ βλαβεραὶ τρυφαὶ αἱ προειρημέναι βασάνους καὶ τιμωρίας αὐτοῖς περιποιοῦνται· ἐὰν δὲ ἐπιμένωσι καὶ μὴ μετανοήσωσι, θάνατον ἑαυτοῖς περιποιοῦνται.

Παραβολὴ ζ’ 
Μετὰ ἡμέρας ὀλίγας εἰδον αὐτὸν εἰς τὸ πεδίον τὸ αὐτό, ὅπου καὶ τοὺ ποιμένας ἐωράκειν, καὶ λέγει μοι· Τί ἐπιζητεῖς; Πάρειμι, μημί, κύριε, ἵνα τὸν ποιμένα τὸν τιμωρητὴν κελεύσῃς ἐκ τοῦ οἴκου μου ἐξελθεῖν, ὅτι λίαν με θλίβειΔεῖ σε, φησί, θλιβῆναι· οὕτω γάρ, φησί, προσέταξεν ὁ ἔνδοξος ἄγγελος τὰ περὶ σοῦ· θέλει γάρ σε πειρασθῆναιΤί γάρ, φημί, κύριε, ἐποίησα οὕτω πονηρόν, ἵνα τῷ ἀγγέλῳ τούτῳ παραδοθῶ; Ἄκουε, φησίν· αἱ μὲν ἁμαρτίαι σου πολλαί, ἀλλ’ οὐ τοσαῦται, ὥστε τῷ ἀγγέλῳ τούτῳ παραδοθῆναι· ἀλλ’ ὁ οἶκός σου μεγάλας ἀνομίας καὶ ἁμαρτίας εἰργάσατο, καὶ παρεπικράνθη ὁ ἔνδοξος ἄγγελος ἐπὶ τοῖς ἔργοις αὐτῶν καὶ διὰ τοῦτο ἐκέλευσέ σε χρόνον τινὰ θλιβῆναι, ἵνα κἀκεῖνοι μετανοήσωσι καὶ καθαρίσωσιν ἑαυτοὺς ἀπὸ´πάσης ἐπιθυμίας τοῦ αἰῶνος τούτουὅταν οὖν μετανοήσωσι καὶ καθαρισθῶσι, τότε ἀποστήσεται ἀπὸ σοῦ ὁ ἄγγελος τῆς τιμωρίαςλέγω αὐτῷ· Κύριε, εἰ ἐκεῖνοι τοιαῦτα εἰργάσαντο, ἵνα παραπικρανθῇ ὁ ἔνδοξος ἄγγελος, τί ἐγὼ ἐποίησα; Ἄλλως, φησίν, οὐ δύνανται ἐκεῖνοι θλιβῆναι, ἐὰν μὴ σὺ ἡ κεφαλὴ τοῦ οἴκου θλιβῇς· σοῦ γὰρ θλιβομένου ἐξ ἀνάγκης κἀκεῖνοι θλιβήσονται, εὐσταθοῦντος δὲ σοῦ οὐδεμίαν δύνανται θλῖψιν ἔχεινἈλλ’ ἰδού, φημί, κύριε, μετανενοήκασιν ἐξ ὅλης καρδίας αὐτῶνΟἶδα, φησί, κἀγώ, ὅτι μετανενοήκασιν ἐξ ὅλης καρδίας αὐτῶν· τῶν οὖν μετανοούντων εὐθὺς δοκεῖς τὰς ἁμαρτίας ἀφίεσθαι; οὐ παντελῶς· ἀλλὰ δεῖ τὸν μετανοοῦντα βασανίσαι τὴν ἑαυτοῦ ψυχὴν καὶ ταπεινοφρονῆσαι ἐν πάσῃ πράξει αὐτοῦ ἰσχυρῶς καὶ θλιβῆναι ἐν πάσαις θλίψεσι ποικίλαις· καὶ ἐὰν ὑπενέγκῃ τὰς θλίψεις τὰς ἐπερχομένας αὐτῷ, πάντως σπλαγχνισθήσεται ὁ τὰ πάντα κτίσας καὶ ἐνδυναμώσας καὶ ἴασίν τινα δώσει αὐτῷ· καὶ τοῦτο πάντως, ἐὰν ἴδῃ τὴν καρδίαν τοῦ μετανοοῦντος καθαρὰν ἀπὸ παντὸς πονηροῦ πράγματοςσοὶ δὲ συμφέρον ἐστὶ καὶ τῷ οἴκῳ σου νῦν θλιβῆναιτί δέ σοι πολλὰ λέγω; θλιβῆναί σε δεῖ, καθὼς προσέταξεν ὁ ἄγγελος κυρίου ἐκεῖνος, ὁ παραδιδούς σε ἐμοί· καὶ τοῦτο εὐχαρίστει τῷ κυρίῳ, ὅτι ἄξιόν σε ἡγήσατο τοῦ προδηλῶσαί σοι τὴν θλῖψιν, ἵνα προγνοὺς αὐτὴν ὑπενέγκῃς ἰσχυρῶςλέγω αὐτῷ· Κύριε, σὺ μετ’ ἐμοῦ γίνου, καὶ δυνήσομαι πᾶσαν θλῖψιν ὐπενεγκεῖνἘγώ, φησίν, ἔσομαι μετὰ σοῦ· ἐρωτήσω δέ καὶ τὸν ἄγγελον τὸν τιμωρητήν, ἵινα σε ἐλαφροτέρως θλίψῃ· ἀλλ’ ὀλίγον χρόνον θλιβήσῃ καὶ´πάλιν ἀποκατασταθήσῃ εἰς τὸν οἶκόν σουμόνον παράμεινον ταπεινοφρονῶν καὶ λειτουργῶν τῷ κυρίῳ ἐν πάσῃ καθαρᾷ καρδίᾳ, καὶ τὰ τέκνα σου καὶ ὁ οἶκός σου, καὶ πορεύου ἐν ταῖς ἐντολαῖς μου αἷς σοι ἐντέλλομαι, καὶ δυνήσεταί σου ἡ μετάνοια ἰσχυρὰ καὶ καθαρὰ εἶναι· καὶ ἐὰν ταύτας φυλάξῃς μετὰ τοῦ οἴκου σου, ἀποστήσεται πᾶσα θλῖψις ἀπὸ σοῦ· καὶ ἀπὸ πάτων δέ, φησίν, ἀποστήσεται θλῖψιςὅσοι ἐὰν ἐν ταῖς ἐντολαῖς μου ταύταις πορευθῶσιν.

Παραβολὴ η’
Ἔδειξέ μοι ἰτέαν μεγάλην, σκεπάζουσαν πεδία καὶ ὄρη, καὶ ὑπὸ τὴν σκέπην τῆς ἰτέας πάντες ἐληλύθασιν οἱ κεκλημένοι ἐν ὀνόματι κυρίουεἱστήκει δὲ ἄγγελος κυρίου ἔνδοξος λίαν ὑψηλὸς παρὰ τὴν ἰτέαν, δρέπανον ἔχων μέγα, καὶ ἔκοπτε κλάδους ἀπὸ τῆς ἰτέας, καὶ ἐπεδίδου τῷ λαῷ τῷ σκεπαζομένῳ ὑπὸ τῆς ἰτέας· μικρὰ δὲ ῥαβδία ἐπεδίδου αὐτοῖς, ὡσεὶ πηχυαῖαμετὰ τὸ πάντας λαβεῖν τὰ ῥαβδία ἔθηκε τὸ δρέπανον ὁ ἄγγελος, καὶ τὸ δένδρον ἐκεῖνο ὑγιὲς ἦν, οἷον καὶ ἑωράκειν αὐτόἐθαύμαζον δὲ ἐγὼ ἐν ἐμαυτῷ λέγων· Πῶς τοσούτων κλάδων κεκομμένων τὸ δένδρον τοῦτο ὑγιὲς ἔμεινε τοσούτων κλάδων κοπέντων· ἐὰν δέ, φησί, πάντα ἴδῃς, σοι δηλωθήσεται τὸ τί ἐστινὁ ἄγγελος ὁ ἐπιδεδωκὼς τῷ λαῷ τὰς ῥάβδους πάλιν ἀπῄτει αὐτούς· καὶ καθὼς ἔλαβον, οὕτω καὶ ἐκαλοῦντο πρὸς αὐτόν, καὶ εἷς ἕκαστος αὐτῶν ἀπεδίδου τὰς ῥάβδους ξηρὰς καὶ βεβρωμένας ὡς ὑπὸ σητός· ἐκέλευσεν ὁ ἄγγελος τοὺς τὰς τοιαύτας ῥάβδους ἐπιδεδωκότας χωρὶς ἱστάνεσθαιἕτεροι δὲ ἐπεδίδοσαν ξηράς, ἀλλ’ οὐκ ἦσαν βεβρωμέναι ὑπὸ σητός· καὶ τούτους ἐκέλευσε χωρὶς ἱστάνεσθαιἕτεροι δὲ ἐπεδίδουν ἡμιξήρους· καὶ οὗτοι χωρὶς ἱστάνοντοἕτεροι δὲ ἐπεδίδουν τὰς ῥάβδους αὐτῶν ἡμιξήρους καὶ σχισμὰς ἐχούσας· καὶ οὗτοι χωρὶς ἵσταντοἕτεροι δὲ ἐπεδίδουν τὰς ῥάβδους αὐτῶν χλωρὰς καὶ σχισμὰς ἐχούσας· καὶ οὗτοι χωρὶς ἱστάνοντοἕτεροι δὲ ἐπεδίδουν τὰς ῥάβδους τὸ ἥμισυ ξηρὸν καὶ τὸ ἥμισυ μέρος χλωρόν καὶ οὗτοι χωρὶς ἱστάνοντοἕτεροι δὲ προσέφερον τὰς ῥάβδους αὐτῶν τὰ δύο μέρη τῆς ῥάβδου χλωρά, τὸ δὲ τρίτον ξηρόν· καὶ οὗτοι χωρὶς ἱστάνοντοἕτεροι δὲ ἐπεδίδουν τὰ δύο μέρη ξηρά, τὸ δὲ τρίτον χλωρόν· καὶ οὗτοι χωρὶς ἱστάνοντοἕτεροι δὲ ἐπεδίδουν τὰς ῥάβδους αὐτῶν παρὰ μικρὸν ὅλας χλωράς, ἐλάχιστον δὲ τῶν ῥάβδων αὐτῶν ξηρὸν ἦν, αὐτὸ τὸ ἄκρον· σχισμὰς δὲ εἶχον ἐν αὐταῖς· καὶ οὗτοι χωρὶς ἱστάνοντοἑτέρων δὲ ἦν ἐλάχοντο χλωρόν, τὰ δὲ λοιπὰ τῶν ῥάβδων ξηρά· καὶ οὗτοι χωρὶς ἱστάνοντοἕτεροι δὲ ἤρχοντο τὰς ῥάβδους χλωρὰς φέροντες ὡς ἔλαβον παρὰ τοῦ ἀγγέλου· τὸ δὲ πλεῖον μέρος τοῦ ὄχλου τοιαύτας ῥάβδους ἐπεδίδουνὁ δὲ ἄγγελος ἐπὶ τούτοις ἐχάρη λίαν· καὶ οὗτοι χωρὶς ἱστάνοντοἕτεροι δὲ ἐπεδίδουν τὰς ῥάβδους αὐτῶν χλωρὰς καὶ παραφυάδας ἐχούσας· καὶ οὗτοι χωρὶς ἵσταντο· καὶ ἐπὶ τούτοις ὁ ἀγγελος λίαν ἐχάρηἕτεροι δὲ ἐπεδίδουν τὰς ῥάβδους αὐτῶν χλωρὰς καὶ παραφυάδας ἐχούσας· αἱ δὲ παραφυάδες αὐτῶν ὡσεὶ καρπόν τινα εἶχον· καὶ λίαν ἱλαροὶ ἦσαν οἱ ἄνθρωποι ἐκεῖνοι, ὧν αἱ ῥάβδοι τοιαῦται εὑρέθησανκαὶ ὁ ἄγγελος ἐπὶ τούτοις ἠγαλλιᾶτο, καὶ ὁ ποιμὴν λίαν ἱλαρὸς ἦν ἐπὶ τούτοις.
Ἐκέλευσε δὲ ὁ ἄγγελος κυρίου στεφάνου ἐνεχθῆναικαὶ ἐνέχθησαν στέφανοι ὡσει ἐκ φοινίκων γεγονότες, καὶ ἐσεφάνωσε τοὺς ἄνδρας τους ἐπιδεδωκότας τὰ ῥάβδους τὰς ἐχούσας τὰς παραφυάδας καὶ καρπόν τινα καὶ ἀπέλυσεν αὐτοὺς εἶς τὸν πύργονκαὶ τοὺς ἄλλους δὲ ἀπέστειλεν εἰς τὸν πύργον, τοὺς τὰς ῥάβδοὺς τὰς χλωρὰς ἐπιδεδωκότας καὶ παραφυάδας ἐχούσας, καρπὸν δὲ μὴ ἐχούσας τὰς παραφυάδας, δοὺς αὐτοῖς σφραγῖδαςἱματισμὸν δὲ τὸν αὐτὸν πάντες εἶχον λευκὸν ὡσεὶ χιόνα, οἱ πορευόμενοι εἰς τὸν πύργονκαὶ τοὺς τὰς ῥάβδους ἐπιδεδωκότας χλωρὰς ὡς ἔλαβον ἀπέλυσε, δοὺς αὐτοῖς ἱματισμὸν καὶ σφραγῖδαςμετὰ τὸ ταῦτα τελέσαι τὸν ἄγγελον λέγει τῷ ποιμένι· Ἐγὼ ὑπάγω· σὺ δὲ τούτους ἀπολύσεις εἰς τὰ τείχη, καθὼς ἄξιός ἐστί τις κατοικεῖνκατανόησον δὲ τὰς ῥάβδους αὐτῶν ἐπιμελῶς καὶ οὕτως ἀπόλυσον· ἐπιμελῶς δὲ κατανόησονβλέπε, μή τίς σε παρέλθῃ, φησίν, ἐὰν δὲ τίς σε παρέλθῃ, ἐγὼ αὐτοὺς ἐπὶ τὸ θυσιαστήριον δοκιμάσωταῦτα εἰπὼν τῷ ποιμένι ἀπῆλθεκαὶ μετὰ τὸ ἀπελθεῖν τὸν ἄγγελον λέγει μοι ὁ ποιμήν· Λάβωμεν πάντων τὰς ῥάβδους καὶ φυτεύσωμεν αὐτάς, εἴ τινες ἐξ αὐτῶν δυνήσονται ζῆσαιλέγω αὐτῷ· Κύριε τὰ ξηρὰ ταῦτα πῶς δύναται ζῆσαι; ἀποκριθείς μοι λέγει· Τὸ δένδρον τοῦτο ἰτέα ἐστὶ καὶ φιλόζωον τὸ γένος· ἐὰν οὐν φυτευθῶσι καὶ μικρὰν ἰκμάδα λαμβάνωσιν αἱ ῥάβδοι, ζήσονται πολλαὶ ἐξ αὐτῶν· εἶτα δὲ πειράσωμεν καὶ ὕδωρ αὐταῖς παραχέεινἐὰν τις αὐτῶν δυνηθῇ ζῆσαι, συγχαρήσομαι αὐταῖς· ἐὰν δὲ μὴ ζήσῃ, οὐχ εὑρεθήσομαι ἐγὼ ἀμέλήςἐκέλευσε δέ μοι ὁ ποιμὴν καλέσαι, καθώς τις αὐτῶν ἐστάθηἦλθον τάγματα τάγματα καὶ ἐπεδίδουν τὰς ῥάβδους τῷ ποιμένι· ἐλάμβανε δὲ ὁ ποιμὴν τὰς ῥάβδους καὶ κατὰ τάγματα ἐφύτευσεν αὐτὰς καὶ μετὰ τὸ φυτεῦσαι ὕδωρ αὐταῖς πολὺ παρέχεεν, ὥστε ἀπὸ τοῦ ὕδατος μὴ φαίνεσθαι τὰς ῥάβδουςκαὶ μετὰ τὸ ποτίσαι αὐτὸν τὰς ῥάβδους λέγει μοι· Ἄγωμεν καὶ μετ’ ὀλίγας ἡμέρας ἐπανέλθωμεν καὶ ἐπισκεψώμεθα τὰς ῥάβδους´πάσας· ὁ γὰρ κτίσας τὸ δένδρον τοῦτο θέλει´πάντας ζῆν τοὺς λαβόντας ἐκ τοῦ δένδρου τούτου κλάδουςἐλπίζω δὲ κἀγώ, ὅτι λαβόντα τὰ ῥαβδία ταῦτα ἰκμάδα καὶ ποτισθέντα ὕδατι ζήσονται τὸ πλεῖστον μέρος αὐτῶν.
Λέγω αὐτῷ· Κύριε, τὸ δένδρον τοῦτο γνώρισόν μοι τί ἐστιν· ἀποροῦμαι γὰρ περὶ αὐτοῦ, ὅτι τοσούτων κλάδων κοπέντων ὑγιές ἐστι τὸ δένδρον καὶ οὐδὲν φαίνεται κεκομμένον ἀπ’ αὐτοῦ· ἐν τούτῳ οὖν ἀποροῦμαιἌκουε, φησί· τὸ δένδρον τοῦτο τὸ μέγα τὸ σκεπάζον πεδία καὶ ὄρη καὶ πᾶσαν τὴν γῆν νόμος θεοῦ ἐστιν ὁ δοθεὶς εἰς ὅλον τὸν κόσμον· ὁ δὲ νόμος οὗτος υἱὸς θεοῦ ἐστι κηρυχθεὶς εἰς τὰ πέρατα τῆς γῆς· οἱ δὲ ὑπὸ τὴν σκέπην λαοὶ ὄντες οἱ ἀκουσαντες τοῦ κηρύγματος καὶ πιστεύσαντες εἰς αὐτόν· ὁ δὲ ἄγγελος ὁ μέγας καὶ ἔνδοξος Μιχαὴλ ὁ ἔχων τὴν ἐξουσίαν τούτου τοῦ λαοῦ καὶ διακυβερνῶν αὐτούς· οὗτος γάρ ἐστιν ὁ διδοὺς αὐτοῖς τὸν νόμον εἰς τὰς καρδίας τῶν πιστευόντων· ἐπισκέπτεται οὖν αὐτούς, οἷς ἔδωκεν, εἰ ἄρα τετηρήκασιν αὐτόνβλέπεις δὲ ἑνὸς ἑκαστου τὰς ῥάβδους· αἱ γὰρ ῥάβδοι ὁ νόμος ἐστίβλέπεις οὖν πολλὰς ῥάβδους ἠχρειωμένας, γνώσῃ δὲ αὐτοὺς πάντας τοὺς μὴ τηρήσαντας τὸν νόμον· καὶ ὄψει ἑνὸς ἑκάστου τὴν κατοικίανλέγω αὐτῷ· Κύριε, διατί οὓς μὲν ἀπέλυσεν εἰς τὸν πύργον, οὓς δὲ σοὶ κατέλειψεν; Ὅσοι, φησί, παρέβησαν τὸν νόμον, ὃν ἔλαβον παρ’ αὐτοῦ, εἰς τὴν ἐμὴν ἐξουσίαν κατέλιπεν αὐτοὺς εἰς μετάνοιαν· ὅσοι δὲ ἤδη εὐηρέστησαν τῷ νόμῳ καὶ τετηρήκασιν αὐτόν, ὑπὸ τὴν ἰδίαν ἐξουσίαν ἔχει αὐτούςΤίνες οὖν, φημί, κύριε, εἰσὶν οἱ ἐστεφανωμένοι καὶ εἰς τὸν πύργον ὑπάγοντες; Ὅσοι, φησίν, ἀντεπάλαισαν τῷ διαβόλῳ καὶ ἐνικησαν αὐτόν, ἐστεφανωμένοι εἰσίν· οὗτοι χλωρὰς τὰς ῥάβδους ἐπιδεδωκότες καὶ παραφυάδας ἐχούσας, καρπὸν δὲ μὴ ἐχουσας οἱ ὑπὲρ τοῦ νόμου θλιβέντες, μὴ παθόντες δὲ μηδὲ ἀρνησάμενοι τὸν νόμον αὐτῶνοἱ δὲ χλωρὰς ἐπιδεδωκότες, οἵας ἔλαβον, σεμνοὶ καὶ δίκαιοι καὶ λίαν πορευθέντες ἐν καθαρᾷ καρδίᾳ καὶ τὰς ἐντολὰς κυρίου πεφυλακότεςτὰ δὲ λοιπὰ φνώσῃ, ὅταν κατανοήσω τὰς ῥάβδους ταύτας τὰς πεφυτευμένας καὶ πεποτισμένας.
Καὶ μετὰ ἡμέρας ὀλίγας ἤλθομεν εἰς τὸν τόπον, καὶ ἐκάθισεν ὁ ποιμὴν εις τὸν τόπον τοῦ ἀγγέλου, κἀγὼ παρεστάθην αὐτῷκαὶ λέγει μοι· Περίζωσαι ὠμόλινον καὶ διακόνει μοιπεριεζωσάμην ὠμόλινον ἐκ σάκκου γεγονὸς καθαρόνἰδὼν δέ με περιεζωσμένον καὶ ἕτοιμον ὄντα τοῦ διακονεῖν αὐτῷ, Κάλει, φησί, τοὺς ἄνδρας, ὧν εἰσὶν αἱ ῥάβδοι πεφυτευμέναι, κατὰ τὸ τάγμα, ὥς τις ἔδωκε τὰς ῥάβδουςκαὶ ἀπῆλθον εἰς τὸ πεδίον καὶ ἐκάλεσα πάντας· καὶ ἔστησαν πάντες τάγματα τάματαλέγει αὐτοῖς· Ἕκαστος τὰς ἰδίας ῥάβδους ἐκτιλάτω καὶ φερέτω πρός μεπρῶτοι ἐπέδωκαν οἱ τὰς ξηρὰς καὶ κεκομμένας ἐσχηκότες, καὶ ὡς αὗτα εὑρέθησαν ξηραὶ καὶ κεκομμέναι, ἐκέλευσεν αὐτοὺς χωρὶς σταθῆναιεἶτα ἐπέδωκαν οἱ τὰς ξηρὰς καὶ μὴ κεκομμένας ἔχοντες· τινὲς δὲ ἐξ αὐτῶν ἐπέδωκαν τὰς ῥάβδους χλωράς, τινὲς δὲ ξηρὰς καὶ κεκομμένας ὡς ὑπὸ σητόςτοὺς ἐπιδεδωκότας οὖν χλωρὰς ἐκέλευσε ἐπιδεδωκότας ἐκέλευσε μετὰ τῶν πρώτων σταθῆναιεἶτα ἐπέδωκαν οἱ τὰς ἡμιξήρους καὶ σχισμὰς· τινὲς δὲ χλωρὰς καὶ παραφυάδας ἐχούσας καὶ εἰς τὰς παραφυάδας καρπούς, οἵους εἶχον εἰς τὸν πύργον πορευθέντες ἐστεφανωμένοιτινὲς δὲ ἐπέδωκαν ξηρὰς καὶ βεβρωμένας, τινὲς δὲ ξηρὰς καὶ ἀβρώτους, τινὲς δὲ οἷαι ἦσαν ἡμίξηροι καὶ σχισμὰς ἔχουσαιἐκέλευσεν αὐτοὺς ἕνα ἕκαστον χωρὶς σταθῆναι, τοὺς μὲν πρὸς τὰ ἴδια τάγματα, τοὺς δὲ χωρίς.
Εἶτα ἐπεδίδουν οἱ τὰς ῥάβδους χλωρὰς μὲν ἔχονταες, σχισμὰς δὲ ἐχούσας· οὗτοι´πάντες χλωρὰς ἐπέδωκαν καὶ ἔστησαν εἰς τὸ ἴδιον τάγμαἐχάρη δὲ ὁ ποιμὴν ἐπὶ τούτοις, ὅτι´πάντες ἠλλοιώθησαν καὶ ἀπέθεντο τὰς σχισμὰς αὐτῶνἐπέδωκαν δὲ καὶ οἱ τὸ ἥμισυ χλωρόν, τὸ δὲ ἥμισυ ξηρὸν ἔχοντες· τινῶν οὖν εὑρέθησαν αἱ ξηραὶ καὶ βεβρωμέναι, τινῶν δὲ χλωραὶ καὶ παραφυάδας ἔχουσαι· οὗτοι πάντες ἀπελύθησαν ἕκαστος πρὸς τὸ τάγμα αὐτοῦεἶτα ἐπέδωκαν οἱ τὰ δύο μέρη χλωρὰ ἔχοντες, τὸ δὲ τρίτον ξηρόνπολλοὶ ἐξ αὐτῶν χλωρὰς ἐπέδωκαν, πολλοὶ δὲ ἡμιξήρους, ἕτεροι δὲ ξηρὰς καὶ βεβρωμένας· οὗτοι πάντες ἔστησαν εἰς τὸ ἴδιον τάγμαεἶτα ἐπέδωκαν οἱ τὰ δύο μέρη ξηρὰ ἔχοντες, τὸ δὲ τρίτον χλωρόν· πολλοὶ ἐξ αὐτῶν ἡμιξήρους ἐπέδωκαν, τινες δὲ ξηρὰς καὶ βεβρωμένας, ἕτεροι δὲ ἡμιξήρους, καὶ σχισμὰς ἐχούσας, ὀλίγοι δὲ χλωράς· οὗτοι πάντες ἔστησαν εἰς τὸ ἴδιον τάγμαἐπέδωκαν δὲ οἱ τὰς ῥάβδους αὐτῶν χλωρὰς ἐσχηκότες, ἐλάχιστον δὲ ξηρὸν καὶ σχισμὰς ἐχούσας· ἐκ τούτων τινὲς χλωρὰς ἐπέδωκαν, τινὲς δὲ χλωρὰς καὶ παραφυάδας· ἀπῆλθον καὶ οὗτοι εἰς τὸ τάγμα αὐτῶνεἶτα ἐπέδωκαν οἱ ἐλάχιστον ἔχοντες χλωρόν, τὰ δὲ λοιπὰ μέρη ξηρά· τούτων αἱ ῥάβδοι εὑρέθησαν τὸ πλεῖστον μέρος χλωραὶ καὶ παραφυάδας ἔχουσαι καὶ καρπὸν ἐν ταῖς παραφυάσι, καὶ ἕτεραι χλωραὶ ὅλαιἐπὶ ταύταις ταῖς ῥάβδοις ἐχάρη ὁ ποιμὴν λίαν μεγάλως, ὅτι οὕτως εὑρέθησανἀπῆλθον δὲ οὗτοι ἕκαστος εἰς τὸ ἴδιον τάγμα.
Μετὰ τὸ πάντων κατανοῆσαι τὰς ῥάβδους τὸν ποιμένα λέγει μοι· Εἶπόν σοι, ὅτι τὸ δένδρον μετενόησαν καὶ ἐσώθησαν; βλέπεις, φησί, πόσοι μετενόησαν και ἐσώθησαν; Βλεπω, φημί, κύριεἽνα ἴδῃς, φησί, τὴν πολυευσπλαγχνίαν τοῦ κυρίου, ὅτι μεγάλη καὶ ἔνδοξός ἐστι, καὶ ἔδωκε πνεῦμα τοῖς ἀξίοις οὖσι μετανοίαςΔιατί οὖν, φημί, κύριε, πάντες οὐ μετενόησαν; Ὧν εἶδε, φησί, τὴν καρδίαν μέλλουσαν καθαρὰν γενέσθαι καὶ δουλεύειν αὐτῷ ἐξ ὅλης καρδίας, τούτοις ἔδωκε τὴν μετάνοιαν· ὧν δὲ εἶδε τὴν δολιότητα καὶ πονηρίαν, μελλόντων ἐν ὑποκρίσει μετανοεῖν, ἐκείνοις οὐκ ἔδωκε μετάνοιαν, μήποτε πάλιν βεβηλώσι τὸ ὄνομα αὐτουλέγω αὐτῷ· Κύριε, νῦν οὖν μοι δήλωσον τοὺς τὰς ῥάβδους ἐπιδεδωκότας, ποταπός τις αὐτῶν ἐστί, καὶ τὴν τούτων κατοικίαν, ἵνα ἀκούσαντες οἱ πιστεύσαντες καὶ εἰληφότες τὴν σφραγῖδα καὶ τεθλακότες καὶ εἰληφότες ὑγιῆ, ἐπιγνόντες τὰ ἑαυτῶν ἔργα μενανοήσωσι, λαβόντες ὑπὸ σοῦ σφραγῖδα, καὶ δοξάσωσι τὸν κύριον, ὅτι ἐσπλαγχνίσθη ἐπ’ αὐτοὺς καὶ ἀπέστειλέ σε τοῦ ἀνακαινίσαι τὰ πνεύματα αὐτῶνἌκουε, φησίν· ὧν αἱ ῥάβδοι ξηραὶ καὶ βεβρωμέναι ὑπὸ σητὸς εὑρέθησαν, οὗτοί εἰσιν οἱ ἀποστάται καὶ προδόται τῆς ἐκκλησίας καὶ βλασφημήσαντες ἐν ταῖς ἁμαρτίαις αὐτῶν τὸν κύριον, ἔτι δὲ καὶ ἐπαισχυνθεντες τὸ ὄνομα κυρίου τὸ ἐπικληθὲν ἐπ’ αὐτούςοὗτοι οὖν εἰς τέλος ἀπώλοντο τῷ θεῷβλέπεις δέ, ὅτι οὐδὲ εἷς αὐτῶν μετενόησε, καίπερ ἀκούσαντες τὰ ῥήματα, ἃ ἐλάλησας αὐτοῖς, ἃ σοι ἐνετειλάμην· ἀπὸ τῶν τοιούτων ἡ ζωὴ ἀπέστηοἱ δὲ τὰς ξηρὰς καὶ ἀσήπτους ἐπιδεδωκότες, καὶ οὗτοι ἐγγὺς αὐτῶν· ἦσαν γὰρ ὑποκριταὶ καὶ διδαχὰς ξένας εἰσφέροντες καὶ ἐκστρέφοντες τοὺς δούλους τοῦ θεοῦ, μάλιστα δὲ τοὺς ἡμαρτηκότας, μὴ ἀφιέντες μετανοεῖν αὐτούςοὗτοι οὖν ἔχουσιν ἐλπίδα τοῦ μετανοῆσαιβλέπεις δὲ πολλοὺς ἐξ αὐτῶν καὶ μετανενοηκότας, ἀφ’ ἧς ἐλάλησα αὐτοῖς τὰς ἐντολάς μου· καὶ ἔτι μετανοήσουσινὅσοι δὲ οὐ μετανοήσουσιν, ἀπώλεσαν τὴν ζωὴν αὐτῶνὅσοι δὲ μετενόησαν ἐξ αὐτῶν, ἀγαθοὶ ἐγένοντο, καὶ ἐγένετο ἡ κατοικία αὐτῶν εἰς τὰ τείχη τὰ πρῶτα· τινὲς δὲ καὶ εἰς τὸν πύργον ἀνέβησανβλέπεις οὖν, φησίν, ὅτι ἡ μετάνοια τῶν ἁμαρτιῶν ζωὴν ἔχει, τὸ δὲ μὴ μετανοῆσαι θάνατον.
Ὅσοι δὲ ἡμιξήρους ἐπέδωκαν καὶ ἐν αὐταῖς σχισμὰς εἶχον, ἄκουε καὶ περὶ αὐτῶνὅσων ἦσαν αἱ ῥάβδοι ἡμίξηροι, δίψυχοί εἰσιν· οὔτε γὰρ ζῶσιν οὔτε τεθνήκασινοἱ δὲ ἡμιξήρους ἔχοντες καὶ ἐν αὐταῖς σχισμάς, οὗτοι καὶ δίψυχοι καὶ κατάλαλοί εἰσι καὶ μηδέποτε εἰρηνεύοντες εἰς ἑαυτούς, ἀλλὰ διχοστατοῦντες πάντοτεἀλλὰ καὶ τούτοις, φησίν, ἐπίκειται μετάνοιαβλέπεις, φησί, τινὰς ἐξ αὐτῶν μετανενοηκότες καὶ ἔτι, φησίν ἐξ αὐτῶν μετανενήκασι, τὴν κατοικίαν εἰς τὸν πύργον ἔξουσιν· ὅσοι δὲ ἐξ αὐτῶν βραδύτερον μετανενοήκασιν, ἀλλ’ ἐμμένουσι ταῖς πράξεσιν αὐτῶν, θανάτῳ ἀποθανοῦνταιοἱ δὲ χλωρὰς ἐπιδεδωκότες τὰς ῥάβδους αὐτῶν καὶ σχισμὰς ἐχούσας, πάντοτε οὗτοι πιστοὶ καὶ ἀγαθοὶ ἐγένοντο, ἔχοντες δὲ ζῆλόν τινα ἐν ἀλλήλοις περὶ πρωτείων καὶ περὶ δόξης τινός· ἀλλὰ´πάντες οὗτοι μωροί εἰσιν, ἐν ἀλλήλοις ἔχοντες ζῆλον περὶ πρωτείωνἀλλὰ καὶ οὗτοι ἀδούσαντες τῶν ἐντολῶν μου, ἀγαθοὶ ὄντες, ἐκαθάρισαν ἑαυτοὺς καὶ μετενόησαν ταχύἐγένετο οὖν ἡ κατοίκησις αὐτῶν εἰς τὸν πύργον· ἐὰν δέ τις πάλιν ἐπιστρέψῃ εἰς τὴν διχοστασίαν, ἐκβληθήσεται ἀπό τοῦ πύργου καὶ ἀπολέσει τὴν ζωὴν αὐτοῦἡ ζωὴ πάντων ἐστὶ τῶν τὰς ἐντολὰς τοῦ κυρίου φυλασσόντων· ἐν ταῖς ἐντολαῖς δὲ περὶ πρωτείων ἢ περὶ ταπεινοφρονήσεως ἀνδρόςἐν τοῖς τοιούτοις οὖν ἡ ζωὴ τοῦ κυρίου ἐν τοῖς διχοστάταις δὲ καὶ παρανόμοις θάνατος.
Οἱ δὲ ἐπιδεδωκότες τὰς ῥάβδους ἥμισυ μὲν χλωράς, ἥμισυ δὲ ξηράς, οὗτοί εἰσιν οἱ ἐν ταῖς πραγματείαις ἐμπεφυρμένοι καὶ μὴ κολλώμενοι τοῖς ἁγίοις· διὰ τοῦτο τὸ ἥμισυ αὐτῶν ζῇ, τὸ δὲ ἥμισυ νεκρόν ἐστιπολλοὶ οὖν ἀκούσαντές μου τῶν ἐντολῶν μετενόησανὅσοι γοῦν μετενοησαν, ἡ κατοικία αὐτῶν εἰς τέλος ἀπέστησανοὗτοι οὖν μετάνοιαν οὐκ ἔχουσιν· διὰ γὰρ τὰς πραγματείας αὐτῶν ἐβλασφήμησαν τὸν κύριον καὶ ἀπηρνήσαντοἀπώλεσαν οὖν τὴν ζωὴν αὐτῶν διὰ τὴν πονηρίαν, ἣν ἔπραξανπολλοὶ δὲ ἐξ αὐτῶν ἐδιψύχησανοὗτοι ἔτι ἔχουσι μετάνοιαν, ἐὰν ταχὺ μετανοήσωσι, καὶ ἔσται αὐτῶν ἡ κατοικία εἰς τὸν πύγον· ἐὰν δὲ βραδύτερον μετανοήσωσι, κατοικήσουσιν εἰς τὰ τείχη· ἐὰν δὲ μὴ μετανοήσωσι, καὶ αὐτοὶ ἀπώλεσαν τὴν ζωὴν αὐτῶνοἱ δὲ τὰ δύο μέρη χλωρά, τὸ δὲ τρίτον ξηρὸν ἐπιδεδωκότες, οὗτοί εἰσιν οἱ ἀρνησάμενοι ποικίλαις ἀρνήσεσιπολλοὶ οὖν πετενόησαν ἐξ αὐτῶν, καὶ ἀπῆλθον εἰς τέλος τοῦ θεοῦ· οὗτοι τὸ ζὴν εἰς τέλος ἀπώλεσαντινὲς δὲ ἐξ αὐτῶν ἐδιψύχησαν καὶ εδιχοστάτησαντούτοις οὖν ἐστὶ μετάνοια, ἐὰν ταχὺ μετανοήσωσι καὶ μὴ ἐπιμείνωσι ταῖς ἡδοναῖς αὐτῶν· ἐὰν δὲ ἐπιμείνωσι ταῖς πράξεσιν αὐτῶν, καὶ οὗτοι θάνατον ἑαυτοῖς κατεργάζονται.
Οἱ δὲ ἐπιδεδωκότες τὰς ῥάβδους τὰ μὲν δύο μέρη ξηρά, τὸ δὲ τρίτον χλωρόν, οὗτοί εἰσι πιστοὶ μὲν γεγονότες, πλουτήσαντες δὲ καὶ γενόμενοι ἔνδοξοι παρὰ τοῖς ἔθνεσιν· ὑπερηφανίαν μεγάλην ἐνεδύσαντο καὶ ὑψηλόφρονες ἐγένοντο καὶ κατέλιπον τὴν ἀλήθειαν καὶ οὐκ ἐκολλήθησαν τοῖς δικαίοις, ἀλλὰ μετὰ τῶν ἐθνῶν συνέζησαν, καὶ αὕτη ἡ ὁδὸς ἡδυτέρα αὐτοῖς ἐγένετο· ἀπὸ δὲ τοῦ θεοῦ οὐκ ἀπέστησαν, ἀλλ’ ἐνέμειναν τῇ πίστει, μὴ ἐργαζόμενοι τὰ ἔργα τῆς πίστεωςπολλοὶ οὖν ἐξ αὐτῶν μετενόησαν, καὶ ἐγένετο ἡ κατοίκησις αὐτῶν ἐν τῷ πύργῳἕτεροι δὲ εἰς τέλος μετὰ τῶν ἐθνῶν συζῶντες καὶ φθειρόμενοι ταῖς κενοδοξίαις τῶν ἐθνῶν ἐλογίσθησανἕτεροι δὲ ἐξ αὐτῶν ἐδιψύχησαν μὴ ἐλπίζοντες σωθῆναι διὰ τὰς πράξεις, ἃς ἔπραξαν· ἕτεροι δὲ ἐδιψύχησαν καὶ σχίσματα ἐν ἑαυτοῖς ἐποίησαν, τούτοις οὖν τοῖς διψυχήσασι διὰ τὰς πράξεις αὐτῶν μετάνοια ἔτι εἶναι, ἵνα ἡ κατοικία αὐτῶν γένηται εἰς τὸν πύργον τῶν δὲ μὴ μετανοούντων, ἀλλ’ ἐπιμενόντων ταῖς ἡδοναῖς, ὁ θάνατος ἐγγύς.
Οἱ δὲ τὰς ῥάβδους ἐπιδεδωκότες χλωράς, αὐτὰ δὲ τὰ ἄκρα ξηρὰ καὶ σχισμὰς ἔχοντα, οὗτοι πάντοτε ἀγαθοὶ καὶ πιστοὶ καὶ ἔνδοξοι παρὰ τῷ θεῷ ἐγένοντο, ἐλάχιστον δὲ ἐξήμαρτον διὰ μικρὰς ἐπιθυμίας καὶ μικρὰ κατ’ ἀλλήλων ἔχοντες· ἀλλ’ ἀκούσαντές μου τῶν ῥημάτων τὸ πλεῖστον μέρος ταχὺ μετενόησαν, καὶ ἐγένετο ἡ κατοικία αὐτῶν εἰς τὸν πύργοντινὲς δὲ ἐξ αὐτῶν ἐδιψύχησαν, τινὲς δὲ διψυχήσαντες διχοστασίαν μείζονα ἐποίησανἐν τούτοῖς οὖν ἔνεστι μετανοίας ἐλπίς, ὅτι ἀγαθοὶ πάντοτε ἐγένοντο· δυσκόλως δέ τις αὐτῶν ἀποθανεῖταιοἱ δὲ τὰς ῥάβδους αὐτῶν ξηρὰς ἐπιδεδωκότες, ἐλάχιστον δὲ χλωρὸν ἐχούσας, οὗτοί εἰσιν οἱ πιστεύσαντες μόνον, τὰ δὲ ἔργα τῆς ἀνομίας ἐργασάμενοι· οὐδέποτε δὲ ἀπὸ τοῦ θεοῦ ἀπέστησαν καὶ τὸ ὄνομα ἡδέως δ̓ἐβάτασαν καὶ εἰς τοὺς οἴκους αὐτῶν ἡδέως ὑπεδέξαντο τοὺς δούλους τοῦ θεοῦἀκούσαντες οὖν ταύτην τὴν μετάνοιαν ἀδιστάκτως μετενόησαν, καὶ ἐργάζονται πᾶσαν ἀρετὴν καὶ δικαιοσύνηντινὲς δὲ ἐξ αὐτῶν καὶ φοβοῦνται, γινώσκοντες τὰς πράξεις αὐτῶν, ἃς ἔπραξαντούτων οὖν πάντων ἡ κατοικία εἰς τὸν πύργον ἔσται.
Καὶ μετὰ τὸ συντελέσαι αὐτὸν τὰς ἐπιλύσεις πασῶν τῶν ῥάβδων λέγει μοι· Ὕπαγε καὶ πᾶσιν λέγε, ἵνα μετανοήσωσιν, καὶ ζήσωνται τῷ θεῷ· ὅτι ὁ κύριος ἔπεμψέ με σπλαγχνισθεὶς πᾶσι δοῦναι τὴν μετάνοιαν, καὶπερ τινῶν μὴ ὄντων ἀξίων διὰ τὰ ἔργα αὐτῶν· ἀλλὰ μακρόθυμος ὢν ὁ κύριος θέλει τὴν κλῆσιν τὴν γενομένην διὰ τοῦ υἱοῦ αὐτοῦ σώζεσθαιλέγω αὐτῷ· Κύριε, ἐλπίζω, ὅτι πάντες ἀκούσαντες αὐτὰ μετανοήσουσι· πείθομαι γάρ, ὅτι εἷς ἕκαστος τὰ ἴδια ἔργα ἐπιγνοὺςκαὶ φοβηθεὶς τὸν θεὸν μετανοήσειἀποκριθείς μοι λέγει· Ὅσοι, φησίν, ἐξ ὅλης καρδίας αὐτῶν μετανοήσωσι καὶ καθαρίσωσιν ἑαυτοὺς ἀπὸ τῶν πονηριῶν αὐτῶν τῶν ἁμαρτίαις αὐτῶν, λήψονται ἴασιν παρὰ τοῦ κυρίου τῶν προτέρων ἁμαρτιῶν, ἐὰν μὴ διψυχήσωσιν ἐπὶ ταῖς ἐντολαῖς ταύταις, καὶ ζήσονται τῷ θεῷὅσοι δέ, φησίν, προσθῶσι ταῖς ἁμαρτίαις αὐτῶν καὶ πορευθῶσιν ἐν ταῖς ἐπιθυμίαις τοῦ αἰῶνος τούτου, θανάτῳ ἑαυτοὺς κατακρινοῦσινσὺ δὲ πορεύου ἐν ταῖς ἐντολαῖς μου, καὶ ζήσῃ τῷ θεῷ· καὶ ὅσοι ἂν πορευθῶσιν ἐν αὐταῖς καὶ ἐργάσωνται ὀρθῶς, ζήσονται τῷ θεῷταῦτά μοι διέξας καὶ λαλήσας πάντα λέγει μοι· Τὰ δὲ λοιπὰ ἐπιδείξω μετ’ ὀλίγας ἡμέρας.

Παραβολὴ θ’
Μετὰ τὸ γράψαι με τὰς ἐντολὰς καὶ παραβολὰς τοῦ ποιμένος, τοῦ ἀγγέλου τῆς μετανοίαςἦλθε πρός με καὶ λέγει μοι· Θέλω σοι δεῖξαι, ὅσα σοι ἔδειξε τὸ πνεῦμα τὸ ἅγιον τὸ λαλῆσαν μετὰ σοῦ ἐν μορφῇ τῆς Ἐκκλησίας· ἐκεῖνο γὰρ τὸ πνεῦμα ὁ υἱὸς τοῦ θεοῦ ἐστινἐπειδὴ γὰρ ἀσθενέστερος τͅτῇ σαρκὶ ἦς, οὐκ ἐδηλώθη σοι δι’ ἀγγέλουὅτε οὖν ἐνεδυναμώθης διὰ τοῦ πνεύματος καὶ ἴσχυσας τῇ ἰσχύϊ σου, ὥστε δύνασθαί σε καὶ ἄγγελον ἰδεῖν, τότε μὲν οὖν ἐφανερώθη σοι διὰ τῆς Ἐκκλησίας ἡ οἰκοδομὴ τοῦ πύργου· καλῶς καὶ σεμνῶς πάντα ὡς ὑπὸ παρθένου ἑώρακαςνῦν δὲ ὑπὸ ἀγγέλου βλέπεις διὰ τοῦ αὐτοῦ μὲν πνεύματος· δεῖ δέ σε παρ’ ἐμοῦ ἀκριβέστερον πάντα μαθεῖνεἰς τοῦτο γὰρ καὶ ἐδόθην ὑπὸ τοῦ ἐνδόξου ἀγγέλου εἰς τὸν οἶκόν σου κατοικῆσαι, ἵνα δυνατῶς πάντα ἴδῃς, μηδὲν δειλαινόμενος καὶ ὡς τὸ πρότερονκαὶ ἀπήγαγέ με εἰς τὴν Ἀρκαδίαν, εἰς ὄρος τι μαστῶδες καὶ ἐκάθισέ με ἐπὶ το ἄκρον τοῦ ὄρους καὶ ἔδειξέ μοι πεδίον μέγακύκλῳ δὲ τοῦ πεδίου ὄρη δώδεκα, ἄλλην καὶ ἄλλην ἰδέαν ἔχοντα τὰ ὄρη δώδεκα, ἄλλην καὶ ἄλλην ἰδέαν ἔχοντα τὰ ὄρητὸ πρῶτον ἦν μέλαν ὡς ἀσβόλη· τὸ δὲ δεύτερον ψιλόν, βοτάνας μὴ ἔχον· τὸ δὲ τρίτον ἀκανθῶν καὶ τριβόλων πλῆρες· τὸ δὲ τέταρτον βοτάνας ἔχον ἡμιξήρους, τὰ μὲν ἐπάνω τῶν βοτανῶν χλωρά, τὰ δὲ πρὸς ταῖς ῥίζαις ξηρά· τινὲς δὲ βοτάναι, ὅταν ὁ ἥλιος ἐπικεκαύκει, ξηραὶ ἐγίνοντο· τὸ δὲ πέμπτον ὅρος ἔχον βοτάνας χλωρὰς καὶ τραχὺ ὄντὸ δὲ ἕκτον ὄρος σχισμῶν ὅλως ἔγεμεν, ὧν μὲν μικρῶν, ὧν δὲ μεγάλων· εἶχον δὲ βοτάνας αἱ σχισμαί, οὐ λίαν δὲ ἦσαν εὐθαλεῖς αἱ βοτάναι, μᾶλλον δὲ ὡς μεμαραμμέναι ἦσαντὸ δὲ ἕβδομον ὄρος εἶχε βοτάνας ἱλαράς, καὶ ὅλον τὸ ὄρος εὐθηνοῦν ἦν, καὶ πᾶν γένος κτηνῶν καὶ ὀρνέων ἐνέμοντο εἰς τὸ ὄρος ἐκεῖνο· καὶ ὅσον ἐβόσκοντο τὰ κτήνη καὶ τὰ πετεινά, μᾶλλον καὶ μᾶλλον αἱ βοτάναι τοῦ ὄρους ἐκείνου ἔθαλλοντὸ δὲ ὄγδοον ὄρος πηγῶν πλῆρες ἦν, καὶ πᾶν γένος τῆςκτίσεως τοῦ κυρίου ἐποτίζοντο ἐκ τῶν πηγῶν τοῦ ὄρους ἐκείνουτὸ ἐρημῶδες ἦνεἶχε δὲ ἔννατον ὀρος ὅλως ὕδωρ οὐκ εἶχεν καὶ ὅλον ἐρημῶδες ἦνεἶχε δὲ ἐν αὐτῷ θηρία καὶ ἑρπετὰ θανάσιμα διαφθείροντα ἀνθρώπουςτὸ δὲ δέκατον ὄρος εἶχε δένδρα μέγιστα καὶ ὅλον κατάσκιον ἦν, καὶ ὑπὸ τὴν σκέπην τῶν δένδρων πρόβατα κατέκειντο ἀναπαυόμενα καὶ μαρυκώμενατὸ δὲ ἑνδέκατον ὄρος λίαν σύνδενδρον ἦν, καὶ τὰ δένδρα ἐκεῖνα κατάκαρπα ἦν, ἄλλοις καὶ ἄλλοις καρποῖς κεκοσμημένα, ἵνα ἰδών τις αὐτὰ ἐπιθυμήσῃ φαγεῖν ἐκ τῶν καρπῶν αὐτῶντὸ δὲ δωδέκατον ὄρος ὅλον ἦν λευκόν, καὶ ἡ πρόσοψις αὐτοῦ ἱλαρὰ ἦν· καὶ εὐπρεπέστατον ἦν ἐν αὑτῷ τὸ ὄρος.
Εἰς μέσον δὲ τοῦ πεδίου ἔδειξέ μοι πέτραν μεγάλην λευκὴν ἐκ τοῦ πεδίου ἀναβεβηκυῖανἡ δὲ πέτρα ὑψυλοτέρα ἦν τῶν ὀρέων, τετράγωνος, ὥστε δύνασθαι ὅλον τὸν κόμον χωρῆσαιπαλαιὰ δὲ ἦν ἡ πέτρα ἐκείνη, π́πύλην ἐκκεκομμένην ἔχουσα· ὡς πρόσφατος δὲ ἐδόκει μοι εἶναι ἡ ἐκκόλαψις τῆς πύληςἡ δὲ πύλην οὗτως ἔστιλβεν ὑπὲρ τὸν ἥλιον, ὥστε με θαυμάζειν ἐπὶ τῇ λαμπηδόνι τῆς πύληςκύκλῳ δὲ τῆς πύλης εἱστήκεισαν παρθένοι δώκεκααἱ οὖν τέσσαρες αἱ εἰς τὰς γωνίας ἑστηκυῖαι ἐνδοξότεραί μοι ἐδόκουν εἶναι· καὶ αἱ ἀλλαι δὲ ἔδοξοι ἦσανεἱστήκεισαν δὲ εἰς τὰ τέσσαρα μέρη τῆς πύλης, ἀνὰ μέσον αὐτῶν ἀνὰ δύο παρθένοιἐνδεδυμέναι δὲ ἦσαν λινοῦς χιτῶνας καὶ περιεζωσμέναι ἦσαν εὐπρεπῶς, ἔξω τοὺς ὤμους ἔχουσαι τοὺς δεξιοὺς ὡς μέλλουσαι φορτίον τι βαστάζεινοὕτως ἕτοιμοι ἦσαν· λίαν γὰρ ἱλαραὶ ἦσαν καὶ πρόθυμοιμετὰ τὸ ἰδεῖν με ταῦτα ἐθαύμαζον ἐν ἐμαυτῷ, ὅτι μεγάλα καὶ ἔνδοξα πράγματα βλέπωκαὶ πάλιν διηπόρουν ἐπὶ ταῖς παρθένοις, ὅτι τρυφεραὶ οὕτως οὖσαι ἀνδρείως εἱστήκεισαν ὡς μέλλουσαι ὅλον τὸν οὐρανὸν βαστάζεινκαὶ λέγει μοι ὁ ποιμήν· Τί ἐν σεαυτῷ διαλογίζῃ καὶ διαπορῇ καὶ σεαυτῷ λύπην ἐπισπάσαι; ὅσα γὰρ οὐ δύνασαι νοῆσαι, μὴ ἐπιχείρει, συνετὸς ὤν, ἀλλ’ ἐρώτα τὸν κύριον, ἵνα λαβὼν σύνεσιν νοῇς αὐτάτὰ ὀπίσω σου ἰδεῖν οὐ δύνασαι, ἔασον, καὶ μὴ στρέβλου σεαυτόν· ἃ δὲ βλέπεις, ἐκείνων κατακυρίευε καὶ περὶ τῶν λοιπῶν μὴ περιεργάζου· πάντα δέ σοι ἐγὼ δηλώσω, ὅσα ἄν σοι δείξωἔμβλεπε οὖν τοῖς λοιποῖς.
Εἶδον ἐξ ἄνδρας ἐληλυθότας ὑψηλοὺς καὶ εὐδόξους καὶ ὁμοίους τῇ ἰδέᾳ· καὶ ἐκαλεσαν πλῆθός τι ἀνδρῶνκἀκεῖνοι δὲ οἱ ἐληλυθότες ὑψηλοὶ ἦσαν ἄνδρες καὶ καλοὶ καὶ δυνατοί· καὶ ἐκέλευσαν αὐτοὺς οἱ ἐξ ἄνδρες οἰκοκομεῖν ἐπάνω τῆς πέτρας πύργον τινάἦν δὲ θόρυβος τῶν ἀνδρῶν ἐκείνων μέγας τῶν ἐληλυθότων οἰκοδομεῖν τὸν πύργον ὧδε κἀκεῖσε περιτρεχόντων κύκλῳ τῆς πύληςαἱ δὲ παρθένοι ἑστηκυῖαι κύκλῳ τῆς πύλης ἔλεγον τοῖς ἀνδράσι σπεύδειν τὸ πύργον οἰκοδομεῖσθαι· ἐκπεπετάκεισαν δὲ τὰς χεῖρας αἱ παρθένοι ὡς μέλλουσαί τι λαμβάνειν παρὰ τῶν ἀνδρῶνοἱ δὲ ἐξ ἄνδρες ἐκέλευον ἐκ βυθοῦ τινος λίθους ἀναβαίνειν καὶ ὑπάγειν εἰς τὴν οἰκοδομὴν τοῦ πύργουἀνέβησαν δὲ λίθοι δέκα τετράγωνοι λαμπροί, μὴ λελατομημένοιοἱ δὲ ἓξ ἄνδρες ἐκάλουν τὰς παρθένους καὶ ἐκέλευσαν αὐτὰς τοὺς λίθους πάντας τοὺς μέλλοντας εἰς τὴν οἰκοδομὴν ὑπάγειν τοῦ πύλης καὶ ἐπιδιδόναι τοῖς ἀνδράσι τοῖς μέλλουσιν οἰκοδομεῖν τὸν πύργοναἱ δὲ παρθένοι τοὺς δέκα λίθους τοὺς πρώτους τοὺς ἐκ τοῦ βυθοῦ ἀναβάντας ἐπετίθουν ἀλλήλαις καὶ κατὰ ἕνα λίθον ἐβάσταζον ὁμοῦ.
Καθὼς δὲ ἐστάθησαν ὁμοῦ κυκλῳ τῆς πύλης, οὕτως ἐβάσταζον αἱ δοκοῦσαι δυναταὶ εἶναι καὶ ὑπὸ τὰς γωνίας τοῦ λίθου ὑποδεδυκυῖαι ἦσαναἱ δὲ ἄλλαι ἐκ τῶν πλευρῶν τοῦ λίθου ὑποδεδύκεισαν καὶ οὕτως ἐβάσταζον πάντας τοὺς λίθους· διὰ δὲ τῆς πύλης διέφερον αὐτούς, καθὼς ἐκελεύσθησαν, καὶ ἐπεδίδουν τοῖς ἀνδράσιν εἰς τὸν πύργον· ἐκεῖνοι δὲ ἔχοντες τοὺς λίθους ᾠκοδόμουνἡ οἰκοδομὴ δὲ τοῦ πύργου ἐπὶ τὴν πέτραν τὴν μεγάλην καὶ ἐπάνω τῆς πύληςἡρμόσθησαν οὖν οἱ δέκα λίθοι ἐκεῖνοι καὶ ἐπέπλησαν ὅλην τὴν πετραν· καὶ ἐγένοντο ἐκεῖνοι θεμέλιος τῆς οἰκοδομῆς τοῦ πύγου· ἡ δὲ πέτρα καὶ ἡ πύλη ἦν βαστάζουσα ὅλον τὸν τὸν πύγον· μετὰ δὲ τοὺς δέκα λίθους ἄλλοι ἀνέβησαν ἐκ τοῦ βυθοῦ εἴκοσι λίθοι· καὶ οὗτοι ἡρμόσθησαν εἰς τὴν οἰκοδομὴν τοῦ πύργου, βασταζόμενοι ὑπὸ τῶν παρθένων καθὼς καὶ οἱ πρότεροιμετὰ δὲ τούτους ἀνέβησαν λε’, καὶ οὗτοι πάντες ἐβλήθησαν εἰς τὴν οἰκοδομὴν τοῦ πύργου· ἐγένοντο οὖν στοῖχοι τέσσαρες ἐν τοῖς θεμελίοις τοῦ πύργουκαὶ ἐπαύσαντο δὲ καὶ οἱ οἰκοδομοῦντες μικρόνκαὶ πάλιν ἐπέταξαν οἱ ἓξ ἄνδρες τῷ πλήθει τοῦ ὄχλου ἐκ τῶν ὀρέων παραφέρειν λίθους εἰς τὴν οἰκοδομὴν τοῦ πύργουπαρεφέροντο οὖν ἐκ πάντων τῶν ἀνδρῶν καὶ ἐπεδίδοντο ταῖς παρθένοις· αἱ δὲ παρθένοι διέφερον αὐτοὺς διὰ τῆς πύλης καὶ ἐπεδίδουν εἰς τὴν οἰκοδομὴν τοῦ πύργουκαὶ ὅταν εἰς τὴν οἰκοδομὴν ἐτέθησαν οἱ λίθοι οἱ ποικίλοι, ὅμοιοι ἐγένοντο λευκοὶ καὶ τὰς χρόας τὰς ποικίλας ἤλλασσοντινὲς δὲ λίθοι ἐπεδίδοντο ὑπὸ τῶν ἀνδρῶν εἰς τὴν οἰκοδομὴν καὶ οὐκ ἐγίνοντο λαμπροί, ἀλλ’ οἷοι ἐτέθησαν, τοιοῦτοι καὶ εὑρέθησαν· οὐ γὰρ ἦσαν ὑπὸ τῶν παρθένων ἐπιδεδομένοι οὐδὲ διὰ τῆς πύλης παρενηνεγμένοιοὗτοι οὖν οἱ λίθοι ἀπρεπεῖς ἦσαν ἐν τῇ οἶκοδομνῇ τοῦ πύργουἰδόντες δὲ οἱ ἓξ ἄδρες τοὺς λίθους τοὺς ἀπρεπεῖς ἐν τῇ οἰκοδομῇ ἐκέλευσαν αὐτοὺς ἀρθῆναι καὶ ἀπαχθυῆναι κάτω εἰς τὸν ἴδιον τόπον, ὅθεν ἠνέχθησανκαὶ λέγουσι τοῖς ἀνδράσι τοῖς παρεμφέρουσι τοὺς λίθους· Ὅλως ὑμεῖς μὴ ἐπιδίδοτε εἰς τὴν οἰκοδομὴν λίθους· τίθετε δὲ αὐτοὺς παρὰ τὸν πύργον, ἵνα αἱ παρθένοι διὰ τῆς πύλης παρενέγκωσιν αὐτοὺς καὶ ἐπιδιδῶσιν εἰς τὴν οἰκοδομήνἐὰν γάρ, φασί, διὰ τῶν χειρῶν τῶν παρθένων τούτων μὴ παρενεχθῶσι διὰ τῆς πύλης, τὰς χρόας αὐτῶν ἀλλάξαι οὐ δύνανται· μὴ κοπιᾶτε οὖν, φασίν, εἰς μάτην.
Καὶ ἐτελέσθη τῇ ἡμέρᾳ ἐκείνῃ ἡ οἰκοδομή, οὐκ ἀπετελέσθη δὲ ὁ πύργος· ἔμελλε γὰρ πάλιν ἐποικοδομεῖσθαι· καὶ ἐγένετο ἀνοχὴ τῆς οἰκοδομῆςἐκελευσαν δὲ οἱ ἓξ ἄδρες τοὺς οἰκοδομοῦντας ἀναχωρῆσαι μικρὸν´πάντας καὶ ἀναπαυθῆναι· ταῖς δὲ παρθένοις ἐπέταξαν ἀπὸ τοῦ πύργου μὴ ἀναχωρῆσαιἐδόκει δέ μοι τὰς παρθένους καταλελεῖφθαι τοῦ φυλάσσειν τὸν πύργονμετὰ δὲ τὸ ἀναχωρῆσαι πάντας καὶ ἀναπαυθῆναι λέγω τῷ ποιμένι· Τί ὅτι, φημί, κύριε, οὐ συνετελέσθη ἡ οἰκοδομὴ τοῦ πύργου; Οὔπω, φησί, δύναται ἀποτελεσθῆναι ὁ πύργος, ἐὰν μὴ ἔλθῃ ὁ κύριος αὐτοῦ καὶ δοκιμάσῃ τὴν οἰκοδομὴν ταύτην, ἵνα, ἐάν τινες λίθοι σαπροὶ εὑρεθῶσιν, ἀλλάξῃ αὐτούς· πρὸς γὰρ τὸ ἐκείνου θέλημα οἰκοδομεῖται ὁ πύργοςἬθελον, φημί, κύριε, τούτου τοῦ γνῶναι τί ἐστιν ἡ οἰκοδομὴ αὕτη, καὶ περὶ τῆς πέτρας καὶ πύλης καὶ τῶν ὀρέων καὶ τῶν παρθένων καὶ τῶν λίθων τῶν ἐκ τοῦ βυθοῦ ἀναβεβηκότων καὶ μὴ λελατομημένων, ἀλλ’ οὕτως ἀπελθόντων εἰς τὴν οἰκοδομήνκαὶ διατί πρῶτον εἰς τὰ θεμέλια ι’ λίθοι ἐτέθησαν, εἶτα κ’, εἶτα λε’ εἶτα μ’ θεμέλια ι’ λίθοι ἐτέθησαν, εἶτα κ’, εἶτα λε’ εἶτα μ’, καὶ περὶ τῶν λίθων τῶν ἀπεληλυθότων εἰς τὴν οἰκοδομὴν καὶ πάλιν ἠρμένων καὶ εἰς τόπον ἴδιον ἀποτεθειμένων· περὶ πάντων τούτων ἀνάπαυσον τὴν ψυχήν μου, κύριε, καὶ γνώρισόν μοι αὐτάἘάν, φησί, κενόσπουδος μὴ εὑρεθῇς, πάντα γνώσῃ· μετ’ ὀλίγας γὰρ ἡμέρας ἐλευσόμεθα ἐνθάδε, καὶ τὰ λοιπὰ ὄψει τὰ ἐπερχόμεθα τῷ πύργῳ τούτῳ καὶ πάσας τὰς παραβολὰς ἀκριβῶς γνώσῃκαὶ μετ’ ὀλίγας ἡμέρας ἤλθομεν εἰς τὸν τόπον, οὗ κεκαθίκαμεν, καὶ λέγει μοι· Ἄγωμεν πρὸς τὸν πύργον· ὁ γὰρ αὐθέντης τοῦ πύργου ἔρχεται κατανοῆσαι αὐτόνκαὶ λέγει μοι· Ἄγωμεν πρὸς τὸν πύργον· ὁ γὰρ αὐθέντης τοῦ πύργου ἔρχεται κατανοῆσαι αὐτόνκαὶ ἤλθομεν πρὸς τὸν πύργον· καὶ ὅλως οὐδεὶς ἧν πρὸς αὐτὸν εἰ μὴ αἱ παρθένοι μόναικαὶ ἐπερωτᾷ ὁ ποιμὴν τὰς παρθένους, εἰ ἄρα παρεγεγόνει ὁ δεσπότης τοῦ πύργουαἱ δὲ ἔφησαν μέλλειν αὐτὸν ἔρχεσθαι κατανοῆσαι τὴν οἰκοδομήν.
Καὶ ἰδοὺ μετὰ μικρὸν βλέπω παράταξιν πολλῶν ἀνδρῶν ἐρχομένων· καὶ εἰς τὸ μέσον ἀνήρ τις ὑψηλὸς τῷ μεγέθει, ὥστε τὸν πύργον ὑπερέχεινκαὶ οἱ ἓξ ἄνδρες οἱ εἰς τὴν οἰκοδομὴν ἐφεστῶτες ἐκ δεξιῶν τε καὶ ἀριστερῶν περιεπάτησαν μετ’ αὐτοῦ, καὶ πάντες οἱ εἰς τὴν οἰκοδομὴν ἐργασάμενοι μετ’ αὐτοῦ ἦσαν καὶ ἕτεροι πολλοὶ κύκλῳ αὐτοῦ ἔνδοξοιαἱ δὲ παρθένοι αἱ τηροῦσαι τὸν πύργον προσδραμοῦσαι κατεφίλησαν αὐτὸν καὶ ἤρξαντο ἐγγὺς αὐτοῦ περιπατεῖν κύκλῳ τοῦ πύργουκατενόει δὲ ὁ ἀνὴρ ἐκεῖνος τὴν οἰκοδομὴν ἀκριβῶς, ὥστε αὐτὸν καθ’ ἕνα λίθον ψηλαφᾶνκρατῶν δέ τινα ῥάβδον τῇ χειρὶ κατὰ ἕνα λίθον τῶν ᾠκοδομημένων ἔτυπτεκαὶ ὅταν ἐπάτασσεν, ἐγένοντο αὐτῶν τινὲς μέλανες ὡσεὶ ἀσβόλη, τινὲς δὲ κολοβοί, τινὲς δὲ οὔτε λευκοὶ οὔτε μέλανες, τινὲς δὲ τραχεῖς καὶ μὴ συμφωνοῦντες τοῖς ἑτέροις λίθοις, τινὲς δὲ σπίλους πολλοὺς ἔχοντες· αὗται ἦσαν αἱ ποικιλίαι τῶν λίθων τῶν σαπρῶν εὑρεθέντων εἰς τὴν οἰκοδομήνἐκέλευσεν οὖν πάντας τούτους ἐκ τοῦ πύργου μετενεχθῆναι καὶ τεθῆναι λίθους καὶ ἐμβληθῆναι εἰς τὸν τόπον αὐτῶνκαὶ ἐπηρώτησαν αὐτὸν οἱ οἰκοδομοῦντες, ἐκ τίνος ὄρους θέλῃ ἐνεχθῆναι λίθους καὶ ἐμβληθῆναι εἰς τὸν τόπον αὐτῶνκαὶ ἐκ μὲν τῶν ὀρέων οὐκ ἐκέλευσεν ἐνεχθῆναικαὶ ὠρύγη τὸ πεδίον, καὶ εὑρέθησαν λίτοι λαμπροὶ τετράγωνοι, τινὲς δὲ καὶ στρογγνυλοιὅσοι δέ ποτε ἦσαν λίθοι ἐν τῷ πεδίῳ ἐκείνῳ, πάντες ἠνέχθησαν καὶ διὰ τῆς πύλης ἐβαστάζοντο ὑπὸ τῶν παρθένωνκαὶ ἐλτομήθησαν οἱ τετράγωνοι λίθοι καὶ ἐτέθησαν εἰς τὸν τόπον τῶν ἠρμένων· οἱ δὲ στρογγύλοι οὐκ ἐτέθησαν εἰς τὴν οἰκοδομήν, ὅτι σκληροὶ ἦσαν εἰς τὸ λατομηθῆναι αὐτοὺς καὶ βραδέως ἐγένοντοἐτέθησαν δὲ παρὰ τὸν πύργον, ὡς μελλόντων αὐτῶν λατομεῖσθαι καὶ τίθεσθαι εἰς τὴν οἰκοδομήν· λίαν γὰρ λαμπροὶ ἦσαν.
Ταῦτα οὖν συντελέσας ὁ ἀνὴρ ὁ ἔνδοξος καὶ κύριος ὅλον τοῦ πύργου προσεκαλέσατο τὸν ποιμένα καὶ παρέδωκεν αὐτῷ τοὺς λίθους πάντας τοὺς παρὰ τὸν πύργον κειμένους, τοὺς ἀποβεβλημένους ἐκ τῆς πύργον κειμένους, τοὺς ἀποβεβλημένους ἐκ τὸν οἰκοδομῆς, καὶ λέγει αὐτῳ· Ἐπιμελῶς καθάρισον τοὺς λίθους τούτους καὶ θὲς αὐτοὺς εἰς τὴν οἰκοδομὴν τοῦ πύργου, τοὺς δυναμένους ἁρμόσαι τοῖς λοιποῖς· τοὺς δὲ μὴ ἁρμόζοντας ῥῖψον μακρὰν ἀπὸ τοῦ πύργουταῦτα κελεύσας τῷ ποιμένι ἀπῄθει· αἱ δὲ παρθένοι κύκλῳ τοῦ πύργου εἱστήκεισαν τηροῦσαι αὐτόνλέγω τῷ ποιμένι· Πῶς οὗτοι οἱ λίθοι δύνανται εἰς τὴν οἰκοδομὴν τοῦ πύργου ἀπελθεῖν ἀποδεδοκιμασμένοι; ἀποκριθείς μοι λέγει· Βλέπεις, φησί, τοὺς λίθους τούτους; Βλέπω, φημί, κύριεἘγώ, φησί, τὸ πλεῖστον μέρος τῶν λίθων τούτων λατομήσω καὶ βαλῶ εἰς τὴν οἰκοδομήν, καὶ ἁρμόσουσι μετὰ τῶν λοιπῶν λίθωνΠῶς, φημί, κύριε, δύνανται περικοπέντες τὸν αὐτὸν τόπον πληρῶσαι; ἀποκριθεὶς λέγει μοι· Ὅσοι μικροὶ εὑρεθήσονται, εἰς μέσην τὴν οἰκοδομὴν βληθὴσονται, ὅσοι δὲ μείζονες, ἐξώτεροι τεθήσονται καὶ συγκρατήσουσιν αὐτούςταῦτά μοι λαλήσας λέγει μοι· Ἄγωμεν καὶ μετὰ ἡμέρας δύο ἔλθωμεν καὶ καθαρίσωμεν τοὺς λίθους τούτους καὶ βάλωμεν αὐτοὺς εἰς τὴν οἰκοδομήν· τὰ γὰρ κύκλῳ τοῦ πύγου πάντα καθαρισθῆναι δεῖ, μήποτε ὁ δεσπότης ἐξάπινα ἔλθῃ καὶ τὰ περὶ τὸν πύργον ῥυπαρὰ εὕρῃ καὶ προσοχθίσῃ, καὶ οὗτοι οἱ λίθοι οὐκ ἀπελεύσονται εἰς τὴν οἰκοδομὴν τοῦ πύργου, κἀγὼ ἀμελὴς δόξω εἶναι παρὰ τῷ δεσπότῃκαὶ μετὰ ἡμέρας δύο ἤλθομεν πρὸς τὸν πύργον καὶ λέγει μοι· Κατανοήσωμεν τοὺς λίθους πάντας καὶ ἴδωμεν τοὺς δυναμένους εἰς τὴν οἰκοδομὴν ἀπελθεῖνλέγω αὐτῷ· Κύριε, κατανοήσωμεν.
Καὶ ἀρξάμενοι πρῶτον τοὺς μέλανας κατενοοῦμεν λίθουςκαὶ οἷοι ἐκ τῆς οἰκοδομῆς ἐτέθησαντοιοῦτοι καὶ εὑρέθησανκαὶ ἐκέλευσεν αὐτοὺς ὁ ποιμὴν ἐκ τοῦ πύργου μετενεχθῆναι καὶ χωρισθῆναιεἶτα κατενόησε τοὺς ἐψωριακότας, καὶ λαβὼν ἐλατόμησε πολλοὺς ἐξ αὐτῶν καὶ ἐκέλευσε τὰς παρθένους ἆραι αὐτοὺς καὶ βαλεῖν εἰς τὴν οἰκοδομήνκαὶ ἦραν αὐτοὺς αἱ παρθενοι καὶ ἔθηκαν εἰς τὴν οἰκοδομὴν τοῦ πύργου μέσουτοὺς δὲ λοιποὺς ἐκέλευσε μετὰ τῶν μελάνων τεθῆναι· καὶ γὰρ καὶ οὗτοι μέλανες εὑρέθησανεἶτα κατενόει τοὺς τὰς σχισμὰς ἔχοντας· καὶ ἐκ τούτων πολλοὺς ἐλατόμησε καὶ ἐκέλευσε διὰ τῶν παρθένων εἰς τὴν οἰκοδομὴν ἀπενεχθῆναι· ἐξώτεροι δὲ ἐτέθησαν, ὅτι ὑγιέστεροι εὑρέθησανοἱ δὲ λοιποὶ διὰ τὸ πλῆθος τῶν σχισμάτων οὐκ ἠδυνήθησαν λατομηθῆναι· διὰ ταύτην οὖν τὴν αἰτίαν ἀπεβλήθησαν ἀπὸ τῆς οἰκοδομῆς τοῦ πύργουεἶτα κατενόει τοὺς κολοβούς, καὶ εὑρέθησαν πολλοὶ ἐν αὐτοῖς μέλανες, τινὲς δὲ σχισμὰς μεγάλας πεποιητῶν ἀποβεβλημένωντοὺς δὲ περισσεύοντας αὐτῶν καθαρίσας καὶ λατομήσας ἐκέλευσεν εἰς τὴν οἰκοδομὴν τεθῆναιαἱ δὲ παρθένοι αὐτοὺς ἄρασαι εἰς μέσην τὴν οἰκοδομὴν τοῦ πύργου ἥρμοσαν· ἀσθενέστεροι γὰρ ἦσανεἶτα κατενόει τοὺς ἡμίσεις λευκούς, ἡμίσεις δὲ μέλανας· καὶ πολλοὶ ἐξ αὐτῶν εὑρέθησαν μέλανες· ἐκέλευσε δὲ καὶ τούτους ἀρθῆναι μετὰ τῶν ἀποβεβλημένωνοἱ δὲ λοιποὶ πάντες ἤρθησαν ὑπὸ του τῶν παρθένων· λευκοὶ γὰρ ὄντες ἡρμόσθησαν ὑπ’ αὐτῶν τῶν παρθένων εἰς τὴν οἰκοδομήν· ἐξώτεροι δὲ ἐτέθησαν, ὅτι ὑγεῖς εὑρέθησαν, ὥστε δύνασθαι αὐτοὺς κρατεῖν τοὺς εἰς το μέσον τεθέντας· ὅλως γὰρ ἐξ αὐτῶν οὐδὲν ἐκολοβώθηεἶτα ὀλίγοι ἐξ αὐτῶν ἀπεβλήθησαν διὰ τὸ μὴ δύνασθαι λατομηθῆναι· σκληροὶ γὰρ λίαν εὑρέθησανοἱ δὲ λοιποὶ αὐτῶν ἐλατομήθησαν καὶ ἤρθησαν ὑπὸ τῶν παρθένων καὶ εἰς μέσην τὴν οἰκοδομὴν τοῦ πύργου ἡρμόσθησανἀσθενέστεροι γὰρ ἦσανεἶτα κατενόει τοὺς ἔχοντας τοὺς σπίλους, καὶ ἐκ τούτων ἐλάχιστοι ἐμελάνησαν καὶ ἀπεβλήθησαν πρὸς τοὺς λοιπούςοἱ δὲ περισσεύοντες λαμπροὶ καὶ ὑγεῖς εὑρέθησαν· καὶ οὗτοι ἡρμόσθησαν ὑπὸ τῶν παρθένων εἰς τὴν οἰκοδομήν, ἐξώτεροι δὲ ἐτέθησαν διὰ τὴν ἰσχυρότητα αὐτῶν.
Εἶτα ἦλθε κατανοῆσαι τοὺς λευκοὺς καὶ στρογγύλους λίθους καὶ λέγει μοι· Τί ποιοῦμεν περὶ τούτων τῶν λίθων; Τί, φημί, ἐγὼ γινώσκω, κύριε; Οὐδὲν οὖν ἐπινοεῖς περὶ αὐτῶν; Ἐγώ, φημί, κύριε, ταύτην τὴν τέχνην οὐκ ἔχω, οὐδὲ λατόμος εἰμὶ οὐδὲ δύναμαι νοῆσαιΟὐ βλέπεις αὐτοὺς, φησί, λίαν στρογγύλους ὄντας; καὶ ἐὰν αὐτοὺς θελήσω τετραγώνους ποιῆσαι, πολὺ δεῖ ἀπ’ αὐτῶν ἀποκοπῆναι· δεῖ δὲ ἐξ αὐτῶν ἐξ ἀνάγκης τινὰς εἰς τὴν οἰκοδομὴν τεθῆναιΕἰ οὖν, φημί, κύριε, ἀνάγκη ἐστί, τί σεαυτὸν βασανίζεις καὶ οὐκ ἐκλέγεις εἰς τὴν οἰκοδομὴν οὓς θέλεις καὶ ἁρμόζεις εἰς αὐτήν; ἐξελέξατο ἐξ αὐτῶν τοὺς μείζονας καὶ λαμπροὺς καὶ ἐλατόμησεν αὐτούς· αἱ δὲ παρθένοι ἄρασαι ἥρμοσαν εἰς τὰ ἐξώτερα μέρη τῆς οἰκοδομῆςοἱ δὲ λοιποὶ οἱ περισσεύσαντες ἤρθησαν καὶ ἀπετέθησαν εἰς το πεδίον, ὅθεν ἠνέχθησαν· οὐκ ἀπεβλήθησαν δέ, Ὅτι, φησί, λείπει τῷ πύργῳ ἔτι μικρὸν οἰκοδομηθῆναιπάντας δὲ θέλει ὁ δεσπότης τοῦ πύργου τούτους ἁρμοσθῆναι τοὺς λίθους εἰς τὴν οἰκοδομήν, ὅτι λαμπροί εἰσι λίανἐκλήθησαν δὲ γυναῖκες δώδεκα, εὐειδέσταται τῷ χαρακτῆρι, μέλανα ἐνδεδυμέναι, περιεζωσμέναι καὶ ἔξω τοὺς ὤμους ἔχουσαι καὶ τὰς τρίχας λελυμέναι· ἐδοκοῦσαν δέμοι αἱ γυναῖκες αὗτα ἄγριαι εἶναιἐκέλευσε δὲ αὐτὰς ὁ ποιμὴν ἆραι τοὺς λίθους τοὺς ἀποβεβλημένους ἐκ τῆς οἰκοδομῆς καὶ ἀπενεγκεῖν αὐτοὺς εἰς τὰ ὄρη, ὅθεν καὶ ἠνέχθησαναἱ δὲ ἱλαραὶ ἦραν καὶ ἀπήνεγκαν πάντας τοὺς λίθους καὶ ἔθηκαν, ὅθεν ἐλήφθησανκαὶ μετὰ τὸ ἀρθῆναι πάντας τοὺς λίθους καὶ μηκέτι κεῖσθαι λίθον κύκλῳ τοῦ πύργου, λέγει μοι ὁ ποιμήν· Κυκλώσωμεν τὸν πύργον καὶ ἴδωμεν, μή τι ἐλάττωμά ἐστιν ἐν αὐτῷκαὶ ἐκύκλευον ἐγὼ μετ’ αὐτοῦἰδὼν δὲ ὁ ποιμὴν τὸν πύργον εὐπρεπῆ ὄντα τῇ οἰκοδομῇ λίαν ἱλαρὸς ἦν· ὁ γὰρ πύργος οὕτως ἦν ᾠκοδομημένος, ὥστε με ἰδόντα ἐπιθυμεῖν τὴν οἰκοδομὴν αὐτοῦ· οὕτω γὰρ ἧν ᾠκοδομημένος, ὡσὰν ἐξ ἑνὸς λίθου μὴ ἔχων μίαν ἁρμογὴν ἐν ἑαυτῷἐφαίνετο δὲ ὁ λίθος ὡς ἐκ τῆς πέτρας ἐκκεκολαμμένος· μονόλιθος γάρ μοι ἐδόκει εἶναι.
Κἀγὼ περιπατῶν μετ’ αὐτοῦ ἱλαρὸς ἤμην τοιαῦτα ἀγαθὰ βλέπωνλέγει δέ μοι ὁ ποιμήν· Ὕπαγε καὶ φέρε ἄσβεστον καὶ ὄστρακον λεπτόν, ἵνα τοὺς τύπους τῶν λίθων τῶν ἠρμένων καὶ εἰς τὴν οἰκοδομὴν βεβλημένων ἀναπληρώσω· δεῖ γὰρ τοῦ πύργου τὰ κύκλῳ πάντα ὁμαλὰ γενέσθαιαὐτόνὙπηρέτει μοι, φησί, καὶ ἐγγὺς τὸ ἔργον τελεσθήσεταιἐπλήρωσεν οὖν τοὺς τύπους τῶν λίθων τῶν εἰς τὴν οἰκοδομὴν ἀπεληλυθότων καὶ ἐκέλευσε σαρωθῆναι τὰ κύκλῳ τοῦ πύργου καὶ καθαρὰ γενέσθαι· αἱ δὲ παρθένοι λαβοῦσαι σάρους ἐσάρωσαν ὕδωρ, καὶ ἐγένετο ὁ τόπος ἱλαρὸς καὶ εὐπρεπέστατος τοῦ πύργουλέγει μοι ὁ ποιμήν· Πάντα, φησί, κεκαθάρται· ἐὰν ἔλθη ὁ κύριος ἐπεισκέψασθαι τὸν πύργον, οὐκ ἔχει ἡμῖν οὐδὲν μέμψασθαι τὸν πύργον, οὐκ ἔχει ἡμῖν οὐδὲν μέμψασθαιταῦτα εἰπὼν ἤθελεν ὑπάγεινἐγὼ δὲ ἐπελαβόμην αὐτοῦ τῆς πήρας καὶ ἠρξάμην αὐτὸν ὁρκίζειν κατὰ τοῦ κυρίου, ἵνα μοι ἐπιλύσῃ, ἃ ἔδειξέ μοιλέγει μοιΜικρὸν ἔχω ἀκαιρεθῆναι καὶ πάντα σοι ἐπιλύσω· Κύριε, μόνος ὢν ὧδε ἐγὼ τί ποιήσω; Οὐκ εἶ, φησί, μόνος· αἱ οὖν, φημί, αὐταῖς μεπροσκαλεῖται αὐτὰς ὁ ποιμὴν καὶ λέγει αὐταῖς· Παρατίθεμανι ὑμῖν τοῦτον ἕως ἔρχομαι· καὶ ἀπῆλθενἐγὼ δὲ ἤμην μόνος μετὰ τῶν παρθένων· ἦσαν δὲ ἱλαρώτεραι πρὸ ἐμὲ εὖ εἶχον· μάλιστα δὲ αἱ τέσσαρες αἱ ἐνδοξότεραι αὐτῶν.
Λέγουσι μοι αἱ παρθένοι· Σήμερον ὁ ποιμὴν ὧδε οὐκ ἔρχεταιΤί οὖν, φημί, ποιήσω ἐγώ; Μέχρις ὀψέ, φασίν, περίμεινον αὐτόν· καὶ ἐὰν μεθ’ ἡμῶν ὧδε ἕως ἔρχεταιλέγω αὐταῖς· Ἐκδέξομαι αὐτὸν ἕως ὀψέ· ἐὰν δὲ μὴ ἔλθῃ, ἀπελεύσομαι εἰς τὸν οἶκον καὶ πρωῒ ἐπανήξωαἱ δὲ ἀκοκριθεῖσαι λέγουσί μοι· Ἡμῖν παρεδόθης· οὐ δύνασαι ἀφ’ ἡμῶν ἀναχωρῆσαιΠοῦ οὖν, φημί, μενῶ; Μεθ’ ἡμῶν, φασί, κοιμηθήσῃ ὡς ἀδελφός, καὶ οὐχ ὡς ἀνήρ· ἡμέτερος γὰρ ἀδελφὸς εἶ, καὶ τοῦ λοιποῦ μέλλομεν μετὰ σοῦ κατοικεῖν λίαν γὰρ σε ἀγαπῶμενἐγὼ δὲ ᾐσχυνόμην μετ’ αὐτῶν μένεινκαὶ ἡ δοκοῦσα πρώτη αὐτῶν εἶναι ἤρξατό με καταφιλεῖν καὶ περιάγειν κύκλῳ τοῦ πύργου καὶ παιζειν μετ’ ἐμοῦκἀγὼ ὡσεὶ νεώτερος ἐγεγόνειν καὶ ἠρξάμην καὶ αὐτὸς παίζειν μετ’ αὐτῶν· αἱ μὲν γὰρ ἐχόρευον, αἱ δὲ ὠρχοῦντο, αἱ δὲ ᾖδον· ἐγὼ δὲ γενομένης ἤθελον εἰς τὸν οἶκον ὑπάγειν· αἱ δὲ οὐκ ἀφῆκαν, ἀλλὰ κατέσχον μεκαὶ ἔμεινα μετ’ αὐτῶν τὴν νύκτα καὶ ἐκοιμήθην παρὰ τὸν πύργονἔστρωσαν γὰρ αἱ παρθένοι τοὺς λινοῦς χιτῶνας ἑαυτῶν χαμαὶ καὶ ἐμὲζ̓ἀνέκλιναν εἰς τὸ μέσον αὐτῶν, καὶ οὐδὲν ὅλως ἐποίουν εἰ μὴ προσηύχοντο· κἀγὼ μετ’ αὐτῶν ἀδιαλείπτως προσηυχόμην καὶ οὐκ ἔλασσον ἐκείνωνκαὶ ἔχαιρον αἱ παρθένοι οὕτω μου προσευχομένουκαὶ ἔμεινα ἐκεῖ μέχρι τῆς αὔριον ἕως ὥρας δευτέρας μετὰ τῶν παρθένονεἶτα παρῆν ὁ ποιμήν, καὶ λέγει ταῖς παρθένοις· Μή τινα αὐτῷ ὕβριν πεποιήκατε; Ἐρώτα, φασίν, αὐτόνλέω αὐτῷ· Κύριε, εὐφράνθην μετ’ αὐτῶν μείναςΤί, φησίν, κύριε, ῥήματα κυρίου ὅλην τὴν νύκταΚαλῶς, φησίν, ἔλαβόν σε; Ναί, φημί, κύριεΝῦν, φησί, κύριε, τί θελεῖς πρῶτον ακοῦσαι; Καθώς, φημί, κύριε, ἀπ’ ἀρχῆς ἔδειξας· ἐρωτῶ σε, κύριε, ἵνα, καθὼς ἄν σε ἐπερωτήσω, οὕτω μοι καὶ δηλώσῃςΚαθὼς βούλει, φησίν, οὕτω σοι καὶ ἐπερωτήσω, οὕτω μοι καὶ δηλώσῃςΚαθὼς βούλει, φησίν, οὕτω σοι καὶ ἐπιλύσω, καὶ οὐδὲν ἄλως ἀποκρύψω ἀπὸ σοῦ.
Πρῶτον, φημί, πάντων, κύριε, τοῦτό μοι δήλωσον· ἡ πέτρα καὶ ἡ πύλη τίς ἐστιν; Ἡ πέτρα, φησίν, αὕτη καὶ ἡ πύλη ὁ υἱὸς τοῦ θεοῦ ἐστίΠῶς, φημί, κύριε, ἡ πέτρα παλαιά ἐστιν, ἡ δὲ πύλη καινή; Ἄκουε, φησὶ, καὶ σύνιε, ἀσύνετε, ὁ μὲν υἱὸς τοῦ θεοῦ πάσης τῆς κτίσεως αὐτοῦ προγενέστερός ἐστιν, ὥστε σύμβουλον αὐτὸν γενέσθαι τῷ πατρὶ τῆς κτίσεως αὐτοῦ· διὰ τοῦτο καὶ παλαιὰ ἡ πέτραἩ δὲ πύλη διατί καινή, φημί, κύριε; Ὅτι, φησίν, ἐπ’ ἐσχάτων τῶν ἡμερῶν τῆς συντελείας φανερὸς ἐγένετο, διὰ τοῦτο καινὴ ἐγένετο ἡ πύλη, ἵνα οἱ μέλλοντες σώζεσθαι δι’ αὐτῆς εἰς τὴν βασιλείαν εἰσέλθωσι τοῦ θεοῦεἶδες, φησίν, τοὺς λίθους τοὺς διὰ τῆς πύλης εἰσεληλυθότας εἰς τὴν οἰκοδομὴν τοῦ πύργου βεβλημένους, τοὺς δὲ μὴ εἰσεληλυθότας πάλιν ἀποβεβλημένους εἰς τὸν ἴδιον τόπον; Εἶδον, φημί, κύριεΟὕτω, φησίν, εἰς τὴν βασιλείαν τοῦ θεοῦ οὐδεὶς εἰσελεύσεται, εἰ μὴ λάβοι τὸ ὄνομα τὸ ἅγιον αὐτοῦἐὰν γὰρ εἰς πόλιν θελήσῃς εἰσελθεῖν τινα κἀκείνη ἡ πόλις περιτετειχισμένη κύκλῳ καὶ μίαν ἔχει πύλην, μήτι δύνὴ εἰς ἐκείνην τὴν πόλιν εἰσελθεῖν, εἰ μὴ διὰ τῆς πύλης ἧς ἔχει; Πῶς γάρ, φημί, κύριε, δύναται γενέσθαι ἄλλως; Εἰ οὖν εἰς τὴν πόλιν οὐ δύνῃ εἰσελθεῖν εἰ μὴ διὰ τῆς πύλης ἧς ἔχει τῆν βασιλείαν τοῦ θεοῦ ἄλλως εἰσελθεῖν οὐ δύναται ἄνθρωπος εἰ μὴ διὰ τοῦ ὀνόματος τοῦ υἱοῦ αὐτοῦ τοῦ ἠγαπημένου ὑπ’ αὐτοῦΕἶδες, φησί, τὸν ὄχλον τὸν οἰκοδομοῦντα τὸν πύργον; Εἰδον, φημί, κύριεἘκεῖνοι, φησί, πάντες ἄγγελοι ἔνδοξοί εἰσι· τούτοις οὖν περιτετείχισται ὁ κύριοςἡ δὲ πύλη ὁ υἱὸς τοῦ θεοῦ ἐστιν· αὕτη μία εἴσοδός ἐστι πρὸς τὸν κύριονἄλλως οὖν οὐδεὶς εἰσελεύσεται πρὸς αὐτὸν εἰ μὴ διὰ τοῦ υἱοῦ αὐτοῦΕἶδες, φησί, τοὺς ἓξ ἄδρας καὶ τὸν μέσον αὐτῶν ἐνδοξον καὶ μέγαν ἄνδρα τὸν περιπατοῦντα περὶ τὸν πύργον καὶ τοὺς λίθους ἀποδοκιμάσαντα ἐκ τῆς οἰκοδομῆς; Εἶδον, φημί κύριεὉ ἔνδοξος, φησίν, ἀνὴρ ὁ υἱὸς τοῦ θεοῦ ἐστι, κἀκεινοι οἱ ἓξ οἱ ἔνδοξοι ἄγγελοί εἰσι δεξιὰ καὶ εὐώνυμα συγκρατοῦντες αὐτόντοῦτων, φησί, τῶν ἀγγέλων τῶν ἐνδόξων οὐδεὶς εἰσελεύσεται πρὸς τὸν θεὸν ἄτερ αὐτοῦ· ὃς ἂν τὸ ὄνομα αὐτοῦ μὴ λάβῃ, οὐκ εἰσελεύσεται εἰς τὴν βασιλείαν τοῦ θεοῦ.
Ὁ δὲ πύργος, φημί, τίς ἐστιν; Ὁ πύργος, φησίν, οὗτος ἡ ἐκκλησία ἐστίνΑἱ δὲ παρθένοι αὗται τίνες εἰσίν; Αὗται, φησίν, ἅγια πνεύματά εἰσι· καὶ ἄλλως ἄνθρωπος οὐ δύναται εὑρεθῆναι εἰς τὴν βασιλείαν τοῦ θεοῦ, ἐὰν μὴ αὗται αὐτὸν ἐνδύσωσι τὸ ἔνδυμα αὐτῶν· ἐὰν γὰρ τὸ ὄνομα μόνον λάβῃς, οὐδὲν ὠφελήσῃ· αὗται γὰρ αἱ παρθένοι δυνάμεις εἰσὶ τοῦ υἱοῦ τοῦ θεοῦἐὰν τὸ ὄνομα δυνάμεις εἰσὶ τοῦ υἱοῦ τοῦ θεοῦἐὰν τὸ ὄνομα φορῇς, τὴν δὲ δύναμιν μὴ φορῇς αὐτοῦ, εἰς μάτην ἔσῃ τὸ ὄνομα αὐτοῦ φορῶντοὺς δὲ λίθους, φησίν, οὓς εἶδες ἀπόβεβλημένους, οὗτοι τὸ μὲν ὄνομα ἐφόρεσαν, τὸν δὲ ἱματισμὸν τῶν παρθένων οὐκ ἐνεδύσαντοΠοῖος, φημί, ἱματισμὸς αὐτῶν ἐστί, κύριέ; Αὐτὰ τὰ ὀνόματα, φησίν, ἱματισμός ἐστιν αὐτῶνὃς ἂν τὸ ὄνοματα φορεῖν· καὶ γὰρ αὐτὸς ὁ υἱὸς τὰ ὀνόματα τῶν παρθένων τούτων φορεῖὅσους, φησί, λίθους εἶδες εἰς τὴν οἰκοδομὴν τοῦ πύργου εἰσεληλυθότας, ἐπιδεκομένους διὰ τῶν χειρῶν αὐτῶν καὶ μείναντας εἰς τὴν οἰκοδομήν, τούτων τῶν παρθένων τὴν δύναμιν ἐνδεδυμένοι εἰσίδιὰ τοῦτο βλέπεις τὸν πύργον μονόλιθον γεγονότα μετὰ τῆς πέτρας· οὕτω καὶ οἱ πιστεύσαντες τῷ κυρίῳ διὰ τοῦ υἱοῦ αὐτοῦ καὶ ἐνδιδυσκόμενοι τὰ πνεύματα ταῦτα ἔσονται εἰς ἓν πνεῦμα, ἓν σῶμα, καὶ μία χρόα τῶν ἱματίων αὐτων αὐτῶντῶν τοιούτων δὲ τῶν φορούντων τὰ ὀνόματα τῶν παρθένων ἐστὶν ἡ κατοικία εἰς τὸν πύργονΟἱ οὖν, φημί, κύριε, ἀποβεβλημένοι λίθοι διατί ἀπεβλήθησαν; διῆλθον γὰρ διὰ τῆς πύλης, καὶ διὰ τῶν χειρῶν τῶν παρθένων ἐτέθησαν εἰς τὴν οἰκοδομὴν τοῦ πύργουἘπειδὴ πάντα σοι, φησί, μέλει, καὶ ἀκριβῶς ἐξετάζεις, ἄκουε περὶ τῶν ἀποβεβλημένων λίθωνοὗτοι, φησί, πάντες τὸ ὄνομα τοῦ υἱοῦ τοῦ θεοῦ ἔλαβον, ἔλαβον δὲ καὶ τὴν δύναμιν τῶν παρθένων τούτωνλαβόντες οὖν τὰ πνεύματα ταῦτα ἐνεδυναμώθησαν καὶ ἦσαν μετὰ τῶν δούλων τοῦ θεοῦ, καὶ ἦν αὐτῶν ἓν πνεῦμα καὶ ἓν σῶμα καὶ ἓν ἔνδυμα· τὰ γὰρ αὐτὰ ἐφρόνουν καὶ δικαιοσύνην εἰργάζοντομετὰ οὖν χρόνον τινὰ ἀνεπείσθησαν ὑπὸ τῶν γυναικῶν ὧν εἶδες μέλανα ἱμάτια ἐνδυμένςν, τοὺς ὤμους ἔξω ἐχουσῶν καὶ τὰς τρίχας λελυμένας καί εὐμόφων· ταύτας ἰδόντες ἐπεθύμησαν αὐτῶν καὶ ἐνεδύσαντο τὴν δύναμιν αὐτῶν, τῶν δὲ παρθένων ἀπεδύσαντο τὴν δύναμιν αὐτῶν, τῶν δὲ παρθένων ἀπεδύσαντο τὸ ἔνδυμα καὶ τὴν δύναμινοὗτοι οὖν ἀπεβλήθησαν ἀπὸ τοῦ οἴκου τοῦ θεοῦ καὶ κάλλει τῶν γυναικῶν τούτων ἔμειναν ἐν τῷ οἴκῳ τοῦ θεοῦἔχεις, φησί, τὴν ἐπίλυσιν τῶν ἀποβεβλημένων.
Τί οὖν, φημί, κύριε, ἐὰν οὗτοι οἱ ἄνθρωποι, τοιοῦτοι ὄντες, μετανοήσωσι καὶ ἀποβάλωσι τὰς ἐπιθυμίας τῶν γυναικῶν τούτων, καὶ ἐπανακάμψωσιν ἐπὶ τὰς παρθένους καὶ ἐν τῇ δυνάμει αὐτῶν καὶ ἐν τοῖς ἔργοις αὐτῶν πορευθῶσιν, οὐκ εἰσελεύσονται εἰς τὸν οἶκον τοῦ θεοῦ; Εἰσελεύσονται, φησίν, ἐὰν τούτων τῶν γυναικῶν ἀπολάβωσι τὴν δύναμιν καὶ ἐν τοῖς ἔργοις αὐτῶν πορευθῶσι· διὰ τοῦτο γὰρ καὶ τῆς οἰκοδομῆς ἀνοχὴ ἐγένετο, ἵνα, ἐὰν μετανοήσωσιν οὗτοι, ἀπέλθωσιν εἰς τὴν οἰκοδομὴν τοῦ πύργουἐὰν δὲ μὴ μετανοήσωσι, τότε ἄλλοι εἰσελεύσονται, καὶ οὗτοι εἰς τέλος ἐκβληθήσονταιἐπὶ τούτοις πᾶσιν ηὐχαρίστησα τῷ κυρίῳ, ὅτι ἐσπλαγχνίσθη ἐπὶ πᾶσι τοῖς ἐπικαλουμένοις τῷ ὀνόματι αὐτοῦ καὶ ἐξαπέστειλε τὸν ἄγγελον τῆς μετανοίας εἰς ἡμᾶς τοὺς ἁμαρτήσαντας εἰς αὐτὸν καὶ ἀνεκαίνισεν ἡμῶν τὸ πνεῦμα καὶ ἤδη κατεφθαρμένων ἡμῶν καὶ μὴ ἐχόντων ἐλπίδα τοῦ ζῆν ἀνενέωσε τὴν ζωὴν ἡμῶνΝῦν, φημί, κύριε, δήλωσόν μοι, διατί ὁ πύργος χαμαὶ οὐκ ᾠκοδόμηται, ἀλλ’ ἐπὶ τὴν πέτραν καὶ ἐπὶ τὴν πύληνἜτι, φησίν, ἄφρων εἶ καὶ ἀσύνετος; Ἀνάγκην ἔχω, φημί, κύριε, πάντα ἐπερωτᾶν σε, ὅτι οὐδ’ ὅλως ἔνδοξά ἐστι καὶ δυσνόητα τοῖς ἀνθρώποιςἌκουε, φησί· τὸ ὄνομα τοῦ υἱοῦ τοῦ θεοῦ μέγα ἐστὶ καὶ ἀχώρητον καὶ τὸν κόσμον ὅλον βαστάζειεἰ οὖν πᾶσα ἡ κτίσις διὰ τοῦ υἱοῦ τοῦ θεοῦ βαστάζεται, τί δοκεῖς τοὺς κεκλμένους ὑπ’ αὐτοῦ καὶ τὸ ὄνομα φοροῦντας τοῦ υἱοῦ τοῦ θεοῦ καὶ πορευμένους ταῖς ἐντολαῖς αὐτοῦ; βλέπεις οὖν, ποίους βαστάζει; τοὺς ἐξ ὅλης καρδίας φοροῦτας τὸ ὄνομα αὐτοῦαὐτὸς οὖν θεμέλιος αὐτοῖς ἐγένετο καὶ ἡδέως αὐτοὺς βαστάζει, ὅτι οὐκ ἐπαισχύνονται αὐτοῦ φορεῖν.
Δήλωσόν μοι, φημί, κύριε, τῶν παρθένων τὰ ὀνόματα καὶ τῶν γυναικῶν τῶν τὰ μέλανα ἱμάτια ἐνδεδυμένωνἌκουε, φησίν, τῶν παρθένων τὰ ὀνόματα τῶν ἰσχυροτέρων, τῶν εἰς τὰς γωνίας σταθεισῶνἡ μὲν πρώτη Πίστις, ἡ δὲ δευτέρα Ἐγκράτεια, ἡ δὲ τρίτη Δύναμις, ἡ δὲ τετάρτη Μακροθυμία· αἱ δὲ ἕτεραι ἀνὰ μέσον τούτων σταθεῖσαι ταῦτα ἔχουσι τὰ ὀνόματα· Ἁπλότης, Ἀκακία, Ἁγνεία, Ἱλαρότης, Ἀλήθεια, Σύνεσις, Ὁμόνοια, Ἀγάπηταῦτα τὰ ὀνόματα ὁ φορῶν καὶ τὸ ὄνοματα τῶν γυναικῶν τῶν τὰ ἱμάτια μέλανα ἐχουσῶνκαὶ ἐκ τούτων τέσσαρές εἰσι δυνατώτεραι· ἡ πρώτη Ἀπιστία, ἡ δευτέρα Ἀκρασία, ἡ δὲ τρίτη Ἀπείθεια, ἡ δὲ τετάρτη ἈπάτηἈσέλγεια, Ὀξυχολία, Ψεῦδος, Ἀφροσύνη, Καταλαλιά, Μῖσοςταῦτα τὰ ὀνόματα ὁ φορῶν τοῦ θεοῦ δοῦλος τὴν βασιλείαν μὲν ὄψεται τοῦ θεοῦ, εἰς αὐτὴν δὲ οὐκ εἰσελεύσεταιΟἱ λίθοι δέ, φημί, κύριε, οἱ ἐκ τοῦ βυθοῦ ἡρμοσμένοι εἰς τὴν οἰκοδομὴν τίνες εἰσίν; Οἱ μὲν πφῶτοι, φησίν, οἱ ι’ οἱ εἰς τὰ θεμέλια τεθειμένοι, πρώτη γενεά· οἱ δὲ κε’ δευτέρα γενεὰ ἀνδρῶν δικαίων· οἱ δὲ μ’ ἀπόστολοι καὶ διδάσκαλοι τοῦ κηρύγματος τοῦ υἱοῦ τοῦ θεοῦΔιατί οὖν, φημί, κύριε, αἱ παρθένοι καί τούτους τοὺς λίθους ἐπέδωκαν εἰς τὴν οἰκοδομὴν τοῦ πύργου, διενέγκασαι διὰ τῆς πύλη; Οὗτοι γάρ, φησί, πρῶτοι ταῦτα τὰ πνεύματα ἐφόρεσαν καὶ ὅλως ἀπ’ ἀλλήλων οὐκ ἀπέστησαν, οὔτε τὰ πνεύματα ἀπὸ τῶν ἀνθρωποι ἀπὸ τῶν πνεύματων, ἀλλὰ παρέμειναν τὰ πνεύματα αὐτοῖς μέχρι τῆς κοιμήσεως αὐτῶνκαὶ εἰ μὴ ταῦτα τὰ´πνεύματα μετ’ αὐτῶν ἐσχήκεισαν, οὐκ ἂν εὔχρηστοι γεγόνεισαν τῇ οἰκοδομῇ τοῦ πύργου τούτου.
Ἔτι μοι, φημί, κύριε, δήλωσονΤί, φησίν, ἐπιζητεῖς; Διατί, φημί, κύριε, οἱ λίθοι ἐκ τοῦ πύργου ἐτέθησαν, πεφορηκότες τὰ πνεύματα ταῦτα; Ἀνάγκην, φησίν, εἶχον δι’ ὕδατος ἀναβῆναι, ἵνα ζωοποιηθῶσιν· οὐκ ἠδύνατο γὰρ ἄλλως εἰσελθεῖν εἰς τὴν βασιλείαν τοῦ θεοῦ, εἰ μὴ τὴν νέκρωσιν ἀπέθεντο τῆς ζωῆς αὐτῶν τῆς προτέραςἔλαβον οὖν καὶ οὗτοι οἱ κεκοιμημένοι τὴν σφραγῖδα τοῦ υἱοῦ τοῦ θεοῦ καὶ εἰσῆλθον εἰς τὴν βασιλείαν τοῦ θεοῦ· πρὶν γάρ, φησί, φορέσαι τὸν ἄνθρωπον τὸ ὄνομα τοῦ υἱοῦ τοῦ θεοῦ, νεκρός ἐστιν· ὅταν δὲ λάβῃ τὴν σφραγῖδα, ἀποτίθεται τὴν νέκρωσιν καὶ ἀναλαμβάνει τὴν ζωήνἡ σφραγὶς οὖν τὸ ὕδωρ ἐστίν· εἰς τὸ ὕδωρ οὖν καταβαινουσι νεκροὶ καὶ ἀναβαίνουσι ζῶντεςκἀκείνοις οὖν ἐκηρύχθη ἡ σφραγὶς αὕτη καὶ ἐχρήσαντο αὐτῇ, ἵνα εἰσέλθωσιν εἰς τὴν βασιλείαν τοῦ θεοῦΔιατί, φημί, κύριε, καὶ οἱ μ’ λίθοι μετ’ αὐτῶν ἀνέβησαν ἐκ τοῦ βυθοῦ, ἤδη ἐσχηκότες τὴν σφραγῖδα; Ὅτι, φησίν, οὗτοι οἱ ἀπόστολοι καὶ οἱ διδάσκαλοι οἱ κηρύξαντες τὸ ὄνομα τοῦ υἱοῦ τοῦ θεοῦ, κοιμηθέντες ἐν δύναμει καὶ πίστει τοῦ υἱοῦ τοῦ θεοῦ ἐκήρυξαν καὶ τοῖς πορκεκοιμημένοις καὶ αὐτοὶ ἔδωκαν αὐτοῖς τὴ σφραγῖδα τοῦ κηρύγματοςκατέβησαν καὶ ζῶντες κατέβησαν καὶ ζῶντες ἀνέβησαν· ἐκεῖνοι δὲ οἱ προκεκοιμημένοι νεκροὶ κατέβησαν, ζῶντες δὲ ἀνέβησανδιὰ τούτων οὖν ἐζωοποιήθησαν καὶ ἐπέγνωσαν τὸ ὄνομα τοῦ υἱοῦ τοῦ θεοῦ· διὰ τοῦτο καὶ συνανέβησαν μετ’ αὐτῶν, καὶ συνηρμόσθησαν εἰς τὴν οἰκοδομὴν τοῦ πύργου, καὶ ἀλατόμητοι συνῳκοδομήθησαν· ἐν δικαιοσύνῃ γὰρ ἐκοιμήθησαν καὶ ἐν μεγάλῃ ἁγνείᾳ· μόνον δὲ τὴν σφραγῖδα ταύτην οὐκ εἶχονἔχεις οὖν καὶ τὴν τούτων ἐπίλυσινἜχω, φημί, κύριε.
Νῦν οὖν κύριε, περὶ τῶν ὀρέων μοι δήλωσον· διατί ἄλλαι καὶ ἄλλαι εἰσὶν αἱ ἰδέαι και ποικίλαι; Ἄκουε, φησί· τὰ ὄρη ταῦτα τὰ δώδεκα φυλαί εἰσιν αἱ κατοικοῦσαι ὅλον τὸν κόσμονἐκηρύχθη οὖν εἰς ταύτας ὁ υἱὸς τοῦ θεοῦ διὰ τῶν ἀποστόλωνΔιατί δὲ ποικίλα καὶ ἄλλη καὶ ἄλλη ἰδέα ἐστὶ τὰ ὄρη, δήλωσόν μοι, κύριεἌκουε, φησίν· αἱ δώδεκα φυλαὶ αὗται αἱ κατοικοῦσαι ὅλον τὸν κόσμον δώδεκα ἔθνη εἰσί· ποικίλα δέ εἰσι τῇ φρονήσει καὶ τῷ νοΐ· οἷα οὖν εἶδες τὰ ὄρη ποικίλα, τοιαῦταί εἰσι καὶ τούτων αἱ ποικιλίαι τοῦ νοὸς τῶν ἐθνῶν καὶ ἡ φρόνησιςδηλώσω δέ σοι καὶ ἑνὸς ἑκάστο τὴν πρᾶξινΠρῶτον, φημί, κύριε, τοῦτο δήλωσον, διατί οὕτω ποικίλα ὄντα τὰ ὄρη, εἰς τὴν οἰκοδομὴν ὅταν ἐτέθησαν οἱ λίθοι αὐτῶν, μιᾷ χρόᾳ ἐγένοντο λαμπροί, ὡς καὶ οἱ ἐκ τοῦ βυθοῦ ἀναβεβηκότες λίθοι; Ὅτι, φησί, πάντα τὰ ἔθνη τὰ ὑπὸ τὸν οὐρανὸν κατοικούντα, ἀκούσαντα καὶ πιστευσαντα ἐπὶ τῷ ὀνόματι ἐκλήθησαν τοῦ υἱοῦ τοῦ θεοῦλαβόντες οὐν τὴν σφραγῖδα μίαν φρον́νησιν ἔσχον καὶ ἕνα νοῦν, καὶ μία πίστις αὐτῶν ἐγένετο καὶ μία ἀγάπη, καὶ τὰ πνεύματα τῶν παρθένων μετὰ τοῦ ὀνόματος ἐφόρεσαν· διὰ τοῦτο ἡ οἰκοδομὴ τοῦ πύργου μιᾷ χροᾳ ἐγένετο λαμπρὰ ὡς ὁ ἥλιοςμετὰ δὲ τὸ εἰσελθεῖν αὐτοὺς ἐπὶ τὸ αὐτὸ καὶ γενέσθαι ἓν σῶμα τινὲς ἐξ αὐτῶν ἐμίαναν ἑαυτοὺς καὶ ἐξεβλήθησαν ἐκ τοῦ γένους τῶν δικαίων καὶ πάλιν ἐγένοντο, οἷοι πρότερον ἦσαν, μᾶλλον δὲ καὶ χείρονες.
Πῶς φηί, κύριε, ἐγένοντο χείρονες, θεὸν ἐπεγνωκότες; Ὁ μὴ γινώσκων, φησί, θεὸν καὶ πονηρευόμενος ἔχει κολασίν τινα τῆς πονηρίας αὐτοῦ, ὁ δὲ θεὸν ἐπιγνοὺς οὐκέτι ὀφείλει πονηρεύεσθαι, ἀλλ’ ἀγαθοποιεῖνἐὰν οὖν ὁ ὀφείλων ἀγαθοποιεῖν πονηρεύηται, οὐ δοκεῖ πλείονα πονηρίαν ποιεῖν παρὰ τὸν μὴ γινώσκοντα τὸν θεόν; διὰ τοῦτο οἱ μὴ ἐγνωκότες θεὸν καὶ πονηρευόμενοι κεκριμένοι εἰσὶν εἰς θάνατον, οἱ δὲ τὸν θεὸν ἐγνωκότες καὶ τὰ μεγαλεῖα αὐτοῦ ἑωρακότες καὶ πονηρευόμενοι δισσῶς κολασθήσονται καὶ ἀποθανοῦνται εἰς τὸν αἰῶναοὕτως οὖν καθαρισθήσεται ἡ ἐκκλησία τοῦ θεοῦὡς δὲ εἶδες ἐκ τοῦ πύργου τοὺς λίθους ἠρμένους καὶ παραδεδομένους τοῖς πνεύμασι τοῖς πονηροῖς καὶ ἐκεῖθεν ἐκβληθέντας· (καὶ ἔσται ἓν σῶμα τῶν κεκαθαρμένων, ὥσπερ καὶ ὁ πύργος ἐγένετο ὡς ἐξ ἑνὸς λίθου γεγονὼς μετὰ τὸ καθαρισθῆναι αὐτόν·) οὕτως ἔσται καὶ ἡ ἐκκλησία τοῦ θεοῦ μετὰ τὸ καθαρισθῆναι καὶ ἀποβληθῆναι τοὺς πονηροὺς καὶ ὑποκριτὰς καὶ βλασφήμους καὶ διψύχους καὶ πονηρευομένους ποικίλαις πονηρίαιςμετὰ τὸ τούτους ἀποβληθῆναι ἔσται ἡ ἐκκλησία τοῦ θεοῦ ἓν σῶμα, μία φρόνησις, εἷς νοῦς, μία πίςτις, μία ἀγάπη· καὶ τότε ὁ υἱὸς τοῦ θεοῦ ἀγαλλιάσετα καὶ εὐφρανθήσεται ἐν αὐτοῖς ἀπειληφὼς τὸν λαὸν αὐτοῦ καθαρόνΜεγάλως, φημί, κύριε, τῶν ὀρέων ἑνὸς ἑκάστου δήλωσόν μοι τὴν δύναμιν καὶ τὰς πράξεις, ἵνα πᾶσα ψυχὴ πεποιθυῖα ἐπὶ τὸν κύριον ἀκούσασα δοξάσῃ τὸ μέγα καὶ θαυμαστὸν καὶ ἔνδοξον ὄνομα αὐτοῦἌκουε, φησί, τῶν ὀρέων τὴν ποικιλίαν καὶ τῶν δώδεκα ἐθνῶν.
Ἐκ τοῦ πρώτου ὄρους τοῦ μέλανος οἱ πιστεύσαντες τοιοῦτοί εἰσιν· ἀποστάται καὶ βλάσφημοι εἰς τὸν κύριον καὶ προδόται τῶν δούλων τοῦ θεοῦτούτοις δὲ μετάνοια οὐκ ἔστι, θάνατος ἔστι, καὶ διὰ τοῦτο καὶ μέλανές εἰσι· καὶ γὰρ τὸ γένος αὐτῶν ἄνομόν ἐστινἐκ δὲ τοῦ δευτέρου ὄρους τοῦ ψιλοῦ οἱ πιστευσαντες τοιοῦτοί εἰσιν· ὑποκριταὶ καὶ διδάσκαλοι πονηρίαςκαὶ οὗτοι οὖν τοῖς προτέροις ὅμοιοί εἰσι, μὴ ἔχοντες καρπὸν δικαιοσύνης· ὡς γὰρ τὸ ὄρος αὐτῶν ἄκαρπον, οὕτω καὶ οἱ ἄνθρωποι οἱ τοιοῦτοι ὄνομα μὲν ἔχουσιν, ἀπὸ δὲ τῆς πίστεως κενοί εἰσι καὶ οὐδεὶς ἐν αὐτοῖς, καρπὸς ἀληθείαςτούτοις οὖν μετάνοια κεῖται, ἐὰν ταχὺ μετανοήσωσιν· ἐὰν δὲ βραδύνωσι, μετὰ τῶν προτέρων ἔσται ὁ θάνατος αὐτῶνΔιατί, φημί, κύριε, τούτοις μετάνοιά ἐστι, τοῖς δὲ πρώτοις οὐκ ἔστι; παρὰ τι γὰρ αἱ πράξεις αὐτῶν εἰσίΔιὰ τοῦτο, φησί, τούτοις μετάνοια κεῖται, ὅτι οὐκ ἐβλασφήμησαν τὸν κύριον αὐτῶν οὐδὲ ἐγένοντο προδόται τῶν δούλων τοῦ θεοῦ· διὰ δὲ τὴν ἐπιθυμίαν τοῦ λήμματος ὑεκρίθησαν καὶ ἐδίδαξεν ἕκαστος κατὰ τὰς ἐπιθυμίας τῶν ἀνθρώπων τῶν ἁμαρτανόντωνἀλλὰ τίσουσι δίκην τινά· κεῖται δὲ αὐτοῖς μετάνοια διὰ τὸ μὴ γενέσθαι αὐτοὺς βλασφήμους μηδὲ προδότας.
Ἐκ δὲ τοῦ ὄρους τοῦ τρίτου τοῦ ἔχοντος ἀκάνθας καὶ τριβόλους οἱ μὲν πλούσιοι, οἱ δὲ πραγματείαις πολλαῖς ἐμπεφυρμένοιοἱ μὲν τριβολοί εἰσιν οἱ πλούσιοι, αἱ δὲ ἄκανθαι οἱ ἐν ταῖς πραγματείαις ταῖς ποικίλαις καὶ ποικίλαις πραγματείαις ἐμπεφυρμένοι, οὐ κολλῶνται τοῖς δούλοις τοῦ θεοῦ, ἀλλ’ ἀποπλανῶνται πνιγόμενοι ὑπὸ τῶν πράξεων αὐτῶν· οἱ δὲ πλούσιοι δυσκόλως κολλῶνται τοῖς δούλοις τοῦ θεοῦ, φοβούμενοι, μή τι αἰτισθῶσιν ὑπ’ αὐτῶν· οἱ τοιοῦτοι οὖν δυσκόλως εἰσελεύσονται εἰς τὴν βασιλείαν τοῦ θεοῦὥσπερ γὰρ ἐν τριβόλοις γυμνοῖς ποσὶ περιπατεῖν δύσκολόν ἐστιν, οὕτω καὶ τοῖς τοιούτοις δύσκολόν ἐστιν εἰς τὴν βασιλείαν τοῦ θεοῦ εἰσελθεῖνἀλλὰ τούτοις πᾶσι μετάνοιά ἐστι, ταχινὴ δέ, ἵν’ ὃ τοῖς προτέροις χρόνοις οὐκ εἰργάσαντο νῦν ἀναδράμωσιν ταῖς ἡμέραις καὶ ἀγαθόν τι ποιήσωσινἐὰν οὖν μετανοήσωσι καὶ ἀγαθόν τι ποιήσωσι, ζήσονται τῷ θεῷ· ἐὰν δὲ ἐπιμείνωσι ταῖς πράξεσιν αὐτῶν, παραδοθήσονται ταῖς γυναιξὶν ἐκείναις, αἵτινες αὐτοὺς θανατώσουσιν.
Ἐκ δὲ τοῦ τετάρτου ὄρους τοῦ ἔχοντος βοτάνας πολλάς, τὰ μὲν ἐπάνω τῶν βοτανῶν χλωρά, τὰ δὲ πρὸς ταῖς ῥίζαις ξηρά, τινὲς δὲ καὶ ἀπὸ τοῦ ἡλίου ξηραινόμεναι, οἱ πιεστεύσαντες τοιοῦτοί εἰσιν· οἱ μὲν δίψυχοι, οἱ δὲ τὸν κύριον ἔχοντεςδιὰ τοῦτο τὰ θεμέλια αὐτῶν ξηρά ἐστι καὶ δύναμιν μὴ ἔχοντα, καὶ τὰ ῥήματα αὐτῶν μόνα ζῶσι, τὰ δὲ ἔργα αὐτῶν νεκρά ἐστινοἱ τοιοῦτοι οὔτε ζῶσιν οὔτε τεθνήκασινὅμοιοι οὖν εἰσὶ τοῖς διψύχοις· καὶ γὰρ οἱ δίψυχοι οὔτε χλωροί εἰσιν οὔτε ξηροί· οὔτε γὰρ ζῶσιν οὔτε τεθνήκασινὥσπερ γὰρ αὗται αἱ βοτάναι ἥλιον ἰδοῦσαι ἐξηράνθησαν, οὕτω καὶ οἱ δίψυχοι, ὅταν θλῖψιν ἀκούσωσι, διὰ τὴν δειλίαν αὐτῶν εἰδωλολοατροῦσι καὶ τὸ ὄνομα ἐπαισχύνονται τοῦ κυρίου αὐτῶνοἱ τοιοῦτοι οὖν οὔτε ζῶσιν οὔτε τεθνήκασινἀλλὰ καὶ οὗτοι ἐὰν ταχὺ μετανοήσωσιν, δυνήσονται ζῆσαι· ἐὰν δὲ μὴ μετανοήσωσιν, ἤδη παραδεδομένοι εἰσὶ ταῖς γυναιξὶ ταῖς ἀποφερομέναις τὴν ζωὴν αὐτῶν.
Ἐκ δὲ τοῦ ὄρους τοῦ πέμπτου τοῦ ἔχοντος βοτάνας χλωρὰς καὶ τρασχέος ὄντος οἱ πιστεύσαντες τοιοῦτοί εἰσι· πιστοὶ μέν, δυσμαθεῖς δὲ καὶ αὐθάδεις καὶ ἑαυτοῖς ἀρέσκοντες, θέλοντες πάντα γινώσκειν, καὶ οὐδὲν ὅλως γινώσκουσι, διὰ τὴν αὐθάδειαν αὐτῶν ταύτην ἀπέστη ἀπ’ αὐτῶν ἡ σύνεσις, καὶ εἰσῆλθεν εἰς αὐτοὺς ἀφροσύνη μωράἐπαινοῦσι δὲ ἑαυτοὺς ὡςσύνεσιν ἔχοντας καὶ θέλουσιν ἐθελοδιδάσκαλοι εἶναι, ἄφρονες ὄντεςδιὰ ταύτην οὖν τὴν ὑψηλοφροσύνην πολλοὶ ἐκενώθησαν ὑψοῦντες ἑαυτούς· μέγα γὰρ δαιμόνιόν ἐστιν ἡ αὐθάδεια καὶ ἡ κενὴ πεποίθησις· ἐκ τούτων οὖν πολλοὶ ἀπεβλήθησαν, τινὲς δὲ μετενόησαν καὶ ἐπίστευσαν καὶ ὑπέταξαν ἑαυτοὺς τοῖς ἔχουσι σύνεσιν, γνόντες τὴν ἑαυτῶν ἀφροσύνηνκαὶ τοῖς λοιποῖς δὲ τοῖς τοιούτοις κεῖται μετάνοια· οὐκ ἐγένοντο γὰρ πονηροί, μᾶλλον δὲ μωροὶ καὶ ἀσύνετοιοὗτοι οὖν ἐὰν μετανοήσωσι, ζήσονται τῷ θεῷ· ἐὰν δὲ μὴ μετανοήσωσι, καοικήσουσι μετὰ τῶν γυναικῶν τῶν πονηρευομένων εἰς αὐτούς.
Οἱ δὲ ἐκ τοῦ ἕκτου τοῦ ἔχοντος σχισμὰς μεγάλας καὶ μικρὰς καὶ ἐν ταῖς σχισμαῖς βοτάνας μεμαραμμένας πειστεύσαντες τοιοῦτοί εἰσινοἱ μὲν τὰς σχισμὰς τὰς μικρὰς ἔχοντες, οὗτοί εἰσιν οἱ κατ’ ἀλλήλων ἔχοντες, καὶ ἀπὸ τῶν καταλαλιῶν ἑαυτῶν μεμαραμμένοι εἰσὶν ἐν τῇ πίστει· ἀλλὰ μετενόησαν ἐκ τούτων πολλοίκαὶ οἱ λοιποὶ δὲ μετανοήσουσιν, ὅταν ἀκούσωσί μου τὰς ἐντοάς· μικραὶ γὰρ αὐτῶν εἰσιν αἱ καταλαλιαί, καὶ ταχὺ μετανοήσουσινοἱ δὲ μεγάλας ἔχοντες σχισμάς, οὗτοι παράμονοί εἰσι ταῖς καταλαλιαῖς αὐτῶν καὶ μνησίκακοι γίνονται μηνιῶντες ἀλλήλοις· οὗτοι οὖν ἀπὸ τοῦ πύργου ἀπερρίφησαν καὶ ἀπεδοκιμάσθησαν τῆς οἰκοδομῆς αὐτοῦοἱ τοιοῦτοι οὖν δυσκόλως ζήσονταιεἰ ὁ θεὸς καὶ ὁ κύριος ἡμῶν ὁ πάντων κυριεύων καὶ ἔχων πάσης τῆς κτίσεως αὐτοῦ τὴν ἐξουσίαν οὐ μνησικακεῖ τοῖς ἐξομολογουμένοις τὰς ἁμαρτίας αὐτῶν, ἀλλ’ ἵλεως γίνεται, ἄνθρωπος φθαρτὸς ὢν καὶ πλήρης ἁμαρτιῶν ἀνθρώπῳ μνησικακεῖ ὡς δυνάμενος ἀπολέσαι ἢ σῶσαι αὐτόν; λέγω δὲ ὑμῖν, ὁ ἄγγελος τῆς μετανοίας· ὅσοι ταύτην ἔχετε τὴν αἵρεσιν ἀπόθεσθε αὐτὴν καὶ μετανήσατε, καὶ ὁ κύριος ἰάσεται ὑμῶν τὰ πρότερα ἁμαρτήματα, ἐὰν καθαρίσητε ἑαυτοὺς ἀπὸ τούτου τοῦ δαιμονίου· εἰ δὲ μή, παραδοθήσεσθε αὐτῷ εἰς θάνατον.
Ἐκ δὲ τοῦ ἑβδόμου ὄρους, ἐν ᾧ βοτάναι χλωραὶ καὶ ἱλαραί, καὶ ὅλον τὸ ὄρος εὐθηνοῦν καὶ πᾶν γένος κτηνῶν καὶ τὰ πετεινὰ τοῦ οὐρανοῦ ἐνέμοντο τὰς βοτάνας ἐκ τούτῳ τῳ ὄρει, καὶ αἱ βοτάναι, ἃς ἐνέμοντο, μᾶλλον εὐθαλεῖς ἐγίνοντο, οἱ πιστεύσαντες τοιοῦτοί εἰσιπάντοτε ἁπλοῖ καὶ ἄκακοι καὶ μακάριοι ἐγίνοντο, μηδὲν κατ’ ἀλλήλων ἔχοντες, ἀλλὰ πάντοτε ἀγαλλιώμενοι ἐπὶ τοῖς δούλοις τοῦ θεοῦ καὶ ἐνδεδυμένοι τὸ πνεῦμα τὸ ἅγιον τούτων τῶν παρθένων καὶ πάντοτε σπλάγχνον ἔχοντες ἐπὶ πάντα ἄνθρωπον, καὶ ἐκ τῶν κόπων αὐτῶν παντὶ ἀνθρώπῳ ἐχορήγησαν ἀνονειδίστως καὶ ἀδιστάκτωςὁ οὖν κύριος ἰδὼν τὴν ἁπλότητα αὐτῶν καὶ πᾶσαν νηπιότητα ἐπλήθυνεν αὐτοὺς ἐν τοῖς κόποις τῶν χειρῶν αὐτῶν καὶ ἐχαρίτωσεν αὐτοὺς ἐν πάσῃ πράξει αὐτῶνλέγω δὲ ὑμῖν τοῖς τοιούτοις οὖσιν ἐγὼ ὁ ἄγγελος τῆς μετανοίας· διαμείνατε τοιοῦτοι, καὶ οὐκ ἐξαλειφθήσεται τὸ σπέρμα ὑμῶν ἕως αἰῶνος· ἐδοκίμασε γὰρ ὑμᾶς ὁ κύριος καὶ ἐνέγραψεν ὑμᾶς εἰς τὸν ἀριθμὸν τὸν ἡμέτερον, καὶ ὅλον τὸ σπέρμα ὑμῶν κατοικήσει μετὰ τοῦ υἱοῦ τοῦ θεοῦ· ἐκ γὰρ τοῦ πνεύματος αὐτοῦ ἐλάβετε
Ἐκ δὲ τοῦ ὄρους τοῦ ὀγδόου, οὗ ἦσαν αἱ πολλαὶ πηγαὶ καὶ πᾶσα ἡ κτίσις τοῦ κυρίου ἐποτίζετο ἐκ τῶν πηγῶν, οἱ πιστεύσαντες τοιοῦτοί εἰσιν· ἀπόστολοι καὶ διδάσκαλοι οἱ κηρυξαντες εἰς ὅλον τὸν κόσμον καὶ οἱ διδάξαντες σεμνῶς καὶ ἁγνῶς τὸν λόγον τοῦ κυρίου καὶ μηδὲν ὅλως νοσφισάμενοι εἰς ἐπιθυμίαν πονηράν, ἀλλὰ πάντοτε ἐν δικαιοσύνῃ καὶ ἀληθείᾳ πορευθέντες, καθὼς καὶ παρέλαβον τὸ πνεῦμα τὸ ἅγιοντῶν τοιούτων οὖν ἡ πάροδος μετὰ τῶν ἀγγέλων ἐστίν.
Ἐκ δὲ τοῦ ὄρους του ἐνάτου τοῦ ἐρημώδους, τοῦ τὰ ἑρπετὰ καὶ θηρία ἐν αὐτῷ ἔχοντος τὰ διαφθείροντα τοὺς ἀνθρώπους, οἱ πιστεύσαντες τοιοῦτοί εἰσιν· οἱ μὲν τοὺς σπίλους ἔχοντες διάκονοί εἰσι κακῶς διακονήσαντες καὶ διαρπάσαντες χηρῶν καὶ ὀρφανῶν τὴν ζωὴν καὶ ἑαυτοῖς περιποιησάμενοι ἐκ τῆς διακονίας ἧς ἔλαβον διακονῆσαι· ἐὰν οὖν ἐπιμείνωσι τῇ αὐτῇ ἐπιθυμίᾳ, ἀπέθανον καὶ οὐδεμία αὐτοῖς ἐλπὶς ζωῆς· ἐὰν δὲ ἐπιστρέψωσι καὶ ἁγνῶς τελειώσωσι τὴν διακονίαν αὐτῶν, δυνήσονται ζῆσαιοἱ δὲ ἐψωριακότες ἐπὶ τὸν κύριον ἑαυτῶν, ἀλλὰ χερσωθέντες καὶ γενόμενοι ἐρημώδεις· μὴ κολλώμενοι τοῖς δούλοις τοῦ θεοῦ, ἀλλὰ μονάζοντες ἀπολλύουσι τὰς ἑαυτῶν ψυχὰςὡς γὰρ ἄμπελος ἐν φραγμῷ τινι καταλειφθεῖσα ἀμελείας τυγχάνουσα καταφθείρεται καὶ ὑπὸ τῶν βοτανῶν ἐρημοῦται καὶ τῷ χρόνῳ ἀγρία γίνεται, καὶ οὐκέτι εὔχρηστός ἐστι τῷ δεσπότῃ ἑαυτῆς, οὕτω καὶ οἱ τοιοῦτοι ἄνθρωποι ἑαυτοὺς ἀπεγνώκασι καὶ γίνονται ἄχρηστοι τῷ κυρίῳ ἑαυτῶν ἀγριωθέντεςτούτοις οὖν μετάνοια γίνεται, ἐὰν δὲ ἐκ καρδίας εὑρεθῇ ἠρνημένος τις, οὐκ οἶδα, εἰ δύναται ζῆσαικαὶ τοῦτο οὐκ εἰς ταύτας τὰς ἡμέρας λέγω, ἵνα τις ἀρνησάμενος μετάνοιαν λάβῃ· ἀδύνατον γάρ ἐστι σωθῆναι τὸν μέλλοντα νῦν ἀρνεῖσθαι τὸν κύριον ἑαυτοῦ· ἀλλ’ ἐκείνοις τοῖς πάλαι ἠρνημένοις δοκεῖ κεῖσθαι μετάνοιαεἴ τις οὖν μέλλει μετανοεῖν, ταχινὸς γενέσθω πρὶν τὸν πύργον ἀποτελεσθῆναι· εἰ δὲ μή, ὑπὸ τῶν γυναικῶν κααφθαρήσεται εἰς θάνατονκαὶ οἱ κολοβοί, οὗτοι δόλιοί εἰσι καὶ κατάλαλοι· καὶ τὰ θηρία, ἃ εἶδες εἰς τὸ ὄρος, οὗτοί εἰσινὥσπερ γὰρ τὰ θηρία διαφθείρει τῷ ἑαυτῶν ἰῷ τὸν ἄνθρώπων τὰ ῥήματα διαφθείρει τόν καὶ ἀπολλύειοὗτοι οὖν κολοβοί εἰσιν ἀπὸ τῆς πίστεως αὐτῶν διὰ τὴν πρᾶξιν, ἣν ἔχουσιν ἐν ἑαυτοῖς· τινὲς δὲ μετενόησαν καὶ ἐσώθησανκαὶ οἱ λοιποὶ οἱ τοιοῦτοι ὄντες δύνανται σωθῆναι, ἐὰν μετανοήσωσιν· ἐὰν δὲ μὴ μετανοήσωσιν, ἀπὸ τῶν γυναικῶν ἐκείνων, ὧν τὴν δύναμιν ἔχουσιν, ἀποθανοῦνται.
Ἐκ δὲ τοῦ ὄρους τοῦ δεκάτου οὗ ἦσαν δένδρα σκεπάζοντα πρόβατά τινα, οἱ πιστεύσαντες τοιοῦτοί εἰσιν· ἐπίσκοποι καὶ φιλόξενοι, οἵτινες ἡδέως εἰς τοὺς οἴκους ἑαυτῶν πάντοτε ὑπεδέξαντο τοὺς δούλους τοῦ θεοῦ ἄτερ ὑποκρίσεως· οἱ δὲ ἐπίσκοποιπ́πάντοτε τοὺς ὑστερημένους καὶ τὰς χήρας τῇ διακονίᾳ ἑαυτῶν ἀδιαλείπτως ἐσκέπασαν καὶ ἁγνῶς ἀνεστράφησαν πάντοτεοὗτοι οὖν πάντες σκεπασθήσονται ὑπὸ τοῦ κυρίου διαπαντόςοἱ οὖν ταῦτα ἐργασάμενοι ἔνδοξοί εἰσι παρὰ τῷ θεῷ καὶ ἤδη ὁ τόπος αὐτῶν μετὰ τῶν ἀγγέλων ἐστίν, ἐὰν ἐπιμείνωσιν ἕως τέλους λειτοργοῦντες τῷ κυρίῳ.
Ἐκ δὲ τοῦ ὄρους τοῦ ἑνδεκάτου, οὗ ἦσαν δένδρα καρπῶν πλήρη, ἄλλοις καὶ ἄλλοις καρποῖς κεκοσμημένα, οἱ πιεστεύσαντες τοιοῦτοί εἰσινοἱ παθόντες ὑπὲρ τοῦ ὀνόματος τοῦ υἱοῦ τοῦ θεοῦ, οἳ καὶ προθύμως ἔπαθον ἐξ ὅλης τῆς καρδίας καὶ παρέδωκαν τὰς ψυχὰς αὐτῶνΔιατί οὖν, φημί, κύριε, πάντα μὲν τὰ δένδρα καροὺς ἔχει, τινὲς δὲ ἐξ αὐτῶν καρποὶ εὐειδέστεροί εἰσιν; Ἄκουε, φησίν· ὅσοι ποτὲ ἔπαθον διὰ τὸ ὄνομα, ἐνδοξοί εἰσι παρὰ τῷ θεῷ, καὶ πάντων αἱ ἁμαρτίαι ἀφῃρέθησαν, ὅτι ἔπαθον διὰ τὸ ὄνομα τοῦ υἱοῦ τοῦ θεοῦδιατί δὲ οἱ καρποὶ αὐτῶν ποικίλοι εἰσίν, τινὲς δὲ ὑπερέχοντες, ἄκουεὅσοι, φησίν, ἐπ’ ἐξωουσίαν ἀχθέντες ἐξητάσθησαν καὶ οὐκ ἠρήσαντο, ἀλλ’ ἔπαθον προθύμως, οὗτοι μᾶλλον ἐνδοξότεροί εἰσι παρὰ τῷ κυρίῳ· τούτων ὁ καρπός ἐστιν ὁ ὑπερέχων· ὅσοι δὲ δεῖοὶ καὶ ἐν δισταγμῷ ἐγένοντο καὶ ἐλογίσαντω ἐν ταῖς καρδίαις αὐτῶν, πότερον ἀρνήσονται ἢ ὁμολογήσουσι, καὶ ἔπαθον, τούτων οἱ καρποὶ ἐλάττους εἰσίν, ὅτι ἀνέβη ἐπὶ τὴν καρδίαν αὐτῶν ἡ βουλὴ αὕτη· πονηρὰ γὰρ ἡ βουλὴ αὕτη, ἵνα δοῦλος κύριον ἴδιον ἀρνήσηταιβλέπετε οὖν ὑμεῖς οἱ ταῦτα βουλευόμενοι, μήποτε ἡ βουλὴ αὕτη διαμείνῃ ἐν ταῖς καρδίαις ὑμῶν καὶ ἀποθάνητε τῷ θεῷὑμεῖς δὲ οἱ πάσχοντες ἕνεκεν τοῦ ὀνόματος δοξάζειν ὀφείλετε τὸν θεόν, ὅτι ἀξιους ὑμᾶς ἡγήσατο ὁ θεός, ἵνα τοῦτο τὸ ὄνομα βαστάζητε καὶ πᾶσαι ὑμῶν αἱ ἁμαρτίαι ἰαθῶσινοὐκοῦν μακαρίζετε ἑαυτούς· ἀλλὰ δοκεῖτε ἔργον μέγα πεποιηκέναι, ἐάν τις ὑμῶν διὰ τὸν θεὸν πάθῃζωὴν ὑμῖν ὁ κύριος χαρίζεται, καὶ οὐ νοεῖτε· αἱ γὰρ ἁμαρτίαι ὑμῶν κατεβάρησαν, καὶ εἰ μὴ πεπόνθατε ἕνεκεν τοῦ ὀνόματος κυρίου, διὰ τὰς ἁμαρτίας ὑμῶν τεθνήκειτε ἂν τῷ θεῷταῦτα ὑμῖν λέγω τοῖς διστάζουσι περὶ ἀρνήσεως ἢ ὁμολογήσεως· ὁμολογεῖτε, ὅτι κύριον ἔχετε, μήποτε ἀρνούμενοι παραδοθήσησθε εἰς δεσμωτήριονεἰ τὰ ἔθνη τοὺς δούλους αὐτῶν κολάζουσιν, ἐὰν τις ἀρνήσηται τὸν κύριον ἑαυτοῦ, τί δοκεῖτε ποιήσει ὁ κύριος ὑμῖν, ὃς ἔχει πάντων τὴν ἐξουσίαν; ἄρατε τὰς βουλὰς ταύτας ἀπὸ τῶν καρδιῶν ὑμῶν, ἵνα διαπαντὸς ζήσητε τῷ θεῷ.
Ἐκ δὲ τοῦ ὄρους τοῦ δωδεκάτου τοῦ λευκοῦ οἱ πιστεύσαντεσ τοιοῦτοί εἰσιν· ὡς νήπια βρέφη εἰσίν, οἷς οὐδεμία κακια ἀναβαίνει ἐπὶ τὴν καρδίαν οὐδὲ ἔγνωσαν, τί ἐστι πονηρία, ἀλλὰ πάντοτε ἐν νηπιότητι διέμεινανοἱ τοιοῦτοι οὖν ἀδιστάκτως κατοικήσουσιν ἐν τῇ βασιλείᾳ τοῦ θεοῦ, ὅτι ἐν οὐδενὶ πράγματι ἐμίαναν τὰς ἐντολὰς τοῦ θεοῦ, ἀλλὰ μετὰ νηπιότητος διέμειναν πάσας τὰς ἡμέρας τῆς ζωῆς αὐτῶν ἐν τῇ αὑτῇ φρονήσειὅοι οὖν διαμενεῖτε, φησί, καὶ ἔσεσθε ὡς τὰ βρέφη, κακίαν μὴ ἔχοντες, πάντων τῶν προειρημένων ἐνδοξότεροι ἔσεσθε·´πάντα γὰρ τὰ βρέφη αὐτῷμακάριοι οὖν ὑμεῖς, ὅσοι ἂν ἄρνητε ἀφ’ ἑαυτῶν τὴν πονηρίαν, ἐνδύσησθε δὲ τὴν ἀκακίαν· πρῶτοι πάντων ζήσεσθε τῷ θεῷμετὰ τὰ συντελέσαι αὐτὸν τὰς παραβολὰς τῶν ὀρέων λέγω αὐτῷ· Κύριε, νῦν μοι δηλωσον περὶ τῶν λίθων τῶν ἠρμένων ἐκ τοῦ πεδίου καὶ εἰς τὴν οἰκοδομὴν τεθειμένων ἀντὶ τῶν στρογγύλων τῶν τεθέντων εἰς τὴν οἰκοδομήν, καὶ τῶν ἔτι στρογγύλων ὄντων.
Ἄκουε, φησί, καὶ περὶ τούτων πάντωνοἱ λίθοι οἱ τοῦ πεδίου ἠρμένοι καὶ τεθειμένοι εἰς τὴν οἰκοδομὴν τοῦ πύργου ἀντὶ τῶν ἀποβεβλημένων, αἱ ῥίζαι εἰσὶ τοῦ ὄρους τοῦ λευκοῦἐπεὶ οὖν οἱ πιστεύσαντες, ἐκ τοῦ ὄρους τοῦ λευκοῦ πάντες ἄκακοι εὑρέθησαν, ἐκέλευσεν ὁ κύριος τοῦ πύργου τούτους ἐκ τῶν ῥιζῶν τοῦ ὄρους τούτο βληθῆναι εἰς τὴν οἰκοδομὴν τοῦ πύργου· ἔγνω γάρ, ὅτι, ἐὰν ἀπελθωσιν εἰς τὴν οἰκοδομὴν τοῦ πύργου οἱ λίθοι οὗτοι, διαμενοῦσι λαμπροὶ καὶ οὐδεὶς αὐτῶν μελανήσειQuodsi de ceteris montibus adiecisset, necesse habuisset rurus visitare eam turrem atque purgareHi autem omnew candidi inventi sunt, πιστεύσαντες καὶ οἱ μέλλοντες πιστεύειν· ἐκ τοῦ αὐτου γὰρ γένους εἰσίνμακάριον τὸ γένος τοῦτο, ὅτι ἄκακόν ἐστινἄκουε νῦν καὶ περὶ τῶν λίθων τῶν στρογύλων καὶ λαμπρῶνκαὶ αὐτοὶ πάντες ἐκ τοῦ ὄρους τοῦ λευκοῦ εἰσίνAudi autem, quare rotundi sunt repertiDivitiae suae eos pusillum obscuraverunt a veritate atque obfuscaverunt, a deo vero nunquam recesserunt, nec ullum verbum malum processit de ore eorum, sed omnis aequitas et virtus veritatisHorum ergo mentem cum vidisset dominus posse eos veritati favere, bonos quoque permanere, iussit opes eorum circumcidi, non enim in totum eorum tolli, ut possint aliquid boni facere de eo, quod eis relictum est, et vivent deo, quoniam ex bono genere suntIdeo ergo pusillum circumcisi sunt et positi sunt in structuram turris huius.
Ceteri vero, qui adhuc rotundi remanserunt neque aptati sunt in eam structuram, quia nondum acceperunt sigillum, repositi sunt suo loco; valde cumcidi hoc saeculum ab illis et vanitates opum suarum, et tunc convenient in dei regnumNecesse est enim eos intrare in dei regnum; hoc enim genus innocuum benedixit dominusEx hoc ergo genere non intercidet quisquamEtenim licet quis eorum temtatus a nequissimo diabolo aliquid deliquerit, cito recurret ad dominum suumFelices vos iudico omnes, ego nuntius paenitentiae, quicumque bona est et honorata apud deumDico autem omnibus, vobis, quicumque sigillum hoc accepistis, simplicitatem habere neque offensarum memores esse neque in malitia vestra permanere aut in memoria offensarum amaritudinis, in unum quemque spiritum fieri et has malas scissuras permediare ac tollere a vobis, ut dominus pecorum gaudeat de hisχαρησεται δέ, ἐὰν πάντα ὑγιῆ εὑρεθῇ, καὶ μὴ διαπεπτωκότα, οὐαὶ τοῖς ποιμέσιν ἔσταιἐὰν δὲ καὶ αὐτοὶ οἱ ποιμένες εὑρεθῶσι διαπεπτωκότες, τί ἐροῦσι τῷ δεσπότῃ τοῦ ποιμνίου; ὅτι ἀπὸ τῶν προβάτων διέπεσαν; οὐ πιστευθήσονται· ἄπιστον γὰρ πρᾶγμά ἐστι ποιμένα ὑπὸ προβάτων παθεῖν τι· μᾶλλον δὲ κολασθήσονται διὰ τὸ ψεῦδος αὐτῶνEt ego sum pastor, et validissime oportet me de vobis reddere rationem.
Remediate ergo vos, dum adhuc turris aedificaturDominus habitat in viris amantibus pacem; ei enimvero pax cara est; a litigiosis vero et perditis malitiae longe abestReddite igitur ei spiritum integrum, sicut accepistisSi enim dederis fulloni versimentum novum integrum idque integrum iterum vis recipere, fullo autem scissum tibi illud reddet, recipies illud? Nonne statim scandesces et eum convicio persequeris, dicens: Vestimentum integrum tibi dedi; quare scidisti illud et inutile redegisti? Et propter scissuram, quam in eo fecisti, in usu esse non potestNonne haec omnia verba dices fulloni ergo et de scissura, quam in vestimento tuo fecerit? Si sic igitur tu doles de vestimento tuo et quereris, quod non illud integrum recipias, quid putas dominum tibi facturum, qui spiritum integrum tibi dedit, et tu eum totum inutilem redegisti, ita ut in nullo usu esse possit domino suo? Inutilis enim esse coepit usus eius, eum sit corruptus a teNonne ititur dominus spiritus eius propter hoc factum tuum morte te adficiet? Plane, inquam, omnes eos, quoscumque invenerit in memoria offensarum permanere, adficietClementiam, inquit, eius calcare nolite, sed potius honorificate eum, quod tam patiens est ad delicta vestra et non est sicut vosAgite enim paenitentiam utilem vobis.
Haec omnia, quae supra scripta sunt, ego pastor nuntius paenitentiae ostendi et locutus sum dei servisSi credideritis ergo et audieritis verba mea et ambulaveritis in his et correxeritis itenera vestra, vivere potnritisSin autem permanseritis in malitia et memoria offensarum, nullus ex huiusmodi vivet deoHaec omnia a me dicenda dicta sunt vobisAit mihi ipse pastor: Omnia a me interrogasti? Et dixi: Ita, domineQuare ergo non interrogasti me de forma lapidum in structura repositorum, uod expelvimus formas? Et dixi: Oblitus sum, domineAudi nunc, inquit, de illisHi sunt qui nunc mandata mea audierun et ex totis praecordiis bonam atque puram esse paenitentiam eorum et posse eos in ea permanere, iussit priora peccata eorum deleriHae enim formae peccata erant eorum, et exaequata sunt, ne apparerent.
imilitudo X
Postquam perscripseram librum hunc, venit nuntius ille, qui me tradiderat huic pastori, in domum, in qua eram, et sonsedit supra lectum, et adstitit ad dexteram hic pasotrDeinde vocavit me et hae mihi dixit; Tradidi te, inquit, et domum tuam huic pastori, ut ab eo protegi, inquit, ab omni vexatione et ab omni saevitia, successum autem habere in omni opere bono atque verbo et omnem virtutem aequitatis, in mandatis huius ingredere, quae dedi tibi, et poteris dominari omni nequitiaeCustodienti enim tibi mandata huius subiecta erit omniw cupiditas et dulcedo saeculi huius, successus vero in omni bono te sequiturMaturitatem huius et modestiam suscipe in to et dic omnibus, in magno honore esse eum et dignitate apud dominum et magnae potestatis eum praesidem esse et potentem in officio suoHuic soli per totam orbem paenitentiae potestas tributa estPotensne tibi videtur esse? Sed vos maturitatem huius et verecundiam, quam in vos habet, despicitis.
Dico ei: Interroga ipsum, domine, ex quo in domo mea est, an aliquid extra ordinem fecerim, ex quo eum offenderimEt ego, inquit, scio nihil extra ordinem fecisse te neque esse facturumEt ideo haec loquor tecum, ut perseveresBene enim de te hic apud me existimavitTu autem ceteris haec verba dices, ut et illi, qui egerunt aut acturi sunt paenitentiam, eadem quae tu sentiant et hic apud me de his bene interpretetur et ego apud dominumEt ego, inquam, domine, omni homini indico magnalia domini; spero autem, quia omnes, qui antea peccaverunt, si haec audiant, libenter acturi sunt paenitentiam vitam recuperantesPermane ergo, inquit, in hoc ministerio et consumma illudQuicumque autem mandata huius efficiunt, habebun vitam, et hic apud dominum magnum honoremquicumque vero huius mandata non servant, fugiunt a sua vita et illum adversus; nec mandata eius sequuntur, sed morti se tradunt et unusquisque eorum reus fit sanguinis suiTibi autem dico, ut servias mandatis his, et remedium peccatorum habebis
Misi autem has virgines, ut habitent tecum; vidi enim eas affabiles tibi esseHabes ergo eas adiutrices, quo magis possis huius mandata servare; non potest enim fieri, ut sine his virginibus haec mandata serventurVideo autem eas libenter esse tecum; sed ego praecipiam eis, omnino a domo tua non discedentTu tantum communda domum tuam; in munda enim sunt atque castae et industriae et omnes habentes gratiam apud dominumIgitur si habuerint domum tuam puram, tecum permanebuntSin autem pusillum aliquid inquinationis acciderit, protinus a domo tua recedent; hae enim virgenes nullum omnino diligunt inquinationemDico ei: Spero me, domine, placiturum eis, ita ut in domo mea libenter habitent semperκαὶ ὥσπερ οὗτος, ᾧ παρέδωκάς με, οὐ μέμφεταί με, οὐδὲ αὗται μέμψονταί μελέγει τῷ ποιμένι· Οἶδα, ὅτι ὁ δοῦλος τοῦ θεοῦ θέλει ζῆν καὶ τηρήσει τὰς ἐντολὰς ταύτας καὶ τὰς παρθένους ἐν καθαρότητι καταστήσειταῦτα εἰπὼν τῷ ποιμένι πάλιν παρέδωκέν με καὶ τὰς παρθένους καλέσας λέγει αὐταῖς· Quoniam video vos libenter in domo huius habitare, commendo eum vobis et domum eius, ut a domo eius non recedatis omninoIllae vero haec verba libenter audierunt.
Ait deinde mihi: Viriliter in ministerio hoc conversare, omni homini indica magnalia domini, et habebis gratiam in hoc ministerioQuicumque ergo in his mandatis ambulaverit, vivet et felix erit in vita sua; quicumque vero neglexerit, non vivet et erit infelix in vita suaDic omnibus, ut non cessent, quicumque recte facere possunt; bona opera exercere utile est illisDico autem, omnen hominem de incommodis eripi oportereEt is enim, qui eget et in cotidiana vita patitur incommoda, in magno tormento est ac necessitateQui igitur huismodi animam eripit de necessitate, magnum gaudium sibi adquiritIs enim, qui huiusmodi vexatur incommodo, pari tormento curciatur atque torquet se qui in vincula estMulti enim propter huiusmodi calamites, cum eas sufferre non possunt, mortem sibi adducuntQui novit igitur calamitatem huiusmodi hominiw et non eripit eum, magnum peccatum admittit et reus fit sanguinis eiusFacite igitur opera bona, quicumque accepistis a domino, ne, dum tardatis facere, consummetur structura turrisPropter vos enim intermissum est opus aedificationis eiusNisi festinetis igitur facere recte, consummabitur turris, et excludeminiPostquam vero locutus est mecum, surrexit de lecto et apprehenso pastore et virginibus abiit, dicens autem mihi, remissurum se pastorem illum et virgines in domum meam.
\section{ΒΑΡΝΑΒΑ ΕΠΙΣΤΟΛΗ}
Χαίρετε υἱοὶ καὶ θυγατέρες, ἐν ὀνόματι κυρίου τοῦ ἀγαπήσαντος ἡμᾶς, ἐν εἰρήνῃ.
Μεγάλων μὲν ὄντων καὶ πλουσίων τῶν τοῦ θεοῦ δικαιωμάτων εἰς ὑμᾶς, ὑπέρ τι καὶ καθ’ ὑπερβολὴν ὑπερευφραίνομαι ἐπὶ τοῖς μακαρίοις καὶ ἐνδόξοις ὑμῶν πνεύμασιν· οὕτως ἔμφυτον τῆς δωρεᾶς πνευματικῆς χάριν εἰλήφατεδιὸ καὶ μᾶλλον συγχαίρω ἐμαυτῷ ἐλπίζων σωθῆναι, ὅτι ἀληθῶς βλέπω ἐν ὑμῖν ἐκκεχυμένον ἀπὸ τοῦ πλουσίου τῆς πηγῆς κυρίου πνεῦμα ἐφ’ ὑμᾶςοὕτω με ἐξέπληξεν ἐπὶ ὑμῶν ἡ ἐμοὶ ἐπιποθήτη ὄψις ὑμῶνπεπεισμένος οὖν τοῦτο καὶ συνειδὼς ἐμαυτῷ, ὅτι ἐν ὑμῖν λαλήσας πολλὰ ἐπίσταμαι, ὅτι ἐμοὶ συνώδευσεν ἐν ὁδῷ δικαιοσύνης κύριος, καὶ πάντως ἀναγκάζομαι κἀγὼ εἰς τοῦτο, ἀγαπᾶν ὑμᾶς ὑπὲρ τὴν ψυχήν μου, ὅτι μεγάλη πίστις καὶ ἀγάπη ἐγκατοικεῖ ἐν ὑμῖν ἐπ’ ἐλπίδι ζωῆς αὐτοῦλογισάμενος οὖν τοῦτο, ὅτι ἐὰν μελήσῃ μοι περὶ ὑμῶν τοῦ μέρος τι μεταδοῦναι ἀφ’ οὗ ἔλαβον, ὅτι ἔσται μοι τοιούτοις πνεύμασιν ὑπηρετήσαντι εἰς μισθόν, ἐσπούδασα κατὰ μικρὸν ὑμῖν πέμπειν, ἵνα μετὰ τῆς πίστεως ὑμῶν τελείαν ἔχητε τὴν γνῶσιν.
Τρία οὖν δόγνατά ἐστιν κυρίου· ζωῆς ἐλπίς, κρίσεως, ἀρχὴ καὶ τέλος πίστεως ἡμῶν· καὶ δικαιοσύνη, καὶ ἀγαλλιάσεως ἔργων δικαιοσύνης μαρτυρίαἐγνώρισεν γὰρ ἡμῖν ὁ δεσπότης διὰ τῶν προφητῶν τὰ παρεληλυθότα καὶ τὰ ἐνεστῶτα, καὶ τῶν μελλόντων δοὺς ἀπαρχὰς ἡμῖν γεύσεως, ὧν τὰ καθ’ ἕκαστα βλέποντες ἐνεργούμενα, καθὼ ἐλάλησεν, ὀφείλομεν πλουσιώτερον καὶ ὑψηλότερον προσάγειν τῷ φόβῳ αὐτοῦἐγὼ δὲ οὐχ ὡς διδάσκαλος, ἀλλ’ ὡς εἷς ἐξ ὑμῶν ὑποδείξω ὀλίγα, δι’ ὧν ἐν τοῖς παροῦσιν εὐφρανθήσεσθε.
Ἡμερῶν οὖν οὐσῶν πονηρῶν καὶ αὐτοῦ τοῦ ἐνεργοῦντος ἔχοντος τὴν ἐξουσίαν, ὀφείλομεν ἑαυτοῖς προσέχοντες ἐκζητεῖν τὰ δικαιώματα κυρίουτῆς οὖν πίστεως ἡμῶν εἰσιν βοηθοὶ φόβος καὶ ὑπομονή, τὰ δὲ συμμαχοῦντα ἡμῖν μακροθυμία καὶ ἐγκράτεια· τούτων οὖν μενόντων τὰ πρὸς κύριον ἁγνῶς συνευφραίνονται αὐτοῖς σοφία, σύνεσις, ἐπιστήμη, γνῶσιςπεφανέρωκεν γὰρ ἡμῖν διὰ πάντων τῶν προφητῶν, ὅτι οὔτε θυσιῶν οὔτε ὁλοκαυτωμάτων οὔτε προσφορῶν ψρῄζει, λέγων ὅτε μέν· Τί μοι πλῆθος τῶν θυσιῶν ὑμῶν; λέγει κύριοςπλήρης εἰμὶ ὁλοκαυτωμάτων, καὶ στέαρ ἀρνῶν καὶ αἷμα ταύρων καὶ τράγων οὐ βούλομαι, οὐδ’ ἂν ἔρχησθε ὀφθῆναί μοιτίς γὰρ ἐξεζήτησεν ταῦτα ἐκ τῶν χειρῶν ὑμῶν; πατεῖν μου τὴν αὐλὴν οὐ προσθήσεσθεἐὰν φέρητε σεμίδαλιν, μάταιον· θυμίαμα βδέλυγμά μοί ἐστιν· τὰς νεομηνίας ὑμῶν Ἰησοῦ Χριστοῦ, ἄνευ ζυγοῦ ἀνάγκης, ὤν, μὴ ἀνθρωποποίητον ἔχῃ τὴν προσφοράνλέγει δὲ πάλιν πρὸς αὐτούς· Μὴ ἐγὼ ἐνετειλάμην τοῖς πατράσιν ὑμῶν ἐκπορευομένοις ἐκ γῆς Αἰγύπτου, προσενέγκαι μοι ὁλοκαυτώματα καὶ θυσίας; ἀλλ’ ἢ τοῦτο ἐντειλάμην αὐτοῖς· ἕκαστος ὑμῶν κατὰ τοῦ πλησίον ἐν τῇ καρδίᾳ ἑαυτοῦ κακίαν μὴ μνησικακέτω, καὶ ὅρκον ψευδῆ μὴ ἀγαπᾶτεαἰσθάνεσθαι οὖν ὀφείλομεν, μὴ ὄντες ἀσύνετοι, τὴν γνώμην τῆς ἀγαθωσύνης τοῦ πατρὸς ἡμῶν, ὅτ’ ἡμῖν λέγει, θέλων ἡμᾶς μὴ ὁμοίως πλανωμένους ἐκείνοις ζητειν, πῶς προσάγωμεν αὐτῷἡμῖν οὖν οὕτως λέγει· Θυσία τῷ κυρίῳ καρδία συντετριμμένη, ὀσμὴ εὐωδίας τῷ κυρίῳ κυρίῳ καρδία δοξάζουσα τὸν πεπλακότα αὐτήνἀκριβεύεσθαι οὖν ὀφείλομεν, ἀδελφοί, περὶ τῆς σωτηρίας ἡμῶν, ἵνα μὴ ὁ πονηρὸς παρείσδυσιν πλάνης ποιήσας ἐν ἡμῖν ἐκσφενδονήσῃ ἡμᾶς ἀπὸ τῆς ζωῆς ἡμῶν.
Λέγει οὖν πάλιν περὶ τούτων πρὸς αὐτούς· Ἱνατί μοι νηστεύετε, λέγει κύριος, ὡς σήμερον ἀκουσθῆναι ἐν κραυγῇ τὴν φωνὴν ὑμῶν; οὐ ταύτην τὴν μηστείαν ἐγὼ ἐξελεξάμην, λέγει κύριος, οὐκ ἄνθρωπον ταπεινοῦντα τὴν ψυχὴν αὐτοῦ, οὐδ’ ἂν κάμψητε ὡς κρίκον τὸν τράχηλον ὑμῶν καὶ σάκκον ἐνδύσησθε καὶ σποδὸν ὑποστρώσητε, οὐδ’οὕτως καλέσετε νηστείαν δεκτήνπρὸς ἡμᾶς δὲ λέγει· Ἰδοὺ αὕτη ἡ νηστεία, ἣν ἐγὼ ἐξελεξάμην, λέγει κύριος· λύε πάντα σύνδεσμον ἀδικίας, διάλυε στραγγαλιὰς βιαίων συναλλαγμάτων, ἀπόστελλε τεθραυσμένους ἐν ἀφέσει καὶ πᾶςαν ἄδικον συγγραφὴν διάσπαδιάθρυπτε πεινῶσιν τὸν ἄρτον σου, καὶ γυμνὸν ἐὰν ἴδῃς περίβαλε· ἀστέγους εἴσαγε εἰς τὸν οἶκον σου, καὶ ἐὰν ἴδῃς ταπεινόν, οὐχ ὑπερόψῃ αὐτόν, οὐδὲ ἀπὸ τῶν οικείων τοῦ σπέρματός σουτότε ῥαγήσεται πρώϊμον τὸ φῶς σου, καὶ τὰ ἱμάτιά σου ταχέως ἀνατελεῖ, καὶ προπορεύσεται ἔμπροσθέν σου ἡ δικαιοσύνη, καὶ προπορεύσεται ἔμπροσθέν σου ἡ δικαιοσύνη, καὶ ἡ δόξα τοῦ θεοῦ περιστελεῖ σετότε βοήσεις, καὶ ὁ θεὸς επακούσεταί σου, ἔτι λαλοῦντός σου ἐρεῖ· Ἰδοὺ πάρειμι· ἐὰν ἀφέλῃς ἀπὸ σοῦ σύνδεσμον καὶ χειροτονίαν καὶ ῥῆμα γογγυσμου, καὶ δῷς πεινῶντι τὸν ἄρτον σου ἐκ ψυχῆς σου καὶ ψυχὴν τεταπεινωμένην ἐλεήσῃςεἰς τοῦτο οὖν, ἀδελφοί, ὁ μακρόθυμος προβλέψας, ὡς ἐν ἀκεραιοσύνῃ πιστεύσει ὁ λαός, ὃν ἡτοίμασεν ἐν τῷ ἠγαπημένῳ αὐτοῦ, προεφανέρωσεν ἡμῖν περὶ πάντων, ἵνα μὴ προσρησσώμεθα ὡς ἐπήλυτοι τῷ ἐκείνων νόμῳ.
Δεῖ οὖν ἡμᾶς περὶ τῶν ἐνεστώτων ἐπιπολὺ ἐραυνῶντας ἐκζητεῖν τὰ δυνάμενα ἡμᾶς σώζεινφύγωμεν οὖν τελείως ἀπὸ πάντων τῶν ἔργων τῆς ἀνομίας, μήποτε καταλάβῃ ἡμᾶς τὰ ἔργα τῆς ἀνομίας· καὶ μισήσωμεν τὴν πλάνην τοῦ νῦν καιροῦ, ἵνα εἰς τὸν μέλλοντα ἀγαπηθῶμενμὴ δῶμεν τῇ ἑαυτῶν ψυχῇ ἄνεσιν, ὥστε ἔχειν αὐτὴν ἐξουσίαν μετὰ ἁμαρτωλῶν καὶ πονηρῶν συντρέχειν, μήποτε ὁμοιωθῶμεν αὐτοῖςτὸ τέλειον σκάνδαλον ἤγγικεν, περὶ οὗ γέγραπται, ὡς Ἐνὼχ λέγειΕἰς τοῦτο γὰρ ὁ δεσπότης συντέτμηκεν τοὺς καιροὺς καὶ τὰς ἡμέρας, ἵνα ταχύνῃ ὁ ἠγαπημένος αὐτοῦ καὶ ἐπὶ τὴν κληρονομίαν ἥξῃ, λέγει δὲ οὕτως καὶ ὁ προφήτης· Βασιλεῖαι δέκα ἐπὶ τῆς γῆς βασιλεύσουσιν, καὶ ἐξαναστήσεται ὄπισθεν μικρὸς βασιλεύς, ὃς ταπεινώσει τρεῖς ὑφ’ ἓν τῶν βασιλέωνὁμοίως περὶ τοῦ αὐτοῦ λέγει Δανιήλ· Καὶ εἶδον τὸ τέταρτον θηρίον τὸ πονηρὸν καὶ ἰσχυρὸν καὶ χαλεπώτερον παρὰ πάντα τὰ θηρία τῆς θαλάσσης, καὶ ὡς ἐξ αὐτοῦ ἀνέτειλεν δέκα κέρατα, καὶ ὡς ἐταπείνωσεν ὑφ’ ἓν τρία τῶν μεγάλων κεράτωνσυιέναι οὖν ὀφείλετεἔτι δὲ καὶ τοῦτο ἐρωτῶ ὑμᾶς ὡς εἷς ἐξ ὑμῶν ὤν, ἰδίως δὲ καὶ πάντας ἀγαπῶν ὑπὲρ τὴν ψυχήν μου, προσέχειν νῦν ἑαυτοῖς καὶ μὴ ὁμοιοῦσθαί τισιν ἐπισωρεύοντας ταῖς ἁμαρτίαις ὑμῶν λέγοντας, ὅτι ἡ διαθήκη ἐκείνων καὶ ἡμῶνἡμῶν μέν· ἀλλ’ ἐκεῖνοι οὕτως εἰς τέλος ἀπωλεσαν αὐτὴν λαβόντος ἤδη τοῦ Μωϋσέωςλέγει γὰρ ἡ γραφή· Καὶ ἦν Μωϋσῆς ἐν τῷ ὄρει νηστεύων ἡμέρας τεσσαράκοντα καὶ νύκτας τεσσαράκοντα, καὶ ἔλαβεν τὴν διαθήκην ἀπὸ τοῦ κυρίου, πλάκας λιθίνας γεγραμμένας τῷ κακτύλῳ τῆς χειρὸς τοῦ κυρίουἀλλὰ ἐπιστραφέντες ἐπὶ τὰ εἴδωλα ἀπώλεσαν αὐτήνλέγει γὰρ οὕτως κύριοςΜωϋσῆ Μωϋσῆ, κατάβηθι τὸ τάχος, ὅτι ἠνόμησεν ὁ λαός σου, οὓς ἐξήγαγες ἐκ γῆς Αἰγύπτου, καὶ συνῆκεν Μωϋσῆς καὶ ἔριψεν τὰς δύο πλάκας ἐκ τῶν χειρῶν αὐτοῦ· καὶ συνετρίβη αὐτῶν ἡ διαθήκη, ἵνα ἡ τοῦ ἠγαπημένου Ἰησοῦ ἐγκατασφραγισθῇ εἰς τὴν καρδίαν ἡμῶν ἐν ἐπίδι τῆς πίστεως αὐτοῦπολλὰ δὲ θέλων γράφειν, οὐχ ὡς διδάσκαλος, ἀλλ’ ὡς πρέπει ἀγαπῶντι ἀφ’ ὧν ἔχομεν μὴ ἐλλείπειν, γράφειν ἐσπούδασα, περίψημα ὑμῶνδιὸ προσέχωμεν ἐν ταῖς ἐσχάταις ἡμέραις· οὐδὲν γὰρ ὠφελήσει ἡμᾶς ὁ πᾶς χρόνος τῆς πίστεως ἡμῶν, ἐὰν μὴ νῦν ἐν τῷ ἀνόμῳ καιρῷ καὶ τοῖς μέλλουσιν σκανδάλοις, ὡς πρέπει υἱοῖς θεοῦ, ἀντιστῶμεν, ἵνα μὴ σχῇ παρείσδυσιν ὁ μέλαςφύγωμεν ἀπὸ´πάσης ματαιότητος, μισήσωμεν τελείως τὰ ἔργα τῆς πονηρᾶς ὁδοῦμὴ καθ’ ἑαυτοὺς ἐνδύνοντες μονάζετε ὡς ἤδη δεδικαιωμένοι, ἀλλ’ ἐπὶ τὸ αὐτὸ συνερχόμενοι συνζητεῖτε περὶ τοῦ κοινῇ συμφέροντοςλέγει γὰρ ἡ γραφή· Οὐαὶ οἱ συνετοὶ ἑαυτοῖς καὶ ἐνώπιον ἑαυτῶν ἐπιστήμονεςγενώμεθα πνευματικοί, γενώμεθα ναὸς τέλειος τῷ θεῷἐφ’ ὅσον ἐστὶν ἐν ἡμῖν, μελετῶμεν τὸν φόβον τοῦ θεοῦ καὶ φυλάσσειν ἀγωνιζώμεθα τὰς ἐντολὰς αὐτοῦ, ἵνα ἐν τοῖς δικαιώμασιν αὐτοῦ εὐφρανθῶμενὁ κύριος ἀποσωπολήμπτως κρινεῖ τὸν κόσμονἕκαστος καθὼς ἐποίησεν κομιεῖταιἐὰν ᾖ ἀγαθός, ἡ πονηρός, ὁ μισθὸς τῆς πονηρίας ἔπροσθεν αὐτοῦ· ἵνα μήποτε ἐπαναπαύομενοι ὡς κλητοὶ ἐπικαθυπνώσωμεν ταῖς ἁμαρτίαις ἡμῶν, καὶ ὁ πονηρὸς ἄρχων λαβὼν τὴν καθ’ ἡμῶν ἐξουσίαν ἀπώσηται ἡμᾶς ἀπὸ τῆς βασιλείας τοῦ κυρίουἔτι δὲ κἀκεῖνο, ἀδελφοί μου, νοεῖτε· ὅταν βλέπετε ματὰ τηλικαῦτα σημεῖα καὶ τέρατα γεγονότα ἐν τῷ Ἰσραήλ, καὶ οὕτως ἐγκαταλελεῖφθαι αὐτούς· προσέχωμεν, μήποτε, ὡς γέγραπται, πολλοὶ κλητοί, ολίγοι δὲ ἐκλεκτοὶ εὑρεθῶμεν
Εἰς τοῦτο γὰρ υπέμεινεν ὁ κύριος παραδοῦναι τὴν σάρκα εἰς καταφθοράν, ἵνα τῇ ἀφέσει τῶν ἁμαρτιῶν ἁγνισθῶμεν, ὅ ἐστιν ἐν τῷ αἵματι τοῦ ῥαντίσματος αὐτοῦγέγραπται γὰρ περὶ αὐτοῦ ἃ μὲν πρὸς τὸν Ἰσραήλ, ἃ δὲ πρὸς ἡμᾶς, λέγει δὲ οὕτως· Ἐτραυματίσθη διὰ τὰς ἀνομίας ἡμῶν καὶ μεμαλάκισται διὰ τὰς ἁμαρίας ἡμῶν· τῷ μώλωπι αὐτοῦ ἡμεῖς ἰάθημεν· ὡς πρόβατον ἐπὶ σφαγὴν ἤχθη, καὶ ὡς ἀμνὸς ἄφωνος ἐνανίον τοῦ κείραντος αὐτόνοὐκοῦν ὑπερευχαρστεῖν οφείλομεν τῷ κυρίῶ, ὅτι καὶ τὰ παρεληλυθότα ἡμῖν ἐγνώρισεν καὶ ἐν τοῖς ἐνεστῶσιν ἡμᾶς ἐσόφισεν, καὶ εἰς τὰ μέλλοντα οὐκ ἐσμὲν ἀσύνετοιλέγει δὲ ἡ γραφή· Οὐκ ἀδίκως ἐκτείνεται δίκτυα πτερωτοῖςτοῦτο λέγει, ὅτι δικαιως ἀπολεῖται ἄνθρωπος, ὃς ἔχων ὁδοῦ δικαιοσύνης γνῶσιν ἑαυτὸν εἰς ὁδὸν σκότους ἀποσυνέχειἔτι δὲ καὶ τοῦτο, ἀδελφοί μου· εἰ ὁ κύριος ὑπέμεινεν παθεῖν περὶ τῆς ψυχῆς ἡμῶν, ὢν παντὸς τοῦ κόσμου κύριος ᾧ εἶπεν ὁ θεὸς ἀπὸ καταβολῆς κόσμου· Ποιήσωμεν ἄνθρωπον κατ’ εἰκόνα καὶ καθ’ ὁμοίωσιν ἡμετέραν· πῶς οὖν ὑπέμεινεν ὑπὸ χειρὸς ἀνθρώπων παθεῖν; μάθετεοἱ προφῆται, ἀπ’ αὐτοῦ ἔχοντες τὴν χάριν, εἰς αὐτὸν ἐπροφήτευσαν· αὐτὸς δέ, ἵνα καταργήσῃ τὸν θάνατον καὶ τὴν ἐκ νεκρῶν ἀνάστασιν δείξῃ, ὅτι ἐν σαρκὶ ἔδει αὐτὸν φανερωθῆναι, ὑπέμεινεν, ἵνα τοῖς πατράσιν τὴν ἐπανγγελίαν ἀποδῷ, καὶ αὐτὸς ἑαυτῷ τὸν λαὸν τὸν καινὸν ἑτοιμάζων ἐπιδείξῃ ἐπὶ τῆς γῆς ὤν, ὅτι τὴν ἀνάστασιν αὐτὸς ποιήσας κρινεῖπέρας γέ τοι διδάσκων τὸν Ἰσραὴλ καὶ τηλικαῦτα τέρατα καὶ σημεῖα ποιῶν ἐκήρυσσεν, καὶ ὑπερηγάπησεν αὐτόνὅτε δὲ τοὺς ἰδίους ἀποστόλους τοὺς μέλλοντας κηρύσσειν τὸ εὐαγγέλιον αὐτοῦ ἐξελέξατο, ὄντας ὑπὲρ πᾶσαν ἁμαρτίαν ἀνομωτέρους, ἵνα δείξῃ, ὅτι οὐκ ἦλθεν καλέσαι δικαίους, ἀλλὰ ἁμαρτωλούς, τότε ἐφανέρωσεν ἐν σαρκί, οὐδ’ ἂν πως οἱ ἄνθρωποι ἐσώθησαν βλέποντες αὐτόν, ὅτε τόν μέλλοντα μὴ εἶναι ἥλιον, ἔργον τῶν χειρῶν αὐτοῦ ὑπάρχοντα, ἐμβλέποντες οὐκ ἰσχύουσιν εἰς τὰς ἀκτῖνας αὐτοῦ ἀντοφθαλμῆσαι; οὐκοῦν ὁ υἱὸς τοῦ θεοῦ εἰς τοῦτο ἐν σαρκὶ ἦλθεν, ἵνα τὸ τέλειον τῶν ἁμαρτιῶν ἀνακεφαλαιώσῃ τοῖς διώξασιν ἐν θανάτῳ τοὺς προφήτας αὐτοῦοὐκοῦν εἰς τοῦτο ὑπέμεινεν, λέγει γὰρ ὁ θεὸς τὴν πληγὴν τῆς σαρκὸς αὐτοῦ ὅτι ἐξ αὐτῶν· Ὅταν πατάξωσιν τὸν ποιμένα ἑαυτῶν, τότε ἀολεῖται τὰ πρόβατα τῆς ποίμνηςαὐτὸς δὲ ἠθέλησεν οὕτω παθεῖν· ἔδει γάρ, ἵνα ἐπὶ ξύλου πάθῃλέγει γὰρ ὁ προφητεύων ἐπ’ αὐτῷΦεῖσαί μου τῆς ψυχῆς ἀπὸ ῥομφαίας, καί· Καθήλωσόν μου τὰς σάρκας, ὅτι πονηρευομένων συναγωγαὶ ἐπανεστησάν μοικαὶ πάλιν λέγει· Ἰδού, τέθεικά μου τὸν νῶτον εἰς μάστιγας, τὰς δὲ σιαγόνας εἰς ῥαπίσματατὸ δὲ πρόσωπόν μου ἔθηκα ὡς στερεὰν πετραν.
Ὅτε οὖν ἐποίησεν τὴν ἐντολήν, τί λέγει; Τίς ὁ κρινόμενός μοι; ἀντιστήτω μοι· ἢ τίς ὁ δικαιούμενός μοι; ἐγγισάτω τῷ παιδὶ κυρίουοὐαὶ ὑμῖν, ὅτι ὑμεῖς πάντες ὡς ἱμάτιον παλαιωθήσεθε, καὶ σὴς καταφάγεται ὑμᾶςκαὶ παλιν λέγει ὁ προφήτης, ἐπεὶ ὡς λίθος ἰσχυρὸς ἐτέθη εἰς συντριβήν· Ἰδού, ἐμβαλῶ εἰς τὰ θεμέλια Σιὼν λίθον πολυτελῆ, ἐκλεκτόν, ἀκρογωναῖον, ἔντιμονεἶτα τί λέγει; Καὶ ὃς ἐλπίσει ἐπ’ αὐτὸν ζήσετα εἰς τὸν αἰῶναἐπὶ λίθον οὖν ἡμῶν ἡ ἐλπίς; μͅμὴ γένοιτο· ἀλλ’ ἐπεὶ ἐν ἰσχΰ τέθεικεν τὴν σάρκα αὐτοῦ κύριοςλέγει γάρ· Καὶ ἔθηκέ με ὡς στερεὰν πέτρανλέγει δὲ πάλιν ὁ προφήτης· Λίθον ὃν ἀπεδοκίμασαν οἱ οἰκοδομοῦντες, οὗτος ἐγενήθη εἰς κεφαλὴν γωνίαςκαὶ πάλιν λέγει· Αὔτη ἐστὶν ἡ ἡμέρα ἡ μεγάλη καὶ θαυματή, ἣν ἐποίησεν ὁ κύριοςἀπολούστερον ὑμῖν γράφω, ἵνα συιῆτε· ἐγὼ περίψημα τῆς ἀγάπης ὑμῶντί οὖν λέγει πάλιν ὁ προφητης; Περιέσχεν με συναγωγὴ πονηρευομένων, ἐκύκλωσάν με ὡσεὶ μέλισσαι κηρίον, καί· Ἐπὶ τὸν ἱματισμόν μου ἔβαλον κλῆρονἐν σαρκὶ οὖν αὐτοῦ μέλλοντος φανεροῦσθαι καὶ πάσχειν, προεφανερώθη τὸ πάθοςλέγει γὰρ ὁ προφήτης ἐπὶ τὸν Ἰσραήλ· Οὐαὶ τῇ ψυχῇ αὐτῶν, ὅτι βεβούλευνται βουλὴν πονηρὰν καθ’ ἑαυτῶν, εἰπόντες· Δήσωμεν τὸν δίκαιον, ὅτι δύσχρηστος ἡμῖν ἐστίντί λέγει ὁ ἄλλος προφήτης Μωϋσῆς αὐτοῖς; Ἰδού, τάδε λέγει κύριος ὁ θεός· Ἐισέλθατε εἰς τὴν γῆν τὴν ἀγαθήν, ἣν ὤμοσεν κύριος τῷ Ἀβραὰμ καὶ Ἰσαὰκ καὶ Ἰακώβ, καὶ κατακληρονομήσατε αὐτήν, γῆν ῥέουσαν γάλα καὶ μέλιτί δὲ λέγει ἡ γνῶσις; μάθετεἐλπίσατε, φησίν, ἐπὶ τὸν ἐν σαρκὶ μέλλοντα φανεροῦσθαι ὑμῖν Ἰησοῦνἄνθρωπος γὰρ γῆς ἐστιν πάσχοῦσα· ἀπὸ προσώπου γὰρ τῆς γ͂ς ἡ πλάσις τοῦ Ἀδὰμ ἐγένετοτί οὖν λέγει Εἰς τὴν γῆν τὴν ἀγαθήν, γῆν ῥέουσαν γάλα καὶ μέλι; εὐλογητὸς ὁ κύριος ἡμῶν, ἀδελφοί, ὁ σοφίαν καὶ νοῦν θέμενος ἐν ἡμῖν τῶν κρυφίων αὐτοῦ· λέγει γὰρ ὁ προφήτης παραβολὴν κυρίου· τίς νοήσει, εἰ μὴ σοφὸς καὶ ἐπιστήμων καὶ ἀγαπῶν τὸν κύριον αὐτοῦ; ἐπεὶ οὖν ἀνακαινίσας ἡμᾶς ἄλλον τύπον, ὡς παιδίων ἔχειν τὴν ψυχήν, ὡς ἂν δὴ ἀναπλάσσοντος αὐτοῦ ἡμᾶςλέγει γὰρ ἡ γραφὴ περί ἡμῶν, ὡς λέγει τῷ υἱῷ· Ποιήσωμεν κατ’ εἰκόνα καὶ καθ’ ὁμοίσιν ἡμῶν τὸν ἄνθρωπον, καὶ ἀρχέτωσαν τῶν θηρίων τῆς γῆς καὶ τῶν πετεινῶν τοῦ οὐρανοῦ καὶ τῶν ἰχθύων τῆς θαλάσσηςκαὶ εἶπεν κύριος, ἰδὼν τὸ καλὸν πλάσμα ἡμῶν· Αὐξάνεσθε καὶ πληθυνέσθε καὶ πληρώσατε τὴν γῆνταῦτα πρὸς τὸν υἱόνπάλιν σοι ἐπιδειξω, πῶς πρὸς ἡμᾶς λέγειδευτέραν πλάσιν ἐπ’ ἐσχάτων ἐποίησενλέγει δὲ κύριος· Ἰδού, ποιῶ τὰ ἔσχατα ὡς τὰ πρῶταεἰς τοῦτο οὖν ἐκήρυξεν ὁ προφήτης· Εἰσέλθατε εἰς γῆν ῥέουσαν γάλα καὶ μέλι καὶ κατακυριεύσατε αὐτῆςἴδε οὖν, ἡμεῖς ἀναπεπλάσμεθα, καθὼς πάλιν ἐν ἑτέρῳ προφήτῃ λέγει· Ἰδού, λέγει κύριος, ἐξελῶ τούτων, τουτέστιν ὧν προέβλεπεν τὸ πνεῦμα κυρίου, τὰς λιθίνας καρδίας καὶ ἐμβαλῶ σαρκίνας· ὅτι αὐτὸς ἐν σαρκὶ ἔμελλεν φανεροῦσθαι καὶ ἐν ἡμῖν κατοικεῖνναὸς γὰρ ἅγιος, ἀδελφοί μου, τῷ κυρίῳ τὸ κατοικητήριον ἡμῶν τῆς καρδίαςλέγει γὰρ κύριος πάλιν· Καὶ ἐν τίνι ὀφθήσομαι τῷ κυρίῳ τῷ θεῷ μου καὶ δοξασθήσομαι; λέγει· Ἐξομολογήσομαί σοι ἐν ἐκκλησίᾳ ἀδελφῶν μου, καὶ ψαλῶ σοι ἀνάμεσον ἐκκλησίας ἁγίωνοὐκοῦν ἡμεῖς ἐσμέν, οὓς εἰσήγαγεν εἰς τὴν γῆν ἀγαθήντί οὖν τὸ γάλα καὶ τὸ μέλι; ὅτι πρῶτον τὸ παιδίον μέλιτι, εἶτα γάλακτι ζωοοιεῖται· οὕτως οὖν καὶ ἡμεῖς τῇ πίστει τῆς ἐπαγγελίας καὶ τῷ λόγῳ ζωοποιούμενοι ζήσομεν κατακυριεύοντες τῆς γῆςπροειρήκαμεν δὲ ἐπάνωΚαὶ αὐξανέσθωσαν καὶ πληθυνέσθωσαν καὶ ἀρχέτωσαν τῶν ἰχθύων ἢ πετεινῶν τοῦ οὐρανοῦ; αἰσθάνεσθαι γὰρ ὀφείλομεν, ὅτι τὸ ἄρχειν ἐξουσίας ἐστίν, ἵνα τις ἐπιτάξας κυριεύσῃεἰ οὖν οὐ γίνεται τοῦτο νῦν, ἄρα ἡμῖν εἴρηκεν, πότε· ὅταν καὶ αὐτοὶ τελειωθῶμεν κληρονόμοι τῆς διαθήκης κυρίου γενέσθαι.
Οὐκοῦν νοεῖτε τέκνα εὐφροσύνης, ὅτι πάντα ὁ καλὸς κύριος προεφανέρωσεν ἡμῶν, ἵνα γνῶμεν, ᾧ κατὰ πάντα εὐχαριστοῦντες ὀφείλομεν αἰνεῖνεἰ οὖν ὁ υἱὸς τοῦ θεοῦ, ὢν κύριος καὶ μέλλων κρίνειν ζῶντας καὶ νεκρούς, ἔπαθεν, ἵνα ἡ πληγὴ αὐτοῦ ζωοποιήσῃ ἡμᾶς· πιστεύσωμεν, ὅτι ὁ υἱὸς τοῦ θεοῦ οὐκ ἠδύνατο παθεῖν εἰ μὴ δι’ ἡμᾶςἀλλὰ καὶ σταρωθεὶς ἐποτιζετο ὄξει καὶ χολῇἀκούσατε, πῶς περὶ τούτου πεφανέρωκαν οἱ ἱερεῖς τοῦ ναοῦγεγραμμένης ἐντολῆς· Ὃς ἂν μὴ νηστεύσῃ τὴν νηστείαν, θανάτω ἐξολεθρευθύσεται, ἐνετείλατο κύριος, ἐπεὶ καὶ αὐτὸς ὑπὲρ τῶν ἡμετέρων ἁμαρτιῶν ἔμελλεν τὸ σκεῦος τοῦ πνευματος προσφέρειν θυσίαν, ἵνα καὶ ὁ τύπος ὁ γενόμενος ἐπὶ Ἰσαὰκ τοῦ προσενεχθέντος ἐπὶ τὸ θυσιαστήριον τελεσθῇτί οὖν λέγει ἐν τῷ προφήτῃ; Καὶ φαγέτωσαν ἐκ τοῦ τράγου τοῦ προσφερομένου τῇ νηστείᾳ ὑπὲρ πασῶν τῶν ἁμαρτιῶνπροσέχετε ἀκριβῶς· Καὶ φαγέτωσαν οἱ ἱερεῖς μόνοι πάντες τὸ ἔντερον ἄπλυτον μετὰ ὄξουςπρὸς τί; ἐπειδὴ ἐμὲ ὑπὲρ ἁμαρτιῶν μέλλοντα τοῦ λαοῦ μου τοῦ καινοῦ προσφέρειν τὴν σάρκα μου μέλλετε ποτίζειν χολὴν μετὰ ὄξους, φάγετε ὑμεῖς μόνοι, τοῦ λαοῦ νηστεύοντος καὶ κοτομένου ἐπὶ σάκκου καὶ σποδοῦἵνα δείξῃ, ὅτι δεῖ αὐτὸν παθεῖν ὑπ’ αὐτῶνἃ ἐνετείλατο, προσέχετε· Λάβετε δύο τράγους καλοὺς καὶ ὁμοίους καὶ προσενέγκατε, καὶ λαβέτω ὁ ἱερεὺς τὸν ἕνα εἰς ὁλοκαύτωμα ὑπὲρ ἁμαρτιῶντὸν δὲ ἕνα τί ποιήσωσιν; Ἐπικατάρατος, φησιν, ὁ εἷςπροσέχετε, πῶς ὁ τύπος τοῦ Ἰησοῦ φανεροῦνται· Καὶ ἐμπτύσατε πάντες καὶ κατακεντήσατε καὶ περίθετε τὸ ἔριον τὸ κόκκινον περὶ τὴν κεφαλὴν αὐτοῦ, καὶ οὕτως εἰς ἔρημον βληθήτωκαὶ ὅταν γένηται οὕτως, ἄγει ὁ βαστάζων τὸν τράγον εἰς τὴν ἔρημον καὶ ἀφαιρεῖ τὸ ἔριον καὶ ἐπιτίθησιν αὐτὸ ἐπὶ φρύγανον τὸ λεγόμενον ῥαχήλ, οὗ καὶ τοὺς βλαστοὺς εἰώθαμεν τρώγειν ἐν τῇ χώρᾳ εὑρίσκοντες· οὕτω μόνης τῆς ῥαχοῦς οἱ καρποὶ γλυκεῖς εἰσιντί οὖν τοῦτό ἐστιν; προσέχετε· Τὸν μὲν ἕνα ἐπὶ τὸ θυσιαστήριον, τὸν δὲ ἕνα ἐπικατάρατον, καὶ ὅτι τὸν ἐπικατάρατον ἐστεφανωμένον; ἐπειδὴ ὄψονται αὐτὸν τότε τῇ ἡμέρᾳ τὸν ποδήρη ἔχοντα τὸν κόκκινον περὶ τὴν σάρκα καὶ ἐροῦσιν· Οὐχ οὗτός ἐστιν, ὅν ποτε ἡμεῖς ἐσταυρώσαμεν ἐξουθενήσαντες καὶ κατακεντήσαντες καὶ ἐμπτύσαντες; ἀληθῶς οὗτος ἦν, ὁ τότε λέγων ἑαυτὸν υἱον θεοῦ εἶναιπῶς γὰρ ὅμοιος ἐκείνῳ; εἰς τοῦτο ὁμοίους τοὺς τράγους, καλούς, ἴσους, ἵινα, ὅταν ἴδωσιν αὐτὸν τότε τράγουοὐκοῦν ἴδε τὸν τύπον τοῦ μέλλοντος πάσχειν Ἰησοῦτί δέ, ὅτι τὸ ἔριον μέσον τῶν ἀκανθῶν τιθέασιν; τύπος ἐστὶν τοῦ Ἰησοῦ τῇ ἐκκλησίᾳ θέμενος, ὅτι ὃς ἐὰν θέλῃ τὸ ἔριον ἆραι τὸ κόκκινον, δεῖ αὐτὸν πολλὰ παθεῖν διὰ τὸ εἶναι φοβερὰν τὴν ἄκανθαν, καὶ θλιβέντα κυριεῦσαι αὐτοῦοὕτω, φησίν, οἱ θέλοντές με ἰδεῖν καὶ ἅψασθαί μου τῆς βασιλείας ὀφείλουσιν θλιβέντες καὶ παθόντες λαβεῖν με
Τίνα δὲ δοκεῖτε τύπον εἶναι, ὅτι ἐντέταλται τῷ Ἰσραὴλ προσφέρειν δάμαλιν τοὺς ἄνδρας, ἐν οἷς εἰσὶν ἁμαρίαι τέλειαι, καὶ σφάξαντας κατακαίεν, καὶ αἴρειν τότε τὴν σποδὸν παιδία καὶ βάλλειν εἰς ἄγγη καὶ περιτιθέναι τὸ ἔριον τὸ κόκκινον ἐπὶ ξύλον (ἴδε πάλιν ὁ τύπος ὁ τοῦ σταυροῦ καὶ τὸ ἔριον τὸ κόκκινον) καὶ τὸ ὕσσωπον, καὶ οὓτως ῥαντίζειν τὰ παιδία καθ’ ἕνα τὸν λαόν, ἵνα ἁγνίζωνται ἀπὸ τῶν ἁμαρτιῶν; νοεῖτε, πῶς ἐν ἁπλότητι λέγει ὑμῖνὁ μόσχος ὁ Ἰησοῦς ἐστίν, οἱ προσφέροντες ἄνδρες ἁμαρτωλοὶ οἱ προσενέγκαντες αὐτὸν ἐπὶ τὴν σφαγήνεἶτα οὐκέτι ἄνδρες, οὐκέτι ἁμαρτωλῶν ἡ δόξαοἱ ῥαντίζοντες παῖδες οἱ εὐαγγελισάμενοι ἡμῖν τὴν ἄφεσιν τῶν ἁμαρτιῶν καὶ τὸν ἁνισμὸν τῆς καρδίας, οἷς ἔδωκεν τοῦ εὐαγγελίου τὴν ἐξουσίαν (οὖσιν δεκάδυο εἰς μαρτύριον τῶν φυλῶν ὅτι δεκάδυο φυλαὶ τοῦ Ἰσραεήλ), εἰς τὸ κηρύσσεινδιὰ τί δὲ τρεῖς παῖδες οἱ ῥαντίζοντες; εἰς μαρτυριον Ἀβραάμ, Ἰσαάκ, Ἰακώβ, ὅτι οὗτοι μεγάλοι τῷ θεῷὅτι δὲ τὸ ἔριον ἐπὶ τὸ ξύλον; ὅτι ἡ βασιλεία Ἰησοῦ ἐπὶ ξύλου, καὶ ὅτι οἱ ἐλπίζοντες ἐπ’ αὐτὸν ζήσονται εἰς τὸν αἰῶναδιὰ τί δὲ ἅμα τὸ ἔριον καὶ τὸ ὕσσωπον; ὅτι ἐν τῇ βασιλείᾳ αὐτοῦ ἡμέραι ἔσονται πονηραὶ καὶ ῥυπαραί, ἐν αἷς ἡμεῖς σωθησόμεθα· ὅτι καὶ ὁ ἀλγῶν σάρκα διὰ τοῦ ῥύπου τοῦ ὑσσύπου ἰᾶται καὶ διὰ τοῦτο οὕτως γενόμενα ἡμῖν μέν ἐστιν φανερά, ἐκείνοις δὲ σκοτεινά, ὅτι οὐκ ἤκουσαν φωνῆς κυρίου.
Λέγει γὰρ πάλιν περὶ τῶν ὠτίων, πῶς περιέτεμεν ἡμῶν τὴν καρδίανλέγει κύριος ἐν τῷ προφήτῃ· Εἰς ἀκοὴν ὠτίου ἡπήκουσάν μουκαὶ πάλιν λέγει· Ἀκοῇ ἀκούσονται οἱ πόρρωθεν, ἃ ἐποίησα γνώσονταικαί· Περιτμήθητε, λέγει κύριος, τὰς καρδίας ὑμῶνκαὶ πάλιν λέγει· Ἄκουε Ἰσραήλ, ὅτι τάδε λέγε κύριος ὁ θεός σουκαὶ πάλιν τὸ πνεῦμα κυρίου προφητεύει· Τίς ἐστιν ὁ θέλων ζῆσαι εἰς τὸν αἰῶνα; ἀκοῇ ἀκουσάτω τῆς φωνῆς τοῦ παιδός μουκαὶ πάλιν λέγει· Ἀκουε οὐρανέ, καὶ ἐνωτίζου γῆ, ὅτι κύριος ἐλάλησεν ταῦτα εἰς μαρτύριονκαὶ πάλιν λέγει· Ἀκούσατε λόγον κυρίου, ἄρχοντες τοῦ λαοῦ τούτουκαὶ πάλιν λέγει· Ἀκούσατε, τέκνα, φωνῆς βοῶντος ἐν τῇ ἐρήμῳοὐκοῦν περιέτεμεν ἡμῶν τὰς ἀκοάς, ἵνα ἀκούσαντες λόγον πιστεύσωμεν ἡμεῖςἀλλὰ καὶ ἡ περιτομή, ἐφ’ ᾗ πεποίθασιν, κατήργηταιπεριτομὴν γὰρ εἰρηκεν οὐ σαρκὸς γενηθῆναι· ἀλλὰ παρέβησαν, ὅτι ἄγγελος πονηρὸς ἐσόφιζεν αὐτούςλέγει πρὸς αὐτούς· Τάδε λέγει κύριος ὁ θεὸς ὑμῶν (ὧδε εὑρίκω ἐντολήν)· Μὴ σπείρητε ἐπ’ ἀκάνθαις, περιτμήθητε τῷ κυρίῳ ὑμῶνκαὶ τί λέγει; Περιτμήθητε καρδίαν ὑμῶν, καὶ τον τράχηλον ὑμῶν οὐ σκληρυνεῖτελάβε πάλιν· Ἰδού, λέγει κύριος,´πάντα τὰ ἔθνη ἀπερίτμητα ἀκροβυστίαν, ὁ δὲ λαὸς οὗτος ἀπερίτμητος καρδίαςἀλλ’ ἐρεῖς· Καὶ μὴν περιτέτμηται ὁ λαὸς εἰς σφραγῖδαἀλλὰ καὶ πᾶς Σύρος καὶ Ἄραψ καὶ πάντες οἱ ἱερεῖς τῶν εἰδώλωνἄρα οὖν κἀκεῖνοι ἐκ τῆς διαθήκης αὐτῶν εἰσίν; ἀλλὰ καὶ οἱ Αἰγύπτιοι ἐν περιτομῇ εἰσίνμάθετε οὖν, τέκνα ἀγάπης, περὶ πάντων πλουσίως, ὅτι Ἀβραάμ, πρῶτος περιτομὴν δούς, ἐν πνεύματι προβλέψας εἰς τὸν Ἰησοῦν περιέτεμεν, λαβὼν τριῶν γραμμάτων δόγματαλέγει γάρ· Καὶ περιέτεμεν Ἀβραὰμ ἐκ τοῦ οἴκου αὐτοῦ ἄνδρας δεκαοκτὼ καὶ τριακοσίουςτίς οὖν ἡ δοθεῖσα αὐτῷ γνῶσις; μάθετε, ὅτι τοὺς δεκαοκτὼ πρώτους, καὶ διάστημα ποιήσας λέγει τριακοσίουςτὸ δεκαοκτὼ ι’ δέκα, η’ ὀκτώ· ἔχεις Ἰησοῦνὅτι δὲ ὁ σταυρὸς ἐν τῷ ταῦ ἤμελλεν ἔχειν τὴν χάριν, λέγει καὶ τοὺς τριακοσίουςδηλοῖ οὖν τὸν μὲν Ἰησοῦν ἐν τοῖς δυσὶν γράμμασιν, καὶ ἐν τῷ ἑνὶ τὸν σταυρόνοἶδεν ὁ τὴν ἔμφυτον δωρεὰν τῆς διδαχῆς αὐτοῦ θέμενος ἐν ἡμῖνοὐδεὶς γνησιώτερον ἔμαθεν ἀπ’ ἐμοῦ λόγον· ἀλλὰ οἶδα, ὅτι ἄξιοί ἐστε ὑμεῖς.
Ὅτι δὲ Μωϋσῆς εἶπεν· Οὐ φάγεσθε χοῖρον οὔτε ἀετὸν οὔτε ὀξύπτερον οὔτε κόρακα οὔτε πάντα ἰχθύν, ὃς οὐκ ἔχει λεπίδα ἐν ἑαυτῷ, τρία ἔλαβεν ἐν τῇ συνέσει δόγματαπέρας γέ τοι λέγει αὐτοῖς ἐν τῷ Δευτερονομίῳ· Καὶ διαθήσομαι πρὸς τὸν λαὸν τοῦτον τὰ δικαιώματά μουἄρα οὖν οὐκ ἔστιν ἐντολὴ θεοῦ τὸ μὴ τρώγειν, Μωϋσῆς δὲ ἐν πνεύματι ἐλάλησεντὸ οὖν χοιρίον πρὸς τοῦτο εἶπεν· οὐ κολληθήσῃ, φησίν, ἄθρώποις τοιούτοις, οἵτινές εἰσιν ὅμοιοι χοίρων· τουτέστιν ὅταν σπαταλῶσιν, ἐπιλανθάνονται τοῦ κυρίου, ὅταν δὲ ὑστεροῦνται, ἐπιγινώσκουσιν τὸν κύριον, ὡς καὶ ὁ χοῖρος ὅταν τρώγει τὸν κύριον οὐκ οἶδεν, ὅταν δὲ πεινᾷ κραυγάζει, καὶ λαβὼν πάλιν σιωπᾷΟὐδὲ φάγῃ τὸν ἀετὸν οὐδὲ τὸν ὀξύπτερον οὐδὲ τὸν ἰκτῖνα οὐδὲ τὸν κόρακα· οὐ μή, φησίν, κολληθήσῃ οὐδὲ ὁμοιωθήσῃ ἀνθρώποις τοιούτοις, οἵτινες οὐκ οἴδασιν διὰ κόπου καὶ ἰδρῶτος προίζειν ἑαυτοῖς τὴν τροφήν, ἀλλὰ ἁρπάζουσιν ὡς καὶ τὰ ἀλλότρια ἐν ἀνομίᾳ αὐτῶν καὶ ἐπιτηροῦσιν ὡς ἐν ἀκεραιοσύνῃ περιπατοῦντες καὶ περιβλέπονται, τίνα ἐκδύσωσιν διὰ τὴν πλεονεξίαν, ὡς καὶ τὰ ὄρνεα ταῦτα μόνα ἑαυτοῖς οὐ προίζει τὴν τροφήν, ἀλλὰ ἀργὰ καθήμενα ἐκζητεῖ, πῶς ἀλλοτρίας σάρκας καταφάγῃ, ὄντα λοιμὰ τῇ πονηρίᾳ αὐτῶνΚαὶ οὐ φάγῃ, φησίν, σμύραιναν οὐδὲ πολύποδα οὐδὲ σηπίαν· οὐ μή, φησίν, ὁμοιωθήσῃ κολλώμενος ἀντρώποις τοιούτοις, οἵτινες εἰς τέλος εἰσὶν ἀσεβεῖς καὶ κεκριμένοι ἤδη τῷ θανάτῳ, ὡς καὶ ταῦτα τὰ ἰχθύδια μόνα ἐπικατάρατα ἐν τῷ βυθῷ νήχεται, μὴ κολθμβῶντα ὡς τὰ λοιπά, ἀλλ’ ἐν τῇ γῇ κάτω τοῦ βυτοῦ κατοικεῖἀλλὰ καὶ τὸν δασύποδα οὐ φάγῃπρὸς τί; οὐ μὴ γένῃ, φησίν, παιδοφθόρος οὐδὲ ὁμοιωθήσῃ τοῖς τοιούτοις, ὅτι ὁ λαγωὸς κατ’ ἐνιαυτὸν πλεονεκτεῖ τὴν ἀφόδευσιν· ὅσα γὰρ ἔτη ζῇ, τοσαύτας ἔχει τρύπαςἀλλὰ οὐδὲ τὴν ὕαιναν φάγῃ· οὐ μή, φησίν, γένῃ μοιχὸς οὐδὲ φθορεὺς οὐδὲ ὁμοιωθήσῃ τοῖς τοιούτοιςπρὸς τί; ὅτι τὸ ζῷον τοῦτο παρ’ ἐνιαυτὸν ἀλλάσσει τὴν φύσιν καὶ ποτὲ μὲν ἄρρεν, ποτὲ δὲ θῆλυ γίνεταιἀλλὰ καὶ τὴν γαλῆν ἐμίσησεν καλῶςοὐ μή, φησίν, γενηθῃς τοιοῦτος, οἵους ἀκούομεν ἀνομίαν ποιούσαις ἐν τῷ στόματιτὸ γὰρ ζῷον τοῦτο τῷ στόματι κύειπερὶ μὲν τῶν βρωμάτων λαβὼν Μωϋσῆς τρία δόγματα οὕτως ἐν πνεύματι ἐλάλησεν· οἱ δὲ κατ’ ἐπιθυμίαν τῆς σαρκὸς ὡς περὶ βρώσεως προσεδέξαντολαμβάνει δὲ τῶν αὐτῶν τριῶν δογμάτων γνῶσιν Δαυείδ καὶ λέγει· Μακάριος ἀνήρ, ὃς οὐκ ἐπορεύθη ἐν βουλῇ ἀσεβῶν, καθὼς καὶ οἱ ἰχθύες πορεύονται ἐν σκότει εἰς τὰ βάθη· καὶ ἐν ὁδῷ ἁμαρτωλῶν οὐκ ἔστη, καθὼς οἱ δοκοῦντες φοβεῖσθαι τὸν κύριον ἁμαρτάνουσιν ὡς ὁ χοῖρος, καὶ ἐπὶ καθέδραν λοιμῶν οὐκ ἐκάθισεν, καθὼς τὰ πετεινα καθήμενα εἰς ἁρπαγήνἔχετε τελείως καὶ περὶ τῆς βρώσεωςπάλιν λέγει Μωϋσῆς· Φάγεσθε πᾶν διχηλοῦν καὶ μαρυκώμενοντί λέγει; ὅτι τὴν τροφὴν λαμβάνων οἶδεν τὸν τρέφοντα αὐτὸν καὶ ἐπ’ αὐτῷ ἀναπαθόμενος εὐφραίνεσθαι δοκεῖκαλῶς εἶπεν βλέπων τὴν ἐντολήντί οὖν λέγει; κολλᾶσθε μετὰ τῶν φοβουμένων τὸν κύριον, μετὰ τῶν μελετώντων ὃ ἔλαβον διάσταλμα ῥήματος ἐν τῇ καρδίᾳ, μετὰ τῶν λαλούντων τὰ διδαιώματα κυρίου καὶ τηρούντων, μετὰ τῶν εἰδότων, ὅτι ἡ μελέτη ἐστὶν ἔργον εὐφροσύνης, καὶ ἀναμαρυκωμένων τὸν λόγον κυρίουτί δὲ τὸ διχηλοῦν; ὅτι ὁ δίκαιος καὶ ἐν τούτῳ τῷ κόσνῳ περιπατεῖ καὶ τὸν ἅγιον αἰῶνα ἐκδέχεταιβλέπετε, πῶς ἐνομοθέτησεν Μωϋσῆς καλῶςἀλλὰ πόθεν ἐκείνοις ταῦτα νοῆσαι ἢ συνιέναι; ἡμεῖς δὲ δικαίως νοήσαντες τὰς ἐντολὰς λαλοῦμεν, ὡς ἠθέλησεν ὁ κύριοςδιὰ τοῦτο περιέτεμεν τὰς ἀκοὰς ἡμῶν καὶ τὰς καρδίας, ἵνα συνιῶμεν ταῦτα.
Ζητήσωμεν δέ, εἰ ἐμέλησεν τῷ κυρίῳ προφανερῶσαι περι τοῦ ὕδατος καὶ περὶ τοῦ σταυροῦπερὶ μὲν τοῦ ὕδατος γέγραπται ἐπὶ τὸν Ἰσραήλπῶς τὸ βάπτισμα τὸ φέρον ἄφεσιν ἁμαρτιῶν οὐ μὴ προσδέξονται, ἀλλ’ ἑαυτοῖς οἰκοδομήσουσινλέγει γὰρ ὁ προφήτης· Ἔστηθι οὐρανέ, καὶ ἐπὶ τούτῳ πλεῖον φριξάτω ἡ γῆ, ὅτι δύο καὶ πονηρὰ ἐποίησεν ὁ λαὸς οὗτος· ἐμὲ ἐγκατέλιπον, πηγὴν ζωῆς, καὶ ἑαυτοῖς ὤρυξαν βόθρον θανάτουΜὴ πέτρα ἔρημός ἐστιν τὸ ὄρος το ἅγιόν μου Σινᾶ; ἔσεσθε γὰρ ὡς πετεινοῦ νοσσιᾶς ἀφῃρημενοικαὶ πάλιν λέγει ὁ προφήτης· Ἐγὼ πορεύσομαι ἔμπροσθέν σου καὶ ὄρη ὁμαλιῶ καὶ πύλας χαλκᾶς συντρίψω καὶ μοχλοὺς σιδηροῦς συγκλάσω, καὶ δώσω σοι θησαυροὺς σκοτεινούς, ἀποκρύους, ἀοράτους, ἵνα γνῶσιν ὅτι ἐγὼ κύριος ὁ θεοςκαί· Κατοικήσεις ἐν ὑψηλῷ σπηλαίῳ πέτρας ἰσχυρᾶςκαί· τὸ ὕδωρ αὐτοῦ πιστόν· βασιλέα μετὰ δόξης ὄψεσθε, καὶ ἡ ψυχὴ ὑμῶν μελετήσει φόβον κυρίουκαὶ πάλιν ἐν ἀλλῳ προφήτῃ λέγει· Και ἔσται ὁ ταῦτα ποιῶν ὡς τὸ ξύλον τὸ πεφυτευμένον παρὰ τὰς διεξόδους τῶν ὑδάτων, ὁ τὸν καρπὸν αὐτοῦ οὐκ απορυήσεται, καὶ πάντα, ὅσα ἂν ποιῇ, κατευοδωθήσεταιοὐχ οὕτως οἱ ἀσεβεῖς, οὐχ οὕτως, αλλ’ ἢ ὡς ὁ χνοῦς, ὃν ἐκρίπτει ὁ ἄνεμος ἀπὸ προσώπου τῆς γῆςδιὰ τοῦτο οὐκ ἀναστήσονται ἀσεβεῖς ἐν κρίσει οὐδὲ ἁμαρτωλοὶ ἐν βουλῇ δικαίων, ὅτι γινώσκει κύριος ὁδὸν δικαίων, καὶ ὁδὸς ἀσεβῶν απολεῖταιαἰσθάνεσθε, πῶς τὸ ὑδωρ καὶ τὸν σταυρὸν ἐπὶ τὸ αὐτὸ ὥρισεντοῦτο γὰρ λέγει· μακάριοι, οἳ ἐπὶ τὸν σταυρὸν ἐλπίσαντες κατέβησαν εἰς τὸ ὕδωρ, ὅτι τὸν μὲν μισθὸν λέγει ἐν καιρῷ αὐτοῦ· τότε, φησίν ἀποδώσωνῦν δὲ ὃ λέγει· τὰ φύλλα οὐκ ἀπορυήσεται, τοῦτο λέγει· ὅτι πᾶν ῥῆμα, ὃ ἐὰν ἐξελεύσεται ἐξ ὑμῶν διὰ τοῦ στόματος ὑμῶν ἐν πίστει καὶ ἀγάπῃ, ἔσται εἰς ἐπιστροφὴν καὶ ἐλπίδα πολλοῖςκαὶ πάλιν ἕτερος προφήτης λέγειΚαὶ ἦν ἡ γῆ τοῦ Ἰακὼβ ἐπαινουμένη παρὰ πᾶσαν τὴν γῆντοῦτο λέγει· τὸ σκεῦος τοῦ πνεύματος αὐτοῦ δοξάζειεἶτα τί λέγει; Καὶ ἦν ποταμὸς ἕλκων ἐκ δεξιῶν, καὶ ἀνέβαινεν ἐξ αὐτοῦ δένδρα ὡραῖα· καὶ ὃς ἂν φάγῃ ἐξ αὐτῶν, ζήσεται εἰς τὸν αἰῶνατοῦτο λέγει ὅτι ἡμεῖς μὲν καταβαίνομεν εἰς τὸ ὕδωρ γέμοντες ἁμαρτιῶν καὶ ῥύπου, καὶ ἀνα βαίνομεν καρποφοροῦντες ἐν τῇ πνεύματι ἔχοντεςΚαὶ ὃς ἂν φάγῃ ἀπὸ τούτων, ζήσεται εἰς τὸν αἰῶνα, τοῦτο λέγει· ὃς ἂν, φησίν, ἀκούσῃ τούτων λαλουμένων καὶ πιστεύσῃ, ζήσεται εἰς τὸν αἰῶνα.
Ὁμοίως πάλιν περὶ τοῦ σταυροῦ ὁρίζει ἐν ἄλλῳ προφήτῃ λέγοντι· Καὶ πότε ταῦτα συντελεσθήται; λέγει κύριος· ὁταν ξύλου αἷμα στάξῃἔχεις πάλιν περὶ τοῦ σταυροῦ καὶ τοῦ σταυροῦσθαι μέλλοντοςλέγει δὲ πάλιν τῷ Μωϋσῇ, πολεμουμένου τοῦ Ἰσραὴλ ὑπὸ τῶν ἀλλοφύλων, καὶ ἵνα ὑπομνήσῃ αὐτοὺς πολεμουμένους, ὅτι διὰ τὰς ἁμαρτίας αὐτῶν παρεδόθησαν εἰς θάνατον· λέγει εἰς τὴν καρδίαν Μωϋσέως τὸ πνεῦμα, ἵνα ποιήσῃ τύπον σταυροῦ καὶ τοῦ μέλλοντος πάσχειν, ὅτι, ἐὰν μή, φησίν, ἐλπίσωσιν ἐπ’ αὐτῷ εἰς τὸν αἰῶνα πολεμηθήσονταιτίθησιν οὖν Μωϋσῆς ἓν ἐφ’ ὅπλον ἐν μέσῳ τῆς πυγμῆς, καὶ ὑψηλότερος σταθεὶς πάντων ἐξέτεινεν τὰς χεῖρας, καὶ οὕτως πάλιν ἐνίκα ὁ Ἰσραήλεἶτα, ὁπόταν καθεῖλεν, ἐθανατοῦντοπρὸς τί; ἱνα γνῶσιν ὅτι οὐ δύναται σωθῆναι, ἐὰν μὴ ἐπ’ αὐτῷ ἐλπίσωσινκαὶ πάλιν ἐν ἑτέρῳ προφήτῃ λέγει· Ὅλην τὴν ἡμέραν ἐξεπέτασα τὰς χεῖρας μου πρὸς λαὸν ἀπειθῆ καὶ ἀντιλέγοντα ὁδῷ δικαίᾳ μουπάλιν Μωϋσῆς ποιεῖ τύον τοῦ Ἰησοῦ, ὅτι δεῖ αὐτὸν παθεῖν, καὶ αὐτὸς ζωοποιήσει, ὃν δόξουσιν ἀπολωλεκέναι, ἐν σημείῳ πίπτοντος τοῦ Ἰσραήλ, (ἐποίησεν γὰρ κύριος πάντα ὄφιν δάκνειν αὐτούς, καὶ ἀπέθνησκον ἐπειδὴ ἡ παράβασις διὰ τὴν παράβασιν αὐτῶν εἰς θλῖψιν θανάτου παραδοθήσονταιπέρας γέ τοι αὐτὸς Μωϋσῆς ἐντειλάμενος· Οὐκ ἔσται ὑμῖν, αὐτὸς ποιεῖ, ἵνα τύπον τοῦ Ἰησοῦ δείξῃποιεῖ οὖν Μωϋσῆς χαλκοῦν ὄφιν καὶ τίθησιν ἐνδόξως καὶ κηρύγματι καλεῖ τὸν λαόνἐλθόντες οὖν ἐπὶ τὸ αὐτὸ ἐδέοντο Μωϋσέως, ἵνα περὶ αὐτῶν ἀνενέγκῃ δέησιν περὶ τῆς ἰάσεως αὐτῶνεἶπεν δὲ πρὸς αὐτοὺς Μωϋσῆς· Ὅταν, φησίν δηχθῇ τις ὑμῶν, ἐλθέτω ἐπὶ τὸν ὄφιν τὸν ἐπὶ τοῦ ξύλου ἐπικείμενον καὶ ἐλπισάτω πιστεύσας, ὅτι αὐτὸς ὢν νεκρὸς δύναται ζωοποιῆσαι, και παραχρῆμα σωθήσεται ἐν οὕτως ἐποίουνἔχεις´πάλιν καὶ ἐν τούτοις τὴν δόξαν τοῦ Ἰησοῦ, ὅτι ἐν αὑτῷ πάντα καὶ εἰς αὐτόντί λέγει πάλιν Μωϋσῆς Ἰησοῦ, υἱῷ Ναυή, ἐπιθεὶς αὐτῳ τοῦτο τὸ ὄνομα, ὄντι προφήτῃ, ἵνα μόνον ἀκούσῃ πᾶς ὁ λαός; ὅτι πάντα ὁ πατὴρ φανεροῖ περὶ τοῦ υἱοῦ Ἰησοῦλέγει οὖν Μωϋσῆς Ἰησοῦ, υἱῷ Ναυή, ἐπιθεὶς τοῦτο τὸ ὄνομα, ὁπότε ἔπεμψεν αὐτὸν κατάσκοπον τῆς γῆς· Λαβε βιβλίον εἰς τὰς χεῖράς σου καὶ γράψον, ἃ λέγει κύριος, ὅτι ἐκκόψει ἐκ ῥιζῶν τὸν οἶκον πάντα τοῦ Ἀμαλὴκ ὁ υἱὸς τοῦ θεοῦ ἐπ’ ἐσχάτων τῶν ἡμερῶνἴδε πάλιν Ἰησοῦς, οὐχὶ υἱὸς ἀνθρωπου, ἀλλὰ υἱὸς τοῦ θεοῦ, τύπῳ δὲ ἐν σαρκὶ φανερωθείςἐπεὶ οὖν μέλλουσιν λέγειν, ὅτι Χριστὸς υἱὸς Δαυείδ ἐστιν, αὐτὸς προφητέει Δαυείδ, φοβούμενος καὶ συνίων τὴν πλάνην τῶν ἁμαρτωλῶν· Εἶπεν κύριος τῷ κυρίῳ μου· Κάθου ἐκ δεξιῶν μου, ἕως ἂν θῶ τοὺς ἐχθρούς σου ὑποπόδιον τῶν ποδῶν σουκαὶ πάλιν λέγει οὕτως Ἡσαΐας· Εἶπεν κύριος τῷ Χριστῷ μου κυρίῳ, οὗ εκράτησα τῆς δεξιᾶς αὐτοῦ, ἐπακοῦσαι ἔμπροσθεν αὐτοῦ ἔθνη, καὶ ἰσχὺν βασιλέων διαρρήξωἴδε, πῶς Δαυεὶδ λέγει αὐτὸν κύριον, καὶ υἱὸν οὐ λέγει.
Ἴδωμεν δὲ εἰ οὗτος ὁ λαὸς κληρονομεῖ ἢ ὁ πρῶτος, καὶ εἰ ἡ διαθήκη εἰς ἡμᾶς ἢ εἰς ἐκείνουςἀκούσατε οὖν περὶ τοῦ λαοῦ τί λέγει ἡ γραπή· Ἐδεῖτο δὲ Ἰσαὰκ περὶ ῾Ρεβέκκας τῆς γυναικὸς αὐτοῦ, ὅτι στεῖρα ἦν· καὶ συνέλαβενεἶτα ἐξῆλθεν ῾Ρεβέκκα πυθέσθαι παρὰ κυρίου, καὶ εἶπεν κύριος πρὸς αὐτήν· Δύο ἔθνη ἐν τῇ γαστρί σου καὶ δύο λαοὶ ἐν τῇ κοιλίᾳ σου, καὶ ὑπερέξει λαὸς λαοῦ καὶ ὁ μείζων δουλεύσει τῷ ἐλάσσονιαἰσθάνεσθαι ὀφείλετε, τίς ὁ Ἰσαὰκ καὶ τίς ἡ ῾Ρεβέκκα, καὶ ἐπὶ τίνων δέδειχεν, ὅτι μείζων ὁ λαὸς οὗτος ἢ ἐκεῖνοςκαὶ ἐν ἄλλῃ προφητείᾳ λέγει φανερώτερον ὁ Ἰακὼβ πρὸς Ἰωσὴφ τὸν υἱὸν αὐτοῦ, λέγων· Ἰδού, οὐκ ἐστέρησέν με κύριος τοῦ προσώπου σου· προσάγαγέ μοι τοὺς υἱούς σου, ἵνα εὐλογήσω αὐτούςκαὶ προσήγαγεν Ἐφραὶμ καὶ Μανασσῆ, τὸν Μανασσῆ θέλων ἵνα εὐλογηθῇ, ὅτι πρεσβύτερος ἦν· ὁ γὰρ Ἰωσὴφ προσήγαγεν εἰς τὴν δεξιὰν χεῖρα τοῦ πατρὸς Ἰακώβεἶδεν δὲ Ἰακὼβ τύπον τῷ πνεύματι τοῦ λαοῦ τοῦ μεταξύ· καὶ τί λέγει; Καὶ ἐποίησεν Ἰακὼβ ἐναλλὰξ τὰς χεῖρας αὐτοῦ καὶ ἐπέθηκεν τὴν δεξιὰν ἐπὶ τὴν κεφαλὴν Ἐφραίμ, τοῦ δευτέρου καὶ νεωτέρου, καὶ εὐλόγησεν αὐτόνκαὶ εἶπεν Ἰωσὴφ πρὸς Ἰακώβ· Μετάθες σου τὴν δεξιὰν ἐπὶ τὴν κεφαλὴν Μανασσῆ, ὅτι πρωτότοκός μου υἱός ἐστινκαὶ εἶπεν Ἰακὼβ πρὸς Ἰωσήφ· Οἶδα, τέκνον, οἶδα· ἀλλ’ ὁ μειζων δουλεύσει τῷ ἐλάσσονι, καὶ οὗτος δὲ εὐλογηθήσεται βλέπετε, ἐπὶ τίνων τέθεικεν, τὸν λαὸν τοῦτον εἶναι πρῶτον καὶ τῆς διαθήκης κληρονόμονεἰ οὖν ἔτι καὶ διὰ τοῦ Ἀβραὰμ ἐμνήσθη, ἀπέχομεν τὸ τέλειον τῆς γνώσεως ἡμῶντί οὖν λέγει τῷ Ἀβραάμ, ὅτε μόνος πιστεύσας ἐτέθη εἰς δικαιοσύνην; Ἰδού, τέθεικά σε, Ἀβραάμ, πατέρα ἐθνῶν τῶν πιστευόντων δι’ ἀκροβυστίας τῷ θεῷ.
Ναίἀλλὰ ἴδωμεν, εἰ ἡ διαθήκη, ἣν ὤμοσεν τοῖς πατράσιν δοῦναι τῷ λαῷ, εἰ δέδωκενδέδωκεν· αὐτοὶ δὲ οὐκ ἐγένοντο ἄξιοι λαβεῖν διὰ τὰς ἁμαρτίας αὐτῶνλέγει γὰρ ὁ προφήτης· Καὶ ἦν Μωϋσῆς νηστεύων ἐν ὄρει Σινᾶ, τοῦ λαβεῖν τὴν διαθήκην κυρίου πρὸς τὸν λαόν, ἡμέρας τεσσεράκοντα καὶ νύκτας τεσσεράκοντακαὶ ἔλαβεν Μωϋσῆς παρὰ κυρίου τὰς δύο πλάκας τὰς γεγραμμένας τῷ δακτύλῳ τῆς χειρὸς κυρίου ἐν πνεύματι· καὶ λαβὼν Μωϋσης κατὲφερεν πρὸς τὸν λαὸν δοῦναικαὶ εἶπεν κύριος πρὸς ΜωϋσηνΜωϋσῆ Μωϋσῆ, κατάβηθι τὸ τάχος, ὅτι ὁ λαός σου, ὃν ἐξήγαγες ἐκ γῆς Αἰγύτου, ἠνόμησενκαὶ συνῆκεν Μωϋσῆς, ὅτι ἐποίησαν ἑαυτοῖς πάλιν χωνέματα, καὶ ἔρριψεν ἐκ τῶν χειρῶν, καὶ συνετρίβησαν αἱ πλάκες τῆς διαθήκης κυρίουΜωϋσῆς θεράπων ὢν ἔλαβεν, αὐτὸς δὲ κύριος ἡμῖν ἔδωκεν εἰς λαὸν κληρονομίας, δι’ ἡμᾶς ὑπομείναςἐφανερώθη δέ, ἵνα κἀκεῖνοι τελειωθῶσιν τοῖς ἁμαρτήμασιν, καὶ ἡμεῖς διὰ τοῦ κληρονομοῦντος διαθήκην κυρίου Ἰησοῦ λάβωμεν, ὃς εἰς τοῦτο ἡτοιμάσθη, ἵνα ἀυτὸς φανείς, τὰς ἤδη δεδαπανημένας ἡμῶν καρδιας τῷ θανάτῳ καὶ παραδεδομένας τῇ τῆς πλάνης ἀνομίᾳ λυτρωσάμενος ἐκ τοῦ σκότους, διάθηται ἐν ἡμῖν διαθήκην λόγῳγέτραπται γάρ, πῶς αὐτῳ ὁ πατὴρ ἐντέλλεται, λυτρωσάμενον ἡμᾶς ἐκ τοῦ σκότους ἑτοιμάσαι ἑαυτῷ λαὸν ἅγιονλέγει οὖν ὁ προφήτης· Ἐγὼ κύριος, ὁ θεός σου, ἐκάλεσά σε ἐν δικαιοσύνῃ καὶ κρατήσω τῆς χειρός σου καὶ ἐνισχύσω σε, καὶ ἔδωκά σε εἰς διαθήκην γένους, εἰς φῶς ἐθνῶν ἀνοῖξαι ὀφθαλμοὺς τυφλῶν καὶ ἐξαγαγεῖν ἐκ δεσμῶν πεπεδημένους καὶ ἐξ οἴκου φυλακῆς καθημένους ἐν σκότειγινώσκομεν οὖν, πόθεν ἐλυτρώθημενπάλιν ὁ προφήτης λέγει· Ἰδού, τέθεικά σε εἰς φῶς ἐθνῶν, τοῦ εἶναί σε εἰς σωτηρίαν ἕως ἐσχάτου τῆς γῆς, οὕτως λέγει κύριος ὁ λυτρωσάμενός σε θεόςκαὶ πάλιν ὁ προφήτης λέγει· Πνεῦμα κυρίου ἐπ’ ἐμέ, οὗ εἵνεκεν ἔχρισέν με εὐαγγελίσασθαι ταπεινοῖς χάριν, ἀπέσταλκέν με ἰάσασθαι τοὺς συντετριμμένους τὴν καρδίαν, κηρῦξαι αἰχμαλώτοις ἄφεσιν καὶ τυφλοῖς ἀνάβλεψιν, καλέσαι ἐνιαυτὸν κυρίου δεκτὸν καὶ ἡμέραν ἀνταποσόσεως, παρακαλέσαι πάντας τοὺς πενθοῦντας.
Ἔτι οὖν καὶ περὶ τοῦ σαββάτου γέγραπται ἐν τοῖς δέκα λόγοις, ἐν οἷς ἐλάλησεν ἐν τῷ ὄρει Σινᾶ πρὸς Μωϋσῆν κατὰ πρόσωπον· Καὶ ἁγιασατε τὸ σάββατον κυρίου χερσὶν καθαραῖς καὶ καρδίᾳ καθαρᾷκαὶ ἐν ἑτέρῳ λέγει· Ἐὰν φυλάξωσιν οἱ υἱοί μου τὸ σάββατον, τότε ἐπιθήσω τὸ ἔλεός μου ἐπ’ αὐτούςτὸ σάββατον λέγει ἐν ἀρχῇ τῆς κτίσεως· Καὶ ἐποίησεν ὁ θεὸς ἐν ἓξ ἡμέραις τὰ ἔργα τῶν χειρῶν αὐτοῦ, καὶ συνετέλεσεν ἐν τῇ ἡμέρᾳ τῇ ἑβδόμῃ καὶ κατέπαυσεν ἐν αὐτῇ καὶ ἡγίασεν αὐτήνπροσέχετε, τέκνα, τί λέγει τὸ συνετέλεσεν ἐν ἓξ ἡμέραιςτοῦτο λέγει, ὅτι ἐν ἑξακισχιλίοις ἔτεσιν συντελέσει κύριος τὰ σύμπαντα· ἡ γὰρ ἡμέρα παρ’ αὐτῷ σημαίνει χίλια ἔτηαὐτὸς δέ μοι μαρτυρεῖ λέγων· Ἰδού, ἡμέρα κυρίου ἔσται ὡς χίλια ἔτηοὐκοῦν, τέκνα, ἐν ἓξ ἡμέραις, ἐν τοῖς ἑξακισχιλίοις ἔτεσιν συντελεσθήσεται τὰ σύμπανταΚαὶ κατέπαυσεν τῇ ἡμέρᾳ τῇ ἑβδόμῃτοῦτο λέγει· ὅταν ἐλτὼν ὁ υἱὸς αὐτοῦ καταργήσει τὸν καιρὸν τοῦ ἀνόμου καὶ κρινεῖ τοὺς ἀσεβεῖς καὶ ἀλλάξει τὸν ἥλιον καὶ τὴν σελήνην καὶ τοὺς ἀστέρας, τότε καλῶς καταπαύσεται ἐν τῇ ἡμέρᾳ τῇ ἑβδόμῃπέρας γέ τοι λέγει· Ἁγιάσεις αὐτὴν χερσὶν καθαραῖς καὶ καρδίᾳ καθαρᾷεἰ οὖν ἣν ὁ θεὸς ἡμέραν ἡγίασεν νῦν τις δύναται ἁγιάσαι καθαρὸς ὢν τῇ καρδίᾳ, ἐν πᾶσιν πεπλανήμεθαἴδε ὅτι ἄρα τότε καλῶς καταπαυόμενοι ἁγιάσομεν αὐτήν, ὅτε δυνησόμεθα αὐτοὶ δικαιωθέντες καὶ ἀπολαβόντες τὴν ἐπαγγελίαν, μηκέτι οὔσης τῆς ἀνομίας, καινῶν δὲ γεγονότων πάντων ὑπὸ κυρίου· τότε δυνησόμεθα αὐτὴν ἁγιάσαι, αὐτοὶ ἁγιασθέντες πρῶτονπέρας γέ τοι λέγει αὐτοῖς· Τὰς νεομηνίας ὑμῶν καὶ τὰ σάββατα οὐκ ἀνέχομαιὁρᾶτε, πῶς λέγει; οὐ τὰ σάββατα ἐμοὶ δεκτά, ἀλλἃ ὃ πεποίηκα, ἐν ᾧ καταπαύσας τὰ πάντα ἀρχὴν ἡμέρας ὀγδόης ποιήσω, ὅ ἐστιν ἄλλου κόσμου ἀρχήνδιὸ καὶ ἄγομεν τὴν ἡμέραν τὴν ὀγόην εἰς εὐφροσύνην, ἐν ᾗ καὶ ὁ Ἰησοῦς ἀνέστη ἐκ νεκρῶν καὶ φανερωθεὶς ἀνέβη εἰς οὐρανοίς
Ἔτι δὲ καὶ περὶ τοῦ ναοῦ ἐρῶ ὑμῖν, ὡς πλανώμενος οἱ ταλαίπωροι εἰς τὴν οἰκοδομὴν ἤλπισαν, καὶ οὐκ ἐπὶ τὸν θεὸν αὐτῶν τὸν ποιήσαντα αὐτούς, ὡς ὄντα οἶκον θεοῦσχεδὸν γὰρ ὡς τὰ ἔθνη ἀφιέρωσαν αὐτὸν ἐν τῷ ναῷἀλλὰ πῶς λέγει κύριος καταργῶν αὐτὸν, μάθετε· Τίς ἐμέτρησεν τὸν οὐρανὸν σπιθαμῇ ἢ τὴν γῆν δρακί; οὐκ ἐγώ; λέγει κύριος· Ὁ οὐρανός μοι θρόνος, ἡ δὲ γῆ ὑποπόδιον τῶν ποδῶν μου· ποῖον οἶκον οἰκοδομήσετέ μοι, ἢ τίς τόπος τῆς καταπαύσεώς μου; ἐγνώκατε, ὅτι ματαία ἡ ἐλπὶς αὐτῶνπέρας γέ τοι πάλιν λέγει· Ἰδού, οἱ καθελόντες τὸν ναὸν τοῦτον αὐτοὶ αὐτὸν οἰκοδομήσουσινγίνεταιδιὰ γὰρ τὸ πολεμεῖν αὐτοὺς καθῃρέθη ὑπὸ τῶν ἐχτρῶν· νῦν καὶ αὐτοι οἱ τῶν ἐχθρῶν ὑπηρέται ἀνοικοδομήσουσιν καὶ ὁ λαὸς Ἰσραὴλ παραδίδοσθαι, ἐφανερώθηλέγει γὰρ ἡ γραφή· Καὶ κύριος τὰ πρόβατα τῆς νομῆς καὶ παραδώσει καὶ τὸν πύργον αὐτῶν εἰς καταφθοράνκαὶ ἐγενετο καθ’ ἃ ἐλάλησεν κύριοςζητήσωμεν δέ, εἰ ἔστιν ναὸς θεοῦἔστιν, ὅπου αὐτὸς λέγει ποιεῖν καὶ καταρτίζεινγέγραπται γάρ· Καὶ ἔσται, τῆς ἑβδομάδος συντελουμένης οἰκοδομηθήσεται ναὸς θεοῦ ἐνδόξως ἐπὶ τῷ ὀνόματι κυρίουεὑρίσκω οὖν, ὅτι ἔτιν ναόςπῶς οὖν οἰκοδομηθήσεται ἐπὶ τῷ ὀνόματι κυρίου, μάθετεπρὸ τοῦ ἡμας πιστεῦσαι τῷ θεῷ ἦν ἡμῶν τὸ κατοικητήριον τῆς καρδίας φθαρτὸν καὶ ἀσθενές, ὡς ἀληθῶς οἰκοδομητὸς ναὸς διὰ χειρός, ὅτι ἦν πλήρης μὲν εἰδωλολοτρείας καὶ ἦν οἶκος δαιμονίων διὰ τὸ ποιεῖν, ὅσα ἦν ἐναντία τῷ θεῷΟἰκοδομηθήσετα δὲ ἐπὶ τῷ ὀνόματι κυρίουπροσέχετε δέ, ἵνα ὁ ναὸς τοῦ κυρίου ἐνδόξως οἰκοδομηθῇπῶς, μάθετελαβόντες τὴν ἄφεσιν τῶν ἁμαρτιῶν καὶ ἐλπίσαντες ἐπὶ τὸ ὄνομα ἐγενόμεθα καινοί,´πάλιν ἐξ ἀρχῆς κτιζόμενοι· διὸ ἐν τῷ κατοικητηρίῳ ἡμῶν ἀληθῶς ὁ θεὸς κατοικεῖ ἐν ἡμῖνπῶς; ὁ λόγος αὐτοῦ τῆς πίστεως, ἡ κλῆσις αὐτοῦ τῆς ἐπαγγελίας, ἡ σοφία τῶν δικαιωμάτων, αἱ ἐντολαὶ τῆς διδαχῆς, αὐτὸς ἐν ἡμῖν προφητεύων, αὐτὸς ἐν ἡμῖν κατοικῶν, τοὺς τῷ θανάτῳ δεδουλωμένους ἀνοιγων ἡμῖν τὴν θύραν τοῦ ναοῦ, ὅ ἐστιν στόμα, μετάνοιαν διδοὺς ημῖν, εἰσάγε εἰσ τὸν ἄφθαρτον ναόνὁ γὰρ ποθῶν σωθῆναι βλέπει οὐκ εἰς τὸν ἄνθρωπον, ἀλλ’ εἰς τὸν ἐν αὐτῷ κατοικοῦντα καὶ λαλοῦντα, ἐπ’ αὐτῷ ἐκπλησσόμενος, ἐπὶ τῷ μηδέποτε μήτε τοῦ λέγοντος τὰ ῥήματα ἀκηκοέναι ἐκ τοῦ στόματος μήτε αὐτός ποτε ἐπιτεθυμηκέναι ἀκούειντοῦτό ἐστιν πνευματικὸς ναὸς οἰκοδομούμενος τῷ κυρίῳ.
Ἐφ’ ὅσον ἦν ἐν δυνατῷ καὶ ἁπλότητι δηλῶσαι ὑμῖν, ἐλπίζει μου ἡ ψυχὴ τῇ ἐπιθυμίᾳ μου μὴ παραλελοιπέναι τι τῶν ἀνηκόντων εἰς σωτηρίανἐὰν γὰρ περὶ τῶν ἐνεστώτων ἢ μελλόντων γράφω ὑμῖν, οὐ μὴ νοήσητε διὰ τὸ ἐν παραβολαῖς κεῖσθαιταῦτα μὲν οὕτως.
Μεταβῶμεν δὲ καὶ ἐπὶ ἑτέραν γνῶσιν καὶ διδαχήνὉδοὶ δύο εἰσὶν διδαχῆς καὶ ἐξουσίας, ἥ τε τοῦ φωτὸς καὶ ἡ τοῦ σκότουςδιαφορὰ δὲ πολλὴ τῶν δύο ὁδῶνἐφ’ ἧς μὲν γάρ εἰσιν τεταγμένοι φωταγωγοὶ ἄγγελοι τοῦ θεοῦ, ἐφ’ ἧς δὲ ἄγγελοι του σατανᾶκαὶ ὁ μέν ἐστιν κύριος ἀπὸ αἰώνων καὶ εἰς τοὺς αἰῶνας, ὁ δὲ ἄρχων καιροῦ τοῦ νῦν τῆς ἀνομίας.
Ἡ οὖν ὁδὸς τοῦ φωτός ἐστιν αὕτη· ἐάν τις θέλων ὁδὸν ὁδεύειν ἐπὶ τὸν ὡρισμένον τόπον, σπεύσῃ τοῖς ἔργοις αὐτοῦἔσιν οὖν ἡ δοθεῖσα ἡμῖν γνῶσις τοῦ περιπατεῖν ἐν αὐτῇ τοιαύτηἀγαπήσεις τὸν ποιήσαντά σε, φοβηθήσῃ τόν σε πλασαντα, δοξάσεις τόν σε λυτρωσάμενον ἐκ θανάτου· ἔσῃ ἁπλοῦς τῇ καρδίᾳ καὶ πλούσιος τῷ πνεύματι· οὐ κολληθήσῃ μετὰ τῶν πορευομένων ἐν ὁδῷ θανάτου, μισήσεις πᾶν, ὃ οὐκ ἔστιν ἀρεστὸν τῷ θεῷ, μισήσεις πᾶσαν ὑπόκρισιν· οὐ μὴ ἐγκαταλιπῃς ἐντολὰς κυρίουοὐχ ὑψώσεις σεαυτόν, ἔσῃ δὲ ταπεινόφρων κατὰ πάντα· οὐκ ἀρεῖς ἐπὶ σεαυτὸν δόξανοὐ λήμψῃ βουλὴν πονηρὰν κατὰ τοῦ πλησίον σου, οὐ δώσεις τῇ ψυχῇ σου θράσοςοὐ πορνεύσεις, οὐ μοιχεύσεις, οὐ παιδοφθορήσειςοὐ μή σου ὁ λόγος τοῦ θεοῦ ἐξέθῃ ἐν ἀκαθαρσίᾳ τινῶνοὐ λήμωψῃ πρόσωπον ἐλέγξαι τινὰ ἐπὶ παραπτώματιἔσῃ πραΰς, ἔσῃ ἡσύχιος, ἔσῃ τρέμων τοὺς λόγους οὓς ἤκουσας, οὐ μνησικακήσεις τῷ ἀδελφῷ σουοὐ μὴ διψυχήσῃς, πότερον ἔσται ἢ οὔοὐ μὴ λάβῃς ἐπὶ ματάῳ τὸ ὄνομα κυρίουἀγαπήσεις τὸν πλησίον σου ὑπὲρ τὴν ψυχήν σουοὐ φονεύσεῖς τέκνον ἐν φθορᾷ, οὐδὲ πάλιν γεννηθὲν ἀποκτενεῖςοὐ μὴ ἄρῃς τὴν χεῖρά σου ἀπὸ τοῦ υἱοῦ σου ἢ ἀπὸ τῆς θυγατρός σου, ἀλλὰ ἀπὸ νεότητος διδάξεις φόβον θεοῦοὐ μὴ γένῃ ἐπιθυμῶν τὰ τοῦ πλησίον σου, οὐ μὴ γένῃ πλεονέκτηςοὐδὲ κολληθήσῃ ἐκ ψυχῆς σου μετὰ ὑψηλῶν, ἀλλὰ μετὰ ταπεινῶν καὶ δικαίων ἀναστραφήσῃ, τὰ συμβαίνοντά σοι ἐνεργήματα ὡς ἀγαθὰ προσδέξῃ, εἰδώς ὅτι ἄνευ θεοῦ οὐδὲν γίνεταιοὐκ ἔσῃ διγνώμων οὐδὲ γλωσσώδης, ὑποταγήσῃ κυρίοις ὡς τύπῳ θεοῦ ἐν αἰσχύνῃ καὶ φόβῳ· οὐ μὴ ἐπιτάξῃς δούλῳ σου ἢ παιδίσκῃ ἐν πικρίᾳ, τοῖς ἐπὶ τὸν αὐτὸν θεὸν ἐλπίζουσιν, μή ποτε οὐ μὴ φοβηθήσονται τὸν ἐπ’ ἀμφοτέροις θεόν· ὅτι οὐκ ἦλθεν κατὰ πρόσωπον καλέσαι, ἀλλ’ ἐφ’ οὓς τὸ πνεῦμα ἡτοίμασενκοινωνήσεις ἐν πᾶσιν τῷ πλησίον σου καὶ οὐκ ἐρεῖς ἴδια εἶναι· εἰ γὰρ ἐν τῷ ἀφθάρτῳ κοινωνοί ἐστε, πόσῳ μᾶλλον ἐν τοῖς φθαρτοῖς; οὐκ ἔσῃ πρόγλωσσος· παγὶς γὰρ τὸ στόμα θανάτουὅσον δύνασαι, ὑπὲρ τῆς ψυχῆς σου ἁγνεύσειςμὴ γίνου πρὸς μὲν τὸ λαβεῖν ἐκτείνων τὰς χεῖρας, πρὸς μὲν τὸ λαβεῖν ἐκτείνων τὰς χεῖρας, πρὸς δὲ τὸ δοῦναι συσπῶνἀγαπήσεις ὡς κόρην τοῦ ὀφθαλμοῦ σου πάντα τὸν λαλοῦντά σοι τὸν λόγον κυρίουμνησθήσῃ ἡμέραν κρίσεως νυκτὸς και ἡμέρας, καὶ ἐκζητήσεις καθ’ ἑκάστην ἡμέραν τὰ πρόσωπα τῶν ἁγίων, ἢ διὰ λόγου κοπιῶν καὶ πορευόμενος εἰς λύτρωσιν ἁμαρτιῶν σουοὐ διστάσεις μισθοῦ καλὸς ἀνταποδότηςφυλάξεις ἃ παρέλαβες, μήτε προστιθεὶς μήτε ἀφαιρῶν, εἰς τέλος μισήσεις τὸ πονηρόνκρινεῖς δικαίωςοὐ ποιήσεις σχίσμα, εἰρηνεύσεις δὲ μαχομένους συαγαγώνἐξομολογήσῃ ἐπὶ ἁμαρτίαις σουοὐ προσήξεις ἐπὶ προσευχὴν ἐν συνειδήσει πονηρᾷαὑτη ἐστὶν ὁδὸς τοῦ φωτός.
Ἧ δὲ τοῦ μέλαος ὁδός ἐστιν σκολιὰ καὶ κατάρας μεστήὁδὸς ἐστιν θανάτου αἰωνίου μετὰ τιμωρίας, ἐν ᾗ ἐστιν τὰ ἀπολλύντα τὴ ψυχὴν αὐτῶν· εἰδωλολατρεία, θρασύτης, ὕψος δυνάμεως, ὑπόκρισις, διπλοκαρδία, μοιχεία, φόνος, ἁρπαγή, ὑπερηφανία, μαγεία, πλεονεξία, ἀφοβία θεοῦ· διῶκται τῶν ἀγαθῶν, μισοῦντες ἀλήθειαν, ἀγαπῶντες ψεῦδος, οὐ γινώσκοντες μισθὸν δικαιοσύνης, οὐ κολλώμενοι ἀγαθῷ, οὐ κρίσει δικαίᾳ, χήρᾳ καὶ ὀρφανῷ οὐ προσέχοντες, ἀγρυπνοῦντες μάταια, διώκοντες ἀνταπόδομα, οὐκ ἐλεῶντες πτωχόν, οὐ πονοῦντες ἐπὶ καταπονουμένῳ, εὐχερεῖς ἐν καταφονεῖς τέκνων, φθορεῖς πλάσματος θεοῦ, ἀποστρεφόμενοι τὸν ἐνδέμενον, καταπονοῦντες τὸν θλιβόμενον, πλουσίων παράκλητοι, πενήτων ἄνομοι κριταί, πανθαμάρτητοι.
Καλὸν οὖν ἐστὶν μαθόντα τὰ δικαιώματα τοῦ κυρίου, ὅσα γέγραπται, ἐν τούτοις περιπατεῖνὁ γὰρ ταῦτα ποιῶν ἐν τῇ βασιλείᾳ τοῦ θεοῦ δοξασθήσεται· ὁ ἐκεῖνα ἐκλεγόμενος μετὰ τῶν ἔργων αὐτοῦ συναπολεῖταιδιὰ τοῦτο ἀνάστασις, διὰ τοῦτο ἀνταπόδομαἐρωτῶ τοὺς ὑπερέχοντας, εἴ τινά μου γνώμης ἀγαθῆς λαμβάνετε συμβουλίαν· ἔχετε μεθ’ ἑαυτῶν εἰς οὓς ἐργάσησθε τὸ καλόν· μὴ ἐλλειπητεἐγγὺς ὁ κύριος καὶ ὁ μισθὸς αὐτοῦἔτι καὶ ἐρωτῶ ὑμᾶς· ἑαυτῶν γίνεσθε νομοθέται ἀγαθοί, ἑαυτῶν μένετε σύμβουλοι πιστοί, ἄρατε ἐξ ὑμῶν πᾶσαν ὑπόκρισινὁ δὲ θεός, ὁ τοῦ παντὸς κόσμου κυριεύων, δῴη ὑμῖν σοφίαν, σύνεσιν, ἐπιστήμην, γνῶσιν τῶν δικαιωμάτων αὐτοῦ, ὑπομονήνγίνεσθε δὲ θεοδίδακτοι, ἐκζητοῦντες τί ζητεῖ κύριος ἀφ’ ὑμῶν, καὶ ποιεῖτε ἵνα εὑρεθῆτε ἐν ἡμέρᾳ κρίσεωςεἰ δὲ τίς ἐστιν ἀγαθοῦ μνεία, μνημονεύετέ μου μελετῶντες ταῦτα, ἵνα καὶ ἡ ἐπιθυμία καὶ ἡ ἀγρυπνία εἴς τι ἀγαθὸν χωρήσῃἐρωτῶ ὑμᾶς, χάριν αἰτούμενοςἕως ἔτι τὸ καλὸν σκεῦός ἐστιν μεθ’ ὑμῶν, μὴ ἐλλείπητε μηδενὶ ἑαυτῶν, ἀλλὰ συνεχῶς ἐκζητεῖτε ταῦτα καὶ ἀναπληρουτε πᾶσαν ἐντολήν· ἔστιν γὰρ ἄξιαδιὸ μᾶλλον ἐσπούδασα γράψαι ἀφ’ ὧν ἠδυνήθην, εἰς τὸ εὐφρᾶναι ὑμᾶςσώζεσθε, ἀγάπης τέκνα καὶ εἰρήνηςὁ κύριος τῆς δόξης καὶ πάσης χάριτος μετὰ τοῦ πνεύματος ὑμῶν.
Ἐπιστολὴ βαρνάβα.
\section{Διδαχη κυριου δια των δωδεκα αποστολων τοις εθνεσιν}
Οδοι δυο εισι, μια της ζωης και μια του θανατου, διαφορα δε πολλη μεταξυ των δυο οδων.
η μεν ουν οδος της ζωης εστιν αυτη, πρωτον αγαπησεις τον θεον τον ποιησαντα σε, δευτερον τον πλησιον σου ως σεαυτον, παντα δε οσα εαν θελησης μη γινηεσθαι σοι, και συ αλλω μη ποιει.
τουτων δε των λογων η διδαχη εστιν αυτη, ευλογειτε τους καταρωμενους υμιν και προσευχεσθε υπερ των εχθρων υμων, νηστευετε δε υπερ των διωκοντων υμας, ποια γαρ χαρις, εαν αγαπατε τους αγαπωντας υμας· ουχι και τα εθνη το αυτο ποιουσιν, υμεις δε αγαπατε τους μισουντας υμας, και ουχ εξετε εχθρον.
απεχου των σαρκικων [και σωματικων] επιθυμιων. εαν τις σοι δω ραπισμα εις την δεξιαν σιαγονα, στρεψον αυτω και την αλλην, και εση τελειος. εαν αγγαρευση σε τις μιλιον εν, υπαγε μετ αυτου δυο, εαν αρη τις το ιματιον σου, δος αυτω και τον χιτωνα, εαν λαβη τις απο σου το σον, μη απαιτει, ουδε γαρ δυνασαι.
παντι τω αιτουντι σε διδου και μη απαιτει, πασι γαρ θελει διδοσθαι ο πατηρ εκ των ιδιων χαρισματων. μακαριος ο διδους κατα την εντολην, αθωος γαρ εστιν. ουαι τω λαμβανοντι, ει μεν γαρ χρειαν εχων λαμβανει τις, αθωος εσται, ο δε μη χρειαν εχων δωσει δικην, ινατι ελαβε και εις τι, εν συνοχη δε γενομενος εξετασθησεται περι ων επραξε και ουκ εξελευσεται εκειθεν, μεχρις ου αποδω τον εσχατον κοδραντην.
αλλα και περι τουτου δε ειρηται, Ιδρωσατω η ελεημοσυνη σου εις τας χειρας σου, μεχρις αν γνως τινι δως.
Δευτερα δε εντολη της διδαχης·
Ου φονευσεις, ου μοιχευσεις, ου παιδοφθορησεις, ου πορνευσεις, ου κλεψεις, ου μαγευσεις, ου φαρμακευσεις, ου φονευσεις τεκνον εν φθορα ουδε γεννηθεν αποκτενεις.
ουκ επιθυμησεις τα του πλησιον, ουκ επιορκησεις, ου ψευδομαρτυρησεις, ου κακολογησεις, ου μνησικακησεις.
ουκ εση διγνωμων ουδε διγλωσσος, παγις γαρ θανατου η διγλωσσια.
ουκ εσται ο λογος σου ψευδης, ου κενος, αλλα μεμεστωμενος πραξει.
ουκ εση πλεονεκτης ουδε αρπαξ ουδε υποκριτης ουδε κακοηθης ουδε υπερηφανος. ου ληψη βουλην πονηραν κατα του πλησιον σου.
ου μισησεις παντα ανθρωπον, αλλα ους μεν ελεγξεις, περι ων δε προσευξη, ους δε αγαπησεις υπερ την ψυχην σου.
Τεκνον μου, φευγε απο παντος πονηρου και απο παντος ομοιου αυτου.
μη γινου οργιλος, οδηγει γαρ η οργη προς τον φονον, μηδε ζηλωτης μηδε εριστικος μηδε θυμικος, εκ γαρ τουτων απαντων φονοι γεννωνται.
τεκνον μου, μη γινου επιθυμητης, οδηγει γαρ η επιθυμια προς την πορνειαν, μηδε αισχρολογος μηδε υψηλοφθαλμος, εκ γαρ τουτων απντων μοιχειαι γεννωνται.
τεκνον μου, μη γινου οιωνοσκοπος, επειδη οδηγει εις την ειδωλολατριαν, μηδε επαοιδος μηδε μαθηματικος μηδε περικαθαιρων, μηδε θελε αυτα βλεπειν [μηδε ακουειν], εκ γαρ τουτων απαντων ειδωλολατρια γενναται.
τεκνον μου, μη γινου ψευστης, επειδη οδηγει το ψευσμα εις την κλοπην, μηδε φιλαργυρος μηδε κενοδοξος, εκ γαρ τουτων απαντων κλοπαι γεννωνται.
τεκνον μου, μη γινου γογγυσος, επειδη οδηγει εις την βλασφημιαν, μηδε αυθαδης μηδε πονηροφρων, εκ γαρ τουτων απαντων βλασφημιαι γεννωνται.
ισθι δε πραυς, επει οι πραεις κληρονομησουσι την γην.
γινου μακροθυμος και ελεημων και ακακος και ησυχιος και αγαθος και τρεμων τους λογους δια παντος, ους ηκουσας.
ουχ υψωσεις σεαυτον ουδε δωσεις τη ψυχη σου θρασος. ου κολληθησεται η ψυχη σου μετα υφηλων, αλλα μετα δικαιων και ταπεινων αναστραφηση.
τα συμβαινοντα σοι ενεργηματα ως αγαθα προσδεξη, ειδως, οτι ατερ θεου ουδεν γινεται.
Τεκνον μου, του λαλουντος σοι τον λογον του θεου μνησθηση νυκτος και ημερας, τιμησεις δε αυτον ως κυριον, οθεν γαρ η κυριοτης λαλειται, εκει κυριος εστιν.
εκζητησεις δε καθ ημεραν τα προσωπα των αγιων, ινα επαναπαης τοις λογοις αυτων.
ου ποιησεις σχισμα, ειρηνευσεις δε μαχομενους, κρινεις δικαιως, ου ληψη προσωπον ελεγξαι επι παραπτωμασιν.
ου διψυχησεις, ποτερον εσται η ου.
μη γινου προς μεν το λαβειν εκτεινων τας χειρας, προς δε το δουναι συσπων.
εαν εχης δια των χειρων σου, δωσεις λυτρωσιν αμαρτιων σου.
ου διστασεις δουναι ουδε διδους γογγυσεις, γνωση γαρ, τις εστιν ο του μισθου καλος ανταποδοτης.
ουκ αποστραφηση τον ενδεομενον, συγκοινωνησεις δε παντα τω αδελφω σου και ουκ ερεις ιδια ειναι, ει γαρ εν τω αθανατω κοινωνοι εστε, ποσω μαλλον εν τοις θνητοις·
ουκ αρεις την χειρα σου απο του υιου σου η απο της θυγατρος σου, αλλα απο νεοτητος διδαξεις τον φοβον του θεου.
ουκ επιταξεις δουλω σου η παιδισκη, τοις επι τον αυτον θεον ελπιζουσιν, εν πικρια σου, μηποτε ου μη φοβηθησονται τον επ αμφοτεροις θεον, ου γαρ ερχεται κατα προσωπον καλεσαι, αλλ εφ ους το πνευμα ητοιμασεν.
υμεις δε [οι] δουλοι υποταγησεσθε τοις κυριοις υμων ως τυπω θεου εν αισχυνη και φοβω.
μισησεις πασαν υποκρισιν και παν ο μη αρεστον τω κυριω.
ου μη εγκαταλιπης εντολας κυριου, φυλαξεις δε α παρελαβες, μητε προστιθεις μητε αφαιρων.
εν εκκλεσια εξομολογηση τα παραπτωματα σου, και ου προσελευση επι προσευχην σου εν συνειδησει πονηρα. αυτη εστιν η οδος της ζωης.
Η δε του θανατου οδος εστιν αυτη, πρωτον παντων πονηρα εστι και καταρας μεστη, φονοι, μοιχειαι, επιθυμιαι, πορνειαι, κλοπαι, ειδωλολατριαι, μαγειαι, φαρμακιαι, αρπαγαι, ψευδομαρτυριαι, υποκρισεις, διπλοκαρδια, δολος, υπερηφανια, κακια, αυθαδεια, πλεονεξια, αισχρολογια, ζηλοτυπια, θρασυτης, υψος, αλαζονεια, [αφοβια].
διωκται αγαθων, μισουντες αληθειαν, αγαπωντες ψευδος, ου γινωσκοντες μισθον δικαιοσυνης, ου κολλωμενοι αγαθω ουδε κρισει δικαια, αγρυπνουντες ουκ εις το αγαθον, αλλ εις το πονηρον, ων μακραν πραυτης και υπομονη, ματαια αγαπωντες, διωκοντες ανταποδομα, ουκ ελεουντες πτωχον, ου πονουντες επι καταπονουμενω, ου γινωσκοντες τον ποιησαντα αυτους, φονεις τεκνων, φθορεις πλασματος θεου, αποστρεφομενοι τον ενδεομενον, καταπονουντες τον θλιβομενον, πλουσιων παρακλητοι, πενητων ανομοι κριται, πανθαμαρτητοι, ρυσθειητε, τεκνα, απο τουτων απαντων.
Ορα, μη τις σε πλανηση απο ταυτης της οδου της διδαχης, επει παρεκτος θεου σε διδασκει.
ει μεν γαρ δυνασαι βαστασαι ολον τον ζυγον του κυριου, τελειος εση, ει δ ου δυνασαι, ο δυνη, τουτο ποιει.
περι δε της βρωσεως, ο δυνασαι βαστασον, απο δε του ειδωλοθυτου λιαν προσεχε, λατρεια γαρ εστι θεων νεκρων.
Περι δε του βαπτισματος, ουτω βαπτισατε, ταυτα παντα προειποντες, βαπτισατε εις το ονομα του πατρος και του υιου και του αγιου πνευματος εν υδατι ζωντι.
εαν δε μη εχης υδωρ ζων, εις αλλο υδωρ βαπτισον, ει δ ου δυνασαι εν ψυχρω, εν θερμω.
εαν δε αμφοτερα μη εχης, εκχεον εις την κεφαλην τρις υδωρ εις ονομα πατρος και υιου καυ αγιου πνευματος.
προ δε του βαπτισματος προνηστευσατω ο βαπτιζων και ο βαπτιζομενος και ει τινες αλλοι δυνανται, κελευεις δε νηστευσαι τον βαπτιζομενον προ μιας η δυο.
Αι δε νηστειαι υμων μη εστωσαν μετα των υποκριτων. νηστευσουσι γαρ δευτερα σαββατων και πεμπτη, υμεις δε νηστευσατε τετραδα και παρασκευην.
μηδε προσευχεσθε ως οι υποκριται, αλλ ως εκελευσεν ο κυριος εν τω ευαγγελιω αυτου, ουτω προσευχεσθε, πατηρ ημων ο εν τω ουρανω, αγιασθητω το ονομα σου, ελθετω η βασιλεια σου, γενηθητω το θελημα σου ως εν ουρανω και επι γης, τον αρτον ημων τον επιουσιον δος ημιν σημερον, και αφες ημιν την οφειλην ημων, ως και ημεις αφιεμεν τοις οφειλειταις ημων, και μη εισενεγκης ημας εις πειρασμον, αλλα ρυσαι ημας απο του πονηρου, οτι σου εστιν η δυναμις και η δοξα εις τους αιωνας.
τρις της ημερας ουτω προσευχεσθε.
Περι δε της ευχαριστιας, ουτως ευχαριστησατε,
πρωτον περι του ποτηριου, ευχαριστουμεν σοι, πατερ ημων, υπερ της αγιας αμπελου Δαυιδ του παιδος σου, ης εγνωρισας ημιν δια Ιησου του παιδος σου, σοι η δοξα εις τους αιωνας.
περι δε του κλασματος, ευχαριστουμεν σοι, πατερ ημων, υπερ της ζωης και γνωσεως, ης εγνωρισας ημιν δια Ιησου του παιδος σου. σοι η δοξα εις τους αιωνας.
ωσπερ ην τουτο [το] κλασμα διεσκορπισμενον επανω των ορεων και συναχθεν εγενετο εν, ουτω συναχθητω σου η εκκλησια απο των περατων της γης εις την σην βασιλειαν, οτι σου εστιν η δοξα και η δυναμις δια Ιησου Χριστου εις τους αιωνας.
μηδεις δε φαγετω μηδε πιετω απο της ευαριστιας υμων, αλλ οι βαπτισθεντες εις ονομα κυριου, και γαρ περι τουτου ειρηκεν ο κυριος. μη δωτε το αγιον τοις κυσι.
Μετα δε το εμπλησθηναι ουτως ευχαριστησατε,
ευχαριστουμεν σοι, πατερ αγιε, υπερ του αγιου ονοματος σου, ου κατεσκηνωσας εν ταις καρδιαις ημων, και υπερ της γνωσεως και πιστεως και αθανασιας, ης εγνωρισας ημιν δια Ιησου του παιδος σου, σοι η δοξα εις τους αιωνας.
συ, δεσποτα παντοκρατορ, εκτισας τα παντα ενεκεν του ονοματος σου, τροφην τε και ποτον εδωκας τοις ανθρωποις εις απολαυσιν, ινα σοι ευχαριστησωσιν, ημιν δε εχαρισω πνευματικην τροφην και ποτον και ζωην αιωνιον δια Ιησου του παιδος σου.
προ παντων ευχαριστουμεν σοι, οτι δυνατος ει, σοι η δοξα εις τους αιωνας.
μνησθητι, κυριε, της εκκλησιας σου του ρυσασθαι αυτην απο παντος πονηρου και τελειωσαι αυτην εν τη αγαπη σου, και συναξον αυτην απο των τεσσαρων ανεμων, την αγιασθεισαν, εις την σην βασιλειαν, ην ητοιμασας αυτη, οτι σου εστιν η δυναμις και η δοξα εις τους αιωνας.
ελθετω χαρις και παρελθετω ο κοσμος ουτος. ωσαννα τω θεω Δαυιδ. ει τις αγιος εστιν, ερχεσθω, ει τις ουκ εστι, μετανοειτω, μαραν αθα, αμην.
τοις δε προφηταις επιτρεπετε ευχαριστειν, οσα θελουσιν.
[περι δε του μυρου ουτως ευχαριστησατε,
ευχαριστουμεν σοι, πατερ ημων αγιε, υπερ του μυρου, ου εγνωρισας ημιν δια Ιησου του παιδος σου, σοι η δοξα εις τους αιωνας, αμην.]
Ος αν ουν ελθων διδαξη υμας ταυτα παντα τα προειρημενα, δεξασθε αυτον,
εαν δε αυτος ο διδασκων στραφεις διδασκη αλλην διδαχην εις το καταλυσαι, μη αυτου ακουσητε, εις δε το προςθειναι δικαιοσυνην και γνωσιν κυριου, δεξασθε αυτον ως κυριον.
περι δε των αποστολων και προφητων, κατα το δογμα του ευαγγελιου ουτω ποιησατε.
πας δε αποστολος ερχομενος προς υμας.
ου μενει [ει μη] ημεραν μιαν, εαν δε η χρεια, και την αλλην, τρεις δε εαν μεινη, ψευδοπροφητης εστιν.
εξερχομενος δε ο αποστολος μηδεν λαμβανετω ει μη αρτον, εως ου αυλισθη, εαν δε αργυριον αιτη, ψευδοπροφητης εστι.
και παντα προφητην λαλουντα εν πνευματι ου πειρασετε ουδε διακρινειτε, πασα γαρ αμαρτια αφεθησεται, αυτη δε η αμαρτια ουκ αφεθησεται.
ου πας δε ο λαλων εν πνευματι προφητης εστιν, αλλ εαν εχη τους τροπους κυριου. απο ουν των τροπων γνωσθησεται ο ψευδοπροφητης και ο προφητης.
και πας προφητης οριζων τραπεζαν εν πνευματι, ου φαγεται απ αυτης, ει δε μηγε ψευδοπροφητης εστι.
πας δε προφητης διδασκων την αληθειαν, ει α διδασκει ου ποιει, ψευδοπροφητης εστι.
πας δε προφητης δεδοκιμασμενος, αληθινος, ποιων εις μυστηριον κοσμικον εκκλησιας, μη διδασκων δε ποιειν, οσα αυτος ποιει, ου κριθησεται εφ υμων, μετα θεου γαρ εχει την κρισιν, ωσαυτως γαρ εποιησαν και οι αρχαιοι προφηται.
ος δ αν ειπη εν πνευματι, δος μοι αργυρια η ετερα τινα, ουκ ακουσεσθε αυτου, εαν δε περι αλλων υστερουντων ειπη δουναι, μηδεις αυτον κρινετω.
Πας δε ο ερχομενος προς υμας εν ονοματι κυριου δεχθητω, επειτα δε δοκιμασαντες αυτον γνωσεσθε, συνεσιν γαρ εχετε δεξιαν και αριστεραν.
ει μεν παροδιος εστιν ο ερχομενος, βοηθειτε αυτω, οσον δυνασθε, ου μενει δε προς υμας ει μη δυο η τρεις ημερας, εαν η αναγκη.
ει δε θελει προς υμας καθησθαι, τεχνιτης ων, εργαζεσθω και φαγετω.
ει δε ουκ εχει τεχνην, κατα την συνεσιν υμων προνοησατε, πως μη αργος μεθ υμων ζησεται Χριστιανος.
ει δ ου θελει ουτω ποιειν, χριστεμπορος εστι, προσεχετε απο των τοιουτων.
Πας δε προφητης αληθινος, θελων καθησθαι προς υμας, αξιος εστι της τροφης αυτου.
ωσαυτως διδακαλος αληθινος εστιν αξιος και αυτος ωσπερ ο εργατης της τροφης αυτου.
πασαν ουν απαρχην γεννηματων ληνου και αλωνος, βοων τε και προβατων λαβων δωσεις την απαρχην τοις προφηταις, αυτοι γαρ εισιν οι αρχιερεις υμων.
εαν δε μη εχητε προφητην, δοτε τοις πτωχοις.
εαν σιτιαν ποιης, την απαρχην λαβων δος κατα την εντολην.
ωσαυτως κεραμιον οινου η ελαιου ανοιξας, την απαρχην λαβων δος τοις προφηταις,
αργυριου δε και ιματισμου και παντος κτηματος λαβων την απαρχην ως αν σοι δοξη, δος κατα την εντολην.
Κατα κυριακην δε κυριου συναχθεντες κλασατε αρτον και ευχαριστησατε, προεξομολογησαμενοι τα παραπτωματα υμων, οπως καθαρα η θυσια υμων η.
πας δε εχων την αμφιβολιαν μετα του εταιρου αυτου μη συνελθετω υμιν, εως ου διαλλαγωσιν, ινα μη κοινωθη η θυσια υμων.
αυτη γαρ εστιν η ρηθεισα υπο κυριου, εν παντι τοπω και χρονω προσφερειν μοι θυσιαν καθαραν, οτι βασιλευς μεγας ειμι, λεγει κυριος, και το ονομα μου θαυμαστον εν τοις εθνεσι.
Χειροτονησατε ουν εαυτοις επισκοπους και διακονους αξιους του κυριου, ανδρας πραεις και αφιλαργυρους και αληθεις και δεδοκιμασμενους, υμιν γαρ λειτουργουσι και αυτοι την λειτουργιαν των προφητων και διδασκαλων.
μη ουν υπεριδητε αυτους, αυτοι γαρ εισιν οι τετιμημενοι υμων μετα των προφητων και διδασκαλων.
ελεγχετε δε αλληλους μη εν οργη, αλλ εν ειρηνη ως εχετε εν τω ευαγγελιω, και παντι αστοχουντι κατα του ετερου μηδεις λαλειτω μηδε παρ υμων ακουετω, εως ου μετανοηση.
τας δε ευχας υμων και τας ελεημοσυνας και πασας τας πραξεις ουτω ποιησατε, ως εχετε εν τω ευαγγελιω του κυριου ημων.
Γρηγορειτε υπερ της ζωης υμων, οι λυχνοι υμων μη σβεσθητωσαν, και αι οσφυες υμων μη εκλυεσθωσαν, αλλα γινεσθε ετοιμοι, ου γαρ οιδατε την ωραν, εν η ο κυριος ημων ερχεται.
πυκνως δε συναχθησεσθε ζητουντες τα ανηκοντα ταις ψυχαις υμων, ου γαρ ωφελησει υμας ο πας χρονος της πιστεως υμων, εαν μη εν τω εσχατω καιρω τελειωθητε.
εν γαρ ταις εσχαταις ημεραις πληθυνθησονται οι ψευδοπροφηται και οι φθορεις, και στραφησονται τα προβατα εις λυκους, και η αγαπη στραφησεται εις μισος.
αυξανουσης γαρ της ανομιας μισησουσιν αλληλους και παραδωσουσι, και τοτε φανησεται ο κοσμοπλανης ως υιος θεου και ποιησει σημεια και τερατα, και η γη παραδοθησεται εις χειρας αυτου, και ποιησει αθεμιτα, α ουδεποτε γεγονεν εξ αιωνος.
τοτε ηξει η κτισις των ανθρωπων εις την πυρωσιν της δοκιμασιας, και σκανδαλισθησονται πολλοι και απολουνται, οι δε υπομειναντες εν τη πιστει αυτων σωθησονται υπ αυτου του καταθεματος.
και τοτε φανησεται τα σημεια της αληθειας, πρωτον σημειον εκπετασεως εν ουρανω, ειτα σημειον φωνης σαλπιγγος, και το τριτον αναστασις νεκρων,
ου παντων δε, αλλ ως ερρεθη, ηξει ο κυριος και παντες οι αγιοι μετ αυτου.
τοτε οψεται ο κοσμος τον κυριον ερχομενον επανω των νεφελων του ουρανου.
\end{document}
